\documentclass[12pt]{article}
\usepackage[left=.75in, right=.75in, top=.75in, bottom = .75in]{geometry}
\usepackage{amssymb, amsmath, amsfonts, mathtools, bbm}
\usepackage{array}
\usepackage{multirow}
\usepackage{cancel}
\usepackage{dashbox} % for dboxed

\newcommand{\hs}{\hspace{1pt}} % 1pt horizontal space
\newcommand{\nhs}{\hspace{-1pt}} % -1pt horizontal space
\newcommand{\hsvec}[1]{\nhs\vec{\hs #1}} % 1pt space with a \vec
\newcommand{\mss}[1]{\text{\scriptsize\(#1\)}} % math scriptsize

\newcommand{\checkedbox}{\mbox{\ooalign{$\checkmark$\cr\hidewidth$\square$\hidewidth\cr}}} % checked box
\newcommand{\crossbox}{\mbox{\ooalign{\ding{55}\cr\hidewidth$\square$\hidewidth\cr}}} % cross box
\newcommand{\dotP}{\boldsymbol \cdot}		% Dot Product
\newcommand{\dboxed}[1]{\dbox{\ensuremath{#1}}}

% \title{Classical Mechanics}
% \author{ringoffire0 }
% \date{November 2022}

\begin{document}

%----------------------------------------------------------------------------------------------------------------------------------
%----------------------------------------------------------------------------------------------------------------------------------
%----------------------------------------------------------------------------------------------------------------------------------
%----------------------------------------------------------------------------------------------------------------------------------
% Curvilinear Coordinates
\section{Curvilinear Coordinates}

% \hat{q}
\(
    % \hsvec{r}
    \boxed{ 
        \begin{aligned}
            \hsvec{r} & = r\cos\phi\sin\theta \hs \hat{x} + r\sin\phi\sin\theta \hs \hat{y} + r\cos\theta \hs \hat{z} \\
            r \hat{r} & = x\hat{x} + y\hat{y} + z\hat{z}
        \end{aligned}    
    }
    \hfill
    % \hat{q} in spherical
    \begin{aligned}
        & \hspace{-3pt} \arraycolsep=3pt\begin{array}{r c r c c c r r}
                \hat{r} & = 
                    & \tfrac{\partial}{\partial r} \hsvec{r} 
                    & = 
                    & \boxed{ \tfrac{\hsvec{r}}{r} }
                    & = 
                    & \nabla r
                    & = \tfrac{\nabla r}{\Vert \nabla r \Vert}
                    \\[5pt]
                \hat{\theta} & = 
                    & \tfrac{1}{r} \tfrac{\partial}{\partial \theta} \hsvec{r}
                    & = 
                    & \boxed{ \tfrac{\partial \hat{r}}{\partial \theta} }
                    & =
                    & r \nabla \theta
                    & = \tfrac{\nabla \theta}{\Vert \nabla \theta \Vert} 
                    \\[5pt]
                \hat{\phi} & = 
                    & \tfrac{1}{r \sin\theta} \tfrac{\partial}{\partial \phi} \hsvec{r}
                    & = 
                    & \boxed{ \tfrac{1}{\sin\theta} \tfrac{\partial \hat{r}}{\partial \phi} }
                    & =
                    & \mss{r \sin\theta} \nabla \phi
                    & = \tfrac{\nabla \phi}{\Vert \nabla \phi \Vert}
            \end{array}
    \end{aligned}
    \hfill\hs
\)

\vspace{15pt}\noindent
% \hat{q} in cartesian
% \(
%     \hfill
%     \mss{ \begin{aligned}
%         \cos\theta \hs \hat{r} - \sin\theta \hs \hat{\theta} & = \cos{2\theta} \hs \hat{z}\\
%         \sin\phi \hs \hat{r} + \cos\phi \hs \hat{\phi} & = \sin\theta \hs \hat{y} + \sin\phi \cos\theta \hs \hat{z}\\
%         \hat{\phi} & = - \sin\phi \hs \hat{x} + \cos\phi \hs \hat{y}
%     \end{aligned} }
%     \hfill
%     \Rightarrow
%     \hfill
%     % Solutions
%     \begin{aligned}
%         \hat{z} & = \boxed{ \tfrac{ \cos\theta \hs \hat{r} - \sin\theta \hs \hat{\theta} }{ \cos{2\theta} } }\\
%         \hat{y} & = \tfrac{ \sin\phi \hs \hat{r} + \cos\phi \hs \hat{\phi} }{\sin\theta}
%             - \tfrac{ \sin\phi \cos\theta }{ \sin\theta \cos{2\theta} } 
%             \left[ \cos\theta \hs \hat{r} - \sin\theta \hs \hat{\theta} \right]
%             \\
%         & = \boxed{ - \tfrac{\sin\phi \sin\theta}{\cos{2\theta}} \hs \hat{r} 
%             + \tfrac{\sin\phi \cos\theta}{\cos{2\theta}} \hs \hat{\theta} 
%             + \tfrac{ \cos\phi }{\sin\theta} \hs \hat{\phi} 
%             }
%             \\
%         \hat{x} & = \boxed{ \cot\phi \hs \hat{y} - \tfrac{ \hat{\phi} }{ \sin\phi } }
%     \end{aligned}   
%     \hfill\hs 
% \)  
\(
    \mss{ \begin{aligned}
        \cos\theta \hs \hat{r} - \sin\theta \hs \hat{\theta} & = \cos{2\theta} \hs \hat{z}\\
        \sin\phi \hs \hat{r} + \cos\phi \hs \hat{\phi} & = \sin\theta \hs \hat{y} + \sin\phi \cos\theta \hs \hat{z}\\
        \hat{\phi} & = - \sin\phi \hs \hat{x} + \cos\phi \hs \hat{y}
    \end{aligned} }
    \hfill \Rightarrow \hfill
    % Solutions
    \begin{aligned}
        1.)\ \hat{z} & = \boxed{ \tfrac{ \cos\theta \hs \hat{r} - \sin\theta \hs \hat{\theta} }{ \cos{2\theta} } } \\
        3.)\ \hat{x} & = \boxed{ \cot\phi \hs \hat{y} - \tfrac{ \hat{\phi} }{ \sin\phi } }
    \end{aligned}
    \hspace{5pt} 
    \begin{aligned}
        2.)\ \hat{y} & = \tfrac{ \sin\phi \hs \hat{r} + \cos\phi \hs \hat{\phi} }{\sin\theta}
            - \tfrac{ \sin\phi \cos\theta }{ \sin\theta \cos{2\theta} } 
            \mss{ \left[ \cos\theta \hs \hat{r} - \sin\theta \hs \hat{\theta} \right] }
            \\
        & = \boxed{ - \tfrac{\sin\phi \sin\theta}{\cos{2\theta}} \hs \hat{r} 
            + \tfrac{\sin\phi \cos\theta}{\cos{2\theta}} \hs \hat{\theta} 
            + \tfrac{ \cos\phi }{\sin\theta} \hs \hat{\phi} 
            }
    \end{aligned}
\)  

\vspace{20pt}\noindent
\begin{minipage}{.56\textwidth}
    % d\dt \hat{q}
    \(\hspace{-4pt}\arraycolsep=2pt\begin{array}{r c l c l c c}
        %r
        \tfrac{ d \hat{r} }{dt} & = 
            & \tfrac{d}{dt} ( \tfrac{ \hsvec{r} }{r} ) 
            & =
            & \tfrac{1}{r} ( \tfrac{d \vec{r}}{dt} - \tfrac{dr}{dt} \hat{r})
            & = 
            & \boxed{ \tfrac{v}{r} \big[ \hat{v} - \mss{(\hat{r} \cdot \hat{v})}\hs \hat{r} \big] }
            \\[5pt]
        & = 
            & \tfrac{d \theta}{dt} \tfrac{\partial \hat{r}}{\partial \theta} 
            & + 
            & \tfrac{d \phi}{dt} \tfrac{\partial \hat{r}}{\partial \phi}
            & = 
            & \boxed{ \tfrac{d\theta}{dt} \hs \hat{\theta} + \sin\theta \hs \tfrac{d\phi}{dt} \hs \hat{\phi} }
            \\[10pt]
        % theta
        \tfrac{ d \hat{\theta} }{dt} & = 
            & \tfrac{d\theta}{dt} \tfrac{\partial}{\partial \theta} \left( \tfrac{\partial \hat{r}}{\partial \theta} \right)
            & + 
            & \tfrac{d\phi}{dt} \tfrac{\partial}{\partial\phi} \left( \tfrac{\partial \hat{r}}{\partial\theta} \right)
            & = 
            & \boxed{ - \tfrac{d\theta}{dt} \hs \hat{r} + \cos\theta \hs \tfrac{d\phi}{dt} \hs \hat{\phi} }
            \\[10pt]
        % phi
        \tfrac{ d \hat{\phi} }{dt} & = 
            & \tfrac{d\theta}{dt} \cancel{ \tfrac{\partial \hat{\phi}}{\partial \theta} }
            & + 
            & \tfrac{d\phi}{dt} \tfrac{\partial}{\partial\phi} 
                \left( \tfrac{1}{\sin\theta} \tfrac{\partial \hat{r}}{\partial \phi} \right)
            & = 
            & - \tfrac{d\phi}{dt} \underbrace{
                    \text{\scriptsize Proj}_\text{\hs xy} \hspace{-2pt} \left( \tfrac{\hat{r}}{\sin\theta} \right)
                }_{\mss{\cos\phi\hat{x} + \sin\phi\hat{y}}}
            \\[-12pt]
        & = 
            & \multicolumn{3}{l}{ 
                - \tfrac{d\phi}{dt} \tfrac{\hat{r} - \cos\theta \hs \hat{z}}{\sin\theta}
                = 
                \boxed{ \tfrac{d\phi}{dt} \tfrac{\sin\theta \hs \hat{r} - \cos\theta \hs \hat{\theta}}{\cos{2\theta}} }
                }
    \end{array}\)
\end{minipage}
\hfill
\vline
\hfill
\begin{minipage}{.39\textwidth}
    % d/dt \vec{r}
    \(\begin{aligned}
        % r
        \tfrac{d \hsvec{r}}{dt} & = \tfrac{dr}{dt} \hat{r} 
            + r (
                \tfrac{d\theta}{dt} \tfrac{\partial \hat{r}}{\partial \theta}
                + \tfrac{d\phi}{dt} \tfrac{\partial \hat{r}}{\partial \phi}
            )
            \\
        \Aboxed{ \hsvec{v} & = \tfrac{dr}{dt} \hat{r} 
                + r \tfrac{d\theta}{dt} \hat{\theta}
                + r \sin\theta \tfrac{d\phi}{dt} \hat{\phi} }
            \\[5pt]
        % theta
        \tfrac{d \hsvec{\theta}}{dt} & = \tfrac{d\theta}{dt} \hat{\theta} 
            + \theta ( 
                - \tfrac{d\theta}{dt} \hs \hat{r} + \cos\theta \hs \tfrac{d\phi}{dt} \hs \hat{\phi}
            )
            \\
        % phi
        \tfrac{d \hsvec{\phi}}{dt} & = \tfrac{d\phi}{dt} \hat{\phi} 
            + \phi \tfrac{d\phi}{dt} ( 
                \tfrac{\sin\theta \hs \hat{r} - \cos\theta \hs \hat{\theta}}{\cos{2\theta}}
            )
            \\
        & \hs \downarrow
    \end{aligned}\)

    \vspace{2pt}\noindent
    % d/dt q
    \(\hspace{-3pt}\arraycolsep=3pt\begin{array}{r c c c l}
        % r
        \tfrac{dr}{dt} & = 
            & \tfrac{d}{dt} \mss{ ( \hsvec{r} \cdot \hsvec{r} ) }^{\frac{1}{2}} 
            & = 
            & \boxed{ \hat{r} \cdot \hsvec{v} = v_{\parallel r} }
            \\[5pt]
        % Theta
        \tfrac{d \theta}{dt} & = 
            & \nabla \theta \cdot \hsvec{v} 
            & = 
            & \boxed{ \tfrac{\hat{\theta} \hs \cdot \hsvec{v} }{r} = \tfrac{v_{\perp\theta}}{r} = \omega_\theta }
            \\[5pt]
        % Phi
        \tfrac{d \phi}{dt} & = 
            & \nabla \phi \cdot \hsvec{v}
            & = 
            & \boxed{ \tfrac{ \hat{\phi} \hs \cdot \hsvec{v} }{r \sin\theta} = \tfrac{v_{\perp\phi}}{r \sin\theta} = \omega_\phi}
    \end{array}\)
\end{minipage}

\vspace{10pt}\noindent
\(
    % Angular Momentum
    \arraycolsep=3pt\begin{array}{r c c c l}
        \hsvec{L} & = 
            & \hsvec{r} \times \hsvec{p} 
            \\[5pt]
        \boxed{ m \hsvec{r} \times \hsvec{v}_\perp } & = 
            & \boxed{ m \hsvec{r} \times \hsvec{v} }
            & = 
            & mr^2 \left( \tfrac{d\theta}{dt} \hat{\phi} - \sin\theta \tfrac{d\phi}{dt} \hat{\theta} \right) 
            \\[5pt] 
        % \underline{m r^2} \hs \tfrac{1}{r} \hs \hat{r} \times \hsvec{v} & = 
        %     & \boxed{ \underline{I} \hsvec{\omega} }
        %     & = 
        %     & \underline{mr^2} \left[ 
        %         \tfrac{v}{r} (\hat{\theta} \cdot \hat{v}) \hat{\phi} 
        %         - \tfrac{v}{r} (\hat{\phi} \cdot \hat{v}) \hat{\theta} 
        %     \right]
        %     \\[5pt]
        % I \hs \underline{ \tfrac{1}{r} \hs \hat{r} \times \hsvec{v}_\perp } & =
        %     & {I} \hs \underline{ \hsvec{\omega} }
        %     & =
        %     & {I} \hs \tfrac{v}{r} 
        %         \left[ (\hat{\theta} \cdot \hat{v}) \hat{\phi} - (\hat{\phi} \cdot \hat{v}) \hat{\theta} \right]
        %     \\[5pt]
        % {I} \hs \underline{ \tfrac{v_\perp}{r} } \hs \hat{r} \times \widehat{v_\perp} & = 
        %     & I \underline{ \omega } \hs \hat{\omega}
        %     & = 
        %     & I \hs \tfrac{v}{r} \hs ( \hat{\theta} \times \hat{\phi} = \hat{r} ) \times \hat{v} 
        %     \\[10pt]
        \boxed{ m \hsvec{r} \times \hsvec{v}_\perp } & = 
            & m \hsvec{r} \times (\hsvec{\omega} \times \hsvec{r})
            & = 
            & m \big[ \hsvec{\omega} \Vert \hsvec{r} \Vert^2 - \hsvec{r} (\hsvec{\omega} \cdot \hsvec{r}) \big]
            \\[5pt]
        &
            & \underline{ \overleftrightarrow{I} } \hsvec{\omega}
            & =
            & \underline{ m \Big[ \Vert \hsvec{r} \Vert^2 \mathbbm{1}_3 - \hsvec{r} \hs \hsvec{r}^T \Big] } \hsvec{\omega} 
    \end{array}
    \hfill
    \vline
    \hfill
    % Cross products of r, v, and w
    \begin{aligned}
        \bullet\ \hsvec{\omega} \times \hsvec{r} % & = \tfrac{1}{r} (\hat{r} \times \hsvec{v}_\perp) \times \hsvec{r} \\
        % & = (\hat{r} \cdot \hat{r}) \hsvec{v}_\perp - \hat{r} (\hsvec{v}_\perp \cdot \hat{r}) \\
        & \equiv \hsvec{v}_\perp
            \\[10pt]
        \bullet\ \hsvec{r}_\perp \times \hsvec{v} & = r_\perp^2 \hsvec{\omega} 
        %     \\[10pt]
        % \bullet\ \hsvec{v} \times \hsvec{\omega} & = 
        %     \tfrac{v_\perp}{r}\hs (v_\perp \hat{v}_\perp + v_\parallel \hat{r}) \times \hat{\omega}
        %     \\
        % & = \tfrac{v_\perp}{r}\hs ( v_\perp \hat{r} - v_\parallel \widehat{v_\perp} )
        %     \\
        % \Vert \hsvec{v} \times \hsvec{\omega} \Vert^2 & = v^2 \tfrac{v^2_\perp}{r^2}
    \end{aligned}
    \hfill\hs
\)

\vspace{15pt}\noindent
\(
    % Inertia Tensor
    \begin{gathered}
        \sum \overleftrightarrow{I} = \left[ \hs \mss{ \arraycolsep=2pt\begin{matrix}
                \sum m(y^2 + z^2) & - \sum m xy & - \sum m xz\\[4pt]
                - \sum m yx & \sum m(x^2 + z^2) & - \sum m yz\\[4pt]
                - \sum m zx  & - \sum m zy & \sum m(x^2 + y^2)
            \end{matrix} } \hs \right]
            \\[10pt]
        L_i = \sum_j I_{ij} \hs \omega^j    
    \end{gathered}
    \hfill
    % Rotational Energy
    \begin{aligned}
        E = \sum_m \frac{\Vert \hsvec{L} \Vert^2}{2I} & = \frac{1}{2} \hsvec{L} \cdot \hsvec{\omega}
            = \frac{1}{2} \sum_{ij} I_{ij} \hs \omega^j \hs \omega^i
            \\[7pt]
        \sum_m \tfrac{1}{2} m \Vert \hsvec{\omega} \times \hsvec{r} \Vert^2 
            & = \tfrac{1}{2} \mss{
            \left[ \begin{matrix}
                \hs\\
                \hs & I & \hs\\
                \hs
            \end{matrix} \right]
            \left[\begin{matrix}
                |\\
                \omega\\
                |
            \end{matrix}\right]
            \cdot
            \left[\begin{matrix}
                |\\
                \omega\\
                |
            \end{matrix}\right]
            }
    \end{aligned}
    \hfill\hs
\)

%----------------------------------------------------------------------------------------------------------------------------------
%
%
%
\newpage
\noindent
% Eccentricity(?)
\(\begin{aligned}
    \tfrac{d}{dt} \big( \hsvec{p} \times \hsvec{L} \big) &
        = \tfrac{d \hsvec{p}}{dt} \times \hsvec{L} 
        = f\mss{(r)} \hs \hat{r} \times (\hsvec{r} \times m \tfrac{d \hsvec{r}}{dt})
        \\
    & = m f\mss{(r)} \left[ 
            \hsvec{r} \hs ( \hat{r} \cdot \tfrac{d \hsvec{r}}{dt} ) 
            - \tfrac{d \hsvec{r}}{dt} ( \hat{r} \cdot \hsvec{r} )
        \right]
        \\
    & = m f\mss{(r)} \left[ 
            \hat{r} \hs \tfrac{1}{2} \tfrac{d}{dt} ( \hsvec{r} \cdot \hsvec{r} )
            - \tfrac{1}{r} \tfrac{d \hsvec{r}}{dt} r^2
        \right]
        \\
    & = m f\mss{(r)} \left[ 
            \hat{r} r \tfrac{dr}{dt}
            - r \tfrac{d \hsvec{r}}{dt}
        \right]
        \\
    & = - \tfrac{m f\mss{(r)} r}{I\mss{(r)}} \left[ 
            - \tfrac{ I\mss{(r)} }{r} \tfrac{dr}{dt} \hsvec{r} 
            + I\mss{(r)} \tfrac{d \hsvec{r}}{dt}
        \right]
        \\
    & = - \tfrac{m f\mss{(r)} r}{I\mss{(r)}} \tfrac{d}{dt} \left[ I\mss{(r)} \hsvec{r} \right]\\
    & = - m \cancel{f\mss{(r)}\hs r^2} \tfrac{d}{dt} \hat{r}
        = m k \tfrac{d}{dt} \hat{r}
        \\
    \tfrac{d}{dt} \big( \tfrac{ \hsvec{p} \times \hsvec{L} }{mk} - \hat{r} \big) & = \tfrac{d}{dt} \hsvec{e}_{\text{ccen}} = 0
\end{aligned}\)
\hfill\vline\hfill
\parbox{9cm}{
    \(
        % Acceleration
        \begin{aligned}
            \hsvec{a} = & \left[ \ddot{r} - r \dot{\theta}^2 + r \dot{\phi}^2 \tfrac{\sin^2\theta}{\cos{2\theta}} \right] \hat{r}\\
            & + \left[ r \ddot{\theta} + 2 \dot{r} \dot{\theta} - r \dot{\phi}^2 \tfrac{\tan{2\theta}}{2} \right] \hat{\theta} \\
            & + \mss{ \left[ 
                2 \dot{r} \dot{\phi} \sin\theta + 2 r \dot{\theta} \dot{\phi} \cos\theta 
                + r \ddot{\phi} \sin\theta 
                \right] } \hat{\phi}
        \end{aligned}
    \)

    \vspace{15pt}
    \(
        % Torque
        \begin{aligned}[t]
            \hsvec{\tau} & = \ \hsvec{r} \times \vec{F} \\
            & = \ (mr^2) \tfrac{ \hat{r} \times \hsvec{a} }{r} = I \hsvec{\alpha}
                \\[10pt]
            \Aboxed{ \tfrac{dv}{dt} & = a (\hat{v} \cdot \hat{a}) = \hat{v} \cdot \hsvec{a} = \tfrac{d}{dt}\Vert \hsvec{v} \Vert}
        \end{aligned}
        \hfill
        \begin{aligned}[t]
            \\[-20pt]
            \tau & = \left[ \mss{ \begin{matrix}
                \tau_x \\
                \tau_y \\ 
                \tau_z 
            \end{matrix} } \right]
                \\[5pt]
            & =^* 
                \left[ \mss{ \arraycolsep=2pt\begin{matrix}
                    0 & \tau_z & -\tau_y \\
                    -\tau_z & 0 & \tau_x \\
                    \tau_y & -\tau_z & 0
                \end{matrix} } \right]
        \end{aligned}
    \)
}

\vspace{20pt}\noindent
% Tangent Vectors/Normal
\(
    \begin{aligned}
        \lVert \hsvec{q} \times \hsvec{p} \rVert^2 & = 
            \mss{ \left| \ \begin{matrix}
                \\[-11pt]
                \hsvec{q} \cdot \hsvec{q} & \hsvec{q} \cdot \hsvec{p} \\[5pt]
                \hsvec{p} \cdot \hsvec{q} & \hsvec{p} \cdot \hsvec{p}
                \\[-12pt]
                \hs
            \end{matrix} \ \right| }
            \\[5pt]
        % & = (\hsvec{q} \times \hsvec{p}) \cdot (\hsvec{q} \times \hsvec{p})
        & = \mss{ \hsvec{q} \cdot \hsvec{p} \times (\hsvec{q} \times \hsvec{p}) }
            \\[5pt]
        \Aboxed{ \tfrac{dt}{ds} & = \tfrac{1}{v} }
    \end{aligned}
    \hfill
    \vline
    \hfill
    \begin{aligned}
        T & = \hat{v} = \tfrac{\hsvec{v}}{v} \\
        \tfrac{dT}{dt} & = \tfrac{ (\hsvec{v} \cdot \hsvec{v}) \hsvec{a} - (\hsvec{v} \cdot \hsvec{a}) \hsvec{v} }{v^3} 
            = \tfrac{ \hsvec{v} \times (\hsvec{a} \times \hsvec{v}) }{v^3}
            = \tfrac{ (\hsvec{v} \times \hsvec{a}) \times \hsvec{v} }{v^3}
            \\
        \lVert \tfrac{dT}{dt} \rVert & = \tfrac{ \sqrt{ v^2 a^2 - (\hsvec{v} \cdot \hsvec{a})^2 } }{v^2}
            = \tfrac{\lVert \hsvec{a} \times \hsvec{v} \rVert }{v^2}
            \hspace{5pt} , \hspace{10pt} \tfrac{dT}{ds} = k \hat{N}
            \\
        \hat{N} & = \tfrac{T'}{\lVert T' \rVert} 
            = \tfrac{ (\hsvec{v} \times \hsvec{a}) \times \hsvec{v} }{\lVert \hsvec{v} \times \hsvec{a} \rVert v} 
            = \hat{B} \times \hat{v}
            \\
        \hat{B} & = \tfrac{\hsvec{v} \times \hsvec{a}}{\lVert \hsvec{v} \times \hsvec{a} \rVert}
            = \widehat{v \times a}
            = \hat{v} \times \hat{N}
            \hspace{15pt} \underline{ \mss{(\hat{B} \cdot \vec{v} = 0)} }
            \\
        \tfrac{d\hat{B}}{dt} & = \tfrac{\hsvec{v} \times \dot{\hsvec{a}}}{\Vert \hsvec{v} \times \hsvec{a} \Vert }
            - \left[ \tfrac{\hsvec{v} \times \dot{\hsvec{a}}}{\Vert \hsvec{v} \times \hsvec{a} \Vert } \cdot \hat{B} \right] \hat{B}
            \hspace{5pt} , \hspace{10pt} \tau = \hat{N} \cdot \tfrac{d\hat{B}}{ds}
    \end{aligned}
    \hfill
    \boxed{
        \begin{aligned}
            \hsvec{a} & = a_T \hat{T} + a_N \hat{N} \\[5pt]
            a_T & = \hsvec{a} \cdot \hat{v} = \tfrac{dv}{dt} \\[5pt]
            a_N & = \tfrac{\Vert \hsvec{a} \times \hsvec{v} \Vert}{v} = \Vert \hsvec{a} \times \hat{v} \Vert \\[5pt]
            a^2 & = a_T^2 + a_N^2 = \Vert \tfrac{d\hsvec{v}}{dt} \Vert^2
        \end{aligned}
    }
\)

\vspace{15pt}\noindent
% Frenet trihedron - when curve is parametrized by arc length
\underline{Frenet Trihedron}\\
\(
    \begin{aligned}
        & \text{\scriptsize Differentiable (in this book)}:\ C^\infty\\
        & \text{\scriptsize No singular pts. Order 0 (Regular)}:\ \hsvec{v}(t) \neq 0\\
        & \mss{ \bullet\ \Vert \hsvec{v}(t) \Vert = c \rightarrow 1
            \ \Rightarrow\ 
            \int_s \Vert \hsvec{v}(t) \Vert \hs dt = t = \Delta s 
            }
            \\
        & \hspace{10pt} \mss{ \rightarrow s:\ \hsvec{x}(t) = \hsvec{x}(s) } \\
        & \mss{ \bullet\ \tfrac{1}{2}\tfrac{d}{dt}(\hsvec{v} \cdot \hsvec{v}) = \boxed{ \hsvec{v} \cdot \hsvec{a} = 0 } } \\
        & \text{\scriptsize No singular pts. Order 1}:\ \hsvec{a}(t) \neq 0\\
        & \mss{ \bullet\ \text{Curvature, } k \neq 0 \text{ (see right)}}
            \hspace{10pt} \mss{ \bullet\ \text{Vertex, } k' = 0 }
    \end{aligned}
    \hfill\vline\hfill
    \begin{aligned}
        & 1 = \Vert \hsvec{t} \Vert = \Vert \hsvec{n} \Vert = \Vert \hsvec{b} \Vert
            \ ,\ \ 
            0 = \hsvec{t} \cdot \hsvec{n} = \hsvec{n} \cdot \hsvec{b} = \hsvec{b} \cdot \hsvec{t}
            \\
        & \bullet\ \hsvec{v}(s) = \hsvec{t}(s) \hspace{20pt} \boxed{ \mss{ (t = n \times b) } } \\
        & \bullet\ \hsvec{a}(s) = \boxed{ \hsvec{t'}(s) = k\mss{(s)} \hsvec{n}\mss{(s)} }
            ,\ k(s) \geq 0
            \hspace{10pt} \begin{gathered}
                \text{\scriptsize(can be L or R-handed)}\\[-8pt]
                \text{\scriptsize(can be neg. if in \(\mathbb{R}^2\))}
            \end{gathered}
            \\
        & \ast\ k(s) > 0 \text{ for well defined curve with \(\hat{n}\)}\\
        & \bullet\ \boxed{ \hsvec{b} = \hsvec{t} \times \hsvec{n} }
            ,\ \ \tfrac{d}{dt} (\hsvec{b} \cdot \hsvec{b}) = \hsvec{b} \cdot \hsvec{b'} = 0
            ,\ \ \ast\ \boxed{ \hsvec{b'}(s) = \tau\mss{(s)} \hsvec{n}\mss{(s)} }
            \\
        & \bullet\ \boxed{ \hsvec{n} = \hsvec{b} \times \hsvec{t} } 
            ,\ \ \ast\ \boxed{ \hsvec{n'}(s) = -k \hsvec{t} - \tau \hsvec{b} }
            ,\ \ \ast\ \mss{ \text{t-n pl.} = \text{osculating pl.} }
    \end{aligned}
\)

% More info
\vspace{10pt} \noindent
\(\begin{aligned}
    & \bullet\ t''(s) = k' n - k^2 t - k\tau b 
        \hspace{20pt} \bullet\ b''(s) = \tau' n - \tau kt - \tau^2 b
        \hspace{20pt} \bullet\ n''(s) = -k't -\tau'b - (k^2 + \tau^2) n
        \\
    & \bullet\ |\tau| = \Vert b' \Vert 
        \hspace{20pt}
        \bullet\ \tau = -\tfrac{ ( t \times t' ) \cdot t'' }{k^2} = -\tfrac{ {t} \cdot ( {t'} \times t'' )}{\Vert t' \Vert^2} 
        \hspace{20pt} 
        \bullet\ k = \Vert t' \Vert = \tfrac{ ( b \times b' ) \cdot b'' }{\tau^2} 
        = \tfrac{ {b} \cdot ( {b'} \times b'' )}{\Vert b' \Vert^2}
        \\
    & \bullet\ n \Rightarrow k,\tau:
        \hspace{20pt}
        \ast\ \Vert n' \Vert^2 = k^2 + \tau^2
        \hspace{20pt}
        \ast\ \tfrac{ ( n \times n' ) \cdot n'' }{\Vert n' \Vert^2} = \tfrac{k' \tau - k \tau'}{k^2 + \tau^2}
        = \tfrac{ \tfrac{d}{ds} (k / \tau) }{(k/\tau)^2 + 1} = \tfrac{d}{ds} \arctan(k/\tau)
\end{aligned}\)

%----------------------------------------------------------------------------------------------------------------------------------
%----------------------------------------------------------------------------------------------------------------------------------
%----------------------------------------------------------------------------------------------------------------------------------
%----------------------------------------------------------------------------------------------------------------------------------
% Lagrangian Equations
\newpage
\section{Lagrangian Equations}

% Newton Lagrangian
\noindent
\fbox{ \(\begin{aligned}[t]
    &\mathcal{L} = T - U \ ,
        \hspace{1cm} p_i \equiv \frac{\partial \mathcal{L}}{\partial \dot{q}_i} \\[10pt]
    &\rightarrow \ F_i \equiv \frac{dp_i}{dt} = \frac{ \partial \mathcal{L} }{\partial q_i}
\end{aligned}\) }
\hspace{3cm}
% Newton Free Particle
\begin{minipage}[t]{8cm}
    \underline{Newton's Laws}:\\[15pt]
    \(\begin{aligned}
        &\mathcal{L} = \tfrac{1}{2} m \dot{ \mathbf{r} }^2 - U( \mathbf{r} ) \ ,
            \hspace{20pt} \vec{p}_r = m \dot{ \mathbf{r} }\\[10pt]
        &\rightarrow \ \boxed{ F = m \ddot{ \mathbf{r} } = - \nabla U }
    \end{aligned}\)    
\end{minipage}

% Angular Lagrangian
\vspace{20pt} \noindent
\underline{Angular}:\\[10pt]
\(\begin{aligned}
    & \mathcal{L} = \tfrac{1}{2} m \dot{r}^2 + \tfrac{1}{2} mr^2 \dot{\phi}^2 - U(r,\phi) 
        \ , \hspace{20pt} \begin{aligned}
            p_r & = m \dot{r}\\
            p_\phi & = mr^2 \dot{\phi} = I \omega = I \tfrac{v_\perp}{r}
        \end{aligned}
        \ , \hspace{20pt} 
        \arraycolsep=2pt\begin{array}{r c c c l}
            - \vec{F} & = 
                & \nabla U 
                & = 
                & \tfrac{\partial U}{\partial r} \hat{r} + \tfrac{1}{r} \tfrac{\partial U}{\partial \phi} \hat{\phi}
                \\[2pt]
            \vec{F} & = 
                & m \mathbf{\ddot{r}} 
                & = 
                & \mss{(F \cdot \hat{r})} \hat{r} + \mss{(F \cdot \hat{\phi})} \hs \hat{\phi}
        \end{array} 
        \\[10pt]
    & \hspace{-3pt} \rightarrow \ 
        \begin{aligned}
            & F_r = \boxed{ - \tfrac{\partial U}{\partial r} + m r \dot{\phi}^2 = m \ddot{r}  }
                \hspace{10pt} \text{\scriptsize(centripetal: \(\tfrac{mv^2}{r} = mr\omega^2\))}
                \\[5pt]
            & F_\phi = \boxed{ 
                    \underbrace{ - \tfrac{\partial U}{\partial \phi} }_{\tau} 
                    = \underbrace{ m r^2 \ddot{\phi} }_{I \alpha}
                    + \underbrace{ 2mr \dot{r} \dot{\phi} }_{\dot{I} \omega}
                }
                \hspace{10pt} \mss{ \left( \begin{gathered}
                    \text{``coriolis'':} \\[-3pt]
                    2m | \vec{\hs \omega} \times \vec{\hs v} | = 2m\dot{\phi}\dot{r}
                \end{gathered} \right) }
        \end{aligned} 
        \hspace{10pt}
        \vline
        \hspace{10pt}
        \mss{ \begin{aligned} 
            \dot{\phi}' & = \dot{\phi} - \omega \ , \ \ r' = r 
                \hspace{10pt} 
                \left( \hat{r} = R(\omega)\hat{r}' \ , \ \ \hat{\phi} = R(\omega)\hat{\phi}' \right)
                \\
            & \hs \downarrow
                \\
            m \mathbf{\ddot{r}} & = m \mathbf{\ddot{r}'} 
                - ( mr\omega^2 + 2mr\dot{\phi}\omega ) \hat{r} 
                + ( 2m\dot{r}\omega + mr\dot{\omega} ) \hs \hat{\phi}
                \\
            m \mathbf{\ddot{r}'} & = m \mathbf{\ddot{r}} 
                + \underbrace{ mr\omega^2 \hs \hat{r} }_{ \begin{gathered}
                    \text{\tiny centrifugal}\\[-6pt]
                    \text{\tiny force}
                \end{gathered} }
                + \underbrace{ 2m\omega (r\dot{\phi}\hs \hat{r} - \dot{r}\hs\hat{\phi}) }_\text{coriolis force}
                - \underbrace{ mr\dot{\omega} \hs \hat{\phi} }_{ \begin{gathered}
                    \text{\tiny Euler}\\[-6pt]
                    \text{\tiny force}
                \end{gathered} }
        \end{aligned} }
\end{aligned}\)

\vspace{15pt}\noindent
Note: \hspace{10pt}\(
    \boxed{
        \begin{aligned}
            \dot{\hat{r}} & = \dot{\phi} \hat{\phi} \\
            \dot{\hat{\phi}} & = - \dot{\phi} \hat{r}
        \end{aligned}
    }
    \ \rightarrow\ 
    \begin{aligned}
        \hsvec{r} & = r \hat{r} 
            \ =\ r \cos\phi \hat{x} + r \sin\phi \hat{y}
            \\
        \dot{\hsvec{r}} & = \dot{r} \hat{r} + r \dot{\phi} \hat{\phi}\\
        \ddot{\hsvec{r}} & = \ddot{r} \hat{r} + 2 \dot{r} \dot{\hat{r}} + r \ddot{\hat{r}} 
            \ =\ ( \ddot{r} - r \dot{\phi} ^2 )\hat{r} + ( 2 \dot{r} \dot{\phi} + r \ddot{\phi} ) \hat{\phi} 
    \end{aligned} 
\)

% Electrodynamic Lagrangian
\vspace{20pt} \noindent
\underline{Electromagnetic}:\\[15pt]
\(\begin{aligned}
    & \mathcal{L} = \tfrac{1}{2} m \dot{ \mathbf{r} }^2 
        - q \Big[ 
            V{\scriptstyle (t,\ \mathbf{r})} 
            - \dot{ \mathbf{r} } \dotP \vec{A}{\scriptstyle (t,\ \mathbf{r})} 
        \Big] \ ,
        \hspace{20pt} p_x = m \dot{x} + q A_x
        \\[10pt]
    & \rightarrow \begin{aligned}[t]
            m \ddot{x} + q \tfrac{d A_x}{d t} & = - q 
                \left[ 
                    \tfrac{\partial V}{\partial x} 
                    - \dot{r} \dotP \tfrac{\partial \vec{A} }{\partial x} 
                \right]
                \\[5pt]
            m \ddot{x} + q \left[ \tfrac{\partial A_x}{\partial t} + \dot{r} \dotP \nabla A_x \right] & 
                = q \left[ - \tfrac{\partial V}{\partial x} 
                + \dot{r} \dotP \tfrac{\partial \vec{A} }{\partial x} \right]
        \end{aligned}
        \hspace{10pt} \rightarrow \hspace{10pt}
        \begin{aligned}[t]
            m \ddot{x} & = q \left( 
                - \tfrac{\partial V}{\partial x} - \tfrac{\partial A_x}{\partial t}
                + \dot{r} \dotP \left[ \tfrac{\partial \vec{A}}{\partial x} - \nabla A_x \right] 
                \right)
                \\[5pt]
            & = q \left[ - \tfrac{\partial V}{\partial x} - \tfrac{\partial A_x}{\partial t} \right]
                + q \dot{y} \left[ \tfrac{\partial A_y }{\partial x} - \tfrac{\partial A_x }{\partial y} \right]
                \\
            & \hspace{15pt} + q \dot{z} \left[ \tfrac{\partial A_z }{\partial x} - \tfrac{\partial A_x }{\partial z} \right]
                \\[5pt]
            & = q E_x + q \dot{y} B_z - q \dot{z} B_y\\[3pt]
            m \ddot{x} & = q E_x + q \left[ \dot{ \mathbf{r} } \times \vec{B} \right]_x\\
            &\downarrow\\
            \Aboxed{ m \ddot{ \mathbf{r} } & = q \left( \vec{E} + \dot{ \mathbf{r} } \times \vec{B} \right) }
        \end{aligned}
\end{aligned}\)

%-----------------------------------------------------------------------------------------------------------------------------
%
%
%
% Special Relativity
\newpage\noindent
\underline{Special Relativity}:\\[15pt]
\(\begin{aligned}
    \mathcal{L} &= - \tfrac{1}{\gamma} mc^2 - U \ , 
        \hspace{4cm} \vec{p} = \gamma m \vec{v} 
        \ \rightarrow \ \gamma m\dot{x} = \frac{\partial \mathcal{L}}{\partial \dot{x}}\\[5pt]
    &= \gamma m v^2 - \gamma mc^2 - U\\[5pt]
    &= m \left( v^2 - c^2 \right) \left(1 - \tfrac{v^2}{c^2}\right)^{-1/2} - U\\[5pt]
    &\approx \tfrac{1}{2}mv^2 - (U + mc^2) \indent\indent \text{\scriptsize(when \(v \ll c\))}
\end{aligned}\)

% Hamiltonian Conservation
\vspace{30pt}\noindent
\underline{Conservation of Energy}:\\[15pt]
\(\begin{aligned}
    \frac{d \mathcal{L}}{dt} &= \sum_i \left( \frac{\partial \mathcal{L}}{\partial q_i} \frac{d q_i}{dt} 
        + \frac{\partial \mathcal{L}}{\partial \dot{q}_i} \frac{d \dot{q}_i}{dt} \right)
        + \frac{\partial \mathcal{L}}{\partial t}\\[5pt]
    &= \sum_i \left( \dot{p}_i \dot{q}_i + p_i \ddot{q}_i \right) + \frac{\partial \mathcal{L}}{\partial t}\\[5pt]
    &= \frac{d}{dt} \left( \sum_i  p_i \dot{q}_i \right) + \frac{\partial \mathcal{L}}{\partial t}
\end{aligned}\)
\hspace{10pt}
\(\rightarrow\)
\hspace{10pt}
\(\begin{aligned}
    \frac{\partial \mathcal{L}}{\partial t} &= - \frac{d}{dt} \left( \sum_i p_i \dot{q}_i - \mathcal{L} \right)\\[5pt]
    & = - \frac{d \mathcal{H}}{dt} \hspace{20pt}
        \parbox{5cm}{
            \scriptsize
            If \(\mathcal{L}\) is explicitly independent of time % 
            (implies coordinates are "na/tural"), %
            then the Hamiltonian is conserved. %   
        }
\end{aligned}\)

% Hamiltonian is Energy Total
\vspace{20pt}\noindent
\(
    \begin{aligned}
        \tfrac{1}{2} \sum_n m & \dot{r}_n^2 = \tfrac{1}{2} \sum_n m \left( \sum_i \frac{\partial r_n}{\partial q_i} \dot{q}_i \right)^2 \\[5pt]
        & = \tfrac{1}{2} \sum_{i,j} \left( m \sum_n \frac{\partial r_n}{\partial q_i} 
            \frac{\partial r_n}{\partial q_j} \right) \dot{q}_i \dot{q}_j\\[5pt]
        & = \tfrac{1}{2} \sum_i \sum_j A_{ij} \dot{q}_i \dot{q}_j\\
        T & = \tfrac{1}{2} \left( 2 \sum_{i \neq j} A_{ij} \dot{q}_i \dot{q}_j + \sum_i A_{ii} \dot{q}_i^2 \right) + ...
    \end{aligned}
    \hspace{5pt}
    \rightarrow
    \hspace{5pt}
    \begin{aligned}
        & \mathcal{L} = \tfrac{1}{2}mv^2 - U \ = \ T{\scriptstyle(\dot{q}_i)} - U{\scriptstyle(q_i)}\ \rightarrow \ \\[5pt]
        & \begin{aligned}[t]
            \mathcal{H} & = \sum_i \frac{\partial T}{\partial \dot{q}_i} \dot{q}_i - \mathcal{L}\\[5pt]
            & = \sum_i \left( \sum_j A_{ij} \dot{q}_j \right) \dot{q}_i 
                - \tfrac{1}{2} m \dot{ \mathbf{r} }^2 + U\\[10pt]
            & = \tfrac{1}{2} m \dot{ \mathbf{r} }^2 + U \indent
                \begin{minipage}{5cm}
                    \scriptsize 
                    If \(\mathcal{L}= \tfrac{1}{2}mv^2 - U\) and \(U\) is independent of \(v\), 
                    then the Hamiltonian is the total energy.   
                \end{minipage}
        \end{aligned}
    \end{aligned}
\)

\vspace{20pt}\noindent
% Lagrange -> Hamiltonian
\parbox[t]{10cm}{
    \underline{Lagrange \((x^i, v^i)\) \(\leftrightarrow\) Hamiltonian \((q^i, p_i)\)}:\\[15pt]
    \(\begin{aligned}
        & v^i\mss{(q^i, p_i)} = \frac{\partial q^i}{\partial t} = \frac{\partial H \mss{(q^i, p_i)}}{\partial p_i}
            \hspace{10pt} \text{\scriptsize(also for Newton. \(\leftarrow\) Hamil.)}
            \\[5pt]
        & \bullet\ \exists p_i \mss{(q^i \rightarrow x^i, v^i)} 
            \ \Leftarrow\ \dboxed{ \left\vert \tfrac{\partial^2 H}{\partial p_i p_j} \right\vert \neq 0 }
            \hspace{10pt} \begin{gathered}
                \text{\scriptsize(invert. + diff.)}\\[-9pt]
                \text{\scriptsize(not req. for uniq. \(H\))}\\
            \end{gathered}
            \\
        & \bullet\ \left\vert \tfrac{\partial^2 H}{\partial p_i p_j} \right\vert 
            = \left\vert \tfrac{\partial^2 L}{\partial v_i v_j} \right\vert^{-1} \neq 0 
            \hspace{10pt} \text{\scriptsize(uniq. cond. for \(L\))}
    \end{aligned}\)
}
% Lagrange Multipliers
\parbox[t]{7cm}{
    \underline{Lagrange Multipliers}:\\[15pt]
    \(\begin{aligned}
        \frac{d}{dt} \left( \frac{\partial \mathcal{L}}{\partial \dot{q}_i} \right)
            & = \frac{\partial \mathcal{L}}{\partial q_i} 
                + \sum_j \lambda_j \frac{\partial f_j}{\partial q_i}
                \\[10pt]
        \dfrac{d p}{dt} & = -\nabla U + \lambda \nabla f\\[10pt]
        F_\text{tot} & = F_\text{ncnstr} + F_\text{cnstr}
    \end{aligned}\)
}

%--------------------------------------------------------------------------------------------------------------------------
%
%
%
\newpage
% Five Examples
\subsection{Examples}

\vspace{10pt}\noindent
\underline{Atwood's Machine (Pulley)}:

\vspace{20pt}\noindent
\underline{Particle Confined to a Cylinder Surface}:

\vspace{20pt}\noindent
\underline{Block Sliding on Wedge}:

\vspace{20pt}\noindent
\underline{Bead on Spinning Wire Hoop}:

\vspace{20pt}\noindent
\underline{Oscillations of Bead Near Equilibriuum}:

%--------------------------------------------------------------------------------------------------------------------------
%--------------------------------------------------------------------------------------------------------------------------
%--------------------------------------------------------------------------------------------------------------------------
%--------------------------------------------------------------------------------------------------------------------------
% Hamiltonian Equations
\newpage
\section{Hamiltonian}
\fbox{ \(\begin{aligned}[t]
    &\mathcal{H} = \sum_i \dot{q_i} p_i - \mathcal{L} \ , 
        \hspace{1cm} p_i = \frac{\partial \mathcal{L}}{\partial \dot{q}}\\[10pt]
    &\rightarrow \ \begin{aligned}
        &\bullet \ \frac{d p_i}{dt} = - \frac{\partial \mathcal{H}}{\partial q_i}\\[10pt]
        &\bullet \ \frac{d q_i}{dt} = \frac{\partial \mathcal{H}}{\partial p_i}
    \end{aligned}
\end{aligned}\) }
\hspace{3cm}
% Newton Hamiltonian
\begin{minipage}[t]{8cm}
    \underline{Newton Particle}:\\[10pt]
    \(\begin{aligned}
        \mathcal{H} &= \dot{x} ( m \dot{x} ) - \tfrac{1}{2} m \dot{x}^2 + U(x) \\[5pt]
        &= \tfrac{1}{2} m \dot{x}^2 + U(x)\\[5pt]
        &= T + U
    \end{aligned}\)    
\end{minipage}

% Angular Hamiltonian
\vspace{20pt} \noindent
\underline{Angular}:\\[10pt]
\(\begin{aligned}
    \mathcal{H} & = m \dot{r}^2 + m r^2 \dot{\theta}^2 - \left( \tfrac{1}{2} m \dot{r}^2 
        + \tfrac{1}{2} mr^2 \dot{\theta}^2 - U(r,\theta) \right) \ ,
        \hspace{20pt} \begin{aligned}
            p_r & = m \dot{r}\\
            p_\theta & = mr^2 \dot{\theta} \ \equiv \ L = I \omega
        \end{aligned}
        \\[-5pt]
    & = \tfrac{1}{2} m \dot{r}^2 + \tfrac{1}{2} mr^2 \dot{\theta}^2 + U(r,\theta)
\end{aligned}\)

% Electrodynamic Hamiltonian
\vspace{20pt} \noindent
\underline{Electromagnetic}:\\[15pt]
\(\begin{aligned}
    & \begin{aligned}
            \mathcal{H} & = \dot{ \mathbf{r} } \dotP \vec{p}_r - \left( \tfrac{1}{2} m \dot{ \mathbf{r} }^2 
                - q \phi{\scriptstyle (t,\ \mathbf{r})} 
                + q \dot{ \mathbf{r} } \dotP \vec{A}{\scriptstyle (t,\ \mathbf{r})} \right) \ ,
                \hspace{2cm} \vec{p}_r = m \dot{ \mathbf{r} } + q \vec{A}\\[5pt]
            & = m \dot{ \mathbf{r} }^2 + q \dot{ \mathbf{r} } \dotP \vec{A}
                - \tfrac{1}{2} m \dot{ \mathbf{r} }^2 + q \phi - q \dot{ \mathbf{r} } \dotP \vec{A}\\[5pt]
            & = \tfrac{1}{2} m \dot{ \mathbf{r} }^2 + q \phi
        \end{aligned}
\end{aligned}\)

% Special Relativity
\vspace{20pt}\noindent
\underline{Special Relativity}:\\[15pt]
\(\begin{aligned}
    \mathcal{H} & = \vec{v} \dotP (\gamma m \vec{v}) - \left( \gamma m v^2 - \gamma mc^2 - U \right) \ ,
        \hspace{3cm} \vec{p} = \gamma m \vec{v}\\[5pt]
    & = \gamma mc^2 + U\\[5pt]
    & \approx \tfrac{1}{2} m v^2 + \left( U + mc^2 \right) \indent\indent \text{\scriptsize(when \(v \ll c\))}
\end{aligned}\)

% Poisson Brackets
\vspace{20pt}\noindent
\underline{Poisson Brackets}\\[10pt]
\(
    \begin{aligned}    
        & \{ f, g \} = 
            \sum_i \frac{\partial f}{\partial q_i} \frac{\partial g}{\partial p_i} 
            - \frac{\partial g}{\partial q_i} \frac{\partial f}{\partial p_i} 
            \\[5pt]
        & \bullet\ \{ q_i , g\mss{(q,p,t)} \} = \tfrac{\partial g}{\partial p_i} 
            \ , \ \{ p_i , g\mss{(q,p,t)} \} = - \tfrac{\partial g}{\partial q_i} 
            \\[5pt]
    \end{aligned}
    \ \Rightarrow\ 
    \begin{aligned}
        \bullet\ \{ f, g \} & = { \sum_i - \{ p_i, f \} \{ q_i, g \} + \{ p_i, g \} \{ q_i, f \} } \\
        & = { \sum_i\ \{ q_i, f \} \{ p_i, g \} - \{ q_i, g \} \{ p_i, f \} } \\
    \end{aligned}
\)

\vspace{5pt}\noindent
% Hamilton's Equations
\underline{\textit{Hamilton Eq.}} : \ 
\( 
    \dot{q_i} = \{q_i, H\} , \hspace{5pt} \dot{p_i} = \{p_i,H\} 
    \ \Rightarrow\  
    \{ f\mss{(q,p,t)}, H \} = 
    \sum_i \frac{\partial f}{\partial q_i} \dot{q}
    + \dot{p} \frac{\partial f}{\partial p_i} 
    = \dot{f} - \tfrac{\partial f}{\partial t}
\)

\vspace{20pt}\noindent
% Canonical Transform with Poisson Brackets
\underline{Canonical Transforms}\\[10pt]
\(
    \begin{aligned}
        & q \rightarrow \bar{q}\mss{(q,p)}\\
        & p \rightarrow \bar{p}\mss{(q,p)}
    \end{aligned} 
    \hspace{7pt} \text{ s.t. } \hspace{7pt}
    \mss{ \begin{gathered}
        \{\bar{q}_i, \bar{q}_j\} = 0 = \{\bar{p}_i, \bar{p}_j\}\\[3pt]
        \{\bar{q}_i, \bar{p}_j\} = \delta_{ij}
    \end{gathered} }
    \hspace{10pt} 
    \left(
    \begin{gathered}
        \text{\scriptsize Point Transforms}\\[-5pt]
        \text{\scriptsize \(\bar{q}\mss{(q)}\) are canonical.}
    \end{gathered}
    \right)
    \vspace{15pt}
    \ \Rightarrow\
    \begin{aligned}
        & \dot{\bar{q}} = \tfrac{\partial H}{\partial \bar{p}}\\
        & \dot{\bar{p}} = - \tfrac{\partial H}{\partial \bar{q}}\\
    \end{aligned}
    \ \ , \ \
    \{ f, g \}_{q,p} = \{ f, g \}_{\bar{q},\bar{p}}    
\)

%--------------------------------------------------------------------------------------------------------------------------
%
%
%
\newpage

% Generator of Transformation
\vspace{20pt}\noindent
\underline{Generator of Transformation}\\[10pt]
\(
    \begin{aligned}
        \{ f &, g \} = \mss{\sum_i} \ \tfrac{\partial f}{\partial q_i} { \tfrac{\partial g}{\partial p_i} } 
            - \tfrac{\partial g}{\partial q_i} { \tfrac{\partial f}{\partial p_i} } 
            \\
        % & = \mss{\sum_i} - \mss{\{ p_i, g \}}  \tfrac{\partial H}{\partial p_i} 
        %     - \dboxed{ \mss{\{ q_i, g \}} } \tfrac{\partial H}{\partial q_i} 
        %     \\
        \hspace{-1pt}\Aboxed{ 
            \tfrac{d f}{d \lambda_g} - \tfrac{\partial f}{\partial t} & \tfrac{\partial t}{\partial \lambda_g} 
            }
            \equiv \mss{\sum_i}\ \tfrac{\partial f}{\partial q_i} { \tfrac{\partial q_i}{\partial \lambda_g} } 
            + { \tfrac{\partial p_i}{\partial \lambda_g} } \tfrac{\partial f}{\partial p_i}
            \\
        \hspace{-1pt}\Aboxed{ 
            \tfrac{\partial g}{\partial t} \tfrac{\partial t}{\partial \lambda_f} - & \tfrac{d g}{d \lambda_f} 
            }
            \equiv \mss{\sum_i}\ { - \tfrac{\partial p_i}{\partial \lambda_f} } \tfrac{\partial g}{\partial p_i} 
            - \tfrac{\partial g}{\partial q_i} { \tfrac{\partial q_i}{\partial \lambda_f} } 
            \\[5pt]
        \{ g, H \} & = \mss{\sum_i} \ \dboxed{ \tfrac{\partial g}{\partial q_i} } \tfrac{\partial H}{\partial p_i} 
            - \dboxed{ \tfrac{\partial g}{\partial p_i} } \tfrac{\partial H}{\partial q_i} 
            \\
        % & = \mss{\sum_i} - \mss{\{ p_i, g \}}  \tfrac{\partial H}{\partial p_i} 
        %     - \dboxed{ \mss{\{ q_i, g \}} } \tfrac{\partial H}{\partial q_i} 
        %     \\
        & \equiv \mss{\sum_i}\ \dboxed{ - \tfrac{\partial p_i}{\partial \lambda} } \tfrac{\partial H}{\partial p_i} 
            - \dboxed{ \tfrac{\partial q_i}{\partial \lambda} } \tfrac{\partial H}{\partial q_i} 
            \\
        \Aboxed{ 
            \dot{g} - \tfrac{\partial g}{\partial t} & 
            = - \tfrac{dH}{d\lambda} + \tfrac{\partial H}{\partial t} \tfrac{\partial t}{\partial \lambda}
            }
            \\
    \end{aligned}
\)
\hfill \vline \hfill
\(
    \begin{aligned}
        & 1.\ \delta H = 0\\[-20pt]
        & 2.\ \begin{aligned}[t]
                & \begin{aligned}[t]
                        \bar{q}_i & = q_i + \delta q_i 
                            \\
                        & \equiv q_i + \epsilon_\lambda \tfrac{\partial g}{\partial p_i} \\
                        & = q_i + \epsilon_\lambda \{q_i, g\}
                    \end{aligned}
                    , \hspace{5pt}
                    \begin{aligned}[t]
                        \bar{p}_i & = p_i + \delta p_i \\
                        & \equiv p_i - \epsilon_\lambda \tfrac{\partial g}{\partial q_i} \\    
                        & = p_i + \epsilon_\lambda \{p_i, g\}
                    \end{aligned}       
            \end{aligned}
            \hspace{-10pt} \Rightarrow\ 
            \begin{gathered}
                \begin{aligned}
                        \delta H & = \epsilon_\lambda \{H, g\} \\[5pt]
                        \Aboxed{ \tfrac{\partial H}{\partial \lambda} & = 0 = - \tfrac{dg}{dt} }
                    \end{aligned}    
                    \\[5pt]
                \mss{( \text{e.g. } g=p \text{ or } g=l_z )}
            \end{gathered}    
            \\[10pt]
        & 3.\ \Rightarrow\ \delta f = \epsilon_\lambda \{f, g\} 
            \ \rightarrow\ \tfrac{d f}{d \lambda_g} - \tfrac{\partial f}{\partial t} \tfrac{\partial t}{\partial \lambda_g} = \{f, g\}
            \\[5pt]
        \hline 
            \\
        & \bullet\ g = l_z 
            \ \Rightarrow\ \begin{aligned}
                    \delta x & = - \epsilon y = - (\delta \theta) y\\
                    \delta y & = \epsilon x = (\delta \theta) x
                \end{aligned}
            \ \Rightarrow\ \boxed{ 
                \begin{aligned}
                    \tfrac{\partial x}{\partial \theta} & = - y\\
                    \tfrac{\partial y}{\partial \theta} & =  x
                \end{aligned} 
            }
    \end{aligned}
\)

%------------------------------------------------------------------------------------------------------------------------------
%------------------------------------------------------------------------------------------------------------------------------
% Hamilton-Jacobi Equations
\vspace{20pt}
\section{Hamilton-Jacobi Equations}

\noindent
\(
    \begin{aligned}
        & \boxed{ K(Q, P, t) \hs \equiv \hs H(q, p, t) + \frac{\partial S \mss{(q, Q, t)}}{\partial t} = 0 }\\[10pt]
        & \dot{q} = \tfrac{\partial H}{\partial p}
            \ \Rightarrow\ \boxed{ q = -\tfrac{\partial S}{\partial p} }
            \hspace{5pt}, \hspace{5pt} 
            \dot{p} = - \tfrac{\partial H}{\partial q}
            \ \Rightarrow\ \boxed{ p = \tfrac{\partial S}{\partial q} }
            \\[10pt]
        & \arraycolsep=2pt \begin{array}{r l l}
                \dot{Q} & = \frac{\partial K}{\partial P} = 0
                    & \ \hs \Rightarrow\ \hs 
                    \boxed{ Q = \tfrac{\partial S}{\partial P} \equiv \alpha_Q \hspace{5pt} \text{\scriptsize(constant)} }
                    \\[10pt]
                \dot{P} & = - \frac{\partial K}{\partial Q} = 0
                    & \ \hs \Rightarrow\ \hs 
                    \boxed{ P = -\tfrac{\partial S}{\partial Q} \equiv \alpha_P \hspace{5pt} \text{\scriptsize(constant)} }
            \end{array}        
            \\
    \end{aligned}
    \hfill\vline\hfill
    \begin{aligned}
        & \underline{ \text{Solve for } S \mss{(q, \alpha_Q, t)} }
            \hspace{10pt} \text{\scriptsize(\(n+1\) variables, nonlinear PDE)}
            \\[5pt]
        & H \left( q, \tfrac{\partial S (q, \hs \alpha_Q, \hs t)}{\partial q}, t \right) 
            + \frac{\partial S \mss{(q, \alpha_Q, t)}}{\partial t} 
            = 0 
    \end{aligned}
    \hfill\hs
\)

\vspace{20pt}\noindent
% If H is independant of time
\(
    \begin{aligned}
        & \tfrac{dS}{dt} = \tfrac{\partial S}{\partial t} + \mss{\sum}\hs\hs p_i \dot{q}_i + 0
            = \cancel{ \tfrac{\partial S}{\partial t} + H } + \mathcal{L}
            \\[5pt]
        & \Rightarrow\ \boxed{ S = \mss{\int} \mathcal{L} \ dt + \text{const.} } 
            \\[10pt]
        & \underline{ \tfrac{\partial H}{\partial t} = 0 }
            \ \Rightarrow\ \boxed{ S(q, Q, t) = W(q,Q) - Et }
            \\
    \end{aligned}
    \hfill\vline\hfill
    \begin{aligned}
        & \underline{ \text{Solve for } W \mss{(q, \alpha_Q)} }
            \hspace{10pt} \text{\scriptsize(\(n\) variables, nonlinear PDE)}
            \\[5pt]
        & H(q, \tfrac{\partial W}{\partial q}) = E \equiv \alpha_Q\\[5pt]
    \end{aligned}
    \hfill\hs
\)

%--------------------------------------------------------------
\vspace{30pt}\noindent
\underline{Harmonic Oscillator}\\[10pt]
\(
    \begin{aligned}
        - \tfrac{\partial S}{\partial t} & = \tfrac{1}{2} p^2 - \tfrac{1}{2} \omega^2 q^2 \\[5pt]
        & = \tfrac{1}{2} \left( \tfrac{\partial S}{\partial q} \right)^2 - \tfrac{1}{2} \omega^2 q^2
            \hspace{15pt} \big( \mss{S = s_1(q) + s_2(t)} \big)
            \\[5pt]
        - \tfrac{\partial s_2(t)}{\partial t} & 
            = \tfrac{1}{2} \left( \tfrac{\partial s_1(q)}{\partial q} \right)^2 - \tfrac{1}{2} \omega^2 q^2
            \equiv \alpha_Q
    \end{aligned}
    \hfill\vline\hfill
    \begin{gathered}
        s_2(t) = - \alpha_Q t + \cancel{ \text{const.} } 
            \hspace{10pt}, \hspace{10pt} 
            s_1(q) = \mss{\int} \sqrt{2a_Q + \omega^2 q^2} \ dq
            \\[10pt]
        Q \equiv \alpha_Q
            \hspace{10pt}, \hspace{10pt} 
            \begin{aligned}[t]
                P & = - \mss{\int} \tfrac{dq}{ \sqrt{2a_Q + \omega^2 q^2} } + t\\[5pt]
                \alpha_P & = - \tfrac{1}{\omega} \sin^{-1} \mss{ \left[ q \tfrac{\omega}{ \sqrt{2 \alpha_Q} } \right] } + t
            \end{aligned}
            \\[10pt]
        \boxed{ q(t) = \tfrac{ \sqrt{2 \alpha_Q} }{\omega} \sin \left[ \omega(t - \alpha_P) \right] }
    \end{gathered}
\)



%-----------------------------------------------------------------------------------------------------------------------------------
%-----------------------------------------------------------------------------------------------------------------------------------
%-----------------------------------------------------------------------------------------------------------------------------------
%-----------------------------------------------------------------------------------------------------------------------------------
% Kinematics
\newpage
\section{Kinematics}
% \fbox{
\begin{minipage}[t]{.6\textwidth}
    % Elastic Collisions
    \underline{Elastic Collisions}: \ \ \(
        \begin{aligned}
            p_0 & = p_1 + p_2\\
            \tfrac{p_0^2}{2m_0} & = \tfrac{p_1^2}{2m_1} + \tfrac{p_2^2}{2m_2}\\
        \end{aligned}
    \)

    \vspace{7pt}
    \(\Rightarrow\ \boxed{ 
        \tfrac{p_0^2}{2m_0} (m_0 - m_{12}) + \tfrac{p_{21}^2}{2m_{21}} (m_1 + m_2) = p_0 \cdot p_{21}
    }\)

    \vspace{10pt}\noindent
    \(
        \bullet\ \ \boxed{ \begin{aligned}[t]
            & m v_0 \ =\ m v_1 + M v_2 
                \ =\ 
                m v_0 \left( 1 - \tfrac{2M}{m+M} \right) + M v_0 \left( \tfrac{2m}{m + M} \right) 
                \\[5pt]
            & \rightarrow\
                M \in (\infty,\ m,\ 0] 
                \ \Rightarrow\ 
                v_1 \in (-v_0,\ 0,\ v_0]
        \end{aligned} }
    \)
\end{minipage}
\hfill
\begin{minipage}[t]{.35\textwidth}
    % Inelastic Collision
    \underline{Inelastic Collision}: \ \ \(E_0 = \tfrac{1}{2} m v_0^2\)

    \vspace{7pt} 
    \(\bullet\ \ \boxed{ \begin{aligned}[t]
        & m v_0 = (m + M) v_1 \\[5pt]
        & \rightarrow\ E_1 = \left( \tfrac{m}{m+M} \right) E_0
    \end{aligned}} \)
\end{minipage}

%-----------------------------------------------------------------------------------------------------------------------------------
%-----------------------------------------------------------------------------------------------------------------------------------
% Orbits
\vspace{10pt}
\section{Orbits}

% Lagrangian
\(\displaystyle
    \begin{aligned}
        \text{\underline{Lagrangian}}:\ \mathcal{L} & 
            = \mss{\sum_{1,2}}\hs \tfrac{1}{2} m_i \Vert \dot{r_i} \Vert^2
            - U\mss{({\scriptstyle \Vert r_1 - r_2 \Vert})}
            \\
        \mss{(MR = \sum m_i r_i)}\ & = \tfrac{1}{2}M \Vert R \Vert^2 + \tfrac{1}{2} \mu \Vert \dot{r} \Vert^2
            - U\mss{(\Vert r\Vert)}
            \\
    \end{aligned}
    = \tfrac{1}{2}MR^2 - U\mss{({\scriptstyle \Vert r \Vert})} +
    \mss{ \begin{aligned}
        & \tfrac{1}{2} m_1 \Vert \dot{\overline{r_1}} \Vert^2 
            + \tfrac{1}{2} m_2 \Vert \dot{\overline{r_2}} \Vert^2 
            \\
        & 
                \sum \tfrac{1}{2} m_i \dot{\overline{r_i}}^2 
                + \tfrac{1}{2} m_i \overline{r_i}^2 
                \left[ {\scriptstyle \dot{\theta_i}^2 + \sin^2{\theta_i} \dot{\phi_i}^2 } \right]
            \\[-5pt]
        & 
                \sum \tfrac{1}{2} m_i \dot{\overline{r_i}}^2 
                + \tfrac{1}{2} m_i \overline{r_i}^2 \Vert w_i \Vert^2 
                \hspace{5pt} ( \text{coplanar \(w_i\)} )
            \\
    \end{aligned} }
\)\\[10pt]
\(
    % Angular Frequency and Constrained rotation
    \bullet\ \begin{aligned}
        \overline{r_1} & = R + \tfrac{m_2}{M} r\\
        \overline{r_2} & = R - \tfrac{m_1}{M} r\\ 
    \end{aligned}
    \begin{aligned}
        % Reduced Mass Ang Momentum
        & \indent\bullet\ l = I \omega_r = \underline{ \mu r^2 } \dot{\theta}_{r}
            = \underline {
                \tfrac{m_1 m_2}{M = {\scriptstyle m_1 + m_2}} 
                \Vert {\scriptstyle r_1 - r_2} \Vert^2 
            }
            \dot{\theta}_{r}
            \\
        % Radial
        & \indent\bullet\ m\ddot{r} = - \tfrac{\partial}{\partial r} U_\text{eff} 
            = - \tfrac{\partial}{\partial r} \left[ \tfrac{l^2}{2\mu r^2} + U\mss{(\Vert r\Vert)} \right]
            \\
        % L_z and other ang momentum
        & \indent \bullet\ { 
                L_{z} = \sum m_i \overline{r_i}^2 \sin^2\theta_i \dot{ \phi_i }
                \ \Rightarrow\ \underline{L_x, L_y}^*
            }
    \end{aligned}
    \hfill
    \parbox{6cm}{\scriptsize
        * angles about the 3 axes can't be treated simultaneously as gen. coord., %
        since not independent; 2 angles per point suffice to determine position. %
        Fully describing a rigid body needs 3 trans. DOF %
        and also 3 rot. DOF. But these can't %
        be defined as rotations about Cartesian axes (see Euler angles).%
    }
\)

% Energy Hamiltonian
\vspace{15pt}\noindent
\underline{Hamiltonian}:\ \ \(E = \frac{p_r^2}{2m} + \frac{l^2}{2mr^2} + U(r)\)\\[10pt]
\indent\(\begin{aligned}
    & \bullet\ \text{Inf. Energy to get to \(r=0\) unless \(l=0\)} \\[5pt]
    & \bullet\ U \sim 1/r
\end{aligned}\)

% Orbit Types
\vspace{20pt}\noindent
\begin{minipage}[t]{.30\textwidth}
    \underline{Orbit Types}:\\[10pt]
    \(\begin{aligned}
        E > 0: & \ \ \text{Hyperbola} \\[5pt]
        E = 0: & \ \ \text{Parabola}\\[10pt]
        E < 0: & \ \ \text{Ellipse}\\[5pt]
        E = \text{Min}(U_\text{eff}): & \ \ \text{Circle}
    \end{aligned}\)    
\end{minipage}
\hfill
% Kepler's Laws
\begin{minipage}[t]{.60\textwidth}
    \underline{Kepler's Laws}:\\[10pt]
    \(\begin{aligned}
        \text{1st Law}:&\ \ \text{Elliptical Orbits} 
        \hspace{.5cm} \text{\scriptsize(Sun [at/orbiting] focus)}\\[5pt]
        \text{2nd Law}:&\ \ \text{Equal Area Sweep} 
            \hspace{.5cm} {\scriptstyle(r^2 d\theta \ =\ \frac{l}{m} dt)}\\[5pt]
        \text{3rd Law}:&\ \ T^2 = k^2a^3 
            \hspace{.5cm} {\scriptstyle \begin{aligned}[t]
                &T,\ \text{Period}\\[-5pt]
                &a,\ \text{Semi-major axis}\\[-3pt]
                &k,\ \text{``constant''} \ \left( \tfrac{2\pi}{\sqrt{G[m_\text{planet} + M_\text{sun}]}} \right)
            \end{aligned}}
    \end{aligned}\)
\end{minipage}

%-----------------------------------------------------------------------------------------------------------------------------------
%-----------------------------------------------------------------------------------------------------------------------------------
% Fluid Mechanics
\vspace{10pt}\noindent
\section{Fluid Mechanics}

\(
    \begin{array}{r c l}
        \text{Bernoulli's Principle}: & \frac{\rho v^2}{2} + \rho gz + P_\text{res}
            &=\ \text{constant \ {\scriptsize[Energy Density]}}\\[5pt]
        \text{Fluid Conservation}: & \rho A v
            &=\ \text{constant \ {\scriptsize[Mass Flow Rate]}} \\[5pt]
        \text{Bouyant Force}: & F = \rho V g & \ \ \ \ \text{\scriptsize(\(\rho, V\),\ of displaced liquid)}
    \end{array}
\)
\hspace{30pt}
\(
    \begin{aligned}
        & \underline{\text{Water Facts}}:\\[10pt]
        & \bullet\ \text{1 L} = \text{1 kg}
    \end{aligned}
\)

%-----------------------------------------------------------------------------------------------------------------------------------
%
%
%-----------------------------------------------------------------------------------------------------------------------------------
% Oscillators
\newpage
\section{Oscillators}
\subsection{Homogenous}
% Full equation
\(\begin{gathered}
    \begin{gathered}
        (F = m\ddot{x}) = -kx - \begin{gathered}[b]
                \text{\scriptsize(damp)}\\[-5pt]
                b\dot{x}
            \end{gathered}\\
        \downarrow\\
        \boxed{\ddot{x} + 2\beta \dot{x} + \omega_0^2 x = 0}\\[20pt]
    \end{gathered}
    \hspace{20pt} 
    \rule[-37pt]{.5pt}{80pt} 
    \hspace{20pt}
    \begin{aligned}
        &z_\text{tr}(t) = \tilde{C} e^{rt} + [\tilde{D}_\text{opt.}\ te^{rt}]: 
            \hspace{20pt} \underline{\text{\scriptsize \(x(t) = \text{Re} \big[ z(t) \big]\) is the real solution.}}\\[12pt]
        &\hspace{8pt} (r^2 + 2\beta r + \omega_0^2) e^{r t} = 0 \\[5pt]
        &\hspace{10pt} r = - \beta \pm \sqrt{\beta^2 - \omega_0^2}
    \end{aligned}
\end{gathered}\)

\vspace{10pt}\noindent
\begin{minipage}[t]{.49\textwidth}
    % Normal Undamped
    \underline{Normal (Undamped)}: \ \( \begin{aligned}[t]
        &\big( F = -kx \big) \ \Rightarrow \\
        &\big( \ddot{x} = -\omega_0^2 x = - \tfrac{k}{m} x \big)
    \end{aligned}\) \\[10pt]
    \indent\(\boxed{ z_\text{tr}(t) = \tilde{C}_{1} e^{i\omega_0 t} + \tilde{C}_{2} e^{-i\omega_0 t} }\)
    
    % Critically Damped
    \vspace{20pt}\noindent
    \underline{Critically Damped}:\ \(\big(\beta = \omega_0\big)\)\\[15pt]
    \indent\(\boxed{ z_\text{tr}(t) = \begin{aligned}[t]
        &\big( \tilde{C}_{1} + \tilde{C}_{2} t \big) \underline{ e^{-\beta t} }\\
        &\text{\scriptsize Decay rate is maximized at \(\beta = \omega_0\)}
    \end{aligned} }\)
\end{minipage}
\begin{minipage}[t]{.49\textwidth}
    % Underdamped
    \underline{Underdamped}:\ \(\big(\beta < \omega_0\big)\)\\[20pt]
    \indent\(\boxed{ z_\text{tr}(t) = \left( \tilde{C}_{1} e^{i\sqrt{\omega_0^2 - \beta^2} t} 
        + \tilde{C}_{2} e^{-i\sqrt{\omega_0^2 - \beta^2} t} \right)
        \underline{ e^{- \beta t} } }\)

    % Overdamped
    \vspace{20pt}\noindent
    \underline{Overdamped}:\ \(\big(\beta > \omega_0\big)\)\\[15pt]
    \indent\(\boxed{ z_\text{tr}(t) = \begin{gathered}[t]
            \underline{ \tilde{C}_{1} e^{- \left(\beta - \sqrt{\beta^2 - \omega_0^2} \right) t} }\\
            \text{\scriptsize(smaller, lasts longer)}
        \end{gathered}
        + \tilde{C}_{2} e^{- \left(\beta + \sqrt{\beta^2 - \omega_0^2}\right) t} }\) 
\end{minipage}

%------------------------------------------------------------------
% Inhomogenous
\subsection{Inhomogenous (Driven)}

% Full equation
\(\begin{gathered}
    \begin{gathered}
        m\ddot{x} = -kx - b\dot{x} + F_\text{dr}\\
        \downarrow\\
        \begin{aligned}
            \ddot{x} + 2\beta \dot{x} + \omega_0^2 x &= f_0 \cos{\omega t}\\[5pt]
            \bullet\ L\ddot{q} + R\dot{q} + \tfrac{1}{C} q &= \mathcal{E}(t)
        \end{aligned}
    \end{gathered}
    \hspace{20pt} 
    \rule[-37pt]{.5pt}{80pt} 
    \hspace{20pt}
    \begin{aligned}
        &\boxed{ z(t) = z_\text{st}(t) + z_\text{tr}(t) }\\[5pt]
        &\boxed{ z_\text{st}(t) = \tilde{C} e^{i\omega t} = A e^{i(\omega t - \delta)} }: 
            \hspace{20pt} \underline{\text{\scriptsize \(x(t) = \text{Re} \big[ z(t) \big]\) is the real solution.}}\\[12pt]
        &\hspace{10pt} (-\omega^2 + 2i \beta \omega + \omega_0^2) \tilde{C} e^{i\omega t} = f_0 e^{i\omega t}\\[10pt]
        &\hspace{10pt} \tilde{C} = \tfrac{f_0}{\omega_0^2-\omega^2 + 2i \beta \omega} = A e^{-i\delta}\\[5pt]
        &\hspace{9pt} \boxed{ A^2 = \frac{f_0^2}{(\omega_0^2-\omega^2)^2 + 4\beta^2\omega^2}\ , \ \ 
            \delta = \arctan{\left( \tfrac{2\beta\omega}{\omega_0^2-\omega^2} \right)} }
    \end{aligned}
\end{gathered}\)

% Extra info
\vspace{20pt}\noindent
\(\begin{array}{r l}
    \text{Resonance {\scriptsize(Max \(A^2\))} with fixed \(\omega\)}:    &\boxed{ \omega_0 = \omega }\\[10pt]
    \text{Resonance {\scriptsize(Max \(A^2\))} with fixed \(\omega_0\)}:  &\boxed{ \omega = \sqrt{\omega_0^2 - 2\beta^2} }
        \indent\indent \text{\scriptsize(usually \(\beta \ll \omega\))}\\[10pt]
    \text{Full Width at Half Max, \(A^2(\omega)\)}:  &\text{FWHM} \ \approx\ 2\beta\\[10pt]
    \text{Quality Factor (Sharpness)}:  &Q = \frac{\omega_0}{2\beta}
        = \left( \pi\frac{1 / \beta}{2\pi / \omega_0} = \pi \frac{\text{decay time}}{\text{period}} \right)
        = \left( 2\pi \frac{\text{Energy stored}}{\text{Energy Dissipated}} \right)
\end{array}\)

%----------------------------------------------------------------------------------------------------------------------------------
\newpage
% Parallel and Series
\subsection{Parallel and Series}

\vspace{10pt}\noindent
\underline{Series, \(k_1 {\scriptstyle+} k_2 {\scriptstyle+} m\)}:\ \
    \(\displaystyle\frac{1}{K_\text{eq}} = \frac{1}{k_1} + \frac{1}{k_2}\)\\[10pt]
\underline{Parallel, \(k_1 k_2 {\scriptstyle+} m\)}:\ \ \(K_\text{eq} = k_1 + k_2\)

%-----------------------------------------------------------------
% Normal Modes
\vspace{15pt}
\subsection{Normal Modes: 3 Springs + 2 Masses, 
    \(k_1 {\scriptstyle+} m_1 {\scriptstyle+} k_2 {\scriptstyle+} m_2 {\scriptstyle+} k_3\)}

% Start with x
\vspace{15pt}\noindent
\(\begin{aligned}    
    1.)\ \ m_1\ddot{x_1} &= -k_1 x_1 - k_2 x_1 + k_2 x_2 \\[5pt]
        &= -(k_1 + k_2) x_1 + k_2 x_2\\[15pt]
    m_2\ddot{x_2} &= k_2 x_1 - k_2 x_2 - k_3 x_2  \\[5pt]
        &= k_2 x_1 - (k_2 + k_3) x_2
\end{aligned}
\hspace{20pt}
\rightarrow
\hspace{20pt}
\begin{aligned}
    \mathbf{M\ddot{x}} &= \mathbf{-Kx}\\[10pt]
    \left(\begin{matrix}
        m_1 & 0 \\
        0 & m_2
    \end{matrix}\right)
    \left(\begin{matrix}
        \ddot{x_1}\\
        \ddot{x_2}
    \end{matrix}\right) &= 
    -\left(\begin{matrix}
        k_1 + k_2 & -k_2\\
        -k_2 & k_2 + k_3
    \end{matrix}\right)
    \left(\begin{matrix}
        x_1\\
        x_2
    \end{matrix}\right)
\end{aligned}\)

% Assume z
\vspace{15pt}\noindent
\(\begin{aligned}
    2.)\ \ \mathbf{z}(t) &= \mathbf{a} e^{i\omega t} 
    = \left(\begin{matrix}
        \tilde{a}_1\\
        \tilde{a}_2
    \end{matrix}\right) e^{i\omega t} \\[10pt]
    &= \left(\begin{matrix}
        a_1 e^{-i\delta_1 t}\\
        a_2 e^{-i\delta_2 t}
    \end{matrix}\right) e^{i\omega t}
\end{aligned}
\hspace{20pt}
\rightarrow
\hspace{20pt}
\begin{gathered}
    \begin{aligned}
        \mathbf{M\ddot{z}} &\ =\ \mathbf{-Kz}\\[5pt]
        -\omega^2 \mathbf{Ma} e^{i\omega t} &\ =\ - \mathbf{Ka} e^{i\omega t}
    \end{aligned}\\[15pt]
    \begin{aligned}
        (\mathbf{K} - \omega^2 \mathbf{M}) \mathbf{a} \ =\ 0\\[5pt]
        \boxed{ \text{det}(\mathbf{K} - \omega^2 \mathbf{M}) \ =\ 0 }
    \end{aligned}
\end{gathered}
\indent\indent \underline{\text{\scriptsize \(x(t) = \text{Re} \big[ z(t) \big]\) is the real solution.}}\)

% Special case: Same M and K
\vspace{20pt}\noindent
\underline{Same \(m\) and \(k\)} \\[5pt]
\indent \(
    \left(\begin{matrix}
        - \omega^2 m & 0 \\
        0 & - \omega^2 m
    \end{matrix}\right)
    = 
    -\left(\begin{matrix}
        2k & -k\\
        -k & 2k
    \end{matrix}\right) \ \ \rightarrow \ \ 
    \begin{aligned}
        &\boxed{ \omega = \sqrt{\frac{k}{m}}, \sqrt{\frac{3k}{m}} }
            \indent \begin{gathered}
                \text{\scriptsize Smaller \(\omega_1\) is most symmetric motion}\\[-10pt]
                \text{\scriptsize (both swing in phase)}\\[-5pt]
                \text{\scriptsize Larger \(\omega_2\) swings out of phase}
            \end{gathered}\\[10pt]
        &\boxed{ z(t) = \tilde{A}_1 \left(\begin{matrix}
                1\\
                1
            \end{matrix}\right) e^{i\omega_1 t} + 
            \tilde{A}_2 \left(\begin{matrix}
                1\\
                -1
            \end{matrix}\right) e^{i\omega_2 t} }
    \end{aligned}
\)

\vspace{20pt}\noindent
\underline{Weak Coupling}

%-----------------------------------------------------------------------------------------------------------------------------------
\newpage
% Single Pendulum
\subsection{Single Pendulum (Use Lagrangian)}

\vspace{10pt}\noindent
\hspace{10pt}\(\begin{aligned}
        \bullet\ T &= \tfrac{1}{2} m R^2 \dot{\theta}^2\\[5pt]
        \bullet\ U &= mg(R - R\cos{\theta})
    \end{aligned}
    \hspace{10pt}
    \rightarrow
    \hspace{10pt}
    \begin{aligned}
        m R^2 \ddot{\theta} &\ =\ - mgR \sin{\theta}\\
        &\ \approx\ -mgR \theta
    \end{aligned}
    \hspace{10pt}
    \rightarrow
    \hspace{10pt}
    \boxed{
        \begin{aligned}
            \ddot{\theta} & = - \left(\tfrac{g}{I/mR}\right) \theta = -\omega^2 \theta\\[5pt]
            \theta(t) &= \text{Re} \big[ C_1 e^{i\omega t} + C_2 e^{-i\omega t} \big]
        \end{aligned}
    }
\)

% Double Pendulum
\vspace{15pt}
\subsection{Double Pendulum (Use Lagrangian)}

\vspace{10pt}\noindent
\hspace{10pt}\(
    \begin{aligned}[t]
        \bullet\ T &= \tfrac{1}{2}m_1 L_1^2 \dot{\theta_1}^2 
            + \tfrac{1}{2}m_2 (L_1 \dot{\vec{\theta_1}} + L_2 \dot{\vec{\theta_2}})^2\\[5pt]
        &= \begin{aligned}[t]
                &\tfrac{1}{2}(m_1+m_2) L_1^2 \dot{\theta_1}^2 + \tfrac{1}{2}m_2 L_2^2 \dot{\theta_2}^2 \\
                &+ m_2 L_2L_2 \dot{\theta_1} \dot{\theta_2} \cos{(\theta_2-\theta_1)}
            \end{aligned}
    \end{aligned}
    \hspace{1.5cm}
    \begin{aligned}[t]
        \bullet\ U &= \begin{aligned}[t]
            &m_1 g (L_1 - L_1\cos{\theta_1}) \\
            &+ m_2 g (L_1 + L_2 - L_2\cos{\theta_2} - L_1\cos{\theta_1}) 
        \end{aligned}
    \end{aligned}
\)

% Matrix Equation
\vspace{15pt}\noindent
\(\rightarrow \begin{aligned}[t]
    \mathbf{M\ddot{\theta}} &\ = \ -\mathbf{K\theta} \indent\indent \text{\scriptsize(small angle quadratic approx.)}\\[5pt]
    \left(\begin{matrix}
        (m_1+m_2)L_1^2 & m_2 L_1 L_2 \\
        m_2 L_1 L_2 & m_2 L_2^2
    \end{matrix}\right)
    \left(\begin{matrix}
        \ddot{\theta_1}\\
        \ddot{\theta_2}
    \end{matrix}\right) &= 
        -\left(\begin{matrix}
            (m_1+m_2) g L_1 + k_2 & 0\\
            0 & m_2 g L_2
        \end{matrix}\right)
        \left(\begin{matrix}
            \theta_1\\
            \theta_2
        \end{matrix}\right)
\end{aligned}\)

\vspace{20pt}
\(
    \begin{aligned}
        & \bullet\    
            \left( \begin{matrix}
                1 & 0 \\
                -v & 1
            \end{matrix} \right)
            \left( \begin{matrix}
                t-t\\
                x_1-x_2
            \end{matrix} \right)
            = 
            \left( \begin{matrix}
                0\\
                x_1-x_2
            \end{matrix} \right)
            \\
        & \bullet\ 
            m_1 \left( \begin{matrix}
                1\\
                v_1
            \end{matrix} \right)
            + 
            m_2 \left( \begin{matrix}
                1\\
                v_2
            \end{matrix} \right)
            =
            m_1' \left( \begin{matrix}
                1\\
                v_1'
            \end{matrix} \right)
            + 
            m_2' \left( \begin{matrix}
                1\\
                v_2'
            \end{matrix} \right)
        \end{aligned}
\)


\end{document}
