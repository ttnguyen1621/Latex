\documentclass[12pt]{article}
\usepackage{enumitem}
\usepackage{amssymb}
\usepackage{amsmath}
\usepackage{esint}
\usepackage{cancel}
\usepackage[left=.75in, right=.75in, top=1in, bottom = 1in]{geometry}

\newcommand\italicmath{\mathversion{italic}}
\DeclareMathVersion{italic}
\SetSymbolFont{operators}{italic}{OT1}{cmr} {m}{it}
\SetSymbolFont{letters}  {italic}{OML}{cmm} {m}{it}
\SetSymbolFont{symbols}  {italic}{OMS}{cmsy}{m}{n}
\SetMathAlphabet\mathsf{italic}{OT1}{cmss}{m}{sl}
\SetMathAlphabet\mathit{italic}{OT1}{cmr}{m}{it}

\newcommand\bitalicmath{\mathversion{bitalic}}
\DeclareMathVersion{bitalic}
\SetSymbolFont{operators}{bitalic}{OT1}{cmr} {bx}{it}
\SetSymbolFont{letters}  {bitalic}{OML}{cmm} {b}{it}
\SetSymbolFont{symbols}  {bitalic}{OMS}{cmsy}{b}{n}
\SetMathAlphabet\mathsf{bitalic}{OT1}{cmss}{bx}{sl}
\SetMathAlphabet\mathit{bitalic}{OT1}{cmr}{bx}{it}

\newcommand*{\dotP}{\boldsymbol \cdot}				% Dot Product
\newcommand*{\contra}[1]{\boldsymbol {#1}^{\mu}}	% Contravariant vector
\newcommand*{\covar}[1]{\boldsymbol {#1}_{\mu}}		% Covariant vector
% \newcommand{\tabitem}{~~\llap{\textbullet}~~}
\newcommand{\checkedbox}{\mbox{\ooalign{$\checkmark$\cr\hidewidth$\square$\hidewidth\cr}}} % checked box

\begin{document}

\section{Lorentz Transformation}

\vspace*{10pt}
% Galilean Transform
\(\begin{aligned}[t]
	&\text{\underline{Galilean Transform}}\\[10pt]
	&\indent\begin{aligned}
		t_0' &= t_0\\[5pt]
		x_0' &= x_0 - vt_0
	\end{aligned} \ \ \ , \ \ \
	\begin{aligned}
		t_0 &= t'_0\\[5pt]
		x_0 &= x'_0 + vt'_0
	\end{aligned}
\end{aligned}
\hspace{1cm}
\rightarrow
\hspace{1cm}
% Lorentz Transform
\begin{aligned}[t]
	&\text{\underline{Lorentz Transform}} \hspace{20pt} 
		\gamma \equiv \tfrac{1}{\sqrt{1 - v^2/c^2}} = \tfrac{1}{\sqrt{1 - \beta^2}}\\[5pt]
	&\indent\boxed{\begin{aligned}
		x' &= \gamma (x - \beta ct)\\[5pt]
		ct' &= \gamma (ct - \beta x)
	\end{aligned}\ \ \ , \ \ \
	\begin{aligned}
		x &= \gamma (x' + \beta ct')\\[5pt]
		ct &= \gamma (ct' + \beta x')
	\end{aligned}}
\end{aligned}\)

\vspace{15pt}\noindent
% Define variables
\begin{minipage}[t]{.46\textwidth}
	\scriptsize
	\(x\) is the position of a point/\underline{event} occuring on the number line, and 
	\(t\) is the \underline{time on a clock} at \(x\).\\[5pt]
	\(x'\) is the position of the \underline{same} point/\underline{event} on the other number line, and
	\(t'\) is the \underline{time on the clock} at \(x'\).\\[5pt]
	\(x=0\) and \(x'=0\) are the positions of the line/reference frame origins, and 
	\(t_0\) and \(t'_0\) are the times on the origin clocks.\\[5pt]
	At \(t_0 = t'_0 = 0\), both origin's coincide at \(x=x'=0\).\\[5pt]
	All points at \(x\) see their clock run the same as \(t = t_0\), but see a different \(t'\) at the adjacent \(x'\).
\end{minipage}
% Lorentz Transform Matrix
\hfill
\begin{minipage}[t]{.49\textwidth}
	\setlength{\parindent}{.5cm}
	\noindent
	\underline{Transform Matrix}\\[10pt]
	\indent\({x^\mu}' = 
		\left( \begin{matrix}
			{x^0}' = ct'\\
			{x^1}'\\
			{x^2}'\\
			{x^3}'
		\end{matrix} \right) =
		\left(\begin{matrix} 
			\gamma & -\gamma\beta 	& 0 & 0\\
			\gamma\beta & \gamma 	& 0 & 0\\
			0 & 0 & 1 & 0\\
			0 & 0 & 0 & 1
		\end{matrix}\right) x^\mu\)
\end{minipage}

% Time slows down
\vspace{20pt}\noindent
\underline{Time Slows:} \ \ \(\boxed{ \Delta t' = \Delta t / \gamma }\)\\[10pt]
\indent\(\begin{aligned}
	&1.)\ t'(x,t_0) \text{ for a Clock at } x = X_0+vt\\[5pt]
	&\Rightarrow \ \ \begin{aligned}[t]
			ct' &= \gamma \left( ct - \beta[X_0 + vt] \right) \\[5pt]
			&= \left( \gamma ct (1 - \beta^2) - \gamma \beta X_0 \right) \\[5pt]
			&= \left( \frac{ct}{\gamma} - \gamma\beta X_0 \right)
				= \left( \frac{ct}{\gamma} + cT_0 \right)
		\end{aligned}
		\ \ \Rightarrow \ \ \begin{aligned}[t]
			&\boxed{\begin{aligned}[t]
					t'(X_0,t) &= \tfrac{t}{\gamma} + T_0 \\[5pt]
					&= \tfrac{t}{\gamma} - \gamma\beta X_0
				\end{aligned}} \indent \begin{gathered}[t]
					{\setlength{\arraycolsep}{3pt}\begin{array}[t]{r l}
							\scriptstyle T_0 > 0 : & \scriptstyle X_0 < 0\\[-5pt]
							\scriptstyle T_0 < 0 : & \scriptstyle X_0 > 0
						\end{array}}\\
					\boxed{ \begin{gathered}
						\text{\scriptsize(No \(t'\) is simultaneous to}\\[-10pt]
						\text{\scriptsize \(x\) unless in same position)}
					\end{gathered} }
				\end{gathered} \\[10pt]
			&\boxed{ \tfrac{dt'}{dt} = \tfrac{dt'}{dt_0}= \tfrac{1}{\gamma} }
				\indent\indent \boxed{ \begin{gathered}
					\text{\scriptsize(for each \(\Delta t\), then \(\Delta t' = \frac{\Delta t}{\gamma}\))}\\
					\text{\scriptsize(Clocks at \(x'\) look like they}\\[-10pt]
					\text{\scriptsize tick slower by factor \(\gamma\))}
				\end{gathered} }
		\end{aligned}\\
	&2.)\ \Delta t'_0 \text{ given } \Delta t_0\\[5pt]
	&\Rightarrow \ \ c \Delta t_0 = \gamma (c \Delta t'_0 - \cancel{ \beta x'_{=0} }) \ \ \Rightarrow \ \
		\boxed{ \Delta t'_0 = \Delta t_0 / \gamma } \indent \text{\scriptsize(Same conclusion of slowed clocks)}
\end{aligned}\)

\vspace{20pt}\noindent
\begin{minipage}[t]{.54\textwidth}
	% Length Contraction
	\underline{Length Contraction}: \ \ \(\boxed{ \Delta x' = \gamma \Delta x }\)\\[10pt]
	\indent \(1.)\ \begin{aligned}[t]
		\Delta x' = x'_2 - x'_1 \ &= \ \gamma (x_2 - \cancel{\beta ct}) - \gamma (x_1 - \cancel{\beta ct}) \\[5pt]
			&=\ \gamma (x_2 - x_1) \ =\ \gamma \Delta x \ \ \ \ \checkedbox
	\end{aligned}\)

	% Velocity addition
	\vspace{20pt}\noindent
	\underline{Velocity Addition (1-D)}: \ \ \(\boxed{ w = \frac{v + u}{1 + vu/c^2} }\)	
\end{minipage}
\begin{minipage}[t]{.45\textwidth}
	\setlength{\parindent}{.5cm}
	% Pythagorean Triples
	\noindent
	\underline{Pythag. Triples}\\[5pt]
	\indent \(\setlength{\arraycolsep}{.3pt} \begin{array}[t]{l l}
		\beta = 3/5 	&\ :\ \gamma = 5/4 = 1.25\\[5pt]
		\beta = 4/5 	&\ :\ \gamma = 5/3\\[5pt]
		\beta = 5/13 	&\ :\ \gamma = 13/12\\[5pt]
		\beta = 7/25	&\ :\ \gamma = 25/24
	\end{array}\)

	% Doppler Shift
	\vspace{15pt}\noindent
	\underline{Doppler Shift}\\[5pt]
	\(f_\text{rec} = \sqrt{\frac{1+\beta}{1-\beta}}\ f_\text{emit} 
		\indent\indent {\scriptstyle(v \text{ is } [+] \text{ if } \rightarrow \leftarrow)}\)

\end{minipage}

%------------------------------------------------------------------------------------------------------------------------------------
%------------------------------------------------------------------------------------------------------------------------------------
%------------------------------------------------------------------------------------------------------------------------------------
%------------------------------------------------------------------------------------------------------------------------------------
% 4 vectors
\newpage
\section{4-Vectors}

% 3 Vectors
\begin{minipage}[t]{0.45\textwidth}
	\textbf{3-Vectors}
	\begin{align*}
		\vec{p} &\ =\ \boxed{ \gamma m \vec{v} } \\[10pt]
		\vec{F} &\ =\ \tfrac{d\vec{p}}{dt} \ =\ m\tfrac{d(\gamma \vec{v})}{dt}\\[2pt]
		&\ =\ \gamma m\vec{a}+\gamma^3 \tfrac{(m\vec{a}\dotP\vec{v})\vec{v}}{c^2}\\[2pt]
		&\ =\ \boxed{ \gamma^3 m(\vec{a} - \tfrac{\vec{v}\dotP\vec{v}}{c^2}\vec{a} + \tfrac{\vec{a} \dotP \vec{v}}{c^2}\vec{v}) }
	\end{align*}
\end{minipage}
\hfill
% Scalars
\begin{minipage}[t]{0.49\textwidth}
	\textbf{Scalars}
	\begin{align*}
		E &\ \ =\ \ \boxed{ \sqrt{p^2c^2 + m^2c^4} \ \ =\ \ \gamma mc^2 }\\[3pt]
		T &\ \ =\ \ E - E_0 \ \ =\ \ \boxed{(\gamma-1)mc^2}\\[3pt]
		P_\text{ow} &\ \ =\ \ \tfrac{dE}{dt} \ \ =\ \ mc^2 \tfrac{d\gamma}{dt} \ \ =\ \ \tfrac{d\vec{p}}{dt} \dotP \vec{v}\\[2pt]
		&\ \ = \ \ \boxed{ \vec{F} \dotP \vec{v} \ \ = \ \ \gamma^3 m\vec{a} \dotP \vec{v} }\\[3pt]
		W &\ \ = \ \ \int \vec{F} \dotP \vec{v}\ dt \ \ =\ \ \boxed{ \int \gamma^3 m\vec{a} \dotP \vec{v}\ dt }
	\end{align*}
\end{minipage}

% Line Divider
\vspace{10pt}\noindent
\rule{1\textwidth}{.5pt}

%--------------------------------------------------------------------------------------------------------------------------------------
% Position
\vspace{-10pt}
\subsection{Position}

\vspace{10pt}
\hspace{.75cm}\(
	\begin{aligned}
		\contra{x} &\ =\ (x^0, \vec{x}) \\
		&\ =\ (ct, \vec{r}) 	
	\end{aligned}
	\ \ \Rightarrow	\ \ 
	\boxed{ \Delta \contra{x} \ =\ \contra{x}_A - \contra{x}_B } \hspace{1cm} \text{(for Event \(A,\ B)\)}
\)

\vspace{10pt}
% Position Dot Product
\indent\(\begin{aligned}
	(\Delta \contra{x})^2 &= (\Delta\contra{x})(\Delta\covar{x})\\[5pt]
	&= c^2\tau^2 \\[5pt]
	&= \boxed{ ct^2 - \vec{r}^{\ 2} }
\end{aligned}
\hspace{20pt}
\left\{\hspace{10pt}
% Time/Space Like
\begin{array}{r l}
	\text{Timelike} : 	& \ (\Delta \contra{x})^2 > 0	
		\indent \begin{gathered}
			\text{\scriptsize(\(\exists\) an inertial frame where \(A,\ B\) occur at} \\[-5pt]
			\text{\scriptsize the \underline{same spacial place} but diff. time, e.g,}\\[-2pt]
			\text{\scriptsize \((c\Delta t,0,0,0)\ \rightarrow\ (\Delta x)^2\ =\ c^2 (\Delta t)^2 > 0\))}
		\end{gathered}\\[15pt]
	\text{Spacelike} : 	& \ (\Delta \contra{x})^2 < 0	
		\indent \begin{gathered}
			\text{\scriptsize(\(\exists\) an inertial frame where \(A,\ B\) occur at} \\[-5pt]
			\text{\scriptsize the \underline{same time} but non-casual space)}
		\end{gathered}\\[10pt]
	\text{Lightlike} : 	& \ (\Delta \contra{x})^2 = 0	
		\indent \text{\scriptsize(\(A\) and \(B\) lie on a trajectory moving at \(c\))}\\
\end{array}\right.\)

% Dot Product Invariance
\vspace{15pt}
\fbox{ Relativistic Dot Products are Ref. Frame Invariant (not necessarily Conserved) }

%--------------------------------------------------------------------------------------------------------------------------------------
% Momentum
\subsection{Momentum}
\vspace{-5pt}
\begin{minipage}{0.45\textwidth}
	\begin{eqnarray*}
	\contra{p} &=& (p^0, \vec{p}) \\[5pt]
		&=& m\frac{d\contra{x}}{d\tau} \ =\  m \contra{\eta} \\[5pt]
		&=& \boxed{ (\gamma mc, \gamma m\vec{v}) \ =\  \left( \tfrac{E}{c}, \vec{p} \right) }
	\end{eqnarray*}
\end{minipage}
\hfill
% Momentum Dot Product
\begin{minipage}{0.49\textwidth}
	\(\pm \contra{p}\covar{p} \ =\ \boxed{ {\boldsymbol{p}}^2 \ =\ m^2c^2 \ =\  \left( \tfrac{E}{c} \right)^2 - \vec{p}^{\ 2}}\)

	\vspace{20pt}
	\fbox{\begin{minipage}{.8\textwidth}
		Momentum Conservation means each \\
		vector component is individually \\
		conserved
	\end{minipage}}
\end{minipage}

% Example
\vspace{20pt}
\indent \underline{Particle Decay Example: Rest \(M_1 \rightarrow m_2 + m_3\)}\\[10pt]
\indent \(\begin{aligned}
	p_3^2 &= (p_1 - p_2)^2 \\[5pt]
	&= p_1^2 + p_2^2 - 2 p_1 \dotP p_2\\[5pt]
	m_3^2c^2 &= M_1^2c^2 + m_2^2c^2 - 2 (M_1 c,0) \dotP (E_2/c,\vec{p}_2)\\[5pt]
	&= M_1^2c^2 + m_2^2c^2 - 2 M_1 E_2
\end{aligned} \ \ \Rightarrow \ \
\boxed{ \begin{aligned}
	E_2 &= \dfrac{M_1^2c^2 + m_2^2c^2 - m_3^2c^2 }{2M_1}\\[5pt]
	E_3 &= \dfrac{M_1^2c^2 + m_3^2c^2 - m_2^2c^2 }{2M_1}
\end{aligned} }\)

%--------------------------------------------------------------------------------------------------------------------------------------
% Acceleration and Force
\subsection{Acceleration and Force}
\vspace{-5pt}
\begin{minipage}{0.49\textwidth}
	\begin{eqnarray*}
	\contra{K} &=& (K^0, \vec{K}) \\
		&=& m\contra{\alpha} \ =\ m\frac{d\contra{\eta}}{d\tau} \ =\ \frac{d\contra{p}}{d\tau} \ = \ \gamma \frac{d\contra{p}}{dt}\\[5pt]
		&=& \left( \gamma \frac{d(\gamma mc)}{dt}, \gamma \frac{d(\gamma m\vec{v})}{dt} \right) \\[5pt]
		&=& \left( \tfrac{\gamma P_\text{ow}}{c}, \gamma \vec{F} \right) \ =\ 
			\left( \tfrac{\gamma \vec{F} \dotP \vec{v}}{c}, \gamma \vec{F} \right)\\[5pt]
		&=& \left( \gamma^4 \tfrac{m\vec{a} \dotP \vec{v}}{c} , \ 
			\gamma^2 m\vec{a} + \gamma^4 \tfrac{(m\vec{a}\dotP \vec{v})\vec{v}}{c^2} \right)
	\end{eqnarray*}
\end{minipage}
\hfill
\begin{minipage}{0.49\textwidth}
	\begin{eqnarray*}
	\mp \contra{\alpha}\covar{\alpha} &=& \gamma^6 \tfrac{(\vec{a} \dotP \vec{v})^2}{c^2} + \gamma^4 \vec{a}^{\ 2} \\[5pt]
	\mp \contra{K}\covar{K} &=& - \gamma^2 \tfrac{(\vec{F} \dotP \vec{v})^2}{c^2} + \gamma^2 \vec{F}^{\ 2}\\[5pt]
		&=& \gamma^2 \vec{F}^{\ 2} \left( 1-\tfrac{\vec{v}\dotP\vec{v}}{c^2}\cos{\theta_{v,F}} \right) \\[5pt]
	\contra{\alpha}\covar{\eta} &=& \frac{d\contra{\eta}}{d\tau}\covar{\eta} = \frac{1}{2} \frac{d(\contra{\eta}\covar{\eta})}{d\tau} = 0 \\[5pt]
	\contra{K}\covar{p} &=& m^2\contra{\alpha}\covar{\eta} = 0
 	\end{eqnarray*}
\end{minipage}

%--------------------------------------------------------------------------------------------------------------------------------------
% Current Density and Vector Potential
\subsection{Current Density and Vector Potential}
\begin{minipage}{0.45\textwidth}
	\begin{eqnarray*}
	\contra{J} &=& (J^0, \vec{J}) \\[5pt]
		&=& \rho_0 \frac{d\contra{x}}{d\tau} \ =\  \rho_0 \contra{\eta} \\
		&=& (\gamma c \rho_0, \gamma \rho_0 \vec{v}) \ =\  (c \rho, \vec{J}) \\ \\
	\contra{A} &=& (A^0, \vec{A}) \ =\ \left( \tfrac{V}{c}, \vec{A} \right) \\ \\
		&=& \contra{A} + \frac{\partial \lambda}{\partial \contra{x}}
	\end{eqnarray*}
\end{minipage}
\hfill
\begin{minipage}{0.49\textwidth}
	\begin{eqnarray*}
	\frac{\partial \contra{J}}{\partial \contra{x}} &=& \frac{\partial}{\partial \contra{x}} \dotP \contra{J} \ =\ \frac{\partial \rho}{dt}+\nabla \vec{J}\ = 0 \\ \\
	\Box^2 \contra{A} &=& \left( \nabla^2 - \frac{1}{c^2}\frac{\partial^2}{\partial t^2} \right) \contra{A} \ =\ -\mu_0 \contra{J}
 	\end{eqnarray*}
\end{minipage}

\end{document}