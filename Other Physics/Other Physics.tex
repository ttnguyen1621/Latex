\documentclass[12pt]{article}
\usepackage{enumitem}
\usepackage{amssymb}
\usepackage{amsmath}
\usepackage{esint}
\usepackage{cancel}
\usepackage[left=.75in, right=.75in, top=1in, bottom = 1in]{geometry}

\newcommand\italicmath{\mathversion{italic}}
\DeclareMathVersion{italic}
\SetSymbolFont{operators}{italic}{OT1}{cmr} {m}{it}
\SetSymbolFont{letters}  {italic}{OML}{cmm} {m}{it}
\SetSymbolFont{symbols}  {italic}{OMS}{cmsy}{m}{n}
\SetMathAlphabet\mathsf{italic}{OT1}{cmss}{m}{sl}
\SetMathAlphabet\mathit{italic}{OT1}{cmr}{m}{it}

\newcommand\bitalicmath{\mathversion{bitalic}}
\DeclareMathVersion{bitalic}
\SetSymbolFont{operators}{bitalic}{OT1}{cmr} {bx}{it}
\SetSymbolFont{letters}  {bitalic}{OML}{cmm} {b}{it}
\SetSymbolFont{symbols}  {bitalic}{OMS}{cmsy}{b}{n}
\SetMathAlphabet\mathsf{bitalic}{OT1}{cmss}{bx}{sl}
\SetMathAlphabet\mathit{bitalic}{OT1}{cmr}{bx}{it}

\newcommand*{\dotP}{\boldsymbol \cdot}				% Dot Product
\newcommand*{\contra}[1]{\boldsymbol {#1}^{\mu}}	% Contravariant vector
\newcommand*{\covar}[1]{\boldsymbol {#1}_{\mu}}		% Covariant vector
% \newcommand{\tabitem}{~~\llap{\textbullet}~~}
\newcommand{\checkedbox}{\mbox{\ooalign{$\checkmark$\cr\hidewidth$\square$\hidewidth\cr}}} % checked box

\begin{document}

% Charged Particle to Matter Interactions
\noindent
\underline{\(\{\pm\}\)-\{Matter\} Interactions}\\[10pt]
\(\begin{aligned}
    &\bullet\ \text{Cross Section of Ball is \(\pi R^2 \ \sim\ \) Probability of Collision} \\[5pt]
    &\bullet\ \text{\(\alpha\) Range/Path Length is \(10^{-5}\) m}\\[5pt]
    &\bullet\ \text{\(e\) Range/Path Length is \(10^{-3}\) m}\\[5pt]
    &\bullet\ \text{\(\alpha\) almost always interact with atomic-\(e\)}\\[5pt]
    &\bullet\ \text{\(e\) mostly interact with atomic-\(e\) and sometimes atomic nuclei}\\[5pt]
    &\bullet\ \text{\(\alpha\) travel in straighter lines}\\[5pt]
    &\bullet\ \text{\(e\) bounce and scatter more often}\\[5pt]
    &\bullet\ \text{\(\alpha\) lose small fractions of energy on collisions}\\[5pt]
    &\bullet\ \text{\(e\) lose large fractions of energy from collisions and bremsstrahlung \(\propto Z\)}\\[5pt]
    &\bullet\ \text{At rel. speeds \((> mc^2)\), all part. lose minimum \(\sim\)1 KeV/cm in air (min. ionizing particle)}
\end{aligned}\)

% Photon Interactions
\vspace{20pt}\noindent
\underline{Photon Interactions (Emits Electrons)}

\vspace{10pt}
\indent\(\begin{array}{c l|l|c|c|c}
    % Photoelectric Effect
    \text{Low}\ E_\gamma
        & \text{\scriptsize(few KeV)}
        & Z^4 
        & \gamma\text{-\scriptsize \{Atom\}}
        & \text{Photoelectric Absorption} 
        & E_\text{emit, \(e\)} = E_\gamma - E_\text{bind}
        \\[5pt]
    \hline
    & & & &\\[-10pt]
    % Compton Scattering
    \text{Med}\ E_\gamma
        & \begin{gathered}
                \text{\scriptsize(tens KeV to}\\[-7pt]
                \text{\scriptsize few MeV)}
            \end{gathered}
        & Z 
        & \gamma\text{-\scriptsize \{Electron\}}
        & \text{Compton Scattering}       
        & \Delta \lambda_\gamma = \frac{h}{mc} (1-\cos{\theta})
        \\[5pt]
    \hline
    & & & &\\[-5pt]
    % Pair Production
    \begin{gathered}[t]
            \text{High}\ E_\gamma\\
            \text{\scriptsize\(\big( E_\gamma > 2m_e c^2 \big)\)}
        \end{gathered} 
        & \text{\scriptsize(\(>\) tens MeV)}
        & Z^2
        & \gamma\text{-\scriptsize \{Nucleus\}}
        & \text{Pair Production} 
        & \gamma \rightarrow e^{-} + e^{+}
\end{array}\)

% Particle Energy Detection
\vspace{25pt}\noindent
\underline{Particle Energy Detection}\\[10pt]
\(\begin{aligned}
    &\bullet\ \begin{minipage}[t]{.96\textwidth}
            \(\{\pm\}\) (mainly electrons) goes past a scintillator with a high \(Z\) to increase the \(\gamma\)-interaction 
            cross section (e.g., NaI/Tl, where I has a high \(Z\))
        \end{minipage}\\[5pt]
    &\bullet\ \text{\(e\) produces visible light propto its energy when it goes past the scintillator.}\\[5pt]
    &\bullet\ \text{The visible light is directed into a photomultiplier tube, producing a current and voltage.}\\[5pt]
    &\bullet\ \text{An analyzer reads the voltage, ideally propto the original photon that produced the electron.}\\[5pt]
    &\bullet\ \text{The detector is calibrated by irradiating it with a light source of known energy.}
\end{aligned}\)

% Radioactive Decay
\vspace{20pt}\noindent
\underline{Radioactive Decay}\\[10pt]
\(\begin{aligned}
    &\bullet\ N = N_0 e^{-t/\tau}\\[5pt]
    &\bullet\ \tfrac{1}{\tau} = \sum_i \tfrac{1}{\tau_i} \indent \text{\scriptsize(multiple decay ``channels'')}
\end{aligned}\)

%--------------------------------------------------------------------------------------------------------------------------------
\newpage

% Lasers
\noindent
\underline{Lasers}\\[10pt]
\(\begin{aligned}
    &\bullet\ \begin{minipage}[t]{.96\textwidth}
            Generally, an Optical Pump provides energy to push a particle state from ground state (0) to (3), which quickly 
            drops to metastable state (2), and more slowly drops to (1). As it drops from (2) to (1), its photon is absorbed 
            by another particle in (2), which emits two photons to be absorbed by another in (2), which multiplies exponentially. 
            All photons end up with the same freq. and phase. If (0) and (1) are close enough, it's called 3-level 
            instead of 4-level.
        \end{minipage}\\[5pt]
    &\bullet\ \text{\underline{Solid State (Nd:YAG)} - Y in Al Garnet, with Nd (sometimes replacing the Y) having 4-levels}\\[5pt]
    &\bullet\ \begin{minipage}[t]{.96\textwidth}
            \underline{Collisional Gas (He-Ne)} - Collisions between the two are the level transitions, of which there can be many.
        \end{minipage}\\[5pt]
    &\bullet\ \text{\underline{Molecular Gas (CO\(_2\))} - Triangle shape provides vibrational energy levels; also is cheap}\\[10pt]
    &\bullet\ \text{Dye Laser - A dye dissolved in liquid. Chain of carbon transfers electrons (also gives color).}\\[5pt]
    &\bullet\ \begin{minipage}[t]{.96\textwidth}
            Semiconductor - Pumping excites \(e\) from the bottom of the conduction band to the holes at the top of the 
            valence band, releasing recombination radiation.
        \end{minipage}\\[5pt]
    &\bullet\ \begin{minipage}[t]{.96\textwidth}
            Free Electron - \(e\) accelerate back and forth between an \(E\)-field, releasing bremsstrahlung. No energy levels
            but there's still amplification.
        \end{minipage}\\[5pt]
    &\bullet\ \text{Interferometers - Split light in two, recombine to see interference from diff. path lengths.}
\end{aligned}\)

% Decay Types
\vspace{20pt}\noindent
\underline{Decay Types}\\[10pt]
\(\begin{array}{r l}
    \text{Strong Force -}   & \text{Usually \(10^{-23}\) sec.}\\[5pt]
    \text{EM Force -}       & \text{Usually \(10^{-18}\) - \(10^{-16}\) sec. and photons}\\[5pt]
    \text{Weak Force -}     & \text{Usually \(10^{-10}\) - \(10^{-8}\) sec. and neutrinos}
\end{array}\)

% Nuclear Forces
\vspace{20pt}\noindent
\underline{Nuclear Forces}\\[10pt]
\(\begin{aligned}
    &\bullet\ \text{Quarks are \textit{confined} to color neutral \textit{color singlets}.}\\[5pt]
    &\bullet\ \text{First proton excited state is}\ \Delta^{+} \text{.}\\[5pt]
    &\bullet\ \text{Mesons have int. spin (0,1) and Baryons have half-spins (1/2, 3/2).}\\[5pt]
    &\bullet\ \big( n \rightarrow p^{+} e^{-}\ \overline{\nu}_{e} \big) \ 
        \text{Neutron Decay takes} \sim \text{15 min., except with Protons/SF in Nucleus}\\[5pt]
    &\bullet\ {\setlength{\tabcolsep}{5pt}\begin{tabular}[t]{r c r l}
        Nuclear Diameter        &\(\sim\) &femptometers &(10\(^{-15}\) m).\\
        Atomic Diameter         &\(\sim\) &picometers   &(10\(^{-12}\) m).\\
        Solid Atoms Distance    &\(\sim\) &angstroms    &(10\(^{-10}\) m).\\
        Air Atoms Distance      &\(\sim\) &nanometers   &(10\(^{-9}\) m).\\
        Bacteria Width          &\(\sim\) &micrometers  &(10\(^{-6}\) m).
    \end{tabular}}\\[5pt]
    &\bullet\ \text{Gamma radiation comes from excited nuclei states and don't change their composition.}\\[5pt]
    &\bullet\ \text{Heavy Nuclei Decay to Lead.}
\end{aligned}\)

%--------------------------------------------------------
\newpage

% Symmetry/Cons/Other
\noindent
\underline{Symmetries, Conservation, and Recent Discoveries}\\[10pt]
\(\begin{aligned}
    &\bullet\ \text{Conservation of Baryon Number and Each of the Three Lepton Numbers.}\\[5pt]
    &\bullet\ \text{Charge, Baryon, and Lepton cons. are symmetries by some multiplication of phase \(e^{i\alpha}\).}\\[5pt]
    &\bullet\ \{\alpha\}\ \text{is cont.}\ \rightarrow\ \text{it is a Continuous Symmetry.}\\[5pt]
    &\bullet\ \alpha \neq \alpha(x) \rightarrow\ \text{it is a Global Symmetry.}\\[5pt]
    &\bullet\ \alpha = \alpha(x)\ \rightarrow\ \text{it is a Guage Symmetry.}\\[5pt]
    &\bullet\ C,P,T\ \text{ are Discrete Symmetries.}\\[5pt]
    &\bullet\ \text{All Lorentz Invariant Theories must be symmetric under \(CPT\).}\\[5pt]
    &\bullet\ \text{Weak Interaction is Maximumally Parity-Violating (only L-hand decays, shown in \(^{60}\)Co decay).}\\[15pt]
    &\bullet\ \text{Higgs Boson Mass}\ =\ \text{125 GeV.}\\[5pt]
    &\bullet\ \text{Neutrino Oscillations show Mass Differences (not exact values).}\\[5pt]
    &\bullet\ \begin{minipage}[t]{.96\textwidth}
            Dark Matter might be WIMPS and might interact with WF; galaxies spin faster + Bullet Cluster.
        \end{minipage}
\end{aligned}\)

% Crystal Structure
\vspace{20pt}\noindent
\underline{Crystal Structure (fuckin hell)}\\[10pt]
\(\begin{aligned}
    &\bullet\ \text{14 Bravias Lattices, but only these three conventional unit cells are important here:}\\[5pt]
    &\begin{tabular}{r c c c}
                                        & Vertex Dist. & Primative Unit (idgi) 
                & Volume (idrk) \\[5pt]
            \hline\\[-7pt]
            Simple Cubic               & \(a\)                         & Cube  
                & \(V=a^3\) \\[5pt]
            Body-centered cubic (BCC)  & \(\tfrac{\sqrt{3}}{2} \ a \)  & Octahedron (2 square pyramids) 
                & \(V/2\) \\[5pt]
            Face-centered cubic (FCC)  & \(\tfrac{\sqrt{2}}{2} \ a\)   & Parallelpiped                     
                & \(V/4\)
        \end{tabular}\\[10pt]
    &\bullet\ \text{The Fourier Transform of a lattice is it's \textit{reciprocal} or \textit{dual}.}\\[5pt]
    &\bullet\ \text{Dual Side Length } = \frac{2\pi}{a}\\[5pt]
    &\bullet\ \text{Dual(Simple Cubic) = Itself}\\[5pt]
    &\bullet\ \text{Dual(BCC) = FCC; Dual(FCC) = BCC.}\\[5pt]
    &\bullet\ \text{Dual(Hexagonal Lattice [HL]) = HL rotated through a 30\(^\circ\) angle.}\\[5pt]
    &\bullet\ \text{Primitive Unit of Dual\((X)\) is called the \textit{(1st) Brillouin Zone} of \(X\).}
\end{aligned}\)

%----------------------------------------------------------------------------
\newpage
% Electrons in Metals
\noindent
\underline{Electrons in Metals (Fermi)}\\[10pt]
\(\begin{aligned}
    &\bullet\ \text{\(e\) have Momentum in Fermi Sphere/Sea, and free \(e\) in Fermi Surface with \(|k| = k_F\).}\\[5pt]
    &\bullet\ k_F = (3\pi^2)^{1/3} n^{1/3} \indent {\scriptstyle(n\ =\ \text{\scriptsize electron density})}\\[5pt]
    &\bullet\ \text{Fermi Energy:}\ \ E_F = \frac{\hbar^2 k_F^2}{2m}\\[5pt]
    &\bullet\ \begin{gathered}[t]
        \text{Density of States:}\\
        \text{\scriptsize(\# \(e\) States for E)}
    \end{gathered} \ \ \
        \rho(E) = \left( \tfrac{V \sqrt{2}}{\pi^2 \hbar^3} \right) m^{3/2}\ E^{1/2}\\[5pt]
    &\bullet\ \text{Total \# \(e\):}\ \ N = \int_0^{E_F} \rho\ dE \ \ \Rightarrow\ \ 
        \rho(E_F) = \frac{3}{2} \frac{N}{E_F}\\[5pt]
    &\bullet\ \text{\# Conducting \(e\):}\ \ N_C \ \approx\ \rho(E_F) k_b T \ \sim\ \frac{N k_b T}{E_F}
\end{aligned}\)

% Semiconductors/Superconductors
\vspace{20pt}\noindent
\underline{Semiconductors/Superconductors}\\[10pt]
\(\begin{aligned}
    &\bullet\ \text{Energy is needed for \(e\) to cross the band gap from valence to conduction.}\\[5pt]
    &\bullet\ \text{Higher \(T\) thus needed to increase conductivity and electron energy/density.}\\[5pt]
    &\bullet\ \begin{minipage}[t]{.96\textwidth}
            n-type doping/junction: 4 \(e^{-}\) valence in Si/Ge/Sn (group 14), 
            doped with 5 \(e\) excess P/As/Sb (group 15).
        \end{minipage}\\[5pt]
    &\bullet\ \text{p-type doping/junction: \(e\) Hole\(^+\) excess by doping (group 14) 
        with 3 valence B/Al (group 13).}\\[5pt]
    &\bullet\ \begin{minipage}[t]{.96\textwidth}
            Meissner Effect: Persistant \(I\) on surface cancel internal \(B\) field exponentially from surface. Akin to
            giving a photon mass to decrease the EM force strength.
        \end{minipage}\\[5pt]
    &\bullet\ \text{Cooper Pair (BCS Theory): \(e\) are Paired together (like bosons) to get lower-than-\(E_F\) energies.}
\end{aligned}\)

% Astrophysics
\vspace{20pt}\noindent
\underline{Astrophysics}\\[10pt]
\(\begin{aligned}
    &\bullet\ ds^2 = dt^2 - a(t)^2 \big( dx^2 + dy^2 + dz^2 \big) 
        \indent\indent \begin{gathered}
            \scriptstyle a\ \text{\scriptsize is scale factor of COSMO, not matter-filled space} \\[-10pt]
            \scriptstyle(\text{\scriptsize space expanding}\ \rightarrow\ a\ \text{\scriptsize is increasing func. of } t)
        \end{gathered}\\[5pt]
    &\bullet\ \text{Cosmo-Expansion \(\gamma\) Redshift:}\ \ 
        \frac{\lambda_0}{\lambda_T} = \frac{a(\text{today})}{a(T)} 
        \indent\indent T =\ \text{time before}\\[5pt]
    &\bullet\ \text{Blackbody Cooling:}\ \ 2\lambda_\text{max} \ \propto\ \frac{1}{T/2} 
        \indent\indent \text{\scriptsize(cosmos 2x means temp. 1/2x)}\\[5pt]
    &\bullet\ \text{Hubble's Law:}\ \ v_\text{recess.} = H_0{\scriptstyle(t)} D
        \indent\indent \text{\scriptsize(2x distance is 2x speed)}\\[5pt]
    &\bullet\ \text{Redshift as Time:}\ \ z(T) = \tfrac{\lambda_0}{\lambda(T)} - 1
        \indent\indent \text{\scriptsize(at ``redshift 3'' means time in past when \(\lambda\) was 400\% smaller)}\\[5pt]
    &\bullet\ \text{Cosmo \(\neq\) Doppler Redshift:}\ \ \begin{aligned}[t]
            \text{Speed} &\leftrightarrow \text{Doppler}\\
            \text{Distance} &\leftrightarrow \text{Cosmo}
        \end{aligned}
\end{aligned}\)

%-----------------------------------------------------------------
\newpage
% Nobel Prizes
\noindent
\underline{Nobel Prizes}\\[10pt]
\begin{tabular}{l}
    \(\bullet\) 2017 (Gravity Wave Detection): LIGO with 10\(^{-18}\)m shift with laser crossing many times.\\[5pt]
    \(\bullet\) 1993 (Gravity Wave Indirect): Measured Pulsar Pair Orbit energy loss from GW.\\[5pt]
    \(\bullet\) \begin{minipage}[t]{.96\textwidth}
            2015 (Neutrino Oscillations): WF Hamiltonian (Flavor) doesn't commute with Free Particle (momentum/mass). 
            Sun makes \(\nu_e\), but less are detected on Earth.
        \end{minipage}\\[20pt]
    \(\bullet\) 2014 (Blue LED): GaN semiconductor has direct band gap of (3.4 \(>\) 3) eV, for UV light. 
\end{tabular}

% Numbers to Memorize
\vspace{20pt}\noindent
\underline{Memorize Numbers/Facts}\\[10pt]
\(\begin{aligned}[t]
    &\bullet\ c = \text{3E8 m/s}\\
    &\bullet\ h = \text{6.626 E-34 Js}\\
    &\bullet\ \hbar c = \text{200 MeV} \dotP \text{fm}\\
    &\bullet\ \alpha = 1/137 = .0073\\
    &\bullet\ a_0 = \text{ 42.9 pm }\\
    &\bullet\ m_e = \text{.511 MeV} / c^2\\
    &\bullet\ E_n = -13.6 Z \tfrac{Z^2}{n^2}\ \text{eV} \\
    &\bullet\ n_A = \text{6.02E23}\\
    &\bullet\ k_b: \text{300K} = \tfrac{1}{40} \text{eV}\\
    &\bullet\ \text{1 atm}\ \approx \ \text{100 kPa}\\
    &\bullet\ \text{1 atm},\ \text{300K}\ \leftrightarrow\ \text{22.4 L}\\
    &\bullet\ \text{1 L} = \text{1E3 cm\(^3\) or mL}\\
    &\bullet\ 1\ \tfrac{\text{g}}{\text{cm}^3} = \text{1E3}\ \tfrac{\text{kg}}{\text{m}^3}\\
    &\bullet\ c_\text{water} = 4.18\ \tfrac{\text{J/K}}{g}\\
    &\bullet\ 1\ \text{cal} = 4.18\ \text{J}\\
    &\bullet\ v_\text{sound} = \text{340 m/s}\\[3pt]
    &\bullet\ \text{Rel. IG}:\ C_V = 3 k_b\\
    &\bullet\ T_\text{CMB} = \text{2.725 K}\ ; \ \ \lambda_\text{CMB} = \text{1.9 mm}\\
    &\bullet\ T_\text{Sun} = 5000\ \text{K} ; \ \ \lambda_\text{Sun} = \text{550 nm}\\
    &\bullet\ \text{BE}_\text{Nucleon} \ \sim \ \text{MeV}; \ \ E_\text{part. accel} \ \sim \ \text{GeV}\\
    &\bullet\ E_\text{UV} \ \sim \ \text{eV}; \ \ E_\text{X-ray} \ \sim \ \text{keV}
\end{aligned}
\hspace{2cm}
\begin{aligned}[t]
    &\bullet\ E_n = \tfrac{-\alpha^2 (Zm_e)c^2}{2} \tfrac{Z^2}{n^2} = \tfrac{\hbar^2 k_n^2}{2m_e} Z \tfrac{Z^2}{n^2}\\
    &\bullet\ \alpha = \tfrac{kqq}{\hbar c}\\
    &\bullet\ a_0 = \tfrac{\hbar^2}{m_e kqq}\\
    &\bullet\ T^2 \ \propto\ a^3\\
    &\bullet\ D\sin{\theta} = 1.22\lambda\\
    &\bullet\ \lambda_\text{max} = \tfrac{\text{2.9E-3 m\(\dotP\)K}}{T}\\
    &\bullet\ K = C + 273.15^\circ\\
    &\bullet\ v_\text{sound}^2 = {\tfrac{K}{\rho}} = {\tfrac{\gamma P}{\rho}} 
        = {\gamma \tfrac{NkT/V}{Nm/V}} = \tfrac{\gamma kT}{m}
\end{aligned}\)

\vfill\noindent
\rule{1\textwidth}{.5pt}
\end{document}