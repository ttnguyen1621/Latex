\documentclass[12pt]{article}
\usepackage[left=.75in, right=.75in, top=1in, bottom = 1in]{geometry}
\usepackage{amssymb, amsmath, amsfonts, mathtools, empheq}
\usepackage{array}
\usepackage{multirow}
\usepackage{cancel,slashed}
\newcommand{\tabitem}{~~\llap{\textbullet}~~}
\newcommand{\checkedbox}{\mbox{\ooalign{$\checkmark$\cr\hidewidth$\square$\hidewidth\cr}}} % checked box
\newcommand*{\dotP}{\boldsymbol \cdot}	% dot product
\newcommand{\hs}{\hspace{1pt}} % 1pt horizontal space
\newcommand{\mathscriptsize}[1]{\text{\scriptsize\(#1\)}}
\newcommand{\mss}[1]{\mathscriptsize{#1}}

% \title{Quantum Mechanics}
% \author{ringoffire0 }
% \date{November 2022}

\begin{document}

% Waves
\section{Wave Function}

\vspace{5pt}\noindent
\(\begin{aligned}
    \Psi_p &= e^{ i( 2 \pi x/ \lambda - 2 \pi t/T ) }\\
    &= e^{ i(k x- \omega t) }\\
    &= e^{ \frac{i}{\hbar} ( p x - E t ) }
\end{aligned}\)
\hspace{27pt} \vline \hspace{15pt}
% Momentum/Energy Operators
\begin{tabular}{c m{.5cm} c}
    \( \breve{p} \Psi_p = p \Psi_p = \hbar k \Psi_p \) 
        &
        & \( \breve{E} \Psi_p = E \Psi_p = \hbar \omega \Psi_p \)
        \\[5pt] 
    \( \boxed{ \breve{p} = \dfrac{\hbar}{i} \frac{\partial}{\partial x} } \) 
        &
        &\( \boxed{ \breve{E} = -\dfrac{\hbar}{i} \frac{\partial}{\partial t} } \)
\end{tabular}

\vspace{10pt}\noindent
\(
    \begin{aligned}
        & \bullet\ \boldsymbol{| f \rangle} \equiv \mss{\int} f\mss{(x)} \hs |x\rangle \hs dx \\[5pt]
        & \bullet\ \boldsymbol{ \langle x | x' \rangle } \equiv \delta\mss{(x - x')} \\[5pt]
        & \bullet\ \boldsymbol{ \langle x | \hat{x} | x' \rangle } \equiv x \langle x | x' \rangle 
    \end{aligned}
    \hfill\vline\hfill
    \begin{aligned}
        % xxf
        & 1.\ \begin{aligned}[t]
                \langle x | \hat{x} | f \rangle & = x f(x)\\
                \Aboxed{ \breve{x} \langle x | f \rangle & \equiv x \langle x | f \rangle }
            \end{aligned}
            \\[5pt]
        % xpx
        & 2.\ \begin{aligned}[t]
                \boldsymbol{ \langle x | \hat{p} | x' \rangle }
                    & \equiv \tfrac{\hbar}{i} \delta'\mss{(x-x')}
                    \\
                & = \tfrac{\hbar}{i} \tfrac{\partial}{\partial x} \langle x | x' \rangle
            \end{aligned}
    \end{aligned}
    \hfill
    \begin{aligned}
        % xpf
        & 3.\ \begin{aligned}[t]
                \langle x | \hat{p} | f \rangle 
                    & = \mss{\int} f\mss{(x')} \hs \delta'\mss{(x-x')} \hs dx'
                    \\
                \Aboxed{ \breve{p} \langle x | f \rangle
                    & \equiv \tfrac{\hbar}{i} \tfrac{\partial}{\partial x} \langle x | f \rangle }
            \end{aligned}
    \end{aligned}
    \hfill
\)

%-----------------------------------------------------------------------------
%-----------------------------------------------------------------------------
% Schrodinger's Equation
\subsection{Schrodinger \( \Psi \)}

\vspace{5pt}\noindent
\[ 
    \begin{aligned}
        \Aboxed{ \breve{E} | \Psi \rangle 
            & = \widehat{H} | \Psi \rangle }
            = ( \widehat{T} + \widehat{V} ) | \Psi \rangle
            \\[5pt]
        i\hbar \frac{\partial}{\partial t} | \Psi \rangle 
            & = \left[ \frac{\hat{p}^2}{2m} + V(\hat{x}, t) \right] | \Psi \rangle
            \\[5pt]
        \Aboxed{ - \breve{E} \langle \Psi | 
            & = \langle \Psi | \hat{H} }
    \end{aligned}
    \hspace{1.5cm}
    \begin{aligned}
        \Aboxed{ \breve{E} \langle x | \Psi \rangle &= \breve{H} \Psi } = ( \breve{T} + \breve{V} ) \Psi
            = \left[ \tfrac{ \breve{p}^2 }{2m} + V(\vec{\mathbf{r}},t) \right] \Psi
            \\
        \Aboxed{ i \hbar \frac{\partial}{\partial t} \Psi(\vec{\mathbf{r}},t) 
            & = \left[ \frac{-\hbar^2}{2m} \nabla^2 
            + V(\breve{\mathbf{r}},t) \right] \Psi(\vec{\mathbf{r}},t) }
            \\
        \Aboxed{ - \breve{E} \langle \Psi | x \rangle
            & = \breve{H} \Psi^* }
    \end{aligned} 
\]

\vspace{15pt}\noindent
% 1-D solution
\begin{minipage}[t]{.5\textwidth}
    \underline{If \(V=V(x)\)}\\[10pt]
    \(\Psi(x,t) = \psi(x)\phi(t) \ \Rightarrow\)

    \vspace{5pt}
    \(\begin{aligned}
        &\bullet\ E_n \phi_n(t) = i \hbar \frac{\partial}{\partial t} \phi_n(t) \ \Rightarrow \ 
            \boxed{ \phi_n(t) = e^{- \frac{i}{\hbar} E_n t} } 
            \\
        &\bullet\ \boxed{ E_n \psi_n(x) 
            = \left( \frac{-\hbar^2}{2m} \partial_x^2 + V(x) \right) \psi_n(x) }
            \\  
        &\hspace{15pt} - \parbox[t]{7cm}{\(\psi\) can be lin. sum of real or complex, so choose real \(\psi\)}
            \\
        &\bullet\ \text{Linear}: \boxed{ 
                \begin{aligned}[t]
                    & \Psi\mss{(x,t)}  = \sum_n \ \psi_n\mss{(x)} \hs e^{- \frac{i}{\hbar} E_n t} \hs c_n  \\
                    & = \sum_n \ \langle x | n \rangle e^{- \frac{i}{\hbar} E_n t} \langle n | \Psi \rangle \\
                    & = \mss{ \int_{x'} }
                        \langle x | 
                        \Big[ \mss{\sum_n} | n \rangle e^{- \frac{i}{\hbar} E_n t} \langle n | \Big]
                        | x' \rangle 
                        \Psi\mss{(x')}
                        \hs\hs dx'
                        \\
                    & = \mss{ \int_{x'} } \ U\mss{(x, t;\hs x', 0)} \hs \Psi\mss{(x')} \hs\hs dx'
                \end{aligned}
            }
            \\
        &\bullet\ \parbox[t]{.9\textwidth}{
            \(\sigma^2_H = \langle H^2\rangle - \langle H \rangle^2 = 0
            \ \Rightarrow \ \) measuring stationary state, \(\Psi_n\), returns one \(E_n\) (determinate state) 
            }
    \end{aligned}\)
\end{minipage}
% \hfill
% % % 3-D solution
% \begin{minipage}[t]{.46\textwidth}
%     \underline{If \( V = V(r) \)}\\[10pt]
%     \( \Psi(\vec{\mathbf{r}}) = R(r) Y^m_l(\theta, \phi) 
%         = R(r) \Theta^m_l(\theta) \Phi_m(\phi) \ \Rightarrow \)
%     \begin{gather*}
%         E u = \left( \frac{\hat{p}_r^2}{2m} + V(r) + \frac{\hat{L^2}}{2(mr^2)} \right) u\\
%         \boxed{ E u = \frac{-\hbar^2}{2m} \partial_r^2 u
%             + \left[ V(r) + \frac{\hbar^2 l(l+1)}{2m r^2} \right] u }
%     \end{gather*}

%     \(\begin{aligned}
%         &\bullet\ u(r) = r R(r) \\[2pt]
%         &\bullet\ \Phi_m(\phi) = e^{i m \phi} \\[2pt]
%         &\bullet\ \Theta^m_l(\theta) = A P^m_l(\cos{\theta}) \\[5pt]
%         &\hspace{5pt} - A = \epsilon \sqrt{ \tfrac{2l+1}{4\pi} \tfrac{(l-|m|)!}{(l+|m|)!} } , \ \
%             \epsilon = \mss{\begin{cases}
%                 \scriptstyle (-1)^m & \scriptstyle (m \geq 0) \\
%                 \scriptstyle1       & \scriptstyle (m \leq 0)
%             \end{cases}}
%             \\[5pt]
%         &\hspace{5pt} - P^m_l(x) = \text{\scriptsize{Assoc. Legendre Func. (see extra)}} \\[2pt]
%         &\bullet\ l \in \mathbb{N}_0, \ m \in \{ -l, ..., -1, 0, 1, ..., l \} \\[2pt]
%         &\bullet\ \widehat{L_i} = ( \vec{r} \times \vec{p} )_i
%     \end{aligned}\)     
% \end{minipage}

%-------------------------------------------------------------------------------------------------------------------------------
%
%
%-------------------------------------------------------------------------------------------------------------------------------
% Usage
\newpage
\subsection{Usage}
% L2 Space
\begin{center}
    \(\arraycolsep=2pt \begin{array}{l r l l p{30pt} l}
        \bullet 
            & \langle f|g \rangle 
            & = 
            & \displaystyle \int_{-\infty}^{\infty} f(x)^* g(x) \ dx
            &
            & \displaystyle \bullet \ \langle f|g \rangle_{ab} = \int_{a}^{b} f(x)^* g(x) \ dx
            \\[20pt]
        \bullet 
            & | f \rangle 
            & = 
            & \displaystyle \int f(x') | x' \rangle \hs dx' \ \sim\ f(x) \equiv \langle x | f \rangle
            &
            & \displaystyle \bullet \ \langle f | = \int f(x)^* [...] \ dx
            \\[15pt]
        \bullet
            & \langle f|f \rangle 
            & = 
            & \multicolumn{3}{l}{ \displaystyle
                    \int_{a}^{b} |f|^2 \hs dx < \infty 
                    \ \Rightarrow \ f \in L_2{\scriptstyle(a,b)} 
                    \hspace{30pt} 
                    \left( \mss{
                        \int_{a}^{b} |f|^p \hs dx < \infty 
                        \ \Rightarrow \ f \in L_p{\scriptstyle(a,b)} 
                    } \right) 
                }
    \end{array}\)  
\end{center}

\noindent\hrulefill

% Psi Decomposition as Series Sum of Basis Vectors
\vspace{20pt} \noindent
\( \forall \{f_n\} \in L_2 \): \hspace{-10pt}
\( \arraycolsep=1pt \begin{array}{c c l}    
    \langle x | \Psi \rangle = 
        & \Psi 
        & = \begin{cases}
                \displaystyle \ \sum_n c_n f_n \\[20pt]
                \displaystyle \ \int_n c_n f_n \ dn
            \end{cases} 
        , \hspace{18pt} 
        \begin{aligned}
            \langle f_m|f_n \rangle &= \Bigg\{ 
                \begin{tabular}{l}
                    \( \delta_{mn} \)\\[5pt]
                    \( \delta {\scriptstyle (m-n)} \)
                \end{tabular} 
                \\[5pt]
            \Rightarrow \ \ &\boxed{ c_n = \langle f_n | \Psi \rangle }
        \end{aligned}
        , \hspace{18pt}
        \begin{gathered}
            \text{\scriptsize(see Born int.)} \\
            |c_n|^2 = \begin{cases}
                P(n) \\[5pt]
                \text{PDF}{\scriptstyle(n)} 
            \end{cases}
        \end{gathered}
        \\[40pt]
    % empty
        & | \Psi \rangle 
        & = \left\{ 
        \setlength{\arraycolsep}{3pt}
        \begin{array}{c c c c c c c c}
            \displaystyle \sum_n c_n | f_n \rangle
                & = & \displaystyle \sum_n \langle f_n | \Psi \rangle \ | f_n \rangle
                & = & \displaystyle \left( \sum_n | f_n \rangle 
                    \langle f_n | \right) | \Psi \rangle 
                & = & \displaystyle | \Psi \rangle 
                \\[20pt]
            \displaystyle \int_n c_n | f_n \rangle \ dn 
                & = & \displaystyle \int_n \langle f_n | \Psi \rangle | f_n \rangle \ dn 
                & = & \displaystyle \left( \int_n | f_n \rangle 
                    \langle f_n | \ dn \right) | \Psi \rangle 
                & = & \displaystyle | \Psi \rangle 
        \end{array} 
        \right.
\end{array}\)

% Different Bases Examples
\vspace{25pt}\noindent
\begin{tabular}{c|c|c}
    % Eigenfunctions
    \( \begin{gathered}[t] 
            \breve{x} \Psi_y = x \Psi_y = y \Psi_y \\[5pt]
            \Rightarrow \ \boxed{ \Psi_y = \delta{\scriptstyle(x-y)} = \langle x | y \rangle } 
        \end{gathered} \)
    & \( \begin{gathered}[t]
            \begin{aligned}
                \langle x | \hat{p} | p \rangle 
                    & = \mss{\int} \langle x | \hat{p} | x' \rangle \langle x' | p \rangle \hs dx'
                    \\
                p \langle x | p \rangle 
                    & = \tfrac{\hbar}{i} \tfrac{\partial}{\partial x} \langle x | p \rangle 
                    \\[5pt]
                \langle x | \hat{p} | p \rangle 
                    & = \breve{p} \Psi_p = p \Psi_p 
            \end{aligned}
                \\[5pt]
            \Rightarrow \ \boxed{ \Psi_p = Ae^{\frac{i}{\hbar} px} = \langle x | p \rangle }
        \end{gathered} \)
    & \( \begin{gathered}[t]
                \\[-19pt]
            \begin{aligned}
                    \langle x | \hat{H} | n \rangle & = E_n \langle x | n \rangle \\
                    \breve{H} \Psi_n & = E_n \Psi_n
                \end{aligned}
                \\
            \text{\scriptsize{(See Potential Examples)}} 
        \end{gathered} \) \\
    && \\ 
    \hline && \\
    % Psi Expansion
    \( \begin{gathered}
            \Psi{\scriptstyle(x,t)} = \int_{-\infty}^{\infty} \Psi_y c_y \mss{(t)} \ dy \\[10pt]
            = \int_{-\infty}^{\infty} \delta{\scriptstyle(x-y)} \hs \Psi{\scriptstyle(y,t)} \ dy
        \end{gathered} \)
    & \( \begin{gathered}
            \Psi{\scriptstyle(x,t)} = \int_{-\infty}^{\infty} \Psi_p \hs \phi\mathscriptsize{(E_p, t)} \hs c_p \ dp 
                \\[10pt]
            = \int_{-\infty}^{\infty} 
                \frac{e^{\frac{i}{\hbar} px}}{\sqrt{2 \pi \hbar}} 
                \hs e^{-\frac{i}{\hbar} \frac{p^2}{2m} t}
                \hs \Phi{\scriptstyle(p,0)}
                \ dp
        \end{gathered} \)
    & \( \begin{gathered}
            \Psi{\scriptstyle(x,t)} = \int_{-\infty}^{\infty} \Psi_n \hs \phi\mathscriptsize{(E_n, t)} \hs c_n \ dn 
                \\[10pt]
            = \int_{-\infty}^{\infty} 
                \Psi_n 
                \hs e^{\frac{-i}{\hbar}E_n t}
                \hs c_n
                \ dn
        \end{gathered} \) \\ 
    && \\ 
    \hline && \\
    % Transform
    \( \begin{aligned}
            c_x(t) &= \langle \Psi_x | \Psi\mss{(x,t)} \rangle = \langle x | \Psi \rangle \\[5pt]
            \Psi{\scriptstyle(x,t)} &= \int_{-\infty}^{\infty} \delta{\scriptstyle(x-y)} \Psi{\scriptstyle(y,t)}\ dy
        \end{aligned} \)
    & \( \begin{aligned}
            c_p(t) &= \langle \Psi_p | \Psi\mathscriptsize{(x,t)} \rangle = \langle p | \Psi \rangle \\[5pt]
            \Aboxed{ \Phi{\scriptstyle(p,t)} &= \int_{-\infty}^{\infty} 
                \frac{e^{\frac{-i}{\hbar} px}}{\sqrt{2 \pi \hbar}}
                \Psi{\scriptstyle(x,t)} \ dx }
        \end{aligned} \)
    & \( \begin{aligned}
            c_n(t) &= \langle \Psi_n | \Psi\mathscriptsize{(x,t)} \rangle = \langle n | \Psi \rangle\\[5pt]
            \Psi\mss{(n,t)} &= \int_{-\infty}^{\infty} \Psi_n^* \Psi{\scriptstyle(x,t)} \ dx
        \end{aligned} \) \\ 
    && \\ 
\end{tabular}

%-------------------------------------------------------------------------------------------------------
% Probability
\newpage 
\noindent \( \boxed{ \text{Born Interpretation: PDF}(x) = | \Psi(x) |^2 = \Psi^* \Psi } \)

% PDF(x)
\vspace{10pt}\noindent
\begin{minipage}[t]{0.5\textwidth}
    \underline{\( P{\scriptstyle(a<x<b)} = \int_{a}^{b} | \Psi |^2 dx \equiv \langle \Psi | \Psi \rangle_{ab} \)}\\[5pt]
    \(\begin{aligned}
        & -\ \boxed{ \langle \Psi | \Psi \rangle = 1 } \ \ \ \text{\scriptsize(physical, bound states only)} \\[5pt]
        & \bullet\ \Psi(\pm \infty) = 0 \\[5pt]
        & \bullet\ \text{Min}(V) \leq E_\Psi \in \mathbb{R} \\[5pt]
        & \bullet\ \langle \Psi_n | \Psi_n \rangle \rightarrow \infty \ \Rightarrow \ \Psi_n \text{ not PHYSICAL}
            \\
        & \hspace{15pt} \text{ sol. but \( \Psi = \int c_n \Psi_n \) can if }
            \langle \Psi|\Psi \rangle = 1
    \end{aligned}\)
\end{minipage} 
% Boundary conditions
\begin{minipage}[t]{0.49\textwidth}
    \underline{Boundary Conditions:}\\[5pt]
    \(\begin{aligned}
        & \bullet\ \Psi(x) \text{ isn't always cont. (see extra)} \\[5pt]
        & \bullet\ \tfrac{\partial \Psi(x)}{\partial x} \text{ is cont. except at } V = \infty \\[5pt]
        & \hspace{15pt} \lim_{\epsilon \rightarrow 0} \ \mss{ \int_{-\epsilon}^{\epsilon} } E \Psi dx 
            = \mss{ \int_{-\epsilon}^{\epsilon} } \widehat{H} \Psi dx \ \Rightarrow
            \\[5pt]
        & \hspace{15pt} \lim_{\epsilon \rightarrow 0} \ \tfrac{\hbar^2}{2m} \Delta ( \tfrac{d \Psi}{dx} )
            = \mss{ \int_{-\epsilon}^{\epsilon} } V \Psi dx 
    \end{aligned}\)
\end{minipage}

% Expectation of a function of x, f(x)
\vspace{5pt}\noindent \( \hspace{2pt} \bullet \ \ E[f{\scriptstyle(x)}] 
    = \int_{-\infty}^{\infty} f{\scriptstyle(x)} \ \text{\scriptsize PDF}{\scriptstyle(x)} \ dx
    = \int_{-\infty}^{\infty} f{\scriptstyle(x)} \ | \Psi{\scriptstyle(x)} |^2 \ dx
    = \int_{-\infty}^{\infty} \Psi{\scriptstyle(x)}^* f{\scriptstyle(x)} \Psi{\scriptstyle(x)} \ dx 
    = \boxed{ \langle \Psi | f \Psi \rangle \equiv \langle f{\scriptstyle(x)} \rangle}
\)

% c_n transform meaning
\vspace{5pt}\noindent \( \hspace{2pt} \bullet \ \begin{aligned}[t]
    \int_x \Psi^* \Psi \ dx 
        &= \int_x 
        \left( {\scriptstyle\int}_{n} \ c_n^*{\scriptstyle(t)} \Psi_n^*{\scriptstyle(x)} \ {\scriptstyle dn} \right)
        \left( {\scriptstyle\int}_{n'} \ c_{n'}{\scriptstyle(t)} \Psi_{n'}{\scriptstyle(x)} \ {\scriptstyle dn'} \right)
        \ dx \\[5pt]
    &= \int_{n} c_n^*{\scriptstyle(t)} \int_{n'} 
        c_{n'}{\scriptstyle(t)} \ \delta{\scriptstyle(n-n')} \ dn' \ dn 
        = \int_n | c_n{\scriptstyle(t)} |^2 \ dn
        \ \Rightarrow \ \Aboxed{ \text{\scriptsize PDF}(n) = | c_n |^2 = c_n^* c_n }
\end{aligned} \)

% Adjoint Def
\vspace{20pt} \noindent
\underline{Adjoint {\scriptsize(herm. adj./herm. conj.)}: 
    \( \big\{ A^\dagger : \langle f | A f \rangle = \langle A^\dagger f | f \rangle \big\} \)}
    \(\ \ \Rightarrow \ \ \langle h | \hat{A} g \rangle = \langle \hat{A}^\dagger h | g \rangle
    \hspace{18pt} {\scriptstyle(\text{let} \ f = h + g, \ f = h + ig)}\) \\[10pt]
% Hermitian Operators
\underline{Hermitian Operator: \( \big\{ A : \hat{A}^\dagger = \hat{A} \big\} \)}\\[5pt]
\hspace{18pt}
\(\begin{aligned}
    \bullet \ &\boxed{ \exists \{ \Psi_n \} : \ \hat{A} \Psi_n{\scriptstyle(x)} = a_n \Psi_n{\scriptstyle(x)} }
        \ \ \text{\scriptsize(spectral theorem)}
        \hspace{18pt} \bullet \ \langle a \rangle = a \in \mathbb{R} 
        \ \Rightarrow \ \hat{A} \ \text{can be an observable}\\[5pt]
    \bullet \ &\boxed{ \langle \Psi_m | \Psi_n \rangle \in \{ \delta_{mn}, \ \delta {\scriptstyle (m-n)} \} } 
        \hspace{3.5cm} \bullet \ \boxed{ \text{Axiom: \( \{ \Psi_n \} \) for \(\hat{A}\) are complete } }\\[5pt]
    &\underline{\text{Non-degenerate:}} \hspace{18pt} 
        (m \neq n), \ (a_m \neq a_n) \ \Rightarrow \ 
        \langle \Psi_m | \Psi_n \rangle \in \{ \delta_{mn}, \ \delta {\scriptstyle (m-n)} \} \\[5pt]
    &\underline{\text{Degenerate:}} \hspace{18pt}
        \begin{aligned}[t]
            &(m \neq n), \ (a_m = a_n), \ (\Psi_m \neq \Psi_n), \langle \Psi_m | \Psi_n \rangle \neq 0
                \ \Rightarrow \ \text{Use Gram-Schmidt} \\[5pt]
            &\text{to find orthogonal} \hspace{18pt} \langle \Psi_m' | \Psi_n' \rangle 
                = \langle a\Psi_m+b\Psi_n | \ c\Psi_m+d\Psi_n \rangle = 0 \\[3pt]        
        \end{aligned}
\end{aligned}\)\\[5pt]
% Expectation Value
\noindent
\underline{Expectation: \( E [ \hat{A} {\scriptstyle(x,p)} ] \)}\\[5pt]
\hspace{18pt} \(\bullet \ \begin{aligned}[t]
    \int_{-\infty}^{\infty} \hat{A}{\scriptstyle(x,p)}^* \ \Psi^* \Psi \ dx 
        &= \langle \hat{A} \Psi | \Psi \rangle
        = \boxed{ \langle \Psi | \hat{A} \Psi \rangle 
        \equiv \langle \hat{A}{\scriptstyle(x,p)} \rangle } 
        \hspace{18pt} \text{\scriptsize(won't work if \(\int A \ |\Psi|^2 \ dx\))}\\[5pt]
    \langle \Psi | \hat{A} \Psi \rangle &= \int_{-\infty}^{\infty} \Psi^* \hat{A} \Psi \ dx 
        = \int_{-\infty}^{\infty} \big( {\scriptstyle\int}_n \ c_n^* \Psi_n^* \ {\scriptstyle dn} \big) 
        \big( {\scriptstyle\int}_{n'} \ c_{n'} \hat{A} \Psi_{n'} \ {\scriptstyle dn'} \big) \ dx\\
    &= \int_n a_n | c_{n} |^2 \ dn = E[a] \equiv \langle a \rangle 
        \hspace{18pt} \hspace{18pt} c_n = \text{\scriptsize PDF}{\scriptstyle(n)} \ \ 
        \text{\scriptsize(see above and Momentum Space)}\\[5pt]
    \Aboxed{ \langle a \rangle 
        &= \langle \Psi | \hat{A} \Psi \rangle = \langle \Psi | \hat{A} | \Psi \rangle 
        = \langle A \rangle}
\end{aligned} \)\\[10pt]
\hspace{18pt} \(\bullet \ \boxed{ \langle \sigma^2_a \rangle = \langle a^2 \rangle - \langle a \rangle^2 }
    \ \Rightarrow \ \sigma_A^2=0 \ \ \ \text{ for } \Psi_n \hspace{18pt} \text{\scriptsize(determinate state)}\)

%-----------------------------------------------------------------------------------------------------------------------
% Matrix Operators
\newpage \noindent
\underline{Matrix Operators:} \\[10pt]
Given complete \(\{e_n\} : \ \langle e_m | e_n \rangle = \delta_{mn}\) 

% 1. How Q interacts
\vspace{20pt}\noindent 
\(\boldsymbol{1.)} \ \ \boxed{ Q_{mn}^{(e)} \equiv 
    \big\langle e_m \big| \widehat{Q}{\scriptstyle(x,p)} \big| e_n \big\rangle }\) \\[5pt]
\( \displaystyle 
    | \beta \rangle = \widehat{Q} | \alpha \rangle \boldsymbol{=} \sum_m | e_m \rangle
    \left[ \begin{aligned}
        \langle e_m | \beta \rangle 
            &= \big\langle e_m \big| \widehat{Q} \big| \alpha \big\rangle 
            \\[10pt]
        \sum_n b_n \langle e_m | e_n \rangle 
            &= \sum_n a_n \hs\hs \boxed{ \big\langle e_m \big| \widehat{Q} \big| e_n \big\rangle } 
            \\
        \Aboxed{ b_m &= \sum_n \Big( Q_m^{(e)} \Big)_n \ a_n }
    \end{aligned} \right]
    =
    \begin{aligned}
        \sum_m b_m | e_m \rangle 
            & = \sum_{n,m} \langle e_n | \alpha \rangle \ Q_{mn}^{(e)} | e_m \rangle 
            \\
        & = \sum_{n,m} Q_{mn}^{(e)} | e_m \rangle \langle e_n |\alpha \rangle 
            \\
        \Rightarrow \ \Aboxed{ \widehat{Q}
            & = \sum_{m,n} Q_{mn}^{(e)} | e_m \rangle \langle e_n | }     
    \end{aligned} 
\)

% 2. Pre-math for matrix representation of Q
\vspace{15pt} \noindent
\(\boldsymbol{2.)}\) Find \(\widehat{Q}\) as a matrix\\[5pt]
\( \displaystyle
    \begin{aligned}
        \\[-8pt]
        | f \rangle & = \sum_n c_n^{(e)}{\scriptstyle[f]} \ \big| e_n \big\rangle \\[-8pt]
        & \hs \downarrow \\[-4pt]
        f {\scriptstyle(x)} & = \sum_n c_n^{(e)}{\scriptstyle[f]} \ e_n{\scriptstyle(x)} 
    \end{aligned}
    \ \ = \ \ 
    % vector representation of c and e 
    \left( \begin{matrix} 
        \vdots\\
        c_n{\scriptstyle[f]}\\
        \vdots
    \end{matrix} \right)^{(e)} \dotP
    \left( \begin{matrix} 
        \vdots\\
        e_n{\scriptstyle(x)}\\
        \vdots
    \end{matrix} \right)
    \ \ \equiv \ \
    % condensed vector representation of c and e 
    \boxed{ \begin{gathered}
        \vec{c}^{\ (e)}{\scriptstyle[f]} \dotP \vec{e}{\scriptstyle(x)} \\[5pt]
        \int_n c^{(e)}{\scriptstyle[f]}{\scriptstyle(n)} \cdot e{\scriptstyle(n, x)} \ dn
    \end{gathered} } 
    \hspace{10pt} , \hspace{10pt}
    % how to find c_n
    \boxed{ c_n^{(e)}{\scriptstyle[f]} = \langle e_n | f \rangle }
\)

% Calculate Matrix representation of Q
\vspace{10pt}\noindent
\( 
    \begin{aligned}
        &\widehat{Q} | f \rangle \\
        &= \Big( \text{\scriptsize\(\sum_{m,n'}\)} \
            Q_{mn'}^{(e)} | e_m \rangle \langle e_{n'} | \Big) 
            \text{\scriptsize\(\sum_{n}\)} \ c_n^{(e)} | e_n \rangle\\
        &= \text{\scriptsize\(\sum_{m,n}\)}
            \Big( \text{\scriptsize\(\sum_{n'}\)} \ Q_{mn'}^{(e)} \ c_n^{(e)} 
            \langle e_{n'} | e_n \rangle \Big) | e_m \rangle \\
        &= \text{\scriptsize\(\sum_{m}\)} \Big( \text{\scriptsize\(\sum_{n}\)} \
            \big( Q_{m}^{(e)} \big)_n c_n^{(e)} \Big) | e_m \rangle
    \end{aligned}
\) 
\hfill
\vline
\hfill
% Interpret math to get matrix representation of Q
\fbox{ \(
    \begin{gathered}
        \begin{aligned}
            &\widehat{Q} \left[
                \left(\begin{matrix} 
                    |\\
                    c\\
                    |
                \end{matrix}\right)^{(e)} \dotP 
                \left( \begin{matrix} 
                    |\\
                    e\\
                    |      
                \end{matrix}\right) \right]
                = \left[ 
                    \left(\begin{matrix} 
                            & \vdots &\\
                        -   & Q_{m} & -\\
                            & \vdots & 
                    \end{matrix}\right)^{(e)} 
                    \left(\begin{matrix} 
                        |\\
                        c\\
                        |
                    \end{matrix}\right)^{(e)}  
                \right] \dotP 
                \left( \begin{matrix} 
                    |\\
                    e\\
                    |      
                \end{matrix}\right) \\[5pt]
            &\widehat{Q} \ | f \rangle = \boxed{ \widehat{Q} \ \big[ \ \vec{c}^{\ (e)}{\scriptstyle[f]} \dotP \vec{e} \ \big]
                = \left[^{\ } \overline{Q}^{(e)} \ \vec{c}^{\ (e)}{\scriptstyle[f]} \right] \dotP \vec{e} } 
        \end{aligned} \\[5pt]
        \text{e.g.} \ \begin{aligned}
            \langle x | \widehat{Q} | f \rangle &= \int_{m} \left[ \overline{Q}^{(\delta)} f \right]{\scriptstyle(m)} 
                \cdot \delta{\scriptstyle(x-m)} \ dm\\[5pt]
            &= \int_{m} \left[ \int_n Q_m^{(\delta)} {\scriptstyle(n)} \cdot f{\scriptstyle(n)}\ dn \right] 
                \delta{\scriptstyle(x-m)} \ dm = \widehat{Q} f{\scriptstyle(x)} 
        \end{aligned}
    \end{gathered}
\) }

% 3. Terms
\vspace{10pt} \noindent
\(\boldsymbol{3.)}\) Terms\\[10pt]
\begin{minipage}[t]{.61\textwidth}
    \begin{tabular}[t]{l l}
        Diagonalizable: & 
            \(A \equiv PDP^{-1}\)
            \\[10pt]
        {\scriptsize Conj. Transpose}, \(\dagger\): & 
            \(A^\dagger \equiv A^{T*} = A^{*T}\)
            \\[10pt]
        Hermitian, \(H\): & 
            \(\begin{aligned}[t]
                &H = H^\dagger\\
                &H = UDU^{-1} = UDU^\dagger \hspace{20pt} \text{\scriptsize(spectral theorem)}
            \end{aligned}\)
        \\[30pt]
        Unitary, \(U\): & 
            \(\begin{aligned}[t]
                &U: \ UU^{\dagger} = U^{\dagger}U = 1 \\
                &\exists H:\ U = e^{iH} = (U') e^{iD} (U')^{\dagger}
            \end{aligned}\)    
    \end{tabular}    
\end{minipage}
\hfill
\begin{minipage}[t]{.32\textwidth}
    \scriptsize
    \vspace{-.6cm}
    \underline{Hermitian Operator \(\sim\) Hermitian Matrix}\\[5pt]
    \(\begin{aligned}[t]
        &\langle Qx | y \rangle = \langle x | Qy \rangle 
            \hspace{15pt} \parbox{3cm}{\scriptsize(if inf. size then must be in Hilbert Space)}
            \\[5pt]
        &\hspace{2.2cm} \text{(draw it out)}
            \\
        &\rightarrow \ \begin{aligned}[t]
                (\overline{Q} x)^{*T} \dotP y_m|e_m\rangle 
                    &= y_m x^{*T} \dotP ( \overline{Q}^*_m ) \\[5pt]
                &= x^{*T} \dotP \overline{Q}^{*T} y_m|e_m\rangle \\[5pt]
                &= x^{*T} \dotP \overline{Q} y_m|e_m\rangle
            \end{aligned}
            \\[5pt]
        &\rightarrow \ \overline{Q}^\dagger \equiv \overline{Q}^{*T} = \overline{Q} \hspace{20pt} \checkedbox
    \end{aligned}\)
\end{minipage}

%-------------------------------------------------------------------------------------------------------
%
%
%
\newpage 
\noindent
% 4. Operating on Eigenvectors (Eigenvalue / Eigenvector)
\(\boldsymbol{4.)}\) Eigenvalue Equation\\[10pt]
\underline{General Case:}\\
% Initial info
\(\displaystyle
    % Initial
    \begin{aligned}
        \widehat{Q} | q_i \rangle &= q_i | q_i \rangle\\[5pt] 
        | q_i \rangle &= \sum c^{(e)}_n{\scriptstyle[q_i]} \big| e_n \big\rangle 
    \end{aligned}
    \hfill\vline\hfill
    % Diagonalization
    \boxed{\begin{aligned}
        \ \ \overline{Q}^{(e)} & = \ U D U^{\dagger} \hspace{18pt}\hspace{18pt} \text{\scriptsize(Spectral Theorem)} \\[5pt]
        & = \mss{
                \left( \arraycolsep=2pt \begin{array}{c c c}
                    |   & |   &    \\
                    \vec{c}\ {\scriptstyle[q_0]} & \vec{c}\ {\scriptstyle[q_1]} & ... \\
                    |   & |   & 
                \end{array} \right)^{(e)}
                \left(\begin{matrix} 
                    q_0   & 0         & ...\\
                    0           & q_1 & ...\\
                    \vdots      & \vdots    & 
                \end{matrix}\right)
                \left(\begin{matrix} 
                    - \ \vec{c}^{\ *}{\scriptstyle[q_0]} \ - \\
                    - \ \vec{c}^{\ *}{\scriptstyle[q_1]} \ - \\
                    \vdots \\
                \end{matrix}\right)^{(e)}
            }
            \\[5pt]
        & \hspace{18pt} \text{where } \langle \vec{c}_m | \vec{c}_n \rangle = \delta_{mn}
            \text{ \ since \ } Q^\dagger Q = QQ^\dagger \hspace{10pt} \text{\scriptsize(normal)}
    \end{aligned}}
\)

% Diagonalize Q
\vspace{5pt}\noindent
\begin{minipage}[t]{.55\textwidth}
    \(\begin{aligned}[t]
        q_i | q_i \rangle \ & = \ \widehat{Q} | q_i \rangle
            \\[5pt]
        \left[ q_i \ \vec{c}^{\ (e)}{\scriptstyle[q_i]} \right] \dotP \vec{e}{\scriptstyle(x)} 
            \ & = \ 
            \left[^{\ } \overline{Q}^{(e)} \ \vec{c}^{\ (e)}{\scriptstyle[q_i]} \right] 
            \dotP \vec{e}{\scriptstyle(x)}
            \\
        & \Downarrow^*\\ % side note to remove basis vectors
        q_i \ \vec{c}^{\ (e)}{\scriptstyle[q_i]} 
            \ & = \ \overline{Q}^{(e)} \ \vec{c}^{\ (e)}{\scriptstyle[q_i]} 
            \\
        q_i \mss{ \left( \begin{matrix} 
                |\\
                \vec{c}\ {\scriptstyle[q_i]}\\
                |
            \end{matrix} \right)^{(e)} }
            \ & = \ \mss{
                \left( \arraycolsep=2pt \begin{array}{c c c}
                    |   & |   &    \\
                    \vec{c}\ {\scriptstyle[q_0]} & \vec{c}\ {\scriptstyle[q_1]} & ... \\
                    |   & |   & 
                \end{array} \right)^{(e)}
                \left(\begin{matrix} 
                    q_0   & 0         & ...\\
                    0           & q_1 & ...\\
                    \vdots      & \vdots    & 
                \end{matrix}\right)
                \left(\begin{matrix} 
                    - \ \vec{c}^{\ *}{\scriptstyle[q_0]} \ - \\
                    - \ \vec{c}^{\ *}{\scriptstyle[q_1]} \ - \\
                    \vdots \\
                \end{matrix}\right)^{(e)}
                \left(\begin{matrix} 
                    |\\
                    \vec{c}\ {\scriptstyle[q_i]}\\
                    |
                \end{matrix}\right)^{(e)} 
            }
            \hspace{30pt} \boxed{ \begin{aligned}
                & \underline{ q_i\ ,\ \vec{c}^{\ (e)}{\scriptstyle[q_i]} }: \\[7pt]
                & \text{det} \hs \mss{ \left( \overline{Q}^{(e)} - I q_i \right) } = 0
            \end{aligned} }
    \end{aligned}\)
\end{minipage}
% side note on how to seperate the basis vectors
\begin{minipage}[t]{.4\textwidth}
    \scriptsize
    \vspace{10pt}
    \(^* \forall {\scriptstyle n}\ \begin{aligned}[t]
        (q c)_n \ | e_n{\scriptstyle(x)} \rangle &= (Q c)_n \ | e_n{\scriptstyle(x)} \rangle\\[5pt]
        \langle e_n{\scriptstyle(x)} | (q c)_n | e_n{\scriptstyle(x)} \rangle 
            &= \langle e_n{\scriptstyle(x)} | (Q c)_n | e_n{\scriptstyle(x)} \rangle\\
        (q c)_n &= (Q c)_n
    \end{aligned}\)
\end{minipage}

% Special Case: Eigenvectors are the basis
\vspace{15pt} \noindent
\underline{Special Case:}\\[10pt]
\( \begin{aligned}
    | q_n \rangle &= | e_n \rangle\\[5pt]
    \widehat{Q} | e_n \rangle &= q_n | e_n \rangle
\end{aligned} \) 
\hspace{5pt}
\rule[-30pt]{.5pt}{67pt}
\hspace{10pt}
\( 
    \begin{aligned}
        \widehat{Q} | a \rangle &= \sum_n \widehat{Q} | e_n \rangle \langle e_n | a \rangle\\
        &= \Big( \sum_n q_n | e_n \rangle \langle e_n | \Big) | a \rangle
    \end{aligned} 
    \ \Rightarrow \ 
    \begin{gathered}
        \boxed{ \widehat{Q} = \sum_n q_n | e_n \rangle \langle e_n | } \\[5pt]
        Q_{mn}^{(e)} = q_n \delta_{mn}
    \end{gathered}
    \ \Rightarrow \
    \boxed{
        \overline{Q}^{(e)} = 
        \left(\begin{matrix} 
            q_0     & 0         & ...\\
            0       & q_1       & ...\\
            \vdots  & \vdots    & q_i
        \end{matrix}\right)
    }
\) 

\vspace{15pt} \noindent
\(
    % 1st row
    \overline{Q}^{(e)} 
    = \mss{ \left( \begin{matrix} 
            q_0     & 0         & ...\\
            0       & q_1       & ...\\
            \vdots  & \vdots    &
        \end{matrix} \right)^{(e)} }
    = \ \ \mss{
        \left(\begin{matrix} 
            1       & 0     & ... \\
            0       & 1     & ... \\
            \vdots  & \vdots  &  \\
        \end{matrix}\right)^{(e)}
        \left(\begin{matrix} 
            q_0   & 0         & ...\\
            0           & q_1 & ...\\
            \vdots      & \vdots    &
        \end{matrix}\right)
        \left(\begin{matrix} 
            1       & 0     & ... \\
            0       & 1     & ... \\
            \vdots  & \vdots  &  \\
        \end{matrix}\right)^{(e)}
    }
    \hfill
    \boxed{ 
        \vec{c}^{\ (e)}{\scriptstyle[q_i]} 
        = \left( ...\ 0\ 0\ 1_{\scriptstyle(i)}\ 0\ 0\ ... \right)^T 
    }
\)

%--------------------------------------------------------------------
% 5. Unitary Transformation
\vspace{30pt}\noindent
5. Unitary Transformation and Trace\\[10pt]
\(
    \begin{aligned}
        & \bullet\ \boxed{ | b_i \rangle = U | a_i \rangle
            \ \Leftrightarrow\ U = \sum | b_n \rangle \langle a_n | }
            \\[5pt]
        & \bullet\ \boxed{ \overline{U}_{ij} = \langle a_i | U | a_j \rangle 
            = \langle a_i | b_j \rangle }
            \\[5pt]
        & \bullet\ \boxed{ \langle b_i | U \widehat{Q} U^\dagger | b_j \rangle 
            = \langle a_i | \widehat{Q} | a_j \rangle }
            \\[5pt]
        & \bullet\ \boxed{ \begin{aligned}[t]
                (A) | a_i \rangle & = a_i | a_i \rangle \\[5pt]
                (U A U^\dagger) | b_i \rangle & = a_i | b_i \rangle 
            \end{aligned}
            }
    \end{aligned}
    \hfill
    % Trace
    \begin{aligned}
        & \bullet\ { \text{Tr}(Q) = \sum \langle a_i | Q | a_i \rangle 
            = \sum \langle b_i | Q | b_i \rangle }
            \\[5pt]
        & \bullet\ { \text{Tr}(QP) = \text{Tr}(PQ)} \\[5pt]
        & \bullet\ { \text{Tr}(U^\dagger Q U) = \text{Tr}(Q)} \\[5pt]
        & \bullet\ { \text{Tr}( \hs | a_i \rangle \langle a_j | \hs ) = \delta_{ij} } \\[5pt]
        & \bullet\ { \text{Tr}( \hs | b_i \rangle \langle a_i | \hs ) = \langle a_i | b_i \rangle }
    \end{aligned}
    \hfill\hs
\)

%--------------------------------------------------------------------------------------------------------------------------------
%
%
%--------------------------------------------------------------------------------------------------------------------------------
% Momentum Space
\newpage
\noindent
\underline{\(\Phi(p,t)\) - Momentum Space (generalizable Born Interpretation):}\\[10pt]
\( 
    \begin{gathered}[t]
        \begin{aligned}[t]
                \int_x \Psi^* \Psi dx 
                    & = \int_x 
                    \int_{p} c_p^*(t) \Psi_p^*(x) dp
                    \int_{p'} c_{p'}(t) \Psi_{p'}(x) dp' \ dx \\[5pt]
                & = \int_{p} c_p^*(t) \int_{p'} c_{p'}(t)
                    \int_x \Psi_p^*(x) \Psi_{p'}(x) dx \ dp' dp \\[5pt]
                & = \int_{p} \Phi^* \int_{p'} \Phi' \ \delta(p-p') dp' dp \\[5pt]
                & = \int_p \Phi^* \Phi \ dp \ \Rightarrow \ 
                    \boxed{ \text{PDF}(p) = | \Phi |^2 = \Phi^* \Phi } \\[5pt]
                \Aboxed{ \langle \Psi | \Psi \rangle & = \langle \Phi | \Phi \rangle }   
            \end{aligned} 
            \\[10pt]
        \begin{gathered}
                \text{
                    \underline{Anything in \(x\)-space can be done in \(p\)-space}
                    }\\
                \text{(or generalize to any transform, \(c_n\))}
            \end{gathered}
    \end{gathered}
\)
\hfill
\(\begin{aligned}[t]
    % heuristic remember
    & \hspace{10pt} 
        x \Phi = x e^{- \frac{i}{\hbar} px} 
        = - \tfrac{\hbar}{i} \tfrac{\partial}{\partial p} \Phi 
        = \tfrac{\hbar}{i} \tfrac{\partial}{\partial (-p)} \Phi
        \\[5pt]
    % ppp
    & \bullet\ \langle p | \hat{p} | p' \rangle \equiv p \langle p | p' \rangle \equiv p \hs \delta\mss{(p-p')} 
        \\
    % ppf
    & 1.\ \langle p | \hat{p} | f \rangle
        = p f(p) 
        = \boxed{ p \langle p | f \rangle \equiv \breve{p} \langle p | f \rangle }
        \\
    % pxp
    & 2.\ \begin{aligned}[t]
            \langle p | \hat{x} | p' \rangle 
                & = \mss{\iint} \langle p | x \rangle \langle x | \hat{x} | x' \rangle \langle x' | p' \rangle \hs dx dx'
                \\
            & = \tfrac{1}{2\pi\hbar} \mss{\int} x e^{\frac{i}{\hbar} x(p'- p)} dx 
                \\    
            & = - \tfrac{\hbar}{i} \delta'\mss{(p-p')}
                = - \tfrac{\hbar}{i} \tfrac{\partial}{\partial p} \langle p | p' \rangle
        \end{aligned}
        \\
    % pxf
    & 3.\ \begin{aligned}[t]
            \langle p | \hat{x} | f \rangle
                & = \mss{\int} \langle p | \hat{x} | p' \rangle \langle p' | f \rangle \ dp'
                \\
            & = \boxed{ - \tfrac{\hbar}{i} \tfrac{\partial}{\partial p} \langle p | f \rangle
                \equiv \breve{x} \langle p | f \rangle }
        \end{aligned}
        \\[5pt]
    % <p|A|p>
    & \Rightarrow\ \begin{aligned}[t]
            & A(x, \hat{p}_x) \rightarrow A(\hat{x}_p, p)\\
            & \Rightarrow \ \boxed{ \langle a \rangle = \big\langle \Phi \big| 
                A(\hat{x}_p, p) \big| \Phi \big\rangle }
        \end{aligned}
\end{aligned}\)

% Heisenberg Uncertainty
\vfill
\noindent
\begin{minipage}[t]{0.6\textwidth}
    \underline{Heisenberg Uncertainty Proof:} \\[5pt]
    \( \begin{aligned}
        \langle f|g \rangle & \equiv 
            \Big\langle \big( \widehat{A} - \langle a \rangle \big) \Psi 
            \Big| \big( \widehat{B} - \langle b \rangle \big) \Psi \Big\rangle
            \\[5pt]
        & = \Bigl\langle \Psi \Bigl| \big( \widehat{A} - \langle a \rangle \big) \Bigr|
            \bigl( \widehat{B} - \langle b \rangle \bigr) \Psi \Bigr\rangle
            \\[5pt]
        & = \big\langle \Psi | \widehat{A} (\widehat{B} \Psi) \big\rangle 
            - \langle a \rangle \langle b \rangle
            = \big\langle \widehat{A}\widehat{B} \big\rangle 
            - \langle a \rangle \langle b \rangle
            \\[10pt]
        \sigma_A^2 \sigma_B^2 & =
            \Vert ( \widehat{{A}} - \langle {a} \rangle ) {\Psi} \Vert^2
            % \big| ( \widehat{{A}} - \langle {a} \rangle ) {\Psi} \big\rangle 
            \ \Vert \big( \widehat{{B}} - \langle {b} \rangle ) {\Psi} \Vert^2
            % \big| ( \widehat{{B}} - \langle {b} \rangle ) {\Psi} \big\rangle 
            \\[5pt]
        & \equiv \langle f|f \rangle \langle g|g \rangle 
            \geq \left\Vert \langle f|g \rangle \right\Vert^2 
            \hspace{20pt} \text{\scriptsize(see Schwarz Ineq.)}
            \\
        & \geq \ \big[ \text{Im} \big( \langle f|g \rangle \big) \big]^2 \ 
            = \ \left( \frac{1}{2i} \big[ \langle f|g \rangle 
            - \langle f|g \rangle^* \big] \right)^2 
            \\
        & = \left( \frac{1}{2i} \big\langle \widehat{A}\widehat{B} 
            - \widehat{B}\widehat{A} \big\rangle \right)^2 
            \equiv \boxed{ \left( \frac{1}{2i} \left\langle 
            \big[ \widehat{A},\widehat{B} \big] \right\rangle \right)^2 }
    \end{aligned} \)
\end{minipage}
\hfill
% Commutator
\begin{minipage}[t]{0.38\textwidth}
    \underline{Commutator}

    \vspace{10pt}
    \hspace{10pt}\(\begin{aligned}
        & \bullet\ \big[ \widehat{A}, \widehat{B} \big] f \equiv \widehat{A} (\widehat{B} f) 
            - \widehat{B} (\widehat{A} f) 
            \\
        & \bullet\ \big[ A, BC \big] = \big[ A, B \big] C + B \big[ A, C \big] \\
        & \bullet\ \big[ AB, C \big] = A \big[ B, C \big] + \big[ A, C \big] B \\
        & \bullet\ \big[ x, \hat{p} \big] = i \hbar \\
        & \bullet\ \begin{aligned}[t]
                & \boxed{ 
                    \sigma_A \sigma_B
                    \ \geq \ \left\Vert \dfrac{1}{2i} \Big\langle 
                    \big[ \widehat{A}, \widehat{B} \big] \Big\rangle \right\Vert 
                    } 
                    \\[5pt]
                & \hspace{10pt} \Rightarrow \boxed{ \Delta x \Delta p \ \geq \ \hbar / 2 }
            \end{aligned} 
            \\[5pt]
        & \bullet\ \boxed{ 
            \begin{aligned}[t]
                \big[ \hat{p}, f\mss{(x)} \big] & = \hat{p} f\mss{(x)} = \tfrac{\hbar}{i} \tfrac{\partial f}{\partial x} \\[5pt]
                \big[ \hat{x}, g\mss{(p)} \big] & = \hat{x} g\mss{(p)} = - \tfrac{\hbar}{i} \tfrac{\partial g}{\partial p}
            \end{aligned}
            }   
    \end{aligned}\)
\end{minipage}

% Commutator for Hermitian Operators
\vspace{15pt} \noindent
\begin{minipage}[t]{.53\textwidth}
    \underline{Commutator of Hermitian \(\widehat{A}, \widehat{B}\)}
    \begin{itemize}
        \item \( \big[A,B \big]^\dagger = - \big[ A,B \big] \)
        \item \( \exists \Psi_n \ \ \text{s.t.} \ \ \left( \widehat{A} \Psi_n = a \Psi_n \right) \ , \ 
            \left( \widehat{B} \Psi_n = b \Psi_n \right) \) \\[5pt]
        \( \Leftrightarrow \big[ \widehat{A}, \widehat{B} \big] = 0 \) \\[5pt]
        \( \ \Rightarrow \ \) \( \begin{aligned}[t]
            &\boxed{ \sigma_A \sigma_B \ \geq \ 0 \ \ 
                \text{\scriptsize (Both can be measured concurrently)} } \\
            &\boxed{AB = BA}
        \end{aligned} \)
    \end{itemize}    
\end{minipage}
\hfill
% Anti Hermitian Operator
\begin{minipage}[t]{.4\textwidth}
    \underline{Anti-Hermitian Operators}: \ \( A^\dagger = - A \)
    \begin{itemize}
        \item \( \langle A \rangle = ai \), \ \ \ \( a \in \mathbb{R} \)
        \item \( \big[ A,B \big]^\dagger = - \big[ A,B \big] \)
    \end{itemize}
\end{minipage}


%---------------------------------------------------------------------------------------------------------------------------------
%
%
%
\newpage \noindent
% Time Derivative of Operator (Heisenberg Equation)
\underline{Operator Evolution (Heisenberg Equation)}\\[10pt]
\(\displaystyle
    \frac{d}{dt} \Big\langle \Psi(x,t) \Big| Q \Big| \Psi(x,t) \Big\rangle 
        = \Big\langle \frac{\partial \Psi}{\partial t} \Big| Q \Big| \Psi \Big\rangle 
        + \Big\langle \Psi \Big| \frac{\partial Q}{\partial t} \Big| \Psi \Big\rangle 
        + \Big\langle \Psi \Big| Q \Big| \frac{\partial \Psi}{\partial t} \Big\rangle 
\)

\vspace{10pt}\noindent
\hspace{30pt}\(
    \boxed{
    \begin{aligned}
        \tfrac{d}{dt} \langle Q \rangle & 
            = \tfrac{1}{i \hbar} \left\langle \big[ \widehat{Q}, \widehat{H} \big] \right\rangle 
            + \left\langle \tfrac{\partial \widehat{Q}}{\partial t} \right\rangle 
            \\
        i\hbar \hs \tfrac{d}{dt} \langle Q \rangle & 
            = \Big\langle \big[ \widehat{Q}, \widehat{H} \big] \Big\rangle 
            + i \hbar \left\langle \tfrac{\partial \widehat{Q}}{\partial t} \right\rangle 
    \end{aligned}
    }
    \hspace{18pt} \hspace{18pt} \text{\scriptsize(\(Q\) is conserved when this equals 0)}    
\)

% Usage of Time Derivative
\begin{itemize}
    \item Conservations: \hspace{18pt} \( \dfrac{d \langle \Psi | \Psi \rangle}{dt} = 0 \), \ 
    \( \dfrac{d \langle H \rangle}{dt} = 0 \)

    \item Ehrenfest's Theorem: \hspace{18pt} 
    \( m \dfrac{d \langle x \rangle}{dt} = \langle p \rangle \), \ 
    \( \dfrac{d \langle p \rangle}{dt} = - \left\langle \dfrac{\partial V}{\partial x} \right\rangle \)
    \( \ \Rightarrow \) \ other classical eq.

    \item Virial Theorem: \hspace{18pt} \( \displaystyle 
    \begin{aligned}[t]
        \tfrac{d}{dt} \langle xp \rangle
            & = \tfrac{i}{\hbar} \left\langle \big[ H,x \big] p + x \big[ H,p \big] \right\rangle 
            = \left\langle \big[ \tfrac{p^2}{2m} , x \big] p + x \big[ V, p \big] \right\rangle 
            \\[5pt]
        & = \tfrac{i}{\hbar} 
            \left\langle 
            \tfrac{1}{2m} p \big[ p, x \big] p 
            - \tfrac{1}{2m} \big[ p, x \big] p^2 
            - x \big[ p, V \big] 
            \right\rangle 
            \\[5pt]
        \Aboxed{ \frac{d \langle xp \rangle}{dt}
            &= 2 \langle T \rangle - 
            \left\langle x \frac{\partial V}{\partial x} \right\rangle }
            \rightarrow 0 = \tfrac{d}{dt} \Big\langle \Psi_n(x) \Big| Q{\scriptstyle(x,p)} \Big| \Psi_n(x) \Big\rangle 
            \ \ \text{\scriptsize (for stationary states) }
    \end{aligned} \)

    \item \begin{minipage}[t]{.9\textwidth}
        \setlength{\parindent}{.5cm}
        \noindent Energy-Time Uncertainty:
        \( \left( {Q} = {Q}(x,\hat{p}) 
            \neq {Q}(x,\hat{p},t) \right)
            \ \Rightarrow \ \sigma_H \sigma_Q \geq \frac{\hbar}{2} 
            \left| \frac{d \langle Q \rangle}{dt} \right| \) \\[10pt]
        \hspace{18pt} 
        \( \ \Rightarrow \ \)
        \fbox{ 
            \( \begin{aligned}[c]
                \sigma_Q \ \equiv \ \frac{d \langle Q \rangle}{dt} \Delta t
                    \ \approx& \ \ \Delta \langle Q \rangle \\[5pt]
                \sigma_H \left( \frac{\sigma_Q}{ \left| d \langle Q \rangle / dt \right| } \right)
                    & \geq \frac{\hbar}{2} \\[5pt]
                \Delta E \Delta t \geq \frac{\hbar}{2} &
            \end{aligned} \)
            \hspace{18pt} \hspace{18pt}
            \begin{minipage}{6cm}
                \(\Delta t\) is the amount of time it would \\ 
                take \(\langle Q \rangle\) to change ``appreciably'',\\
                or one std. dev. at the constant \\
                rate \( \frac{d}{dt} \langle Q \rangle \)
            \end{minipage}
        }

        \vspace{15pt}
        \hspace{18pt} Mass Lifetime: \\[10pt]
        \hspace{18pt} \hspace{18pt} \( \Delta (mc^2) \Delta t \geq \frac{\hbar}{2} \ \ \checkedbox \)

        \vspace{15pt}
        \hspace{18pt} Orthogonal Time Example: \\[10pt]
        \hspace{18pt} \hspace{18pt} \( \Psi(x,\tau) = \frac{\sqrt{2}}{2} 
            ( \Psi_1 e^{-\frac{i}{\hbar} E_1 \tau} 
            + \Psi_2 e^{-\frac{i}{\hbar} E_2 \tau} ) \) \\[10pt]
        \hspace{18pt} \hspace{18pt} \( \Big\langle \Psi(x,0) \big| \Psi(x,\tau) \Big\rangle = 0 
            = \frac{1}{2} ( e^{-\frac{i}{\hbar} E_1 \tau} 
            + e^{-\frac{i}{\hbar} E_2 \tau} ) \) \\[10pt]
        \hspace{18pt} \hspace{18pt} \( \ \Rightarrow \ \tau \ \frac{E_2 - E_1}{2} 
            = \frac{\pi}{2} \ \hbar \ (\frac{1}{2} + n) \geq \frac{\hbar}{2} \ \ \checkedbox \)
    \end{minipage}
\end{itemize}

% Translation Operator
\newpage \noindent
\underline{Translation Operator}

\vspace{-10pt}
\begin{align*}
    f(x + \Delta x) &\approx f(x) + \frac{df}{dx} \Delta x \\
    &= f(x) + f'(x) \Delta x + \frac{f''(x)}{2!} (\Delta x)^2 + ... 
        = \left\{ \begin{aligned}
            &f(x') = \sum_{n} \frac{ f^{(n)}(a) }{n!} (x'-a)^n \\
            &(x' = x + \Delta x), \ (a = x)
        \end{aligned} \right\} 
        \\
    &= \sum_{n=0}^\infty \frac{ f^{(n)}(x) }{n!} (\Delta x)^n 
        \ = \ \sum_{n=0}^\infty \frac{ (\Delta x \nabla)^n }{n!} f(x) 
        \\[5pt]
    \Aboxed{ f(x + \Delta x) &= e^{\frac{i}{\hbar} (\Delta x) \breve{p}} \ f(x) }
        \ \Leftrightarrow \ \boxed{ f(x) = e^{\frac{i}{\hbar} x \breve{p}} \ f(0) }
        \hspace{10pt} *\ \langle x | e^{\frac{i}{\hbar} x \hat{p}} | x' \rangle 
        =  e^{\frac{i}{\hbar} x \breve{p}} \langle x | x' \rangle \ *
\end{align*}    

% Time Translation
\vspace{20pt} \noindent
\(\begin{gathered}
    \text{Time Translation}:\\[15pt]
    \langle x_N | \hat{U} \mss{(t)} | x_0 \rangle \\
    = \breve{U} \mss{(x_N, t;\hs x_0, 0)}
        \\[10pt]
    \langle x | \hat{U} \mss{(t)} | \Psi \rangle \\
    = \Psi \mss{(x,t)}
\end{gathered}\) \hspace{5pt} \vline \hspace{5pt}
\( \begin{gathered}
    \displaystyle f(t + \Delta t) 
        \ =\ \underline{ f(t) + f'(t) \Delta t } + ... 
        \ = \ \sum_n \frac{ (\Delta t)^n }{n!} 
        \left( \frac{\partial}{\partial t} \right)^n f(t) 
        \\[5pt]
    \begin{gathered}
            i\hbar \frac{\partial}{\partial t} \Psi = \breve{H}\mss{(x,p,t)} \Psi \\[10pt]
            \begin{aligned}
                \frac{\partial f}{\partial t} & = \left[ \tfrac{- i \breve{H}}{\hbar} \right] f \\[10pt]
                \tfrac{\partial^n f}{\partial t^n} & \neq \left[ \tfrac{- i \breve{H}(t)}{\hbar} \right]^n f 
            \end{aligned}
        \end{gathered}
        \Rightarrow \hspace{5pt}
        \left\{ \ \begin{aligned}
            & f(t_0 + \Delta t) \ \approx\ e^{\frac{-i \Delta t}{\hbar} \breve{H}(t_0)} \hs f(t_0) 
                \hspace{15pt} \text{\scriptsize(1st order)}
                \\[5pt]
            & \boxed{ \begin{aligned}
                    f(0 + t) &
                        = \lim_{N\rightarrow\infty} \ \prod^{N-1}_{n=0}\ 
                        e^{\frac{-i}{\hbar} \breve{H}(n \Delta t) \Delta t} \hs f(0)
                        \\[5pt]
                    \sim &
                        \neq e^{\frac{-i}{\hbar} \int \breve{H}(t) \hs\hs dt} \hs f(0)
                        \hspace{15pt} \left(\hs\begin{gathered}
                            \text{\scriptsize since} \ \mss{ 0 \neq } \\[-5pt]
                            \mss{ [H(t_0),H(t_1)]} 
                        \end{gathered}\hs\right) 
                \end{aligned}
                }
            %     \\[2pt]
            % & *\ \langle x | e^{\frac{-i}{\hbar} \int \hat{H}(t) \hs\hs dt} | x' \rangle 
            %     = e^{\frac{-i}{\hbar} \int \breve{H}(t) \hs\hs dt} \langle x | x' \rangle
            %     \ *
        \end{aligned} \right.
\end{gathered} \)

% Pictures
\vspace{30pt}\noindent
\underline{Pictures:} \hspace{18pt} \(
    \big\langle Q \big\rangle {\scriptstyle(t)} = \left\langle \Psi{\scriptstyle(x,t)} 
    \left| \ Q{\scriptstyle (x,p,t)} \ \right| \Psi{\scriptstyle(x,t)} \right\rangle 
\)
\begin{itemize}
    % Schrodinger Picture
    \item \fbox{
        \tabcolsep=2pt\begin{tabular}{p{4cm} l}
            \hfil Schrodinger Picture: & \(
                \big\langle Q \big\rangle {\scriptstyle(t)} = 
                \left\langle 
                e^{\frac{-i}{\hbar} t \widehat{H}} \hs \Psi{\scriptstyle(t=0)} 
                \Bigl| \ Q{\scriptstyle (x,p,t)} \ \Bigr| 
                e^{\frac{-i}{\hbar} t \widehat{H}} \hs \Psi{\scriptstyle(t=0)} 
                \right\rangle 
            \) 
        \end{tabular}

    }\\[15pt]
    % Stationary States for Schrodinger
    \hspace{18pt} \( 
        Q = Q{(x,p)} \ \Rightarrow \  \big\langle Q \big\rangle {\scriptstyle(t)} = 
        \left\langle 
        \sum e^{\frac{-i}{\hbar}E_nt} \hs c_n \Psi_n{\scriptstyle(x)}
        \Bigl| \ Q \ \Bigr| 
        \sum e^{\frac{-i}{\hbar}E_nt} \hs c_n \Psi_n{\scriptstyle(x)} 
        \right\rangle 
    \) \ \ \ {\scriptsize (nice for stationary states)}

    % Heisenberg Picture
    \item \fbox{
        \tabcolsep=2pt\begin{tabular}{p{4cm} l}
            \hfil Heisenberg Picture: & \( 
                \big\langle Q \big\rangle {\scriptstyle(t)} 
                = \left\langle \Psi{\scriptstyle(t=0)} \Bigl| 
                e^{\frac{i}{\hbar} t \widehat{H}} 
                \hs Q \hs 
                e^{\frac{-i}{\hbar} t \widehat{H}} 
                \Bigr| \Psi{\scriptstyle(t=0)} \right\rangle
            \)  
        \end{tabular}
    }

    % Dirac Picture
    \item \fbox{
        \tabcolsep=2pt\begin{tabular}{p{4cm} l}
            \hfil Dirac Picture: & \( 
                \big\langle Q \big\rangle {\scriptstyle(t)} 
                = \left\langle e^{\frac{-i}{\hbar} \int \widehat{H}_1 \mss{(t)} dt}  \hs \Psi{\scriptstyle(t=0)} \left| 
                e^{\frac{i}{\hbar} t \widehat{H}_0} 
                \hs Q \hs
                e^{\frac{-i}{\hbar} t \widehat{H}_0} 
                \right| e^{\frac{-i}{\hbar} \int \widehat{H}_1 \mss{(t)} dt} \hs \Psi{\scriptstyle(t=0)} \right\rangle
            \)
        \end{tabular}
    }
\end{itemize}

\vspace{5pt}
\noindent \( 
    \big\langle Q \big\rangle {\scriptstyle (t + \Delta t)} 
    \ = \ \big\langle Q \big\rangle {\scriptstyle (t)}
    + \frac{d \langle Q \rangle}{dt} \Delta t + ... 
\) 
\hspace{5pt} \(\Rightarrow\) \hspace{5pt}
\begin{minipage}{.5\textwidth}
    A 1st order approximation of \ \( \big\langle Q \big\rangle {\scriptstyle (t + \Delta t)} \) \\[5pt]
    \ should yield \ \( \frac{d \langle Q \rangle}{dt}
        = \frac{1}{i\hbar} \Big\langle \big[ Q, H \big] \Big\rangle + \frac{\partial Q}{\partial t} \)
\end{minipage}

%-----------------------------------------------------------------------------------------------------------------------------------
%
%
%
% Pictures Cont.
\newpage\noindent
\parbox[t]{.49\textwidth}{
    % Schrodinger Picture
    \underline{Schrodinger Picture}

    \vspace{5pt}
    \(
        \begin{aligned}[t]
            1.)\ & i\hbar \hs \tfrac{\partial}{\partial t} \langle Q_S \rangle = \langle \left[ Q_S, H_S \right] \rangle      
                \\[5pt]
            4.)\ & | \Psi_S\mss{(t)} \rangle = U_S \mss{(t,t_0)} | \Psi_S \mss{(t_0)} \rangle
                \\[5pt]
            & \Rightarrow\ \begin{aligned}[t]
                    & i\hbar \hs \tfrac{\partial}{\partial t} | \Psi_S \rangle = H_S \Psi_S
                        = [ H_S^0 + H_S^1 \mss{(t)} ] | \Psi_S \rangle
                        \\[3pt]
                    & = \mss{ \sum } \ E_n | n_S^0 \rangle \hs e^{- \frac{i}{\hbar} E_n t} \hs
                        \langle n_S^1 \mss{(t)} | \Psi \mss{(0)} \rangle
                        \\
                    & \hspace{13pt} + \mss{ \sum } \ | n_S^0 \rangle \hs e^{- \frac{i}{\hbar} E_n t} 
                        \cdot i\hbar \tfrac{\partial}{\partial t} \langle n_S^1 \mss{(t)} | \Psi \mss{(0)} \rangle
                \end{aligned}
                \\[5pt]
            & \Rightarrow\ \begin{aligned}[t]
                    & i\hbar \hs \tfrac{\partial}{\partial t} U_S^0 \mss{(t,t_0)} = H_S^0 U_S^0 \mss{(t,t_0)} \\
                    & \Rightarrow\ U_S^0 \mss{(t,t_0)} = e^{- \tfrac{i}{\hbar} H_S^0 (t-t_0)}
                \end{aligned}
        \end{aligned}
    \)
}
\hfill
\parbox[t]{.45\textwidth}{
    % Heisenberg Picture
    \underline{Heisenberg Picture}

    \vspace{5pt}
    \(
        \begin{aligned}[t]
            % 
            1.)\ & 
                \begin{aligned}[t]
                    & Q_H \mss{(t)} \ \equiv\ U_S^\dagger Q_S U_S\\
                    & \Rightarrow\ i\hbar \hs \tfrac{\partial}{\partial t} Q_H = \left[ Q_H, H_H \right]
                \end{aligned}      
                \hspace{20pt}
                \mss{ \begin{gathered}[t]
                    H_S \neq H_S(t) \\[-5pt]
                    \Downarrow\\[-5pt]
                    H_H = H_S
                \end{gathered} }
                \\[5pt]
            2.)\ & U_H \ \equiv\ U_S^\dagger \mss{(t,t_0)} U_S \mss{(t,t_0)} = \mathbb{I}
                \\[5pt]
            3.)\ & \begin{aligned}[t]
                    & | \hs q_H \mss{(t)} \hs \rangle \ \equiv\ U_S^\dagger \mss{(t,t_0)} \hs | \hs q_S \hs \rangle \\[3pt]
                    & \Rightarrow\ Q_H | \hs q_H \mss{(t)} \hs \rangle = q | \hs q_H\mss{(t)} \hs \rangle\\[3pt]
                    & \Rightarrow\ i\hbar \tfrac{\partial}{\partial t} | \hs q_H \mss{(t)} \hs \rangle
                        = - H_S | \hs q_H \mss{(t)} \hs \rangle
                \end{aligned}
                \\[5pt]
            4.)\ & \begin{aligned}[t]
                    & \begin{aligned}[t]
                            | \Psi_H \rangle & \ \equiv\ U_S^\dagger \mss{(t,t_0)} | \Psi_S \mss{(t)} \rangle 
                                = | \Psi_S \mss{(t_0)} \rangle
                                \\
                            & \ =\ U_H \mss{(t,t_0)} | \Psi_H \mss{(t_0)} \rangle 
                        \end{aligned}
                        \\[5pt]
                    & \Rightarrow\ i\hbar \hs \tfrac{\partial}{\partial t} | \Psi_H \rangle = 0
                \end{aligned}
        \end{aligned}
    \)
}

\vspace{20pt}\noindent
% Dirac/Interaction Picture
\underline{Dirac/Interaction Picture (see transition amplitude)}\\[5pt]
\(
    \begin{aligned}
        1.)\ & Q_I \mss{(t)} \ \equiv\ {U_S^0}^\dagger Q_S U_S^0
            \\
        & \Rightarrow\ i\hbar \hs \tfrac{\partial}{\partial t} Q_I = \left[ Q_I, H_I^0 \right]
            \hspace{15pt} \mss{(H_S^0 = H_I^0 \ \text{ see Heis. pic.} )}
            \\[5pt]
        2.)\ & U_I \mss{(t,t_0)} \ \equiv\ {U_S^0}^\dagger \mss{(t,t_0)} U_S \mss{(t,t_0)}
            \\[5pt]
        3.)\ & \begin{aligned}[t]
                & | \hs q_I \mss{(t)} \hs \rangle \ \equiv\ {U_S^0}^\dagger \mss{(t,t_0)} \hs | \hs q_S \hs \rangle \\[3pt]
                & \Rightarrow\ Q_I | \hs q_I \mss{(t)} \hs \rangle = q | \hs q_I\mss{(t)} \hs \rangle\\[3pt]
                & \Rightarrow\ i\hbar \tfrac{\partial}{\partial t} | \hs q_I \mss{(t)} \hs \rangle
                    = - H_S^0 | \hs q_I \mss{(t)} \hs \rangle
            \end{aligned}
            \\[5pt]
        4.)\ & \begin{aligned}[t]
                | \Psi_I \mss{(t)} \rangle & \ \equiv\ {U_S^0}^\dagger \mss{(t,t_0)} | \Psi_S \mss{(t)} \rangle\\[3pt]
                & \ =\ U_I \mss{(t,t_0)} | \Psi_I \mss{(t_0)} \rangle
                    \hspace{15pt} \text{\scriptsize(since \(| \Psi_I \mss{(t_0)} \rangle = | \Psi_S \mss{(t_0)} \rangle\))}
            \end{aligned}
            \\[5pt]
        & \Rightarrow\ i\hbar \hs \tfrac{\partial}{\partial t} | \Psi_I \rangle 
            = {U_S^0}^\dagger H_S^1 \mss{(t)} U_S^0 | \Psi_I \rangle
            = H_I^1 | \Psi_I \rangle
            \\[5pt]
        & \Rightarrow\ \begin{aligned}[t]
                & i\hbar \hs \tfrac{\partial}{\partial t} U_I \mss{(t,t_0)} = H_I^1 \mss{(t)} U_I \mss{(t,t_0)} \\
                & \Rightarrow\ U_I \mss{(t,t_0)} = \mathbb{I} 
                    + \tfrac{1}{i\hbar} \mss{ \int_{t_0}^t } H_I^1 \mss{(t')} \hs U_I \mss{(t', t_0)} \hs\hs dt'
            \end{aligned}
    \end{aligned}
\)
\hfill
\vline
\hfill
\(
    \begin{aligned}
        % Dirac Picture Propagator
        & \bullet\ \begin{aligned}[t]
                U_I & \mss{(t,t_0)} = \mathbb{I} 
                    + \tfrac{1}{i\hbar} \mss{ \int_{t_0}^t } H_I^1 \mss{(t')} \hs U_I \mss{(t', t_0)} 
                    \hs\hs dt'
                    \\[5pt]
                = & \ \mathbb{I} + \mathcal{O}(H_I^1)\\[5pt]
                = & \ \mathbb{I} + \tfrac{1}{i\hbar} \mss{ \int_{t_0}^t } H_I^1 \mss{(t')} \hs\hs dt' + \mathcal{O}( [H_I^1]^2 )\\
                = & \ \mathbb{I} + \tfrac{1}{i\hbar} \mss{ \int_{t_0}^t } H_I^1 \mss{(t')} \hs\hs dt'\\
                & + \left( \tfrac{1}{i\hbar} \right)^2 \mss{ \int_{t_0}^t \int_{t_0}^{t'} } 
                    H_I^1 \mss{(t')} H_I^1 \mss{(t'')} 
                    \hs\hs dt'' dt'
                    + \dots
            \end{aligned}  
    \end{aligned}
\)

\vspace{20pt}\noindent
% Normal Schrodinger Propagator as Sum of all paths
\(\bullet\ 
    \begin{aligned}[t]
        U_S \mss{(t,t_0)} = & \ U_S^0
            + \tfrac{1}{i\hbar} \mss{ \int_{t_0}^t } 
                U_S^0 \hs H_I^1 \mss{(t')} 
                \hs\hs dt'
            + \left( \tfrac{1}{i\hbar} \right)^2 \mss{ \int_{t_0}^t \int_{t_0}^{t'} } 
                U_S^0 \hs H_I^1 \mss{(t')} \hs H_I^1 \mss{(t'')} 
                \hs\hs dt'' dt'
            + \dots
            \\
        = & \ U^0 \mss{(t,t_0)} \\
        & + \tfrac{1}{i\hbar} \mss{ \int_{t_0}^t } 
            U^0 \mss{(t,t_0)} \hs 
            {U^0}^\dagger \mss{(t',t_0)} \hs H^1 \hs U^0 \mss{(t',t_0)} 
            \hs\hs dt'
            \\
        & + \left( \tfrac{1}{i\hbar} \right)^2 \mss{ \int_{t_0}^t \int_{t_0}^{t'} } 
            U^0 \mss{(t,t_0)} \hs 
            {U^0}^\dagger \mss{(t',t_0)} \hs H^1 \hs U^0 \mss{(t',t_0)} 
            {U^0}^\dagger \mss{(t'',t_0)} \hs H^1 \hs U^0 \mss{(t'',t_0)} 
            \hs\hs dt'' dt'
            + \dots
    \end{aligned}  
\)

%-----------------------------------------------------------------------------------------------------------------------------------
%
%
%
% Infinitismal time Path Integral
\newpage\noindent
\underline{Infinitismal \(t\) Path Integral}

% Action
\vspace{15pt}\noindent
\(\begin{aligned}
    S[x(t)] & 
        = \int_{0}^{t} \mathcal{L}(x, \dot{x}) \hs dt
        \ \rightarrow\
        \mathcal{L} \hs \delta t = 
        \left[ 
            \tfrac{1}{2}m \left( \tfrac{ x_{1} - x_{0} }{\delta t} \right)^2
            - V \mss{ \left( \tfrac{x_{1} + x_{0}}{2}, t_0 + \tfrac{\delta t}{2} \right) }
        \right] \delta t
        \\[5pt]
\end{aligned}\)

% Propagator
\vspace{10pt}\noindent
\(    
    \langle x | \hat{U}\mss{(\epsilon)} | \Psi \rangle
    = \int \langle x| \hat{U}\mss{(\epsilon)} | x' \rangle \langle x' | \Psi\mss{(x,t)} \rangle \hs dx'
    = \int \breve{U}\mss{(x, t + \epsilon;\hs x', t)} \Psi\mss{(x',t)} \hs dx'
    = \Psi\mss{(x,t+\epsilon)}
\)

% Propagator Taylor Expansion
\vspace{10pt}\noindent
\(\hspace{20pt} \bullet\ \begin{aligned}[t]
    \breve{U}\mss{(x_1,\epsilon;\hs\hs x_0, 0)} & 
        = A e^{\tfrac{i}{\hbar} S} 
        = A e^{\tfrac{i}{\hbar} \mathcal{L} \epsilon} 
        \\
    & = A \exp{
            \left\{
                \tfrac{i}{\hbar} 
                \left[ 
                    \tfrac{1}{2}m \tfrac{ \left( x_1 - x_0 \right)^2}{\epsilon} 
                    - \epsilon V \mss{ \left( \tfrac{x_1 + x_0}{2}, 0 + \tfrac{\epsilon}{2} \right) }
                \right]
            \right\}
        }
        \\
    \mss{(\eta = x_0 - x_1)} \hspace{10pt} & = A \exp{
            \left\{
                \tfrac{i}{\hbar} 
                \left[ 
                    \tfrac{1}{2}m \tfrac{\eta^2}{\epsilon} 
                \right]
            \right\}
        } 
        \exp{
            \left\{
                - \tfrac{i}{\hbar}
                \epsilon V \mss{ \left( x_1 + \tfrac{\eta}{2}, 0+ \tfrac{\epsilon}{2} \right) } 
            \right\}
        }
        \\
    & \approx A \exp{
            \left\{
                \tfrac{i}{\hbar} 
                \left[ 
                    \tfrac{1}{2}m \tfrac{\eta^2}{\epsilon} 
                \right]
            \right\}
        } 
        \exp{
            \left\{
                - \tfrac{i}{\hbar}
                \epsilon V \mss{\left( x_1 , 0 \right)} 
            \right\}
        }
        \\
    & \approx A \exp{
            \left\{
                \tfrac{i}{\hbar} 
                \left[ 
                    \tfrac{1}{2}m \tfrac{\eta^2}{\epsilon} 
                \right]
            \right\}
        } 
        \left[ 1 - \tfrac{i}{\hbar} \epsilon V \mss{ \left( x_1 , 0 \right) } \right]
\end{aligned}\)
\hfill
\fbox{
    \begin{minipage}[t]{.3\textwidth}
        % All Paths Explanation
        \scriptsize 

        \(\tfrac{\eta^2}{\epsilon} \lesssim \pi\) \ Explanation
    
        \vspace{10pt}
        The integral involving \(\breve{U}\) is over all \(\eta\). The phase of the complex exponential will vary/oscillate %
        too wildly and destructively interfere if \(\eta^2/\epsilon\) were to grow too big, so \(\eta^2 \sim \epsilon\) %
        is all that matters. %
        This means the integral is over \(\sqrt{\epsilon}\), not \(\epsilon\). Because of this (somehow, Bibl. given), using a %
        finite difference formula for derivatives is legitimate in this case, though not in general.
    \end{minipage}
}    

% Schrodinger Equivalence
\vspace{10pt}\noindent
\(\begin{aligned}
    \Psi\mss{(x,\epsilon)} & = \int_{-\infty}^\infty \breve{U}\mss{(x,\epsilon;\hs x', 0)} \Psi\mss{(x',0)} \hs dx'\\
    & = A \int_{-\infty}^\infty e^{\tfrac{i}{\hbar} S} \Psi\mss{(x',0)} \hs dx'\\
    & = A \int_{-\infty}^\infty 
        \exp{
            \left\{
                \tfrac{i}{\hbar} 
                \left[ 
                    \tfrac{1}{2}m \tfrac{ \left( x - x' \right)^2}{\epsilon} 
                    - \epsilon V \mss{ \left( \tfrac{x + x'}{2}, 0+ \tfrac{\epsilon}{2} \right) }
                \right]
            \right\}
        }
        \Psi\mss{(x',0)} \hs dx'
        \\
    & = A \int_{-\infty}^\infty 
        \exp{
            \left\{
                \tfrac{i}{\hbar} 
                \left[ 
                    \tfrac{1}{2}m \tfrac{\eta^2}{\epsilon} 
                    - \epsilon V \mss{ \left( \tfrac{x + \eta/2}{2}, 0 + \tfrac{\epsilon}{2} \right) }
                \right]
            \right\}
        }
        \Psi\mss{(x + \eta,0)} \hs d\eta
        \\
    & \approx A
        \int_{-\infty}^\infty e^{
            \tfrac{i}{\hbar} \left[ \tfrac{1}{2}m \tfrac{\eta^2}{\epsilon} \right]
        } 
        \left[ 1 - \tfrac{i}{\hbar} \epsilon V \mss{ \left( x , 0 \right) } \right]
        \left[ \Psi\mss{(x,0)} + \eta \Psi'\mss{(x,0)} + \tfrac{\eta^2}{2} \Psi''\mss{(x,0)} \right] d\eta
        \\
    & \approx A
        \int_{-\infty}^\infty e^{
            \tfrac{i}{\hbar} \left[ \tfrac{1}{2}m \tfrac{\eta^2}{\epsilon} \right]
        }
        \left[
            \left( 1 - \tfrac{i}{\hbar} \epsilon V \mss{ \left( x , 0 \right) } \right) \Psi\mss{(x,0)} 
            + \cancel{ \eta \Psi'\mss{(x,0)} }
            + \tfrac{\eta^2}{2} \Psi''\mss{(x,0)} 
        \right] d\eta
        \\
    & = \bcancel{ A \sqrt{ \tfrac{2\hbar\epsilon\pi}{-im} } } 
        \left[
            \left( 1 - \tfrac{i}{\hbar} \epsilon V \mss{ \left( x , 0 \right) } \right) 
            + \tfrac{1}{2} \cdot \tfrac{2\hbar\epsilon }{-i m} \cdot \tfrac{1}{2} \tfrac{\partial^2}{\partial x^2}
        \right] \Psi\mss{(x,0)} 
        \\
    & = \Psi\mss{(x, 0)} - \tfrac{i}{\hbar}\epsilon \breve{H} \Psi\mss{(x, 0)}
\end{aligned}\)

\vspace{5pt}\noindent
\( \boxed{
    i\hbar \ \frac{ \Psi\mss{(x, \epsilon)} - \Psi\mss{(x, 0)} }{\epsilon - 0}
    = \breve{H} \Psi\mss{(x, 0)}
} \)

%-----------------------------------------------------------------------------------------------------------------------------------
%
%
%
% Finite time Path Integral
\newpage
\noindent
\underline{Finite \(t\), Free Particle Propagator}

% Action
\vspace{10pt}\noindent
\( \displaystyle
    S[x(t)]
        = \int_{0}^{t} \mathcal{L}(x, \dot{x}) \hs dt 
        = \lim_{N\rightarrow\infty} \sum_{n=0}^{N-1} \left[ 
            \tfrac{1}{2}m \left( \tfrac{ x_{n+1} - x_{n} }{\delta t} \right)^2
        \right] \delta t
\)

% Propagator
\vspace{10pt}\noindent
\(
    \langle x_N | \hat{U}\mss{(t)} | \Psi\mss{(0)} \rangle 
    = \int \langle x_N | \hat{U}\mss{(t)} | x_0 \rangle \Psi\mss{(x_0, 0)} \ dx_0
    = \int \breve{U}\mss{(x_N, t;\hs x_0, 0)} \Psi\mss{(x_0, 0)} \ dx_0
    = \Psi\mss{(x_N, t)}
\)

\vspace{10pt}\noindent
\(\Rightarrow\ \begin{aligned}[t]
    \langle x_N | & \hat{U}\mss{(t)} | x_0 \rangle 
        = \langle x_N | e^{- \frac{i}{\hbar}H t} e^{\frac{i}{\hbar}H t_0} | x_0 \rangle 
        = \langle x_N, t_N \hs |\hs x_0, t_0 \rangle 
        \\[5pt]
    \langle x_N | & \hat{U}\mss{(t)} | x_0 \rangle 
        = \lim_{N\rightarrow\infty} \hs
        \langle x_N | \hat{U}^{N} \hspace{-1pt} \mss{(\epsilon)} | x_0 \rangle
        = \lim_{N\rightarrow\infty} \hs
            \langle x_N | 
            \hat{U}\mss{(\epsilon)} \dots \hat{U}\mss{(\epsilon)} \hat{U}\mss{(\epsilon)} 
            | x_0 \rangle
        \\[5pt]
    & = \lim_{N\rightarrow\infty}
        \int_{-\infty}^{\infty} \dots \int_{-\infty}^{\infty} \int_{-\infty}^{\infty} 
        \langle x_N | \hat{U}\mss{(\epsilon)} | x_{N-1} \rangle \dots 
        \langle x_2 | \hat{U}\mss{(\epsilon)} | x_1 \rangle 
        \langle x_1 | \hat{U}\mss{(\epsilon)} | x_0 \rangle  
        \ dx_1 \hs dx_2 \dots dx_{N-1}
\end{aligned}\)

\vspace{15pt}\noindent
\( \displaystyle \breve{U}\mss{(x_N, t;\hs x_0, 0)} = \int_{x_0}^{x_N} A e^{\tfrac{i}{\hbar}S} \hs\hs \mathcal{D}[x\mss{(t)}] \)

\vspace{5pt}\noindent
\hspace{10pt} \(\begin{aligned}
    & = \lim_{N \rightarrow \infty} 
        A 
        \int_{-\infty}^{\infty} \dots \int_{-\infty}^{\infty} 
        e^{ 
            \tfrac{i}{\hbar} 
            \left[ 
                \tfrac{1}{2}m \tfrac{ ( x_{1} - x_{0} )^2 }{\epsilon} 
                + \tfrac{1}{2}m \tfrac{ ( x_{2} - x_{1} )^2}{\epsilon} 
                + \tfrac{1}{2}m \tfrac{ ( x_{3} - x_{2} )^2 }{\epsilon}
                + \dots
            \right]
        }
        dx_1 \hs dx_2 \dots \hs dx_{N-1}
        \\[5pt]
    & = \lim_{N \rightarrow \infty} 
        A \sqrt{ \tfrac{2 \hbar \epsilon}{m} }^{N-1}
        \int \dots \int_{-\infty}^{\infty} 
        e^{ 
            - \tfrac{ ( y_{1} - y_{0} )^2 }{i}
            - \tfrac{ ( y_{2} - y_{1} )^2 }{i}
        }
        \hs\hs dy_1 
        \hs 
        e^{ 
            \left[ 
                - \tfrac{ ( y_{3} - y_{2} )^2 }{i}
                \dots 
            \right]
        }
        \hs dy_2 \dots
        \\[5pt]
    & = \lim_{N \rightarrow \infty} 
        A \sqrt{ \tfrac{2 \hbar \epsilon}{m} }^{N-1}
        \int \dots \int_{-\infty}^{\infty} 
        e^{ 
            - \tfrac{ y^2_{1} + y^2_{1} - 2y_1 (y_2 + y_0) + y^2_{2}+ y^2_{0} }{i}
        }
        \hs\hs dy_1 \dots
        \\[5pt]
    & = \lim_{N \rightarrow \infty} 
        A \sqrt{ \tfrac{2 \hbar \epsilon}{m} }^{N-1}
        \int \dots \int_{-\infty}^{\infty} 
        e^{ 
            - \tfrac{1}{i} \left(  
                2 \left[ y^2_{1} - y_1 (y_0 + y_2) + \tfrac{(y_0 + y_2)^2}{4} \right]
                - \tfrac{(y_2 + y_0)^2}{2} 
                + y^2_{2}+ y^2_{0} 
            \right)
        }
        \hs\hs dy_1 \dots
        \\[5pt]
    & = \lim_{N \rightarrow \infty} 
        A \sqrt{ \tfrac{2 \hbar \epsilon}{m} }^{N-1}
        \int \dots \int_{-\infty}^{\infty} 
        e^{ 
            - \tfrac{2}{i} 
            \left[ y_{1} - \cancel{ \tfrac{y_2 + y_0}{2} } \right]^2
        }
        \hs\hs dy_1 
        \hs e^{ - \tfrac{(y_2 - y_0)^2}{2i} } e^{ - \tfrac{ ( y_{3} - y_{2} )^2 }{i} \dots }
        \hs\hs dy_2
        \hs\hs e^{ \left[ \dots \right] }
        \dots
        \\[5pt]
    & = \lim_{N \rightarrow \infty} 
        A \sqrt{ \tfrac{2 \hbar \epsilon}{m} }^{N-1}
        \int \dots         
        \sqrt{\tfrac{\pi i}{2}}
        \int_{-\infty}^{\infty} 
        e^{ - \tfrac{(y_2 - y_0)^2}{2i} } e^{ - \tfrac{ ( y_{3} - y_{2} )^2 }{i} \dots }
        \hs\hs dy_2
        \dots
        \\[5pt]
    & = \lim_{N \rightarrow \infty} 
        A \sqrt{ \tfrac{2 \hbar \epsilon}{m} }^{N-1}
        \int \dots         
        \sqrt{\tfrac{(\pi i)^2}{3}}
        \int_{-\infty}^{\infty} 
        e^{ - \tfrac{(y_3 - y_0)^2}{3i} } e^{ - \tfrac{ ( y_{4} - y_{3} )^2 }{i} \dots }
        \hs\hs dy_3
        \dots
        \\[5pt]
    & = \lim_{N \rightarrow \infty} 
        A \sqrt{ \tfrac{2 \hbar \epsilon}{m} }^{N-1} 
        \sqrt{\tfrac{(\pi i)^{N-1}}{N}} \ e^{- \tfrac{(y_N - y_0)^2}{Ni}}
        \ =\ \lim_{N \rightarrow \infty} 
        \bcancel{ A \sqrt{ \tfrac{2 \hbar \epsilon \pi}{-i m} }^{N} }
        \sqrt{\tfrac{-i m}{2 \hbar \epsilon N \pi}} 
        \ e^{\tfrac{i}{\hbar} \tfrac{m}{2} \tfrac{(x_N - x_0)^2}{N\epsilon}}
\end{aligned}\)

\vspace{10pt}\noindent
\(\begin{aligned}
    & \boxed{ 
        \text{Free Particle}:\ 
        \breve{U}\mss{(x_N, t;\hs x_0, 0)}
        = \sqrt{\tfrac{-i m}{2 \hbar t \pi}} \ e^{\tfrac{i}{\hbar} \tfrac{m}{2} \tfrac{(x_N - x_0)^2}{t}} 
        }
        \\[5pt] 
    & \boxed{ 
        \int_{x_0}^{x_N} \mathcal{D}[x\mss{(t)}]
        = \lim_{N\rightarrow\infty} 
        \sqrt{ \tfrac{-i m}{2 \hbar \epsilon \pi} }
        \int_{-\infty}^{\infty} \sqrt{ \tfrac{-i m}{2 \hbar \epsilon \pi} } \hs\hs dx_1
        \dots
        \int_{-\infty}^{\infty} \sqrt{ \tfrac{-i m}{2 \hbar \epsilon \pi} } \hs\hs dx_{N-1} 
        }
\end{aligned}\)

%-----------------------------------------------------------------------------------------------------------------------------------
%
%
%
\newpage\noindent
\fbox{
\begin{minipage}[t]{.4\textwidth}
    % All Paths Explanation
    \underline{All Paths Explanation}

    % \scriptsize 
    \vspace{10pt}
    \(S[x] = S[x_{cl}] + \cancel{S'[x_{cl}]} \eta + \mathcal{O}(\eta^2)\)

    \vspace{10pt}
    1\(^\text{st}\) order variation of \(S\) from \(x_{cl}\) equals 0. This means propagator integrand for paths %
    near \(x_{cl}\) will have about the same phase, and will add constructively. Paths very different from \(x_{cl}\) %
    (like those with faster than light motion) will vary in action, and because \(\hbar\) is so small their phases %
    will vary wildly, meaning the sum will destructively interfere. The result is that only paths near the classical path %
    will be important, with \(S[x]/\hbar \lesssim \pi\). 
\end{minipage}
}
\hfill
\begin{minipage}[t]{.54\textwidth}
    % Action-Energy Relationship
    % \Delta S_cl = -H(t_f) * \Delta t_f
    \underline{Action-Energy Relationship}

    \vspace{10pt}\noindent
    \(\displaystyle
        \begin{aligned}
            & S \mss{(x_{cl} + \Delta x_{cl} , \dot{x} + \Delta \dot{x}_{cl}, \tau + \Delta \tau)} 
                = S_{cl} + \Delta S_{cl} 
                \\
            & \begin{aligned}
                    & = \int_0^{\tau + \Delta \tau} 
                        \hspace{-20pt} \mathcal{L}(
                        \mss{ x_{cl} + \Delta x_{cl}, \hs \dot{x}_{cl} + \Delta \dot{x}_{cl}, \hs t }
                        ) \hs\hs dt
                        \\[7pt]
                    dS & = \tfrac{\partial S}{\partial \tau} d \tau 
                        + \left[ 
                            \tfrac{\partial S}{\partial x} dx 
                            + \tfrac{\partial S}{\partial \dot{x}} d\dot{x} 
                            \mss{ \ =\int (dL) dt }
                        \right]
                        \hspace{12pt} \mss{ \left(\eta = dx, \hs \eta(0)=0 \right) }   
                \end{aligned} 
        \end{aligned}
    \)

    \vspace{10pt}\noindent
    \(\begin{aligned}
        \Delta S_{cl} & = \mathcal{L}\mss{(\tau)} \hs \Delta \tau
            + \int_0^{\tau} \left. \tfrac{\partial \mathcal{L}}{\partial x} \right|_{cl} \hs \eta
            + \left. \tfrac{\partial \mathcal{L}}{\partial \dot{x}} \right|_{cl} \hs \dot{\eta}
            \hs\hs dt
            \\[5pt]
        & = \mathcal{L}\mss{(\tau)} \hs \Delta \tau
            + \int_0^{\tau} \cancel{ \left[ 
                \tfrac{\partial \mathcal{L}}{\partial x}
                - \tfrac{d}{dt} \tfrac{\partial \mathcal{L}}{\partial \dot{x}}
            \right]_{cl} } \hs \eta
            + \tfrac{d}{dt} \left[ \tfrac{\partial \mathcal{L}}{\partial \dot{x}_{cl}} \eta \right]
            dt
            \\[5pt]
        & = \left[ \mathcal{L} + \tfrac{\partial \mathcal{L}}{\partial \dot{x}} \dot{x} \right]_{cl, \tau} \hs \Delta \tau
            \\[5pt]
        \Aboxed{ \Delta S_{cl} & = -H\mss{(t_f)} \hs \Delta t_f }
    \end{aligned}\)
\end{minipage}

%-------------------------------------------------------------------
% Hamiltonian-Lagrangian Propagator Relationship
\vspace{20pt}\noindent
\underline{(Time-Independent) Hamiltonian-Lagrangian Propagator Relationship / Finite Path Integral}

\vspace{10pt}\noindent
\(\begin{aligned}
    \breve{U} & \mss{(x_N, t;\hs x_0, 0)} = \langle x_N | e^{- \tfrac{i}{\hbar} H t} | x_0 \rangle 
        = \langle x_N | [ e^{- \tfrac{i}{\hbar} H \tfrac{t}{N}} ]^N | x_0 \rangle 
        = \lim_{N\rightarrow\infty} 
        \langle x_N | 
        [ 
            e^{- \tfrac{i}{\hbar} \tfrac{\hat{p}^2}{2m} \epsilon} 
            e^{- \tfrac{i}{\hbar} V\mss{(\hat{x})} \epsilon} 
        ]^N 
        | x_0 \rangle 
        \hspace{15pt} \text{\scriptsize(not trivial)}
        \\[5pt]
    & = \lim_{N\rightarrow\infty} 
        \int_{x \dots}
        \langle x_N | 
            e^{- \tfrac{i}{\hbar} \tfrac{\hat{p}^2}{2m} \epsilon} 
            e^{- \tfrac{i}{\hbar} V\mss{(\hat{x})} \epsilon}
        | x_{N-1} \rangle 
        \dots
        \langle x_1 | 
            e^{- \tfrac{i}{\hbar} \tfrac{\hat{p}^2}{2m} \epsilon} 
            e^{- \tfrac{i}{\hbar} V\mss{(\hat{x})} \epsilon}
        | x_{0} \rangle 
        \hs\hs dx \dots
        \\[5pt]
    & = \lim_{N\rightarrow\infty}
        \int_{x \dots}
        .\hs\hs .\hs\hs .
        \Big[
            \underbrace{
                \mss{\int} \langle x_1 | 
                e^{- \tfrac{i}{\hbar} \tfrac{\hat{p}^2}{2m} \epsilon} 
                | p \rangle \langle p | x_{0} \rangle 
                \hs \mss{dp}
            }_\text{free part. prop.}
            \hs e^{- \tfrac{i}{\hbar} V\mss{(x_{0})} \epsilon}
        \Big]
        dx \dots
        \\[5pt]
    % Phase Space Propagator
    1. & = \lim_{N\rightarrow\infty}
        \int_{x \dots}
        .\hs\hs .\hs\hs .
        \Big[
            \mss{\int}  
            \tfrac{e^{\tfrac{i}{\hbar} p (x_1 - x_0)}}{2\pi\hbar}
            e^{- \tfrac{i}{\hbar} \tfrac{p^2}{2m} \epsilon} 
            e^{- \tfrac{i}{\hbar} V\mss{(x_{0})} \epsilon}
            \hs \mss{dp}
        \Big]
        dx \dots
        \ = \boxed{ 
            \int e^{\tfrac{i}{\hbar} \int p \dot{x} - H\mss{(x,p)} \hs dt} \hs\hs [\mathcal{D}x \mathcal{D}p]
            \hspace{15pt} \text{\scriptsize(Phase Space)} 
        }
        \\[5pt]
    % Configuration Space Propagator
    2. & = \lim_{N\rightarrow\infty}
        \int_{x \dots}
        .\hs\hs .\hs\hs . \
        \left[ 
            \sqrt{\tfrac{-im}{2\pi\hbar\epsilon}} 
            \ e^{\tfrac{im(x_1 - x_0)^2}{\epsilon}} 
            \hs e^{- \tfrac{i}{\hbar} V\mss{(x_{0})} \epsilon}
        \right]
        dx \dots
        = \lim_{N\rightarrow\infty}
        \int_{x \dots}
        .\hs\hs .\hs\hs . \
        \sqrt{\tfrac{-im}{2\pi\hbar\epsilon}} \hs e^{\tfrac{i}{\hbar}\mathcal{L}\epsilon}
        \hs\hs dx_1 \dots
        \\[5pt]
    & = \boxed{ 
            \int_{x_0}^{x_N} e^{\tfrac{i}{\hbar} S} \hs\hs [\mathcal{D}x]
            = \int_{x_0}^{x_N} e^{\tfrac{i}{\hbar} \int \mathcal{L} \hs dt} \hs\hs [\mathcal{D}x]
            \hspace{15pt} \text{\scriptsize(Configuration Space)}
        }
        \hspace{15pt} \text{\scriptsize(above is only integrable if \(p\) is quadratic in \(H\))}
\end{aligned}\)

%-----------------------------------------------------------
% Trace of Propagator
\vspace{20pt}\noindent
\underline{Trace of Propagator}\\[5pt]
\(\begin{aligned}
    G\mss{(t)} & = \int \langle x | e^{- \frac{i}{\hbar} H t} | x\rangle \hs d^3 x\\
    & = \sum_n \int \langle x | n \rangle e^{- \frac{i}{\hbar} E_n t} \langle n | x\rangle \hs dx\\
    G\mss{(t)} & = \sum_n e^{- \frac{i}{\hbar} E_n t} 
        \ \sim\ \sum_n e^{- \beta E_n} = Z \mss{(\beta)}
        \\
\end{aligned}\)

%-----------------------------------------------------------------------------------------------------------------------------------
%-----------------------------------------------------------------------------------------------------------------------------------
%-----------------------------------------------------------------------------------------------------------------------------------
%-----------------------------------------------------------------------------------------------------------------------------------
% Extra
\newpage \noindent
\subsection{Extra}

\vspace{10pt}\noindent
\(L_2 \subset \) Hilbert Space \(=\) complete inner product space

\vspace{10pt}\noindent
\( \displaystyle
    \rho\mss{(x,t)} \equiv \Vert \Psi \Vert^2 
    \ \ , \ \
    P_a^b(t) = \mss{\int_a^b} \rho \hs\hs dx 
    \ \ , \ \ 
    P(t) = P_{-\infty}^\infty(t)
    \ \ , \ \ 
    \Psi = \sqrt{\rho} \hs e^{\frac{i}{\hbar} S} \ \ \text{e.g., \ \(e^{\tfrac{i}{\hbar} (p \cdot x - Et) }\)}
\)
\begin{itemize}
    \item \(
        \begin{aligned}[t]
            \breve{E} \rho
                & = \breve{E} (\Psi^*\Psi)
                = \Psi^* (\breve{E} \Psi) + \Psi (\breve{E}\Psi^*)
                \\
            & = \Psi^* (\breve{H} \Psi) - \Psi (\breve{H}\Psi^*) \\
            & = \Psi^* (\tfrac{p^2}{2m} + V) \Psi - \Psi (\tfrac{p^2}{2m} + V) \Psi^*\\
            -\tfrac{\hbar}{i} \tfrac{\partial \rho}{\partial t} 
                & = \tfrac{\hbar}{i} \nabla \cdot \left( \Psi^* \tfrac{p}{2m} \Psi - \Psi \tfrac{p}{2m} \Psi^* \right)
        \end{aligned}
        \hfill\vline\hfill
        \begin{gathered}[t]
            \\[-20pt]
            \text{\scriptsize (Probability Current)}\\
            \begin{aligned}
                    \tfrac{\partial \rho}{\partial t} = - \nabla \cdot J &
                        = - \nabla \cdot \left( \Psi^* \tfrac{p}{2m} \Psi - \Psi \tfrac{p}{2m} \Psi^* \right)
                        \\[3pt]
                    & = - \nabla \cdot \tfrac{\rho \nabla S}{m} 
                        \ \ \ ( \text{\scriptsize e.g., \ \( \nabla S = p \)} )
                \end{aligned}
                \\[3pt]
            \left[ \tfrac{d}{dt} P_a^b = J \mss{(a,t)} - J\mss{(b,t)} \right] 
                ,\ \left[ \mss{\int} J \hs dV = \langle \Psi | \tfrac{p}{m} | \Psi \rangle = \tfrac{\langle p \rangle}{m} \right]
        \end{gathered}
        \)
        
    \vspace{5pt}
    \item \begin{tabular}[t]{l c l c l}
        \( (V \in \mathbb{R}) \) & \(\Rightarrow\) 
            & \( \frac{d}{dt} P = 0\) 
            & \(\Rightarrow\) & \( P(t) \equiv 1 \)\\[10pt]
        \( (V = V_0 - i\Gamma) \) & \(\Rightarrow\) 
            & \( \frac{d}{dt} P = \frac{- \ 2\Gamma}{\hbar} P \) 
            & \(\Rightarrow\) & \( P(t) = e^{-2 (\Gamma / \hbar) t}\)
    \end{tabular}

    \vspace{7pt}
    \item \( \langle \Psi_n | \Psi_n \rangle \ , \ \langle \Psi_m | \Psi_m \rangle = 1 
        \ \Rightarrow \ \frac{d}{dt} \langle \Psi_n | \Psi_m \rangle = 0 \)
\end{itemize}

\hfill \break
Schwarz Inequality: \hspace{18pt} 
\( \begin{aligned}
    \left\Vert \int_a^b f^*g \ dx \right\Vert^2 &\leq 
        \left\Vert \int_a^b f^*f \ dx \right\Vert 
        \left\Vert \int_a^b g^*g \ dx \right\Vert \\[5pt]
    \left\Vert \langle f|g \rangle_{ab} \right\Vert^2 &\leq 
        \left\Vert \langle f|f \rangle_{ab} \right\Vert
        \left\Vert \langle g|g \rangle_{ab} \right\Vert
\end{aligned} \)

\vspace{15pt}\noindent
\( \Big[ V(x) = V(-x) \Big] \ \Rightarrow \ \Big[ \Psi(x) \Rightarrow \Psi(-x) \Big] 
    \ \Rightarrow \ \Big[ \Psi(-x) = \Psi(x) \Big] 
    \ \cup \ \Big[ \Psi(-x) = -\Psi(x) \Big] \)

\vspace{15pt}\noindent
Discontinuity in \( \Psi \) means the possiblity of \( \sigma_p \rightarrow \infty \)\\[10pt]
\hspace{18pt} Prob 3.29: \( \Psi(x,0) = 
    \begin{cases}
        \frac{1}{\sqrt{2 n \lambda}} e^{2 \pi i x / \lambda}, & -n\lambda < x < n\lambda\\
        0 & \text{else}
    \end{cases} \)\\[10pt]
\hspace{18pt} \(\sigma_p \rightarrow \infty\) because the integral of \(\delta^2(x)\) is infinite 

\vspace{25pt} \noindent
\( \displaystyle
    \mss{\int_{-\infty}^\infty} f(x) D_1(x) dx = \mss{\int_{-\infty}^\infty} f(x) D_2(x) dx 
    \ \Rightarrow \
    \delta(cx) = \frac{1}{|c|} \delta(x)
\)\\[15pt]
\( \displaystyle
    \delta(x-x') = \frac{1}{{2\pi}} \mss{ \int_{-\infty}^\infty } e^{ik(x-x')} dx' 
    \ \Rightarrow \ 
    F[\delta(x)] = \frac{1}{2\pi}
\)\\[15pt]
\( \displaystyle
    \delta'(x-x') = \frac{1}{{2\pi}} \mss{\int_{-\infty}^\infty} ik e^{ik(x-x')} dx' 
    \ \Rightarrow\ 
    \mss{\int} \delta'(x-x') f(x') dx' = f'(x)
\)

%------------------------------------------------------------------------------------------------------------------------
%
%
%
\newpage 

% Poisson Brackets
\noindent
\parbox[t]{.42\textwidth}{
    \underline{Poisson Brackets}\\[10pt]
    \(\begin{aligned}    
        \{ f, g \} & = 
            \sum_i \frac{\partial f}{\partial q_i} \frac{\partial g}{\partial p_i} 
            - \frac{\partial g}{\partial q_i} \frac{\partial f}{\partial p_i} 
            \\[5pt]
        \{ \omega\mss{(q,p,t)}, H \} & = 
            \sum_i \frac{\partial \omega}{\partial q_i} \dot{q}
            + \dot{p} \frac{\partial \omega}{\partial p_i} 
            = \dot{\omega} - \tfrac{\partial \omega}{\partial t}
    \end{aligned}\)

    \vspace{8pt}\noindent
    \textit{Hamilton Eq.} : \ \(    
        \dot{q} = \{q, H\} , \hspace{5pt} \dot{p} = \{p,H\}
    \)
}
\hfill
% Canonical Transforms
\parbox[t]{.55\textwidth}{
    \underline{Canonical Transforms}\\[10pt]
    \(
        \begin{aligned}
            &q \rightarrow \bar{q}\mss{(q,p)}\\
            &p \rightarrow \bar{p}\mss{(q,p)}
        \end{aligned} 
        \hspace{7pt} \text{ s.t. } \hspace{7pt}
        \mss{ \begin{gathered}
            \{\bar{q}_i, \bar{q}_j\} = 0 = \{\bar{p}_i, \bar{p}_j\}\\[3pt]
            \{\bar{q}_i, \bar{p}_j\} = \delta_{ij}
        \end{gathered} }
        \hspace{15pt} 
        \left(
        \begin{gathered}
            \text{\scriptsize Point Transforms}\\[-5pt]
            \text{\scriptsize \(\bar{q}\mss{(q)}\) are canonical.}
        \end{gathered}
        \right)
    \)

    \vspace{15pt}
    \(
        \ \Rightarrow\
        \begin{aligned}
            &\dot{\bar{q}} = \tfrac{\partial H}{\partial \bar{p}}\\
            &\dot{\bar{p}} = - \tfrac{\partial H}{\partial \bar{q}}\\
        \end{aligned}
        \ \ , \ \
        \{ f, g \}_{q,p} = \{ f, g \}_{\bar{q},\bar{p}}    
    \)
}

% Generator of Transformation
\vspace{20pt}\noindent
\underline{Generator of Transformation}\\[10pt]
\(
    \begin{aligned}
        & 1.\ \delta H = 0\\[-20pt]
        & 2.\ \begin{aligned}[t]
                & \begin{aligned}[t]
                        \bar{q}_i & = q_i + \delta q_i 
                            \\
                        & \equiv q_i + \epsilon_\lambda \tfrac{\partial g}{\partial p_i} \\
                        & = q_i + \epsilon_\lambda \{q_i, g\}
                    \end{aligned}
                    , \hspace{5pt}
                    \begin{aligned}[t]
                        \bar{p}_i & = p_i + \delta p_i \\
                        & \equiv p_i - \epsilon_\lambda \tfrac{\partial g}{\partial q_i} \\    
                        & = p_i + \epsilon_\lambda \{p_i, g\}
                    \end{aligned}       
            \end{aligned}
            \ \Rightarrow\ 
            \begin{gathered}
                \begin{aligned}
                        \delta H & = \epsilon_\lambda \{H, g\} \\[5pt]
                        \Aboxed{ \tfrac{\partial H}{\partial \lambda} & = 0 = \tfrac{dg}{dt} }
                    \end{aligned}    
                    \\[5pt]
                \mss{( \text{e.g. } g=p \text{ or } g=l_z )}
            \end{gathered}    
            \\[10pt]
        & 3.\ \Rightarrow\ \delta f = \epsilon_\lambda \{f, g\} 
            \ \rightarrow\ \tfrac{\partial f}{\partial \lambda} = \{f, g\}
    \end{aligned}
\)
\hspace{35pt}
\(
    \begin{aligned}
        &g = l_z \\
        &\Rightarrow\ \begin{aligned}
                \delta x & = - \epsilon y = - (\delta \theta) y\\
                \delta y & = \epsilon x = (\delta \theta) x
            \end{aligned}
            \\[5pt]
        &\Rightarrow\ \boxed{ 
            \begin{aligned}
                \tfrac{\partial x}{\partial \theta} & = - y\\
                \tfrac{\partial y}{\partial \theta} & =  x
            \end{aligned} 
            }
    \end{aligned}
\)

\vspace{20pt}\noindent
% Tensors and Tensor Operators
\underline{Tensors and Tensor Operators}\\[10pt]
\(\begin{aligned}
    & \text{rank-2 Tensor}:\ \ | t^{(2)} \rangle = \mss{ \sum_{i=1}^{3} \sum_{j=1}^{3} }\ t_{ij} \hs | i \rangle | j \rangle 
        = \mss{ \sum_{i=1}^{3} \sum_{j=1}^{3} }\ | i j \rangle \langle i j | t^{(2)} \rangle
        \\[5pt]
    & \text{rank-\(2\) Carte. Tens. Oper.,}\ T_{ij}:\ \ \text{Set of } 3^{n=2} \text{ Operators}\\[5pt]
    & \text{rank-\(k\) Spher. Tens. Oper.,}\ T^q_k:\ \ \begin{aligned}[t]
            & \text{Set of } 2k+1 \text{ Operators \ s.t. }
                U[R] \hs T^q_k \hs U^\dagger [R] = \mss{ \sum^k_{q'=-k} }\ D^{k}_{q'q} \hs T^{q'}_k
                \\
            & \arraycolsep=2pt\begin{array}[t]{c r c l}
                    \Rightarrow & U T^q_k U^\dagger U | j m \rangle 
                        & = 
                        & \mss{ \displaystyle \sum_{q'} \sum_{m'} }\ D^k_{q'q} D^j_{m'm} \hs T^k_{q'} | j m' \rangle
                        \\[5pt]
                    & \sim\ U |k q \rangle | j m \rangle 
                        & = 
                        & \mss{ \displaystyle \sum_{q'} \sum_{m'} }\ D^k_{q'q} D^j_{m'm} | k q' \rangle | j m' \rangle
                \end{array}
        \end{aligned}
        \\[5pt]
    & \textit{Wigner-Eckhart}:\ \ \langle \alpha_2 j_2 m_2 | \hs T_k^q \hs | \alpha_1 j_1 m_1 \rangle
        = \langle \alpha_2 j_2 |\hs T_k \hs | \alpha_1 j_1 \rangle 
        \cdot \begin{gathered}[b]
            \text{\scriptsize(CG coeff.)}\\
            \langle j_2 m_2 | kq , j_1 m_1 \rangle
        \end{gathered}
\end{aligned}\)

\vspace{20pt}\noindent
% Functions and Polynomials
\underline{Functions}

\vspace{5pt}\noindent
Associated Legendre Functions: \hspace{18pt} \( \displaystyle
    P^m_l \equiv \sqrt{1-x^2}^{\ |m|} \left( \frac{d}{dx} \right)^{|m|} P_l(x) 
\) \hspace{18pt} {\scriptsize (not a polynomial if odd)}\\
Legendre Polynomials: \hspace{18pt} \( \displaystyle
    P_l(x) \equiv \frac{1}{2^l l!} \left( \frac{d}{dx} \right)^l (x^2-l)^l
\)

\vspace{10pt}\noindent
Associated Laguerre Polynomials: \hspace{18pt} \( \displaystyle
    L \equiv 
\) \\[10pt]
Laguerre Polynomials: \hspace{18pt} \( \displaystyle
    L_ \equiv 
\)
% P.23 - Prob. 1.18 

%--------------------------------------------------------------------------------------------
%--------------------------------------------------------------------------------------------
%--------------------------------------------------------------------------------------------
%--------------------------------------------------------------------------------------------
% Simple 1-D Potentials
\newpage
\section{Simple 1D Potentials}

% Inf Square Well
\subsection{Infinite Square Well (1-D)}
\boldmath \[ V(x) = \begin{cases}
    0 & 0<x<a \\
    \infty & \text{otherwise}
\end{cases} \] \unboldmath 

\vspace{15pt}\noindent
\( \displaystyle \Psi_n(x) = \sqrt{\frac{2}{a}} \sin{k_nx} \) \\[20pt]
\( \displaystyle k_n = \frac{2 \pi}{\lambda} = \frac{2 \pi}{2a/n} = \frac{n \pi}{a}\)
    \hspace{18pt} \( \forall n=1, 2, 3, ...\)         
    \hspace{18pt} \hspace{18pt} \fbox{!! \( \hat{p} \Psi_n \neq p \Psi_n \) !!} 
    \hspace{18pt} {\scriptsize wave isn't infinite} \\[20pt]
\( \displaystyle E_n = \frac{p^2}{2m} = \frac{\hbar^2 k_n^2}{2m}\)

\vspace{15pt}
\subsubsection{3-D Rectangular Box}
\( \displaystyle \Psi_{n_x n_y n_z}(x,y,z) = \Psi_{n_x}(x) \Psi_{n_y}(y) \Psi_{n_z}(z) 
    = \sqrt{\frac{8}{a_x a_y a_z}} (\sin{k_{n_x} x}) (\sin{k_{n_y} y}) (\sin{k_{n_z} z}) \) \\[20pt]
\( \displaystyle k_{n_i} = \frac{n_i \pi}{a_i}\)
    \hspace{18pt} \( \forall n_x, n_y, n_z = 1, 2, 3, ...\) \\[20pt]
\( \displaystyle E_{n_x n_y n_z} = \frac{\hbar^2}{2m} (k_{n_x}^2 + k_{n_y}^2 + k_{n_z}^2) \) 

%------------------------------------------------------------------------------------------------------------------------------
%
%
%------------------------------------------------------------------------------------------------------------------------------
\newpage
% Harmonic Oscillator
\subsection{Harmonic Oscillator (1-D):\ \ \boldmath\( V(x) = \frac{1}{2} k x^2 = \frac{1}{2} m \omega^2 x^2 \)\unboldmath}

% Lowering/Rising Operator
\vspace{5pt}\noindent
\( 
    \begin{aligned}
        \tfrac{p^2}{2m} + \tfrac{1}{2} m \omega^2 x^2 & = \tfrac{1}{2m} \left( p^2 + m^2 \omega^2 x^2 \right)\\
        & = \tfrac{1}{2m} \left( -ip + m \omega x \right) \left( i p + m \omega x \right) 
            \sim E \sim \hbar \omega 
    \end{aligned}
    \ \ \Rightarrow \ \
    \boxed{ a = a_{-} = \tfrac{1}{\sqrt{2 m}} \tfrac{1}{\sqrt{\hbar \omega}} \left( i\hat{p} + m \omega x \right) }    
\)

% Hamiltonian
\vfill\noindent
\( 
    \boxed{ \begin{aligned}
        a a^\dagger & = \tfrac{H}{\hbar \omega} + \tfrac{1}{2} \\
        a a^\dagger | n \rangle & = ( \tfrac{E_n}{\hbar \omega} + \tfrac{1}{2} ) | n \rangle 
    \end{aligned} }
    \hspace{5pt} , \hspace{5pt}
    \boxed{ \begin{aligned}
        a^\dagger a & = \tfrac{H}{\hbar \omega} - \tfrac{1}{2} \\
        a^\dagger a | n \rangle & = ( \tfrac{E_n}{\hbar \omega} - \tfrac{1}{2} ) | n \rangle 
    \end{aligned} }
    \ \ \rightarrow \ \ 
    \begin{gathered}
        \boxed{ \big[ a, a^\dagger \big]=1 }\\
        \mss{ \text{or } \big[H, a_{\pm}\big] = (\pm\hbar\omega) a_\pm }
    \end{gathered} 
    \ \ \leftarrow \ \ 
    \boxed{ \begin{aligned}
        H & = \hbar\omega (a^\dagger a + \tfrac{1}{2})\\
        & = \hbar\omega (a a^\dagger - \tfrac{1}{2})
    \end{aligned} }
\)

\vfill\noindent
% Raising and Lowering Operators
\(
    \hfill
    % a^\dagger a
    \begin{aligned}[t]
        (a a^\dagger) a \Psi_n & = a (a^\dagger a) \Psi_n \\
        \left( \tfrac{H}{\hbar \omega} + \tfrac{1}{2} \right) a \Psi_n & 
            = a \left( \tfrac{H}{\hbar \omega} - \tfrac{1}{2} \right) \Psi_n
            \\[5pt]
        \left( \tfrac{ E_{an} }{\hbar \omega} + \tfrac{1}{2} \right) a \Psi_n & 
            = \left( \tfrac{E_n}{\hbar \omega} - \tfrac{1}{2} \right) a \Psi_n
            \\[4pt]
        E_{an} (a \Psi_n) & = (E_n - \hbar \omega) (a \Psi_n) \\
        & \hs \Downarrow
            \\
        E_{n-1} | n-1 \rangle & = (E_n - \hbar \omega) | n-1 \rangle
    \end{aligned}    
    % \hfill\vline\hfill
    \hspace{1.5cm}\hspace{1.5cm}
    % a a^\dagger
    \begin{aligned}[t]
        (a^\dagger a) a^\dagger | n \rangle & = a^\dagger (a a^\dagger) | n \rangle \\
        \left( \tfrac{H}{\hbar \omega} - \tfrac{1}{2} \right) a^\dagger | n \rangle & 
            = a^\dagger \left( \tfrac{H}{\hbar \omega} + \tfrac{1}{2} \right) | n \rangle
            \\
        \left( \tfrac{ E_{a^\dagger n} }{\hbar \omega} - \tfrac{1}{2} \right) \cancel{c_n} | a^\dagger n \rangle & 
            = \left( \tfrac{E_n}{\hbar \omega} + \tfrac{1}{2} \right) \cancel{c_n} | a^\dagger \mss{(n)} \rangle
            \\
        E_{a^\dagger n} | a^\dagger n \rangle & = (E_n + \hbar \omega) | a^\dagger \mss{(n)} \rangle \\
        & \hs \Downarrow\\
        E_{n+1} | n+1 \rangle & = (E_n + \hbar \omega) | n+1 \rangle
    \end{aligned}
    \hfill\hs
\)

\vfill\noindent
% Alternative way that expresses why they are ladder operators
\text{\scriptsize(Why ladders)}:\\[-8pt]
\(
    \begin{aligned}
        \boxed{ \big[\tfrac{H}{\hbar\omega}, a_{\pm}\big] = (\pm 1) a_\pm } &
            \ \Rightarrow\ \begin{gathered}[b]
                \text{\scriptsize(use induction)}\\[-3pt]
                \big[\tfrac{H}{\hbar\omega}, a_{\pm}^m \big] = (\pm 1)m a_\pm
            \end{gathered}
            \\
        \boxed{ \tfrac{H}{\hbar\omega} |n\rangle = \tfrac{E_n}{\hbar\omega} |n\rangle = c_n |n\rangle } &
            \ \Rightarrow\ \tfrac{H}{\hbar\omega} a_{\pm}^m | n \rangle = (\pm 1 \cdot m + c_n ) | n \rangle
    \end{aligned}
    \hspace{10pt} \vline \hspace{10pt}
    % \begin{gathered}
        % \tfrac{H}{\hbar\omega} \underline{ a_\pm |n\rangle } = (c_n \pm 1) \underline{ a_\pm |n\rangle }\\[3pt]
            \begin{aligned}
                H a_\pm^m |n\rangle & = \hbar\omega (c_n \pm m) a_\pm^m |n\rangle \\
                *\ H a_\pm^n |0\rangle & = \hbar\omega (c_0 \pm n) a_\pm^n |0\rangle \ * \\
                & = E_{n0} \cdot a_\pm^n |0\rangle
            \end{aligned}
    % \end{gathered}
\) 

\vfill

\noindent
% Ground State Info
\begin{minipage}{0.51\textwidth}
    \( 
        E_n \geq \text{Min}(V) \ \Rightarrow{} \ a \Psi_0 = 0 
        \hspace{20pt} \text{\scriptsize (else is un-normalizable)} 
    \)

    \vspace{15pt}
    \(
        \begin{aligned}
            0 & = (ip + m \omega x) \Psi_0\\[5pt]
            \hbar \tfrac{d}{dx} \Psi_0 & = - m \omega x \Psi_0 
        \end{aligned}
        \hspace{10pt}\vline\hspace{10pt}
        \boxed{
            \begin{aligned}
                \Psi_0 & = A e^{- \frac{m \omega}{\hbar} \frac{x^2}{2} } \\
                A & = \left( \frac{m \omega }{\pi \hbar} \right)^{1/4} \\[2pt]
                \tfrac{1}{\sigma^2} & = \tfrac{m\omega}{\hbar}
            \end{aligned}
        }
    \)

    \vspace{15pt}
    \hfill
    \(\begin{aligned}
        a^\dagger a | 0 \rangle & = \left( \tfrac{E_0}{\hbar\omega} - \tfrac{1}{2} \right) | 0 \rangle = 0
            \\[5pt]
        E_0 | 0 \rangle & = \tfrac{1}{2} \hbar \omega | 0 \rangle
            \hspace{20pt} \mss{ (c_0 = \tfrac{1}{2}) }
    \end{aligned}\)
    \hfill\hs
\end{minipage}
\vline
\hfill
% Raising Lowering Operator
\(
    \begin{aligned}
        & a a^\dagger | a_+^n \mss{(0)} \rangle 
            = \left( \tfrac{\hbar \omega (n + 1/2)}{\hbar \omega} + \tfrac{1}{2} \right) | a_+^n \mss{(0)} \rangle 
            \\
        & \boxed{ a a^\dagger | a_+^n \mss{(0)} \rangle = (n + 1) | a_+^n \mss{(0)} \rangle }
            \\[5pt]
        & \bullet\ \langle a_+^n \mss{(0)} | a a^\dagger | a_+^n \mss{(0)} \rangle = n+1 \\
        & \bullet\ a^\dagger | a_+^n \mss{(0)} \rangle = \sqrt{n+1} \ | a_+^{n+1} \mss{(0)} \rangle \\%\ \Rightarrow
        & \bullet\ a | a_+^{n+1} \mss{(0)} \rangle = \sqrt{n+1} | a_+^n \mss{(0)} \rangle * * 
            \\[10pt]
        & a^\dagger a | \mss{n,0} \rangle 
            = \left( \tfrac{\hbar \omega (n + 1/2)}{\hbar \omega} - \tfrac{1}{2} \right) | \mss{n,0} \rangle 
            \\
        & \boxed{ a^\dagger a | a_+^n \mss{(0)} \rangle = n | a_+^n \mss{(0)} \rangle } 
    \end{aligned}
\)
\hfill\hs

\vfill

% Energy of Eigenstates
\noindent 
\(
    \begin{aligned}
        \tfrac{H}{\hbar\omega} | a_{+}^n \mss{(0)} \rangle & 
            = \underline{( \tfrac{1}{2} + n)} | a_{+}^n \mss{(0)} \rangle 
            = \underline{c \circ a_{+}^n \mss{(0)}} \hs | a_{+}^n \mss{(0)} \rangle 
            \\[10pt]
        \tfrac{H}{\hbar\omega} a_\pm^m | a_{+}^n \mss{(0)} \rangle &
            = \underline{( \tfrac{1}{2} + n \pm m)} \hs a_\pm^m | a_{+}^n \mss{(0)} \rangle
            \\
        \tfrac{H}{\hbar\omega} | a_{+}^{n \pm m} \mss{(0)} \rangle &
            = \underline{( c \circ a_+^{n \pm m} \mss{(0)} )} | a_{+}^{n \pm m} \mss{(0)} \rangle 
    \end{aligned}
    \hfill\rightarrow\hfill
    \begin{aligned}
        c_n & = \tfrac{1}{2} + n\\ 
        a_+^n \mss{(0)} & = n \\
        a_\pm \mss{(n)} & = n\pm 1
    \end{aligned}
    \hfill\rightarrow\hfill
    \begin{gathered}
        \boxed{ E_n = \hbar\omega (\tfrac{1}{2} + n) }\\
        \boxed{ 
            \begin{aligned}
                a | n \rangle & = \sqrt{n} | n-1 \rangle\\
                a^\dagger | n \rangle & = \sqrt{n+1} | n+1 \rangle
            \end{aligned}
            }
            \\[5pt]    
        \boxed{ \Psi_n = \frac{1}{ \sqrt{n!} }(a^\dagger)^n \Psi_0 }
    \end{gathered} 
\)

\vfill

%------------------------------------------------------------------------------------------------------------------------------
%
%
%
\newpage

% Position/Momentum Operators
\subsubsection{Position/Momentum Operators}
\vspace{5pt}
\[ \boxed{ \displaystyle x = \frac{1}{2} 
        \frac{ \sqrt{2 m} \sqrt{\hbar \omega} }{m \omega} (a + a^\dagger) }
    \hspace{18pt} \hspace{18pt} 
    \boxed{ \hat{p} = \frac{1}{2} 
        \frac{ \sqrt{2 m} \sqrt{\hbar \omega} }{i} (a - a^\dagger) } \]

\vspace{20pt}\noindent
\begin{minipage}[t]{.46\textwidth}
    % Virial Theorem
    \underline{Show Virial Theorem Works}

    \vspace{15pt}
    \hfill \( \boldsymbol{ 2 \langle T \rangle = N \langle V \rangle } \) \hfill\hs
        
    \vspace{10pt}
    \( \begin{aligned} 
        E_n & = 2 \langle V \rangle_n\\
        & = 2 \langle \Psi_n |V| \Psi_n \rangle\\[5pt]
        & = 2 \left\langle \Psi_n \left| \tfrac{1}{2}mw^2 
            \tfrac{2m \hbar \omega}{(2m \omega)^2} 
            (a + a^\dagger)^2 \right| \Psi_n \right\rangle \\[5pt]
        & = \tfrac{2m^2 \hbar \omega^3}{(2m \omega)^2} 
            \left( 0 + \left\langle \Psi_n \left| (a a^\dagger + a^\dagger a) \right| 
            \Psi_n \right\rangle + 0 \right)\\[10pt]
        E_n & = (n+1/2) \hbar \omega \hspace{15pt} \checkedbox
    \end{aligned} \)

    \vspace{20pt}
    % Heisenberg Picture
    \underline{Heisenberg Picture}

    \vspace{15pt}
    \(\begin{aligned}
        & \tfrac{da_{\pm}}{dt} = \mp \hs i \omega a_{\pm} \\
        & \Rightarrow \begin{aligned}[t]
                a_{\pm} \mss{(t)} & = a_{\pm} \mss{(0)} e^{\mp\hs i\omega t}\\
                x \mss{(t)} \pm \tfrac{ip \mss{(t)}}{m\omega} & = x \mss{(0)} e^{\mp\hs i\omega t} 
                    \pm \tfrac{i p \mss{(0)}}{m \omega} e^{\mp\hs i\omega t}
            \end{aligned}
            \\[5pt]
        & x \mss{(t)} = x \mss{(0)} \cos{\omega t} + \tfrac{p \mss{(0)}}{m\omega} \sin{\omega t}\\
        & \tfrac{ p \mss{(t)} }{m\omega} = - x \mss{(0)} \sin{\omega t} + \tfrac{p \mss{(0)}}{m\omega} \cos{\omega t}\\
    \end{aligned}\)
\end{minipage}
\hfill
\begin{minipage}[t]{.51\textwidth}
    \setlength{\parindent}{.5cm}
    \noindent
    % Uncertainty Principle
    \underline{Test the Uncertainty Principle}
    
    \vspace{10pt}
    { \setlength{\tabcolsep}{0pt}
    \begin{tabular}{c}
        \( \sigma_x \sigma_p \geq \frac{1}{2} 
            \Big| \Big\langle \big[ x,p \big] \Big\rangle \Big| \)\\[20pt]
        \( \begin{aligned} 
            xp-px &= \tfrac{2 m \hbar \omega}{4 m \omega i}
                \left( \begin{aligned}
                      a^2 - a a^\dagger + a^\dagger a - (a^\dagger)^2 \\
                    - a^2 + a^\dagger a - a a^\dagger + (a^\dagger)^2
                \end{aligned} \right) \\[5pt]
            &= \tfrac{\hbar}{i} (a^\dagger a - a a^\dagger) = i \hbar (n+1 - n)\\[5pt]
            % &= i \hbar (n+1 - n)\\[5pt]
            &\Rightarrow \ \sigma_x \sigma_p \geq \frac{\hbar}{2} \hspace{18pt} \checkedbox
        \end{aligned} \)
    \end{tabular} }
    
    \vspace{15pt} \noindent
    \( \begin{aligned}
        \sigma_x^2 &= \langle x^2 \rangle 
            - \langle x \rangle^2 
            \\[5pt]
        &= \tfrac{ 2 m \hbar \omega }{4 m^2 \omega^2} 
            \left[ \begin{aligned}
                \langle (a + a^\dagger)^2 \rangle \\
                - \langle a + a^\dagger \rangle^2 
            \end{aligned} \right]
            \\[5pt]
        &= \tfrac{\hbar}{2 m \omega} \langle a a^\dagger + a^\dagger a \rangle \\[5pt]
        &= \tfrac{\hbar}{m \omega} (n + \tfrac{1}{2}) 
    \end{aligned} \) 
    \ \ 
    \( \begin{aligned}
        \sigma_p^2 &= \langle p^2 \rangle 
            - \langle p \rangle^2 
            \\[5pt]
        &= \tfrac{ 2 m \hbar \omega }{-4} 
            \left[ \begin{aligned}
                \langle (a - a^\dagger)^2 \rangle \\
                - \langle a - a^\dagger \rangle^2 
            \end{aligned} \right]
            \\[5pt]
        &= \tfrac{\hbar m \omega}{2} \langle a a^\dagger + a^\dagger a \rangle \\[5pt]
        &= \hbar m \omega (n + \tfrac{1}{2}) 
    \end{aligned} \)
    
    \[ \Rightarrow \ \sigma_x \sigma_p = \hbar (n+ \tfrac{1}{2}) \geq \tfrac{\hbar}{2} 
        \hspace{18pt} \checkedbox \]
\end{minipage}

%---------------------------------------------------------------------------------------
% Analytic Method
\vspace{20pt}
\subsubsection{Analytic Method}

\boldmath \[ \Psi_n = A \frac{1}{\sqrt{2^n n!}} H_n(\xi) e^{- \xi^2 / 2} \] \unboldmath

\vspace{5pt}\noindent
\( A = \left( \frac{m \omega }{\pi \hbar} \right)^{1/4} \) \\[10pt]
\( \xi = \sqrt{ \frac{m \omega }{\hbar} } x\) \\[10pt]
Hermite Polynomials: \ \( \begin{aligned}
    &H_n(x) = (-1)^n \ e^{-x^2} \left( \frac{d}{dx} \right)^n e^{x^2}\\[5pt]
    &e^{-z^2 + 2zx} = \sum_{n=0}^\infty \frac{z^n}{n!} H_n(x)
\end{aligned} \)

%-------------------------------------------------------------------------------------------------------------------------------
%
%
%-------------------------------------------------------------------------------------------------------------------------------
% Coherent States
\newpage
\subsubsection{Coherent States}
\[ \boldsymbol{ a | \alpha \rangle = \alpha | \alpha \rangle } \]
\( \boxed{ \sigma_x \sigma_p = \frac{\hbar}{2} } \)\\[5pt]
\noindent
\begin{minipage}{.45\textwidth}
    \( \begin{aligned}
        \langle \alpha | \alpha \rangle 
            &= \langle \alpha | 
            \left( \begin{aligned}
                &\sum_{n=0}^\infty 
                    \langle \Psi_n | \alpha \rangle \ | \Psi_n \rangle =\\
                &\sum_{n=0}^\infty \left\langle \left. 
                    \frac{(a^\dagger)^n}{\sqrt{n!}} \Psi_0 \right| \alpha \right\rangle \ 
                    | \Psi_n \rangle = \\
                &\sum_{n=0}^\infty 
                    \frac{\alpha^n}{\sqrt{n!}} \langle \Psi_0 | \alpha \rangle
                    \ | \Psi_n \rangle\\
            \end{aligned} \right) \\[5pt]
        &= \langle \Psi_0 | \alpha \rangle^2 \sum_{n=0}^\infty
            \frac{ (\alpha^{2})^n}{n!} \ \langle \Psi_n|\Psi_n \rangle\\[5pt]
        &= \langle \Psi_0 | \alpha \rangle^2 \ e^{\alpha^{2}} = 1
    \end{aligned} \) 
\end{minipage}
\begin{minipage}{.51\textwidth}
    \( \displaystyle \Rightarrow \ \boxed{ \big| \alpha \big\rangle 
    = e^{- \alpha^{2} / 2} \ \sum_{n=0}^\infty 
    \tfrac{\alpha^n}{\sqrt{n!}} \ | \Psi_n \rangle } \) 
    \ \(\rightarrow\) \
    \( \big| \alpha {\scriptstyle =} 0 \big\rangle = | \Psi_0 \rangle \)\\[10pt]
    \( \begin{aligned}
        a | \alpha{\scriptstyle(x,t)} \rangle 
            &= e^{- \frac{\alpha^{2}}{2}} \sum_{n=0}^\infty 
            \tfrac{\alpha^n}{\sqrt{n!}} \ e^{\frac{-i}{\hbar} E_n t} a | \Psi_n \rangle\\
        &= e^{- \frac{\alpha^{2}}{2}} \sum_{n=0}^\infty 
            \tfrac{\alpha^n}{\sqrt{n!}} \ e^{\frac{-i}{\hbar} 
            \hbar \omega (\frac{1}{2} + n) t} \sqrt{n} | \Psi_{n-1} \rangle \\
        &= \left( \alpha e^{\frac{-i}{\hbar} \hbar \omega t} \right) e^{- \frac{\alpha^{2}}{2}}
            \sum_{n=0}^\infty \tfrac{\alpha^n}{\sqrt{n!}} \ e^{\frac{-i}{\hbar} 
            \hbar \omega (\frac{1}{2} + n) t} | \Psi_n \rangle\\
        \Aboxed{ a | \alpha{\scriptstyle(x,t)} \rangle 
            &= \left( \alpha e^{-i \omega t} \right) | \alpha{\scriptstyle(x,t)} \rangle }
    \end{aligned} \)
\end{minipage}

\vspace{1cm}
\noindent {\centering\( |\alpha \rangle \) are obviously not orthogonal. They are an overcomplete basis.\par}

%----------------------------------------------------------------
\vspace{10pt}\noindent
\subsubsection{3-D Harmonic Potential}

\[ \boldsymbol{V(r) = \frac{1}{2} k r^2} \]
\[ \boxed{ 
    E_{n_x n_y n_z} = \hbar \omega \left(n_x + n_y + n_z + \tfrac{3}{2}\right) 
    = \begin{gathered}[b]
        \text{\scriptsize(Isotropic)}\\[-1pt]
        \hbar\omega \left( n + \tfrac{3}{2} \right) 
    \end{gathered}
    \hspace{10pt} \mss{l = n - 2k \in \{ n, n-2, ..., 0 \}}
} \]

%------------------------------------------------------------------------------------------------------------------------------
%
%
%
\newpage
% Free Particle
\subsection{Free Particle (1-D)}
\[ \boldsymbol{ V(x) = 0 }\]

\vspace{10pt}\noindent
\hfill \(
    \begin{aligned}[t]
        \Psi\mss{(x,t)} & = \frac{1}{\sqrt{2\pi\hbar}} 
            \int_{-\infty}^\infty 
            \Phi\mss{(x,0)} e^{\frac{i}{\hbar} \left[ px - E\mss{(p)} t \right]} \hs\hs dp
            \\[5pt]
        & = \int 
            \langle x | p \rangle 
            e^{-\tfrac{i}{\hbar}E\mss{(p)} t}
            \langle p | \Psi \rangle
            \hs\hs dp
            \\[5pt]
        \Phi\mss{(x,t)} & = \frac{1}{\sqrt{2\pi\hbar}} 
            \int_{-\infty}^\infty 
            \Psi\mss{(x,0)} e^{\frac{-i}{\hbar} \left[ px + E\mss{(p)} t \right]} \hs\hs dx
    \end{aligned}
    \hspace{1cm}
    \begin{aligned}[t]
        \langle x | U & \mss{(t)} | \Psi \rangle 
            = \iint
            \langle x | p \rangle 
            e^{-\tfrac{i}{\hbar} \tfrac{p^2}{2m} t}
            \langle p | x' \rangle
            \hs dp \hs
            \langle x' | \Psi \rangle
            \hs\hs dx' 
            \\[5pt]
        & = \iint_{-\infty}^{\infty} 
            \tfrac{1}{2\pi\hbar}
            e^{-\tfrac{i}{\hbar} \left[ \tfrac{p^2 t}{2m} - p(x-x') \right]}
            \hs dp 
            \hs \Psi\mss{(x', 0)}
            \hs\hs dx' 
            \\[5pt]
        & = \int 
            \sqrt{\tfrac{-im}{2\pi\hbar t}}
            e^{\tfrac{im (x-x')^2}{2\hbar t}}
            \hs \Psi\mss{(x', 0)}
            \hs\hs dx' 
    \end{aligned}
\) \hfill \hs

\vspace{15pt}\noindent
\hfill \(
    \text{\scriptsize(\(E<0 \rightarrow \Psi = e^{\pm kx}\) is possible and also not normalizable, 
    but solution above is already a complete set)}    
\) \hfill \hs

\vspace{20pt} \noindent
\( 
    \begin{aligned}[t]
        \bullet\ & & E(p) & = \tfrac{p^2}{2m}\\[5pt]
        \bullet\ & & v_\text{wave} &
            = \boxed{ v_\text{phase} = \frac{ \omega{\scriptstyle(k)} }{k} } 
            = \frac{E}{p} 
            = \frac{v_\text{classical}}{2} 
            \\[5pt]
        \bullet\ & & v_\text{particle} & 
            \approx \boxed{ v_\text{group} = \frac{d \omega{\scriptstyle(k)} }{d k} } 
            = 2 v_\text{wave}
            \hspace{15pt} \left( \begin{gathered}
                \text{\scriptsize dispersion}\\[-5pt]
                \text{\scriptsize relation}
            \end{gathered} \right)
    \end{aligned} 
\)
\hfill
% Free Particle Heisenberg Picture
\parbox[t]{6.4cm}{
    \underline{Heisenberg Pic. Free Particle}\\[5pt]
    \(\begin{aligned}
        x_H \mss{(t)} & = x_H \mss{(0)} + \tfrac{p_H \mss{(0)}}{m} t\\
        \left[ x_H \mss{(0)} ,\ x_H \mss{(t)} \right] & 
            = \left[ x_H \mss{(0)} , \tfrac{p_H \mss{(0)}}{m} t \right] 
            = \tfrac{i\hbar t}{m}
            \\[5pt]
        \Aboxed{ \sigma_{x_t} \sigma_{x_0} & \geq \tfrac{\hbar t}{2m} }
    \end{aligned}\)
}
\hfill\hs


%----------------------------------------------------------------
%----------------------------------------------------------------
% Delta Potential
\vspace{10pt}
\subsection{Delta Potential (1-D)}

% Potential Eq
\vspace{15pt}
\textbf{Potential Well:} \hspace{18pt} \hspace{18pt} \hspace{18pt} 
    \( \boldsymbol{ V(x) = - \alpha \delta(x) } \) 
    \hspace{18pt}\hspace{18pt} {\scriptsize(\(\alpha \rightarrow -\alpha\)\ for potential wall)}

\vspace{25pt} \noindent
% Bound State
\begin{minipage}[t]{.45\textwidth}
    \setlength{\parindent}{.5cm}
    \noindent
    \underline{Bound State (\(E<0\)) {\scriptsize[only for Well]}}: \\[10pt]
    \hspace{18pt} \( \Psi = \sqrt{k} e^{k|x|} = \begin{cases} 
        \sqrt{k} e^{kx}  &   x \leq 0\\
        \sqrt{k} e^{-kx} &   x \geq 0
    \end{cases}\)
    
    \vspace{20pt} \noindent
    \(\begin{aligned}
        k &= \frac{m \alpha}{\hbar^2}\\[5pt]
        E &= - \frac{(\hbar k)^2}{2m}
    \end{aligned}\)    
\end{minipage}
% Scattering State
\begin{minipage}[t]{.5\textwidth}
    \setlength{\parindent}{.5cm}
    \noindent
    \underline{Scattering State (\(E>0\)) {\scriptsize[for both]}}:\\[10pt] 
    \hspace{18pt} \(\Psi = \begin{cases}
        A e^{iKx} + B e^{-iKx}  &   x < 0\\
        F e^{iKx}               &   x > 0
    \end{cases}\)
    
    \vspace{20pt} \noindent
    \(\begin{aligned}
        E &= \frac{(\hbar K)^2}{2m} \ , & 
            \beta &\equiv \frac{k}{K} = \frac{m \alpha / \hbar^2}{K}\\[10pt]
        B &= \frac{i\beta}{1 - i\beta} A \ , &
            F &= \frac{1}{1- i\beta} A\\[10pt]
        R &= \frac{|B|^2}{|A|^2} = \frac{\beta^2}{1+\beta^2} \ , & \hspace{18pt} \hspace{18pt}
            T &= \frac{|F|^2}{|A|^2} = \frac{1}{1+\beta^2}
    \end{aligned}\)    

    \vspace{20pt} \noindent
    {\scriptsize Can't normalize. 
    All free particles have ranges of \(p\) and thus \(E\), so \(R\) and \(T\) are approx.
    in the vicinity of \(E\).}
\end{minipage}

%--------------------------------------------------------------
% Finite Square Potential
\subsection{Finite Square Potential (1-D)}
\vspace{5pt}
\subsubsection{Potential Well \hspace{30pt}
    \(
        \boldsymbol{
            V(x) = 
            \begin{cases}
                -V_0 & -a<x<a \\
                0 & \text{otherwise}
            \end{cases} 
        }
    \) \hspace{25pt} \begin{minipage}{5cm}
        \text{\scriptsize(\(V_0 \rightarrow -V_0\) for wall and do cases}\\
        \text{\scriptsize for \(E>V_0, E=V_0, E<V_0\), and}\\
        \text{\scriptsize change to \(\sinh, \cosh\) if needed)}
    \end{minipage}
}

\vspace{5pt}
{
\setlength{\tabcolsep}{2pt}
\begin{tabular}{r c l}
    \(k\ ; K\)\ :   &\ \ \(E\)         &\(= \tfrac{-(\hbar k)^2}{2m} = \tfrac{(\hbar K)^2}{2m}\)\\[10pt]
    \(l\)\ :        &\ \ \(E + V_0\)   &\(= \tfrac{(\hbar l)^2}{2m}\)\\[10pt]
    \(v\)\ :        &\ \ \(V_0\)       &\(= \tfrac{\hbar^2 v^2}{2m}
        = \tfrac{\hbar^2 ( l^2 + k^2 )}{2m} = \tfrac{\hbar^2 ( l^2 - K^2 )}{2m}\)
\end{tabular}
}
\ \ 
\rule[-40pt]{.5pt}{80pt}
\ \ \
\(\begin{aligned}
    &\frac{k_a}{l_a} \equiv \sqrt{ \frac{(ka)^2}{(la)^2} } 
    = \sqrt{ \frac{ (la)^2 + (ka)^2 }{(la)^2} - 1 } \\[5pt]
    &\boxed{ \frac{k_a}{l_a} \equiv \sqrt{ \left( \frac{v_a}{l_a} \right)^2 - 1} }
        \ , \ v_a^2 = \begin{cases} 
            l_a^2 + k_a^2 \\[5pt]
            l_a^2 - K_a^2 
        \end{cases}
\end{aligned}\)

% Bound States
\vspace{15pt} \noindent
\underline{Bound State \ (\(E_n<0\)) {\scriptsize[only for well]}}:\\[15pt]
% Even States
\begin{minipage}[t]{.49\textwidth}
    \setlength{\parindent}{.5cm}
    \(\Psi_\text{even}(x) = \begin{cases}
        \Psi(-x)    &   x < 0\\[5pt]
        D \cos(lx)   &   0 < x < a \\[5pt]
        F e^{-kx}   &   a < x 
    \end{cases}\)
    
    \vspace{20pt}
    \(\bullet \ \ \ F = D \cos(la) e^{ka}\)\\[10pt]
    \hspace{18pt} \(\bullet \ \ \ \begin{aligned}[t]
        &\tfrac{-(\partial_x \Psi)(a)}{\Psi(a)} = k = l \tan(la) \ \Rightarrow\\[5pt]
        &\tan(l_a) = \sqrt{ (v_a / l_a)^2 - 1}
            \hspace{18pt} \ \ \\[5pt]
        &\scriptstyle \text{big } v_a \ \rightarrow \ l\ \approx_< \tfrac{n \pi}{2a}
            \ \rightarrow \ E_n + V_0\ =\ \tfrac{\hbar^2 l^2}{2m}\ ;\ \underline{n\ \text{odd}}
    \end{aligned}\)\\[5pt]
    \hspace{18pt} \(\bullet \ \ \ \boxed{ n_\text{max} = \left\lfloor \dfrac{v_a}{\pi} \right\rfloor + 1} \)
\end{minipage}
% Odd States
\begin{minipage}[t]{.5\textwidth}
    \setlength{\parindent}{.5cm}
    \(\Psi_\text{odd}(x) = \begin{cases}
        - \Psi(-x)    &   x < 0\\[5pt]
        C \sin(lx)   &   0 < x < a \\[5pt]
        F e^{-kx}   &   a < x 
    \end{cases}\)
    
    \vspace{20pt}
    \(\bullet \ \ \ F = D \sin(la) e^{ka}\)\\[10pt]
    \hspace{18pt} \(\bullet \ \ \ \begin{aligned}[t]
        &\tfrac{-(\partial_x \Psi)(a)}{\Psi(a)} = k = - l \cot(la) \ \Rightarrow\\[5pt]
        &-\cot(l_a) = \sqrt{ (v_a / l_a)^2 - 1} \\[5pt]
        &\scriptstyle \text{big } v_a \ \rightarrow \ l\ \approx_< \tfrac{n \pi}{2a}
            \ \rightarrow \ E_n + V_0\ =\ \tfrac{\hbar^2 l^2}{2m}\ ;\ \underline{n\ \text{even}}
    \end{aligned}\)\\[5pt]  
    \hspace{18pt}\(\bullet \ \ \ \boxed{ n_\text{max} = \left\lfloor \dfrac{v_a + \tfrac{\pi}{2}}{\pi} \right\rfloor } \) 
\end{minipage}

% Scattering States
\vspace{20pt} \noindent
\underline{Scattering State \ (\(E>0\)) {\scriptsize[for both]}}:\\[15pt]
\(\Psi = \begin{cases}    
    A e^{iKx} + B e^{-iKx}              &   x < -a\\
    C \sin{lx} + D \cos{lx} \ \ \ \ \ \ &   -a < x < a\\
    F e^{iKx}                           &   a < x 
\end{cases} 
\hspace{18pt} \hspace{18pt} 
\frac{d \Psi}{dx} = \begin{cases}
    iKA e^{iKx} - iKB e^{-iKx} \ \ \ \ \ \  &   x < -a\\
    lC \cos{lx} - lD \sin{lx}               &   -a < x < a\\
    iKF e^{iKx}                             &   a < x
\end{cases}\)

% Coefficient and Transmission
\vspace{15pt}
\(\begin{aligned}[t]
    B &= i \sin(2 l_a) \left( \tfrac{l_a^2 - K_a^2}{2 K_a l_a} \right) \ F\\[10pt]
    F &= \frac{e^{-2iK_a}}{\cos(2 l_a) - i \left( \frac{l_a^2 + K_a^2}{2 K_a l_a} \right) \sin(2 l_a) } \ A\\[5pt]
    &\text{\scriptsize(Can't normalize. See delta potential.)}
\end{aligned}\)
\hspace{18pt} \hspace{18pt} \(\begin{aligned}[t]
    T^{-1} &= 1 + \left( \tfrac{l_a^2 - K_a^2}{2 K_a l_a} \right)^2 \ \sin^2( 2l_a )\\[5pt]
    &= 1 + \frac{V_0^2}{4E(E+V_0)} \ \sin^2 \left( 2a \sqrt{\frac{E + V_0}{\hbar^2 / 2m}}\ \right)\\[5pt]
    &\text{\scriptsize(full transmission at inf. sqr. well 
        \(E_n + V_0 = \tfrac{\hbar^2 l^2}{2m}\ ;\ l = \tfrac{n \pi}{2a}\))}
\end{aligned}\)

% \subsubsection{Potential Barrier}
% \[ \boldsymbol{
%     V(x) = 
%     \begin{cases}
%         V_0 & -a<x<a \\
%         0 & \text{otherwise}
%     \end{cases} 
% } \]

%-------------------------------------------------------------------------------------------------------------------------------
%-------------------------------------------------------------------------------------------------------------------------------
%-------------------------------------------------------------------------------------------------------------------------------
%-------------------------------------------------------------------------------------------------------------------------------
% 2D and 3D Schrodinger Equation
\newpage
\section{2D and 3D Schrodinger Equation}

%--------------------------------------------------------------
%--------------------------------------------------------------
% General Dimensions
\underline{General dimensions, \(D\)}\\[10pt]
\(
    \begin{aligned}
        & \tfrac{1}{r^{D-1}} \tfrac{\partial}{\partial r} \left( r^{D-1} \tfrac{\partial}{\partial r} \right) R(r) 
            = \left[ \tfrac{\partial^2}{\partial r^2} + \tfrac{D-1}{r} \tfrac{\partial}{\partial r} \right] R(r) 
            \\[5pt]
        & = \left[ \tfrac{\partial^2}{\partial r^2} + \tfrac{D-1}{r} \tfrac{\partial}{\partial r} \right] r^n u(r) 
            \\
        % & = \left[
        %         \tfrac{\partial^2}{\partial r^2}    
        %         + \tfrac{D-1 + n + n}{r} \tfrac{\partial}{\partial r} 
        %         + \tfrac{n(D-1) + n(n-1)}{r^2} 
        %     \right] u
        %     \\
        & = \left[
                \tfrac{\partial^2}{\partial r^2}    
                + \tfrac{D-1 + 2n}{r} \tfrac{\partial}{\partial r} 
                + \tfrac{2n(2D - 4 + 2n)}{4r^2} 
            \right] u
            \\
        & = \left[
                \tfrac{\partial^2}{\partial r^2}    
                - \tfrac{(D-1)(D - 3)}{4r^2} 
            \right] u
            \hspace{15pt} \mss{(n = \tfrac{1-D}{2}, \bcancel{ 0, 2-D })}
            \\
    \end{aligned}
    \hfill\vline\hfill
    \begin{gathered}
        R(r) = u(r) / \sqrt{r}^{D-1} \ \sim\ e^{\frac{i}{\hbar} p_r r} / \sqrt{r}^{D-1} 
            % \hspace{15pt} \mss{(\Psi^2 \sim 1 / r^{D-1})}
            \\[10pt]
        L^2 \sim \hbar^2
            \hspace{5pt} , \hspace{10pt} 
            \hat{p}_r = \tfrac{\hbar}{i} \left( \tfrac{\partial}{\partial r} + \tfrac{D-1}{2r} \right)
            \hspace{5pt} , \hspace{10pt} 
            \hat{p'}_r= \tfrac{\hbar}{i} \tfrac{\partial}{\partial r} 
            \\[10pt]
        \begin{aligned}
                E R(r) & = \left[ \tfrac{\hat{p}_r^2}{2M} + V(r) + \tfrac{L^2 - \hbar^2 (D-1)^2 / 4 }{2(Mr^2)} \right] R(r)\\
                E u(r) & = \left[ \tfrac{\hat{p'}_r^2}{2M} + V(r) + \tfrac{L^2 - \hbar^2 (D-1)(D-3) / 4 }{2(Mr^2)} \right] u(r)\\
            \end{aligned}
    \end{gathered}
\)

%--------------------------------------------------------------
%--------------------------------------------------------------
% 2D Schrodinger
\subsection{2D Schrodinger}

\noindent
\begin{minipage}{.5\textwidth}
    \underline{If \( V = V(\rho) \)}\\[10pt]
    \( 
        \Psi(\vec{\mathbf{r}}) = R_m(\rho) \Phi_m(\phi) \ \Rightarrow 
    \)
    \begin{gather*}
        E R = 
            \left[ 
                \frac{-\hbar^2}{2M} \frac{1}{\rho} \frac{\partial}{\partial \rho} \left( \rho \frac{\partial}{\partial \rho} \right) 
                + V(\rho) + \frac{\hbar^2 m^2}{2M \rho^2} 
            \right] R
            \\[5pt]
        \boxed{
                E u = 
                \frac{-\hbar^2}{2M} \frac{\partial^2 u}{\partial \rho^2}
                + \left[ V(\rho) + \frac{\hbar^2 (m^2 + 1/4)}{2M \rho^2} \right] u
            }
            \\
        E u \Phi = \left[ \frac{\hat{p'}_{\rho}^2}{2M} + V(\rho) + \frac{\hat{L_z^2} + \hbar^2/4}{2(M \rho^2)} \right] u \Phi
    \end{gather*}   
\end{minipage}
% \vline
\hfill\vline\hfill
\(\begin{aligned}
    & \bullet\ R_m(\rho) = u_m(\rho) / \sqrt{\rho}
        \hspace{20pt} \mss{\left( \int \Psi r \hs dr d\phi = 1 \right)}
        \\[2pt]
    & \bullet\ \Phi_m(\phi) = e^{i m \phi} \\[2pt]
    & \bullet\ L_z = ( \vec{\hs r} \times \vec{\hs p} \hs )_z = \tfrac{\hbar}{i} \tfrac{\partial}{\partial \phi}
\end{aligned}\)  
\hfill\hs

%----------------------------------------------------------------------------------------------------------------------------------
%
%
%----------------------------------------------------------------------------------------------------------------------------------
\newpage
% 3D Schrodingers
\subsection{3D Schrodinger}

\noindent
\begin{minipage}{.51\textwidth}
    \underline{If \( V = V(r) \)}\\[10pt]
    \( 
        \Psi(\vec{\mathbf{r}}) = R_l(r) Y^m_l(\theta, \phi) 
        = R_l(r) \Theta^m_l(\theta) \Phi_m(\phi) \ \Rightarrow 
    \)
    \begin{gather*}
        E R = \left[ 
                \frac{-\hbar^2}{2M} \frac{1}{r^2} \frac{\partial}{\partial r} \left( r^2 \frac{\partial}{\partial r} \right) 
                + V(r) + \frac{\hbar^2l(l+1)}{2Mr^2} 
            \right] R
            \\
        \boxed{ E u = \frac{-\hbar^2}{2M} \frac{\partial^2 u}{\partial r^2}
            + \left[ V(r) + \frac{\hbar^2 l(l+1)}{2M r^2} \right] u }
            \\
        E u \Theta = \left[ \frac{\hat{p'}_{r}^2}{2M} + V(r) + \frac{\hat{L^2}}{2(Mr^2)} \right] u \Theta
    \end{gather*}   
\end{minipage}
\hfill\vline\hfill
\(\begin{aligned}
    & \bullet\ R_l(r) = u_l(r) / r
        \hspace{20pt} \mss{\left( \int \Psi r^2 \sin\theta \hs dr d\theta d\phi = 1 \right)}
        \\[2pt]
    & \bullet\ \Phi_m(\phi) = e^{i m \phi} \\[2pt]
    & \bullet\ \Theta^m_l(\theta) = A P^m_l(\cos{\theta}) \\[5pt]
    & \hspace{5pt} - A = \epsilon \sqrt{ \tfrac{2l+1}{4\pi} \tfrac{(l-|m|)!}{(l+|m|)!} } , \ \
        \epsilon = \mss{\begin{cases}
            \scriptstyle (-1)^m & \scriptstyle (m \geq 0) \\
            \scriptstyle1       & \scriptstyle (m \leq 0)
        \end{cases}}
        \\[5pt]
    & \hspace{5pt} - P^m_l(x) = \text{\scriptsize{Assoc. Legendre Func. (see extra)}} \\[2pt]
    & \bullet\ l \in \mathbb{N}_0, \ m \in \{ -l, ..., -1, 0, 1, ..., l \} \\[2pt]
    & \bullet\ \widehat{L_i} = ( \vec{r} \times \vec{p} )_i
\end{aligned}\)  

% Properties of u
\vspace{15pt}\noindent
\(
    % Boundary Conditions
    \begin{aligned}
        & \langle u | \nabla_r^2 u \rangle = \langle \nabla_r^2 u | u \rangle 
            \ \Rightarrow\ \left[ 
                u^* \tfrac{\partial u}{\partial r} 
                - u \tfrac{\partial u^*}{\partial r} 
            \right]_{0 \ \ \leftarrow 2.)}^{\infty \ \leftarrow 1.)} = 0
            \\[5pt]
        & 1.)\ \mss{\int_0^\infty} u^2 \hs dr = 1 \ \Rightarrow\ \boxed{ u \mss{(\infty)} = 0\ \text{\scriptsize or}\ e^{ir} } \\
        & 2.)\ \boxed{ u \mss{(0)} = c = 0 }
            \ \ \mss{ \left\{ 
                c \neq 0 
                \rightarrow \begin{aligned}
                    & \Psi_{l=0} \mss{(r)} \sim \tfrac{c}{r} \\
                    & \nabla^2 (\tfrac{1}{r}) \sim \delta^3 \mss{(r)}
                \end{aligned}
                \rightarrow \begin{gathered}
                    H \Psi \neq E \Psi\\
                    \text{\scriptsize if \(V(r) \neq \delta^3(r)\)} 
                \end{gathered}
            \right\} }
    \end{aligned}
\)
\hfill\vline\hfill
\(
    % Function Properties
    \begin{aligned}
        \mss{ V \sim r^{-2 < a} ,\ l \neq 0 } & 
            :\ \lim_{r \rightarrow 0} u'' \sim \tfrac{l(l+1)}{r^2} u , \ u \sim r^{l+1}
            \\
        \mss{ V \sim r^{-2 < a < -1}, \ E > 0 } &
            :\ \lim_{r \rightarrow \infty} p^2_r u \sim E u , \ u \sim e^{\pm ikr}
            \\
        \mss{ V \sim r^{-1 \leq a} ,\ E > 0 } &
            :\ u \gtrsim re^{\pm ikr}
            \\
        \text{\scriptsize Sim for \(E < 0\)} &
            :\ u \sim e^{\pm kr} \ \text{\scriptsize or}\ re^{\pm kr}\ \text{\scriptsize etc.}
    \end{aligned}
\)

%--------------------------------------------------------------
% 3D Free Particle
\vfill
\subsubsection{3D Free Particle, \ \(V = 0\)}

\vspace{5pt}\noindent
\(
    \hfill
    \begin{aligned}[t]
        - \frac{\hbar^2}{2M} \left[ \frac{\partial^2}{\partial r^2} + \frac{l(l+1)}{r^2} \right] u &
            = \frac{\hbar^2 k^2}{2M} u
        \hspace{10pt} \Rightarrow \hspace{10pt}
        \left[ \frac{\partial^2}{\partial \rho^2} + \frac{l(l+1)}{\rho^2} \right] | l \rangle = | l \rangle
    \end{aligned}
    \hfill\hs
\)

\vspace{15pt}\noindent
\begin{minipage}[t]{.46\textwidth}
    \(
        \hfill
        \begin{gathered}[t]
            a_{l} \equiv \frac{\partial}{\partial \rho} + \frac{l+1}{\rho} 
                \\[20pt]
            \begin{aligned}
                    a_{l} a_{l}^\dagger | l \rangle & = | l \rangle \\
                    a_{l}^\dagger | l \rangle & = e^{i\theta_l} | a_{l}^\dagger \mss{(l)} \rangle
                \end{aligned}
        \end{gathered}  
        \hspace{20pt}
        \begin{gathered}[t]
            a_{l}^\dagger = - \frac{\partial}{\partial \rho} + \frac{l+1}{\rho}
                \\[20pt]
            a_{l}^\dagger a_{l} | l \rangle = a_{l+1} a_{l+1}^\dagger | l \rangle
        \end{gathered}
        \hfill\hs
    \)

    \vspace{20pt}
    \(
        \hfill
        \begin{aligned}    
            a_{l}^\dagger \left( a_{l} a_{l}^\dagger \right) | l \rangle & = \underline{ a_{l}^\dagger | l \rangle } \\
            \left( a_{l}^\dagger a_{l} \right) \underline{ a_{l}^\dagger | l \rangle } & 
                = \left( a_{l+1} a_{l+1}^\dagger \right) \underline{ a_{l}^\dagger | l \rangle }
                \\
            \underline{ a_{l}^\dagger | l \rangle } & = \bcancel{e^{i\theta_l}} \ | l+1 \rangle
        \end{aligned}
        \hfill\hs
    \)
\end{minipage}
\hfill
\vline
\hfill
\begin{minipage}[t]{.48\textwidth}
    % Ground States
    \(
        \boxed{
        \begin{aligned}[t]
            \text{Bessel}: & \ \ r \underline{R_0^B} = u_0^B \sim \sin{(\rho)} = \sin{(kr)}
                \\[5pt]
            \cancel{ \text{Neumann} }: & \ \ r \underline{R_0^N} = u_0^N \sim - \cos{(\rho)}
        \end{aligned}
        }
    \)
    
    \vspace{15pt}
    % Higher Levels
    \(
        \begin{aligned}
            \cancel{ e^{i\theta_l} } \tfrac{\rho}{k} R_{l+1} & = a_{l}^\dagger (\tfrac{\rho}{k} R_l) 
                = \left( - \tfrac{\partial}{\partial \rho} + \tfrac{l+1}{\rho} \right) (\tfrac{\rho}{k} R_l) 
                \\
            R_{l+1} & = \left( - \tfrac{\partial}{\partial \rho} + \tfrac{l}{\rho} \right) R_l 
                = - \rho^l \tfrac{\partial}{\partial \rho} \left( \rho^{-l} R_l \right)
                \\
                \tfrac{ R_{l} }{\rho^{l}} & = - \tfrac{1}{\rho} \tfrac{\partial}{\partial \rho} \tfrac{R_{l-1}}{\rho^{l-1}}
                    = \left( - \tfrac{1}{\rho} \tfrac{\partial}{\partial \rho} \right)^l R_0
                    \\
                \Aboxed{ R_l & = C_l (- \rho)^l \left( \tfrac{1}{\rho} \tfrac{\partial}{\partial \rho} \right)^l R_0 }
        \end{aligned}
    \)  
\end{minipage}

% Infinite Spherical Well
\vspace{20pt}\noindent
\underline{Infinite Spherical Well}: \ \ \(
    V(r) = \begin{cases}
        0 & r \leq a\\
        \infty & r > a
    \end{cases}
    \hspace{10pt} , \hspace{10pt}
    E_n = \tfrac{\hbar^2 k_n^2}{2m}
\)

\vspace{10pt}\noindent
\hspace{20pt}
\underline{Bessel}: \ \( 
    \begin{aligned}
        R_l^B \mss{(\rho)} & = C_l (- \rho)^l \left( \tfrac{1}{\rho} \tfrac{\partial}{\partial \rho} \right)^l R_0^B \mss{(\rho)} \\
        R_0^B \mss{(\rho)} & \sim k_n \sin{(\rho)} / \rho = \sin{(k_n r)} / r
    \end{aligned}
    \ \ \Rightarrow \ \
    \begin{aligned}
        \beta_l^n & \equiv k_n a :\ R_l^B \mss{(\beta_l^n)} = 0 \\[5pt]
        \beta_0^n & = \tfrac{n \pi}{a} \cdot a
    \end{aligned}    
\)

%-------------------------------------------------------------------------------------------------------------------------------
%
%
%-------------------------------------------------------------------------------------------------------------------------------
\newpage
% Hydrogen Atom
\subsubsection{Hydrogen Atom, \ \(V = - \frac{ke^2}{r}\)}
\unskip
% Schroedinger Equation
\begin{gather*}
    E u = \left( \frac{\hat{p}_r^2}{2m} + V(r) + \frac{\hat{L^2}}{2(mr^2)} \right) u
        \hspace{18pt} \hspace{18pt} u(r) = r R(r) \\[10pt]
    \boxed{ E u = \frac{-\hbar^2}{2m} \frac{\partial^2}{\partial r^2} u
        + \left[ - \frac{ke^2}{r} + \frac{\hbar^2 l(l+1)}{2m r^2} \right] u }
\end{gather*}

% Solution
\vspace{10pt} \noindent
\( \boxed{ \Psi_{nlm}(\vec{\mathbf{r}}) = R_{nl}(r) \ Y^m_l(\theta, \phi) 
    = R_{nl}(r) \ \Theta^m_l(\theta) \ \Phi_m(\phi) } \)

\noindent
\begin{minipage}[t]{.47\textwidth}
    % \noindent
    \begin{itemize}
        \item \( \Phi_m(\phi) = e^{i m \phi}\)
        \item \( \Theta^m_l(\theta) = A P^m_l(\cos{\theta}) \)\\[5pt]
        - \( A = \epsilon \sqrt{ \frac{2l+1}{4\pi} \frac{(l-|m|)!}{(l+|m|)!} } \), \ \
            \( \epsilon = \begin{cases}
                \scriptstyle (-1)^m & \scriptstyle (m \geq 0) \\
                \scriptstyle1       & \scriptstyle (m \leq 0)
            \end{cases} \)\\[5pt]
        - \( P^m_l(x) \) \ \ \ {\scriptsize{Assoc. Legendre Func. (see extra)}}
    \end{itemize}   
\end{minipage}
\begin{minipage}[t]{.52\textwidth}
    % \noindent
    \begin{itemize}
        \item \( R_{nl}(r) = \dfrac{B}{r} \rho^{l+1} e^{-\rho} \nu(\rho) \)\\[5pt]
        - \( \rho = k_n r \) , \ \ \( k_n = \frac{1}{a_0 n} \) 
            \ \ \ {\scriptsize(fine structure below)} \\[10pt]
        - \( \nu(\rho) = L^{2l+1}_{n-l-1}(2\rho) \) 
            \ \ \ {\scriptsize{Assoc. Laguerre Pol. (see extra)}} \\[10pt]
        - \( B = \sqrt{2 k_n \ \frac{ ( n-l-1 )! }{ 2n [ (n+l)! ]^3 }} \ 2^{l+1} \) 
    \end{itemize}     
\end{minipage}

% Fine structure and Bohr Radius
\vspace{25pt} \noindent
\( \boxed{ \alpha \equiv \frac{kqq}{\hbar c} 
    = \frac{1}{4 \pi \epsilon_0} \frac{e^2}{\hbar c} \ \approx \ \frac{1}{137} } \)
% Hydrogen Atom Energy
\hspace{1cm}
\( \boxed{ a_0 \equiv \frac{\hbar^2}{m (kqq)} = \frac{4 \pi \epsilon_0 \hbar^2}{m e^2} } \)

\begin{align*}
    \Aboxed{ &E_n = - \frac{\hbar^2 k_n^2}{2m} = - \frac{\hbar^2}{2m a_0^2} \frac{1}{n^2}
        = - \frac{1}{2} \alpha^2 \left( m c^2 \right) \dfrac{1}{n^2}
        \ \approx \ -13.6 \ \dfrac{1}{n^2} \ [\text{eV}] }\\[10pt]
    \Aboxed{ &\frac{1}{\lambda} = \frac{\alpha^2 \left( m c^2 \right) }{2 h c}
        \left( \dfrac{1}{n_f^2} - \dfrac{1}{n_i^2} \right)
        = R \left( \tfrac{1}{n_f^2} - \tfrac{1}{n_i^2} \right) 
        \ , \ \ R = 1.097 \ \text{E} 7 \ [\text{m}^{-1}] }
\end{align*}

% Quantum Numbers
\vspace{20pt} \noindent
\underline{Quantum Numbers - \(n,l,m\)}:
\begin{itemize}
\item \( \begin{gathered}[t]
        \Big( n \in \{ 1, 2, 3, ... \} \Big), \ 
        \Big( l \in \{ 0, 1, 2, ..., n-1 \} \Big), \
        \Big( m \in \{ -l, ..., -1, 0, 1, ..., l \} \Big)
    \end{gathered} \)\\[10pt]
    - \underline{Degeneracy is \(n^2\)}
\end{itemize}      

% Bohr Atom
\vspace{10pt}\noindent
\underline{(outdated) Bohr Model}:\\[10pt]
\(\begin{aligned}
    \hspace{18pt} &\bullet \ L = ( \bar{r} ) ( \bar{p} ) = (a_0 n^2) (\hbar k_n) 
        = n\hbar \hspace{18pt} \text{\scriptsize(not correct!!)}\\
    \hspace{18pt} &\bullet \ \text{Electrons don't radiate about the nucleus}\\
    \hspace{18pt} &\bullet \ \text{Energy diff. follows Rydberg formula}
\end{aligned}\)

% Extra
% P. 133, Prob. 4.1
% \( \Theta^m_l(\theta) = A ln( tan( \theta / 2 ) ) \) is another solution, but blows up at \theta = 0 or \pi
% Infinite Spherical Well (Bessel and Neumann Functions)
% P. 155 (167), Prob. 4.13-4.16 

%---------------------------------------------------------------------------------------------------------------------
%---------------------------------------------------------------------------------------------------------------------
%---------------------------------------------------------------------------------------------------------------------
%---------------------------------------------------------------------------------------------------------------------
% Spin
\newpage
\section{Spin and \(L\)}
\subsection{Hydrogen Atom}

% L_z and L^2 and L_{+-}
\noindent
\(
    \begin{aligned}[t]
        & \text{\underline{Angular Momentum}}:
            \\[5pt]
        & \boxed{ \widehat{L_i} \equiv ( \vec{r} \times \vec{p} )_i }
            \hspace{5pt}, \hspace{5pt} \boxed{ L_z = \tfrac{\hbar}{i} \tfrac{\partial}{\partial \phi}}
            \\
        & \boxed{ \widehat{L}_{\pm} \equiv \widehat{L}_x \pm i \widehat{L}_y } \\
        & \boxed{ \widehat{L}^2 \equiv L_x^2 + L_y^2 + L_z^2 } \\
        & \boxed{ L_\pm L_\mp = \widehat{L}^2 - L_z^2 \pm \hbar L_z  }
    \end{aligned}
\)
\hfill
% Commutation Relations
\begin{minipage}[t]{.6\textwidth}
    \underline{Commutation Relations}:

    % [ L_x , L_y ]
    \vspace{10pt} \noindent \(
        \boxed{ \big[ \hat{x} , L_y \big] = i \hbar \hat{z} } 
        \ , \
        \boxed{ \big[ p_x , L_y \big] = i \hbar p_z }
        \ , \ 
        \boxed{ \big[ L_x , L_y \big] = i \hbar L_z } 
    \) %\ \ {\scriptsize (can't measure concurrently)}

    % [ H , _ ]
    \vspace{10pt} \noindent \(
        \boxed{ \big[ L^2 , L_i \big] = \big[ H , L_i \big] = \big[ H , L^2 \big] = 0}
    \) \ \ {\scriptsize (can measure concurrently)}

    % [ L^2 , L_z ]
    \vspace{10pt} \noindent \( 
        \rightarrow \ 
        \begin{aligned}
            \Aboxed{ L_z Y_{m'} & = \hbar m' Y_{m'} }\\[5pt]
            \Aboxed{ L^2 Y_{m'} & = \hbar^2 \lambda_{m'} Y_{m'} }
        \end{aligned}
        \ \Rightarrow \ 
        \begin{aligned}
            & \langle L^2 - L_z^2 \rangle 
                = \langle L_x^2 + L_y^2 \rangle 
                \geq 0
                \\[5pt]
            & \bullet \ \sqrt{\lambda_{m'}} 
                \geq \ m' \geq -\sqrt{\lambda_{m'}}
        \end{aligned} 
    \)
\end{minipage}

% [ __ , L_{+-} ]
\vspace{20pt} \noindent
\textbf{Let} \ \( (L_\pm)^n Y_\mu \equiv | m \rangle \) \hspace{10pt} {\scriptsize(see harm. osc. for why ladders)}\\[10pt]
\hspace{18pt} 
\begin{tabular}{c c l}
    \(
        \boxed{ \big[ \tfrac{L_z}{\hbar} , L_\pm \big] = (\pm 1) L_\pm }
        \hspace{10pt} \Rightarrow \hspace{10pt}
        \big[ \tfrac{L_z}{\hbar} , (L_\pm)^n \big] = \pm n (L_\pm)^n
        \)
        & \( \ \Rightarrow \ \) 
        & \( \begin{aligned}
                L_z \big[ (L_\pm)^n Y_\mu \big] &= (\mu \pm n )\hbar \big[ (L_\pm)^n Y_\mu \big]\\[5pt]
                \bullet \ L_z | m \rangle 
                    &= (\mu \pm n )\hbar | m \rangle
            \end{aligned} \) 
        \\[.8cm] 
    \( 
        \boxed{ \big[ L^2 , L_\pm \big] = 0 } 
        \hspace{10pt} \Rightarrow \hspace{10pt}
        \big[ L^2 , (L_\pm)^n \big] = 0 
        \)
        & \( \ \Rightarrow \ \)
        & \( \begin{aligned}
            L^2 \big[ (L_\pm)^n Y_\mu \big] &= \lambda_\mu \big[ (L_\pm)^n Y_\mu \big] \\[5pt]
            \bullet \ L^2 | m \rangle
                &= \lambda_\mu | m \rangle
        \end{aligned} \)
\end{tabular}

% Solve for Eigenvalues and Eigenfunctions
\vspace{25pt} \noindent 
\textbf{Then} \ \( 
    \Big( \sqrt{\lambda_\mu} \ \geq \ (\mu \pm n) \ \geq \ -\sqrt{\lambda_\mu} \ \Big)
    \ \Rightarrow 
\) \ \textbf{Let} \ \ \ {\scriptsize (else un-normalizable solution)}

\vspace{10pt}\noindent
\(
    \hfill
    \begin{aligned}
        & \underline{ L_+ | m_t \rangle = 0 } \ , \ \ L_z | m_t \rangle = \hbar l \ , \\[5pt]
        & L^2 | m_t \rangle = \lambda \hbar^2 \ , \ \ L^2 = L_- L_+ + L_z^2 + \hbar L_z \\[10pt]
        & \bullet \ L^2 | m_t \rangle = \hbar^2 l (l+1) | m_t \rangle = \lambda \hbar^2 | m_t \rangle
    \end{aligned}
    \hfill\vline\hfill
    \begin{aligned}
        & \underline{ L_- | m_b \rangle = 0 } \ , \ \ L_z | m_b \rangle = \hbar l' \ , \\[5pt]
        & L^2 | m_b \rangle = \lambda \hbar^2 \ , \ \ L^2 = L_+ L_- + L_z^2 - \hbar L_z \\[10pt]
        & \bullet \ L^2 | m_b \rangle = \hbar^2 l' (l'-1) | m_b \rangle = \lambda \hbar^2 | m_b \rangle    
    \end{aligned}
    \hfill\hs
\)

\vspace{10pt} \noindent
\hfill
\( 
    \Big[ \lambda = l' (l'-1) = l (l+1) \Big]
    \ \Rightarrow \ 
    \big[ l' = -l \big]
    \ \Rightarrow \ 
    \left[ \begin{aligned}
        L_z | m_t \rangle &= \hbar l \hs | m_t \rangle \\
        L_z | m_b \rangle &= - \hbar l \hs | m_b \rangle
    \end{aligned} \right]
\) \ \ \parbox{3.3cm}{\scriptsize (Spherical Harmonics do not allow half-integer \(l\))} 
\hfill\hs

% Solutions
\vspace{15pt} \noindent 
Schrodinger \(Y_l^m\): \ \fbox{ 
    \( \begin{gathered}
        l \in \left\{ 0 , 1 , 2, ... \right\} \\[5pt]
        m \in \{ -l , -l + 1, ... , l-1 , l \}
    \end{gathered} \)
    \hspace{.2cm}
    \rule[-25pt]{.5pt}{55pt}
    \hspace{.2cm}
    \( \begin{aligned}
        L_z \big| Y_l^m \big\rangle & = \hbar m \big| Y_l^m \big\rangle 
            = \tfrac{\hbar}{i} \tfrac{\partial}{\partial \phi} \big| Y_l^m \big\rangle 
            \\[5pt]
        L^2 \big| Y_l^m \big\rangle & = \hbar^2 l (l+1) \big| Y_l^m \big\rangle \\[5pt]
        L_\pm \big| Y_l^m \big\rangle & =
            \hbar \sqrt{\scriptstyle l(l+1) - m(m \pm 1)} \ \big| Y_l^m  \big\rangle
    \end{aligned} \)     
} 

%---------------------------------------------------------------------------------------------------------------------
%
%
%---------------------------------------------------------------------------------------------------------------------
\newpage \noindent
% Generalized Angular Momentum
\subsection{Generalized}
\noindent
\(
    % Angular Momentum
    \begin{aligned}[t]
        & \text{\underline{Angular Momentum}:}
            \\[5pt]
        & \begin{array}[t]{l l}
                \widehat{J}_i \equiv \text{???} & \boxed{ J^2 \equiv J_x^2 + J_y^2 + J_z^2 } \\[10pt]
                \boxed{ J_\pm \equiv J_x \pm i J_y } & \boxed{ J_\pm J_\mp = J^2 - J_z^2 \pm \hbar J_z }
            \end{array}
    \end{aligned}
    \hfill
    % Commutation Relations
    \begin{aligned}[t]
        & \text{\underline{Commutation Relations}:} \\[5pt]
        % [ J_x , J_y ]
        & \boxed{ \big[ J_i , J_j \big] = i \hbar J_k \ \epsilon_{ij} }
            \ \Leftrightarrow\ 
            \boxed{ J \times J = i \hbar J }
            \\[10pt]
        % [ J^2 , J_z ]
        & \boxed{ \big[ J^2 , J_z \big] = 0 } = \big[ H , J_z \big]
            \hspace{10pt} \text{\scriptsize (if spher. symm.)}
    \end{aligned}
    \hfill\hs
\)

% General Eigenvalues
\vspace{15pt} \noindent
General: \ \fbox{ 
    \( 
        \begin{gathered}
            j \in \left\{ 0 , \tfrac{1}{2}, 1 , \tfrac{3}{2}, ... \right\} \\[5pt]
            m \in \{ -j , -j + 1, ... , j-1 , j \}
        \end{gathered}
        \hspace{10pt}\vline\hspace{10pt}
        \begin{aligned}
            J_z \big| j m \big\rangle & = \hbar m \big| j m \big\rangle \\[5pt]
            J^2 \big| j m \big\rangle & = \hbar^2 j (j+1) \big| j m \big\rangle \\[5pt]
            J_\pm \big| j m \big\rangle & =
                \hbar \sqrt{\scriptstyle j(j+1) - m(m \pm 1)} \ \big| j m  \big\rangle
                \\[5pt]
            J_x \big| j m \big\rangle & = \tfrac{ J_{+} + J_{-} }{2} \big| j m \big\rangle
        \end{aligned} 
    \)     
}

\vspace{15pt}\noindent
% Generator of Rotations
\text{\underline{Generator of Rotations}:}\\[10pt]
\(\displaystyle
    U [R \mss{(\theta)}] = e^{- \tfrac{i}{\hbar} \theta \hs \hat{\theta} \cdot L} 
    = \lim_{N \rightarrow \infty} \left[ I - \tfrac{i}{\hbar} \tfrac{\theta}{N} \hs \hat{\theta} \cdot L \right]^N
    \ \Leftrightarrow\ \boxed{ U[R\mss{(\epsilon_z \hs \hat{z})}] = I - \tfrac{i}{\hbar} \epsilon_z L_z }
\)

\vspace{15pt}\noindent
\(
    1.)\ \ \begin{aligned}[t]
        & U[R\mss{(\epsilon_z \hat{z})}] \mss{|x,y \rangle} \equiv \mss{|x - \epsilon_z y , \hs \epsilon_z x + y \rangle}\\
        & \Rightarrow\ \langle \mss{x,y} | U[R\mss{(\epsilon_z \hat{z})}] | \Psi \rangle
            = \Psi \mss{(x + \epsilon_z y , \hs \epsilon_z x - y)}
            \\
        & \Rightarrow\ \underline{ \langle \mss{x,y} | L_z | \Psi \rangle = (XP_y - YP_x) \Psi \mss{(x,y)} }
    \end{aligned}
    \hfill
    % Commutators
    \text{or\ \ } 2.)\ \ \begin{aligned}[t]
        & \begin{aligned}[t]
                U^\dagger X U \equiv X - \epsilon_z Y & \ \Rightarrow\ [X, L_z] = - i\hbar Y \\
                U^\dagger P_y U \equiv \epsilon_z P_x + P_y & \ \Rightarrow\ [P_y, L_z] = i\hbar P_x \\
                \mss{ U^\dagger Y U,\ U^\dagger P_x U, } & \ \Rightarrow\ \dots
            \end{aligned}
            \\[5pt]
        & \Rightarrow\ \underline{L_z = XP_y - YP_x}
    \end{aligned}
\)

\vspace{10pt}\noindent
% Consistency Check
\textit{Consistency Check}:\ \
\(
    \begin{aligned}[t]
        & U[\mss{ R(- \epsilon_z \hat{z}) }] \hs T \mss{ (- \epsilon_x \hat{x} - \epsilon_y \hat{y}) }
            \hs U[\mss{ R(\epsilon_z \hat{z}) }] \hs T \mss{ (\epsilon_x \hat{x} + \epsilon_y \hat{y}) }
            = T \mss{( - \epsilon_y \epsilon_z \hat{x} + \epsilon_x \epsilon_z \hat{y} )}
            \\
        & ? U[\mss{ R(- \epsilon_y \hat{y}) }] U[\mss{ R(- \epsilon_x \hat{x}) }] 
            U[\mss{ R(\epsilon_y \hat{y}) }] U[\mss{ R(\epsilon_x \hat{x}) }] 
            = \begin{gathered}[t]
                I + \tfrac{i}{\hbar} \epsilon_x \epsilon_y L_z \\
                \mss{(L_z = x p_y - y p_x)}
            \end{gathered}
            = U [\mss{ R(-\epsilon_x \epsilon_y \hat{z}) }] 
    \end{aligned}
\)

\vspace{5pt}\noindent
% Tensors and Tensor Operators
\underline{Tensors and Tensor Operators}

\vspace{10pt}\noindent
\(
    \hfill
    \begin{aligned}[t]
        & \textit{rank-2 Tensor}: \\
        & \hspace{10pt} \begin{aligned}
                | t^{(2)} \rangle & = \mss{ \sum_{i=1}^{3} \sum_{j=1}^{3} }\ t_{ij} \hs | i \rangle | j \rangle \\[5pt]
                & = \mss{ \sum_{i=1}^{3} \sum_{j=1}^{3} }\ | i j \rangle \langle i j | t^{(2)} \rangle
            \end{aligned}
            \\[10pt]
        & \textit{rank-\(2\) Cartesian Tens. Oper.,}\ T_{ij}: \\
        & \hspace{10pt} \mss{ \text{Set of } 3^{n=2} \text{ Operators}}\\[5pt]
    \end{aligned}
    \hfill\vline\hfill
    \begin{aligned}[t]
        & \textit{rank-\(k\) Spherical Tensor Operator,}\ T^q_k: \\
        & \hspace{10pt} \mss{ \text{Set of } 2k+1 \text{ Operators \ s.t. } }\\
        & \hspace{10pt} \begin{aligned}[t]
                U[R] \hs T^q_k \hs U^\dagger [R] & = \mss{ \sum^k_{q'=-k} D^{k}_{q'q} \hs T^{q'}_k } \\[-5pt]
                & \hs \downarrow\\
                U T^q_k U^\dagger U | j m \rangle & = 
                    \mss{ \displaystyle \sum_{q'} \sum_{m'} D^k_{q'q} D^j_{m'm} \hs T^k_{q'} | j m' \rangle }
                    \\[5pt]
                \sim\ U |k q \rangle | j m \rangle & = 
                    \mss{ \displaystyle \sum_{q'} \sum_{m'} D^k_{q'q} D^j_{m'm} | k q' \rangle | j m' \rangle }
            \end{aligned}
            \hspace{10pt}
            \bullet\ \arraycolsep=2pt \begin{array}[t]{l c l}
                T_1^{\pm1} & = 
                    & \mp \tfrac{V_x \pm i V_y}{\sqrt{2}}
                    \\[5pt]
                T_1^0 & = 
                    & V_z
            \end{array}
    \end{aligned}
    \hfill\hs
\)

\vspace{15pt}\noindent
% Wigner-Eckhart Theorem
\(
    \textit{Wigner-Eckhart}:\ \ 
    \langle \alpha_2 j_2 m_2 | \hs T_k^q \hs | \alpha_1 j_1 m_1 \rangle
    = \langle \alpha_2 j_2 |\hs T_k \hs | \alpha_1 j_1 \rangle 
    \cdot \begin{gathered}[t]
        \langle j_2 m_2 | kq ; j_1 m_1 \rangle\\[-3pt]
        \text{\scriptsize(CG coeff.)}
    \end{gathered}
\)

\vfill

%-----------------------------------------------------------------------------------------------------------------------------
%
%
%-----------------------------------------------------------------------------------------------------------------------------
% Spin 1/2
\newpage
\subsection{1 Particle w/ Spin, \ \(s = \frac{1}{2}\)}
*Find the Eigenvectors, \(e_i\) , of \(S_z\) and \(S^2\) in the form of \(
    |\chi\rangle = 
    \mss{ \left(\begin{matrix}
        \cos\tfrac{\theta}{2} \hs e^{i\phi_1} \\[3pt]
        \sin\tfrac{\theta}{2} \hs e^{i\phi_2}
    \end{matrix}\right) }
    =
    \mss{ e^{i\gamma} \left(\begin{matrix}
        \cos\tfrac{\theta}{2} \hs e^{-i\phi} \\[3pt]
        \sin\tfrac{\theta}{2} \hs e^{i\phi}
    \end{matrix}\right) }
    \hfill
    \mss{ \begin{aligned}
        & \gamma = \tfrac{\phi_1 + \phi_2}{2}\\
        & \phi = \tfrac{\phi_2 - \phi_1}{2}
    \end{aligned} }
\)

% States
\begin{center}
    * \fbox{ 
        \( 
            e_i \in \left\{ \ \ 
                | \tfrac{1}{2} \ \tfrac{1}{2} \rangle 
                \equiv | \uparrow \ \rangle \equiv 
                \left( \begin{matrix} 
                    1 \\
                    0
                \end{matrix} \right)
                \hspace{.5cm} , \hspace{.5cm}
                | \tfrac{1}{2} \ \tfrac{-1}{2}  \rangle 
                \equiv | \downarrow \ \rangle \equiv 
                \left( \begin{matrix} 
                    0 \\
                    1
                \end{matrix} \right)
            \ \ \right\} 
        \) 
    }
\end{center}

\vspace{5pt} \noindent
\(\begin{array}{c c l}
    % S^2
    \left. \begin{aligned}
            S^2 | \uparrow \ \rangle = \dfrac{3\hbar^2}{4} | \uparrow \ \rangle \\[10pt]
            S^2 | \downarrow \ \rangle = \dfrac{3\hbar^2}{4} | \downarrow \ \rangle
        \end{aligned} \hspace{18pt} \right\}
        & \Rightarrow
        & \begin{aligned}
                S^2 & = \dfrac{3\hbar^2}{4}
                    \left( \begin{matrix} 
                        1 & 0 \\
                        0 & 1
                    \end{matrix} \right) 
                    = \dfrac{3\hbar^2}{4} \sigma_0
                    = * \left( \begin{matrix} 
                        1 & 0 \\
                        0 & 1
                    \end{matrix} \right) 
                    \left( \begin{matrix} 
                        \tfrac{3\hbar^2}{4} & 0 \\
                        0 & \tfrac{3\hbar^2}{4}
                    \end{matrix} \right) 
                    \left( \begin{matrix} 
                        1 & 0 \\
                        0 & 1
                    \end{matrix} \right)^{T*}
                    \\
                & = \tfrac{3 \hbar^2}{4} \mathbb{I} 
                    \hspace{20pt} (\text{\scriptsize only for \(s=1/2\) systems})
            \end{aligned}
        \\[45pt]
    % S_+-
    \left. \begin{gathered}
            S_- | \uparrow \ \rangle = \hbar | \downarrow \ \rangle \\[10pt]
            S_+ | \downarrow \ \rangle = \hbar | \uparrow \ \rangle \\[10pt]
            S_+ | \uparrow \ \rangle = S_- | \downarrow \ \rangle = 0
        \end{gathered} \hspace{18pt} \right\}
        & \Rightarrow
        & \begin{aligned}
                S_+ & = \hbar
                    \left( \begin{matrix} 
                        0 & 1 \\
                        0 & 0
                    \end{matrix} \right)
                    \\[5pt]
                & = \hbar | \uparrow \ \rangle \langle \ \downarrow |
            \end{aligned}
            \hspace{18pt}
            \begin{aligned}
                S_- & = \hbar
                    \left( \begin{matrix} 
                        0 & 0 \\
                        1 & 0
                    \end{matrix} \right)
                    \\[5pt]
                & = \hbar | \downarrow \ \rangle \langle \ \uparrow |
            \end{aligned}
            \ \ \ \ \text{\scriptsize (can't measure)}
        \\[45pt]
    % S_z
    \left. \begin{aligned}
            S_z | \uparrow \ \rangle = \frac{\hbar}{2} | \uparrow \ \rangle \\[10pt]
            S_z | \downarrow \ \rangle = - \frac{\hbar}{2} | \downarrow \ \rangle \\
        \end{aligned} \hspace{18pt} \right\} 
        & \Rightarrow
        & \begin{aligned}
                S_z & = \dfrac{\hbar}{2}
                    \left( \begin{matrix} 
                        1 & 0 \\
                        0 & -1
                    \end{matrix} \right) 
                    = \dfrac{\hbar}{2} \sigma_z =
                    * \left( \begin{matrix} 
                        1 & 0 \\
                        0 & 1
                    \end{matrix} \right)
                    \left( \begin{matrix} 
                        \tfrac{\hbar}{2} & 0 \\
                        0 & -\tfrac{\hbar}{2}
                    \end{matrix} \right)
                    \left( \begin{matrix} 
                        1 & 0 \\
                        0 & 1
                    \end{matrix} \right)^{T*} 
                    \\[5pt]
                & = \tfrac{\hbar}{2} | \uparrow\ \rangle \langle \ \uparrow | 
                    - \tfrac{\hbar}{2} | \downarrow\ \rangle \langle \ \downarrow |
            \end{aligned}
        \\[45pt]
    % S_x, S_y
    \left. \begin{gathered}
            S_x = \frac{1}{2} (S_+ + S_-) \\[10pt]
            S_y = \frac{1}{2i} (S_+ - S_-)
        \end{gathered} \hspace{18pt} \right\} 
        & \Rightarrow
        & \begin{gathered}
            S_x = \dfrac{\hbar}{2}
                \left( \begin{matrix} 
                    0 & 1 \\
                    1 & 0
                \end{matrix} \right) = \dfrac{\hbar}{2} \sigma_x
                \hspace{18pt}
                S_y = \dfrac{\hbar}{2}
                \left( \begin{matrix} 
                    0 & -i \\
                    i & 0
                \end{matrix} \right) = \dfrac{\hbar}{2} \sigma_y 
                \\[5pt]
            % Anticommutator
            \mss{ \boxed{ \left\{ S_i, S_j \right\} = \tfrac{\hbar^2}{2} \delta_{ij} } }
                \hspace{20pt} (\text{\scriptsize only for \(s=1/2\) systems})
            \end{gathered}
\end{array}\)

% General Direction
\vspace{30pt}\noindent
\underline{General Direction, \(\hat{n}\)}\\[5pt]
\(
    \begin{aligned}
        \hat{n} \cdot \vec{S} & = (\cos\phi \sin\theta) S_x + (\sin\phi \sin\theta) S_y + (\cos\theta) S_z 
            \\[5pt]
        & = \tfrac{\hbar}{2} \mss{ \left[\begin{matrix}
                n_z & n_x - i n_y \\
                n_x + i n_y & - n_z
            \end{matrix}\right] }
            = 
            \tfrac{\hbar}{2} \mss{ \left[\begin{matrix}
                \cos\theta & \sin\theta e^{-i\phi} \\
                \sin\theta e^{i\phi} & - \cos\theta
            \end{matrix}\right] }
    \end{aligned}
    \hfill\vline\hfill
    \begin{aligned}
        & \hat{n} \cdot \vec{S} \hs\hs | \chi_\pm \rangle = \mss{\pm} \tfrac{\hbar}{2} \hs | \chi_\pm \rangle
            % \ \ \Rightarrow\ \
            \\[5pt]
        & \boxed{
            | \chi_+ \rangle = \mss{
                e^{i\gamma}
                \left[\begin{matrix}
                    \cos\tfrac{\theta}{2} \hs e^{- i \phi/2}\\[5pt]
                    \sin\tfrac{\theta}{2} \hs e^{i \phi/2}
                \end{matrix}\right]
            }
            \ , \
            | \chi_- \rangle = \mss{
                e^{i\gamma}
                \left[\begin{matrix}
                    - \sin\tfrac{\theta}{2} \hs e^{- i \phi/2}\\[5pt]
                    \cos\tfrac{\theta}{2} \hs e^{i \phi/2}
                \end{matrix}\right]
            }
            }
    \end{aligned}
\)

%--------------------------------------------------------------------------------------------------------------------------------
%
%
%
\newpage
% Pauli Matricies Properties
\vspace{10pt}\noindent
\underline{Properties of Pauli Matrices, \(\sigma_i\)}\\[10pt]
\(\begin{aligned}
    & \bullet\ \sigma_i^2 = I
        = (\hat{n} \cdot \sigma)^2 
        \ \Leftarrow\ ( \hat{n} \cdot \sigma )^2 - I^2
        = ( \hat{n} \cdot \sigma + I ) ( \hat{n} \cdot \sigma - I ) = 0 
        = ( \hat{n} \cdot S + \tfrac{\hbar}{2} ) ( \hat{n} \cdot S - \tfrac{\hbar}{2} )
        \\[5pt]
    & \bullet\ \sigma_i \sigma_j = i \sigma_k = - \sigma_j \sigma_i
        \ \Rightarrow\ 
        \left[ \tfrac{\sigma_i}{2}, \tfrac{\sigma_j}{2} \right] = i \tfrac{\sigma_k}{2}
        \hspace{20pt} \mss{(i \neq j)}
        \\[5pt]
    & \bullet\ (A \cdot \sigma)(B \cdot \sigma) = A \cdot B + i (A \times B) \cdot \sigma
        \hspace{20pt} ( \mss{ \text{\scriptsize if } \left[A_i, \sigma_i\right] = 0 = \left[B_i, \sigma_i\right] } )
        \\[5pt]
    & \bullet\ \text{Tr } \sigma_i = 0
        \ \Rightarrow\ 
        \text{Tr} (\sigma_\alpha \sigma_\beta) = 2 \delta_{\alpha\beta} 
        \hspace{20pt} \mss{\alpha \in (0, x, y, z)}
        \\[5pt]
    & \bullet\ \mss{ \sum } c_\alpha \sigma_\alpha = 0 
        \rightarrow
        c_\alpha = 0
        \ \Rightarrow\ 
        M_{2\times 2} = \mss{ \sum \tfrac{1}{2} \text{\scriptsize Tr}(M \sigma_\alpha) } \hs \sigma_\alpha
        \\[5pt]
    & \bullet\ \mss{ \begin{gathered}
            \text{Rotate \(\hat{k}\) to \(\hat{n}\) by angle \(\theta\)}\\[-3pt]
            \hat{\theta} = (-\sin\phi, \cos\phi, 0) = \hat{k} \times \hat{n} 
        \end{gathered} }
        \hspace{20pt} 
        U[R(\theta)] = e^{ - i \tfrac{\theta}{2} \hat{\theta} \cdot \sigma } 
        = I \cos \tfrac{\theta}{2} - i \big( \hat{\theta} \cdot \sigma \big) \sin \tfrac{\theta}{2}
        = \mss{ \left[\begin{matrix}
            \cos \tfrac{\theta}{2} & - \sin \tfrac{\theta}{2} \hs e^{- i \phi}\\[3pt]
            \sin \tfrac{\theta}{2} \hs e^{i \phi} & \cos \tfrac{\theta}{2}
        \end{matrix}\right] }
        \\[5pt]
    & \bullet\ \mss{ \begin{gathered}
            \text{Rotate about \(\hat{n}\) by angle \(\epsilon\)}\\[-3pt]
            \hat{n} = (n_x, n_y, n_z) 
        \end{gathered} }
        \hspace{20pt} 
        U[R(\psi)] = I \cos \tfrac{\psi}{2} - i \big( \hat{n} \cdot \sigma \big) \sin \tfrac{\psi}{2}
        = \mss{ \left[\begin{matrix}
            \cos \tfrac{\psi}{2} - i n_z \sin \tfrac{\psi}{2}  & (- i n_x - n_y) \sin \tfrac{\psi}{2} \\[3pt]
            (- i n_x + n_y) \sin \tfrac{\psi}{2} & \cos \tfrac{\psi}{2} + i n_z \sin \tfrac{\psi}{2}
        \end{matrix}\right] }
\end{aligned}\)

\vspace{10pt}
%-----------------------------------------------------------------------------------------------------------------------------
%-----------------------------------------------------------------------------------------------------------------------------
% 2 Particles w/ Spin 1/2
\subsection{2 Objects w/ Spin}
{\scriptsize Objects could be orbital momentum, another particle spin, etc.}

\vspace{5pt}
\subsubsection{2 Objects w/ Spin \(\frac{1}{2}\):\ \ \
    \(
        \tfrac{1}{2} \otimes \tfrac{1}{2} = 1 \oplus 0 
        \ \Rightarrow\ \mss{ \displaystyle (2s_1+1)(2s_2+1) = \sum_{s=|s_1-s_2|}^{s_1+s_2} 2s+1 } 
    \) 
}

\vspace{10pt}
*Find Eigenvectors, \(e_i\) , of \( ( S^{ (1,2) } )_z \) and 
    \( ( S^{ (1,2) } )^2 \) in the form of \( | \chi_i \chi_j \rangle \) ( using \( ( S^{ (1,2) } )_\pm \) )

\vspace{15pt}\noindent
% Tensor Product for 2 particles
\[ \boxed{  \chi_i \chi_j \rightarrow | \chi_i \chi_j \rangle \equiv | \chi_i \rangle | \chi_j \rangle 
    \equiv | \chi_i \rangle \otimes | \chi_j \rangle } \]

% Chi for 1 particle
\vspace{20pt} \noindent
Choose \( | \chi_i \rangle \equiv S_z\)-Eigenvector w/ 
    Spin \(\frac{1}{2}\) (e.g, \( | \frac{1}{2} \frac{-1}{2} \rangle 
        = \left( \begin{smallmatrix} 0 \\ 1 \end{smallmatrix} \right) \), 
        as opposed to \( \left( \begin{smallmatrix} .6 \\ .8 \end{smallmatrix} \right) \))

\vspace{1cm} \noindent
\begin{minipage}[t]{.5\textwidth}
    % "Vector" of Matrices
    \begin{center} \( 
        S^{(i)} \equiv 
        \left( \begin{matrix} 
            S^{(i)}_x\\[5pt] 
            S^{(i)}_y\\[5pt] 
            S^{(i)}_z 
        \end{matrix}\right) 
    \) \end{center}
        
    \vspace{5pt}
    % Matrices only work on respective particles
    \( 
        \bullet \ S^{(2)}_z S^{(1)}_x \Big( | \chi_1 \rangle \otimes | \chi_2 \rangle \Big)
        = \left( S^{(1)}_x | \chi_1 \rangle \right) \otimes \left( S^{(2)}_z | \chi_2 \rangle \right) 
    \)\\[10pt]
    % Dot product for "vector" of matrices
    \( \begin{aligned}
        \bullet \ S^{(i)} \dotP S^{(j)} & \equiv \underline{ S^{(i)}_x S^{(j)}_x + S^{(i)}_y S^{(j)}_y } + S^{(i)}_z S^{(j)}_z \\[5pt]
        & = S^{(i)}_z S^{(j)}_z + \underline{ \tfrac{1}{2} S^{(i)}_+ S^{(j)}_- + \tfrac{1}{2} S^{(i)}_- S^{(j)}_+ }
            \\[10pt]
        ( S^{(i)} )^2 & \equiv S^{(i)} \dotP S^{(i)} 
    \end{aligned}\)
\end{minipage}
\hfill\vline\hfill
\begin{minipage}[t]{.45\textwidth}
    % Sum of "Vector" of Matrices
    \begin{center} \( 
        S^{(1,2)} \equiv \left( S^{(1)} + S^{(2)} \right) \equiv 
        \left( \begin{matrix} S^{(1)}_x + S^{(2)}_x \\[5pt]
            S^{(1)}_y + S^{(2)}_y \\[5pt]
            S^{(1)}_z + S^{(2)}_z 
        \end{matrix} \right) 
    \) \end{center}

    \vspace{5pt}
    \( \bullet \ \left( S^{(1,2)} \right)^2 =
        \left( S^{(1)} + S^{(2)} \right) \dotP \left( S^{(1)} + S^{(2)} \right) \)
\end{minipage}


%--------------------------------------------------------------------------------------------------------------------------------
%
%
%
\newpage 
% 1.) Find Eigenvalues for S^(1,2)_z
\noindent
\textbf{1.} \(\left( S^{(1,2)} \right)_z\) \\[10pt]
\begin{minipage}[t]{.55\textwidth}
    \( \begin{aligned}[t] 
        \left( S^{(1,2)} \right)_z \chi_1 \chi_2 
            &= \Big( S^{(1)}_z + S^{(2)}_z \Big) 
                | \chi_1 \rangle | \chi_2 \rangle\\[10pt]
        &= S^{(1)}_z | \chi_1 \rangle \otimes | \chi_2 \rangle 
            + | \chi_1 \rangle \otimes S^{(2)}_z | \chi_2 \rangle \\[10pt]
        \left( S^{(1,2)} \right)_z  | \chi_1 \chi_2 \rangle 
            &= \hbar (m_1 + m_2) \ | \chi_1 \chi_2 \rangle
    \end{aligned} \)
    
    \vspace{10pt}
    \( \Rightarrow \ \underline{ e_i = a_i| \uparrow \uparrow \ \rangle + b_i| \uparrow \downarrow \ \rangle +
    c_i| \downarrow \uparrow \ \rangle + d_i| \downarrow \downarrow \ \rangle } \)    
\end{minipage}
\hfill\vline\hfill
\begin{minipage}[t]{.3\textwidth}
    \( \begin{aligned}[t]
        | \uparrow \uparrow \ \rangle \quad & =
            & | \tfrac{1}{2} \tfrac{1}{2} \rangle 
            & \otimes | \tfrac{1}{2} \tfrac{1}{2} \rangle 
            \\[10pt]
        | \uparrow \downarrow \ \rangle \quad & =
            & | \tfrac{1}{2} \tfrac{1}{2} \rangle 
            & \otimes | \tfrac{1}{2} \tfrac{-1}{2} \rangle 
            \\[10pt]
        | \downarrow \uparrow \ \rangle \quad & =
            & | \tfrac{1}{2} \tfrac{-1}{2} \rangle 
            & \otimes | \tfrac{1}{2} \tfrac{1}{2} \rangle 
            \\[10pt]
        | \downarrow \downarrow \ \rangle \quad & =
            & | \tfrac{1}{2} \tfrac{-1}{2} \rangle 
            & \otimes | \tfrac{1}{2} \tfrac{-1}{2} \rangle
    \end{aligned}  \)
\end{minipage}

% 2.) Use S_+- to guess the Eigenvectors
\vspace{25pt} \noindent
\textbf{2.} Use \( \left( S^{ (1,2) } \right)_\pm \) on 
    \( | \uparrow \ \rangle \otimes | \uparrow \ \rangle \) to GUESS \(e_i\) from ``nice'' bevahior 

\vspace{15pt} \noindent
\begin{minipage}{.40\textwidth}
    {\setlength{\tabcolsep}{3pt}
    \begin{tabular}{l c l}
        \(S_-  \ | \uparrow \uparrow \ \rangle \)
            & \(=\)
            & \( \tfrac{\sqrt{2}}{2} \big( | {\scriptstyle \uparrow \downarrow} \rangle +
                | {\scriptstyle \downarrow \uparrow} \rangle \big) \) \\[5pt]
        \( S_- \left[ \tfrac{\sqrt{2}}{2} \big( | {\scriptstyle \uparrow \downarrow} \rangle +
            | {\scriptstyle \downarrow \uparrow} \rangle \big) \right] \)
            & \(=\)
            & \( | \downarrow \downarrow \ \rangle \) \\[5pt]
        \( S_- \ | \downarrow \downarrow \ \rangle \)
        & \(=\) 
        & \(0\)
    \end{tabular} }

    \vspace{20pt}
    \(S_+\) works too

    \vspace{20pt}
    If \( \tfrac{\sqrt{2}}{2} \big( | {\scriptstyle \uparrow \downarrow} \rangle +
        | {\scriptstyle \downarrow \uparrow} \rangle \big) \) then maybe \\
    \( \tfrac{\sqrt{2}}{2} \big( | {\scriptstyle \uparrow \downarrow} \rangle -
        | {\scriptstyle \downarrow \uparrow} \rangle \big) \) works (try \(S_\pm\) on it).
\end{minipage}
\hfill\vline\hfill
\begin{minipage}{.58\textwidth}
    Guess for \(\{e_i\}\): \\[15pt]
    { \setlength{\tabcolsep}{3pt}
    \begin{tabular}{l c c c c}
        \( | 1 \ 1 \rangle \)
            & \(\equiv\)
            & \( | \tfrac{1}{2} \tfrac{1}{2} \rangle | \tfrac{1}{2} \tfrac{1}{2} \rangle \)
            & \(=\) 
            &\( | \uparrow \uparrow \ \rangle \) \\[10pt]
        \( | 1 \ 0 \rangle \) 
            & \(\equiv\)
            & \( \tfrac{1}{\sqrt{2}} \Big(
                | \tfrac{1}{2} \tfrac{1}{2} \rangle | \tfrac{1}{2} \tfrac{-1}{2} \rangle +
                | \tfrac{1}{2} \tfrac{-1}{2} \rangle | \tfrac{1}{2} \tfrac{1}{2} \rangle \Big) \)
            & \(=\)
            & \( \tfrac{\sqrt{2}}{2} \big( | {\scriptstyle \uparrow \downarrow} \rangle +
                | {\scriptstyle \downarrow \uparrow} \rangle \big) \) \\[10pt]
        \( | 1 \ \text{-}1 \rangle \)
            & \(\equiv\)
            & \( | \tfrac{1}{2} \tfrac{-1}{2} \rangle | \tfrac{1}{2} \tfrac{-1}{2} \rangle \)
            & \(=\)
            & \( | \downarrow \downarrow \ \rangle \) \\[20pt]
        \( | 0 \ 0 \rangle \)
            & \( \equiv \)
            & \( \tfrac{1}{\sqrt{2}} \Big( 
                | \tfrac{1}{2} \tfrac{1}{2} \rangle | \tfrac{1}{2} \tfrac{-1}{2} \rangle -
                | \tfrac{1}{2} \tfrac{-1}{2} \rangle | \tfrac{1}{2} \tfrac{1}{2} \rangle \Big) \)
            & \(=\)
            & \( \tfrac{\sqrt{2}}{2} \big( | {\scriptstyle \uparrow \downarrow} \rangle - 
                | {\scriptstyle \downarrow \uparrow} \rangle \big) \)
    \end{tabular} }
\end{minipage}

% 3.) Check with S^2
\vspace{25pt} \noindent
\textbf{3.} Check if the guesses are eigenvectors of \(\left( S^{(1,2)} \right)^2\) 
    [and do \(\left( S^{(1,2)} \right)_z\) to see eigenvalues]

{\scriptsize (work has been skipped, do it yourself, check answer below)}

\vspace{20pt} \noindent
\( \begin{aligned}[b]
    S^2 | 1 \ 1 \rangle \ &= \ 
        \hbar^2(1)(1+1) | 1 \ 1 \rangle& \hspace{18pt} (s&=1) 
    & \hspace{18pt}\hspace{18pt} 
    S_z | 1 \ 1 \rangle \ &= \ 
        \hbar(1) | 1 \ 1 \rangle& \hspace{18pt} (m&=1)
    \\[5pt]
    S^2 | 1 \ 0 \rangle \ &= \ 
        \hbar^2(1)(1+1) | 1 \ 0 \rangle& \hspace{18pt} (s&=1)
    & \hspace{18pt}\hspace{18pt}
    S_z | 1 \ 0 \rangle \ &= \ 
        \hbar(0) | 1 \ 0 \rangle& \hspace{18pt} (m&=0)
    \\[5pt]
    S^2 | 1 \ \text{-} 1 \rangle \ &= \ 
        \hbar^2(1)(1+1) | 1 \ \text{-} 1 \rangle& \hspace{18pt} (s&=1)
    & \hspace{18pt}\hspace{18pt}
    S_z | 1 \ \text{-} 1 \rangle \ &= \ 
        \hbar(-1) | 1 \ \text{-} 1 \rangle& \hspace{18pt} (m&=-1)
    \\[10pt]
    S^2 | 0 \ 0 \rangle \ &= \ 
        \hbar^2(0)(0+1) | 0 \ 0 \rangle& \hspace{18pt} (s&=0) 
    & \hspace{18pt}\hspace{18pt}
    S_z | 0 \ 0 \rangle \ &= \ 
        \hbar(0) | 0 \ 0 \rangle& \hspace{18pt} (m&=0)
\end{aligned} \hspace{18pt} \checkedbox \)

\vspace{20pt} \noindent
* \ \fbox{ 
\( 
e_i \in \left\{ \begin{gathered}
    \left. 
        { \setlength{\arraycolsep}{3pt} \begin{array}{c c c c c}
            | 1 \ 1 \rangle & =
                & \mss{1 \leftarrow\ } \bcancel{e^{i\phi}} \ | \tfrac{1}{2} \tfrac{1}{2} \rangle | \tfrac{1}{2} \tfrac{1}{2} \rangle 
                & =
                & | \uparrow \uparrow \ \rangle 
                \\[10pt]
            | 1 \ 0 \rangle & =
                & \tfrac{1}{\sqrt{2}} \Big(
                    | \tfrac{1}{2} \tfrac{1}{2} \rangle | \tfrac{1}{2} \tfrac{-1}{2} \rangle +
                    | \tfrac{1}{2} \tfrac{-1}{2} \rangle | \tfrac{1}{2} \tfrac{1}{2} \rangle \Big)
                & =
                & \tfrac{\sqrt{2}}{2} \big( | {\scriptstyle \uparrow \downarrow} \rangle +
                    | {\scriptstyle \downarrow \uparrow} \rangle \big)
                \\[10pt]
            | 1 \ \text{-}1 \rangle & =
                & | \tfrac{1}{2} \tfrac{-1}{2} \rangle | \tfrac{1}{2} \tfrac{-1}{2} \rangle
                & =
                & | \downarrow \downarrow \ \rangle
        \end{array} } 
        \right\} 
        \hspace{18pt} \text{Triplet}: \ s=1 
        \\[15pt]
    \left.
            { \setlength{\arraycolsep}{3pt}
            \begin{array}{c c c c c}
                | 0 \ 0 \rangle & =
                    & \tfrac{1}{\sqrt{2}} \Big( 
                        | \tfrac{1}{2} \tfrac{1}{2} \rangle | \tfrac{1}{2} \tfrac{-1}{2} \rangle -
                        | \tfrac{1}{2} \tfrac{-1}{2} \rangle | \tfrac{1}{2} \tfrac{1}{2} \rangle \Big)
                    & =
                    & \tfrac{\sqrt{2}}{2} \big( | {\scriptstyle \uparrow \downarrow} \rangle - 
                        | {\scriptstyle \downarrow \uparrow} \rangle \big)
            \end{array} }
        \right\} 
        \hspace{18pt} \text{Singlet}: \ s=0 
\end{gathered} \right. 
\)
}

%---------------------------------------------------------------------------------------------------------------------------------
%
%
%---------------------------------------------------------------------------------------------------------------------------------
\newpage
% 2 Objects with General Spin
\subsubsection{2 Objects w/ Any Spin: \ \ \ 
    \(
        \mss{ j_1 \otimes j_2 = (j_1 + j_2) \oplus |j_1 - j_2| }
        \ \Rightarrow\ \mss{ \displaystyle (2j_1+1)(2j_2+1) = \sum_{j=|j_1-j_2|}^{j_1+j_2} 2j+1 } 
    \) 
}

\vspace{5pt} \noindent
\begin{minipage}[t]{.58\textwidth}
    \setlength{\parindent}{.5cm}
    \noindent
    \(\bullet\) \(| \chi_1 \rangle\) has spin, \(j_1\); and \(| \chi_2 \rangle\) has spin, \(j_2\)\\[10pt]
    \(\bullet\) \(j_\text{max} = j_2 + j_1\) and \(j_\text{min} = |j_2 - j_1| \)\\[10pt]
    \(\bullet\) Possible total \(| j \ m \rangle\) must satisfy %
    \begin{center}
        \(\begin{aligned}
            & \boldsymbol{1.)} \ j_\text{min} \leq j \leq j_\text{max},  
                \ \ \ \ \boldsymbol{2.)} \ -j \leq m \leq j, 
                \\[5pt]
            & \boldsymbol{3.)} \ \text{have integer differences}
        \end{aligned}\)
    \end{center}

    \vspace{10pt} If \(j_1\) and \(j_2\) are known from the start,
    \[ \boxed{ 
        \begin{aligned}
            | j \ m \rangle & = \mss{ \sum_{m_1, m_2} 
                | j_{1} \ m_{1} \rangle \otimes | j_{2} \ m_{2} \rangle \
                \langle j_{1} \ m_{1} | \otimes \langle j_{2} \ m_{2} |
                }
                | j \ m \rangle
                \\
            & = \sum_{m_1, m_2} | j_{1} \ m_{1} \rangle \otimes | j_{2} \ m_{2} \rangle \
                C^{j_1 j_2 j}_{m_{1} m_{2} m}
        \end{aligned}
    } \] 
    where the sum is over all poss. int. diff. values that satisfy\\[10pt]
    \(\begin{aligned}
        m_{1} + m_{2} &= m,&  -j_{1} \leq \ &m_{1} \leq j_{1},&   -j_{2} \leq &\ m_{2} \leq j_{2}, 
    \end{aligned}\)\\[10pt]
    and \(C\) are the corresponding Clebsh-Gordan coefficients, whose squared value is
    the probability of measuring the \(\chi_1 \otimes \chi_2\) state represented by that term.
\end{minipage}
\hfill
\fbox{
\begin{minipage}[t]{.33\textwidth}
    \vspace{5pt}
    \begin{center}
        \underline{Possible Combined \( |j \ m\rangle \)}
    \end{center}
    \begin{center} 
        \( \mss{ \begin{aligned}
            (2j_\text{max}+1) & \left\{ \hspace{5pt} \begin{aligned}
                    & \Big| j_\text{max} \ \ j_\text{max} \Big\rangle = \mss{1 \leftarrow\ } \cancel{ e^{i\phi} } \cdot \dots\\
                    & \Big| j_\text{max} \ \ j_\text{max} {\scriptstyle - 1} \Big\rangle \\
                    & \hspace{25pt} \vdots \\
                    & \Big| j_\text{max} \ \ {\scriptstyle -} j_\text{max} \Big\rangle \\
                \end{aligned} \right.
                \\
            (2j_\text{max}-1) & \left\{ \hspace{5pt} \begin{aligned}
                    & \Big| j_\text{max} {\scriptstyle - 1} \ \ j_\text{max} {\scriptstyle - 1} \Big\rangle = 1 \cdot \dots\\
                    & \Big| j_\text{max} {\scriptstyle - 1} \ \ j_\text{max} {\scriptstyle - 2} \Big\rangle \\
                    & \hspace{25pt} \vdots \\
                \end{aligned} \right. 
                \\
            & \hspace{5pt} \vdots
                \\
            (2j_\text{min} + 1) & \left\{ \hspace{5pt} \begin{aligned}
                    & \Big| j_\text{min} \ \ j_\text{min} \Big\rangle = 1 \cdot \dots\\
                    & J_- \Big| j_\text{min} \ \ j_\text{min} \Big\rangle \sim \hs \dots \\
                    & \hspace{25pt} \vdots \\
                    & \Big| j_\text{min} \ \ {\scriptstyle -} j_\text{min} \Big\rangle 
                \end{aligned} \right. 
        \end{aligned} } \) 
    \end{center}
    \vspace{.1cm}
\end{minipage}
}

\vspace{15pt} 
    If the top state in a \(j\)-set (see box above) is known, applying the \(J_-\) lowering operator (and normalizing) %
provides the coefficients for the rest of the set of varying \(m\). %
The coefficients for each top state of a set are (by convention) positive, real, %
and normalized to 1. This makes all of the coefficients real. For the top state of the initial %
\(j_\text{max}\)-set, \( | j_\text{max}\ j_\text{max} \rangle\), %
there is only one product-ket in the sum; its coefficient %
is thus set to 1. For an arbitrary set below the first, the top state has product-ket coefficients such that the state %
is orthogonal to all other previously determined states that have the same \(m\). To reduce some work to solve for them, use %
\[ 
    \begin{aligned}
        C^{j_1 j_2 j}_{m_{1} m_{2} m} & = (-1)^{j_1 + j_2-  j} \cdot C^{j_1 j_2 j}_{-m_{1} -m_{2} -m}\\[5pt]
        \langle j_1 \ m_1 | \langle j_2\ m_2 | j\ m \rangle & 
            = (-1)^{j_1 + j_2-  j} \cdot \langle j_1 \ -m_1 | \langle j_2\ -m_2 | j\ -m\rangle 
    \end{aligned}
\]

\vspace{15pt} 
    If \(m_1\) and \(m_2\) are also known from the start, then \(m = m_1 + m_2\), and
\[ 
    \boxed{ 
        | j_1 \ m_1 \rangle \otimes | j_2 \ m_2 \rangle 
        = \sum_j C^{j_1 j_2 j}_{m_{1} m_{2} m} \ | j \ {\scriptstyle (m_1+m_2)} \rangle 
    } 
\]
where the sum is only over all possible \(s\) as satisfied above - \textbf{1.), 2.) and 3.)}.
In this case, the total z-component, \(m\), is known. The only unknown is the total spin, \(s\), 
whose probability to be measured is \(C^2\).

%--------------------------------------------------------------------------------------------------------------------
%
%
%--------------------------------------------------------------------------------------------------------------------
\newpage
\subsection{Electron in Magnetic Field}

\vspace{10pt} \noindent
\(
    \mu_\text{clas.} = IA = \frac{q}{2 \pi r} v (\pi r^2) 
    = \frac{q}{2 \pi r} \frac{L}{m r} (\pi r^2) 
    = \left( \frac{q}{2m} \right) L 
    \ \rightarrow \ \frac{e\hbar}{2m} \cdot n 
    \hspace{20pt} \text{\scriptsize(Bohr magneton)}
\)\\[10pt]
\(
    \mu_\text{quan.} = \left( \frac{g_e q}{2m} \right) S = \left( \frac{q}{m} \right) S = \gamma S
\)
\[
    \begin{aligned}
        \tau_\mu &= \mu \times B \\[5pt]
        F_\mu &= \nabla(\mu \dotP B)
    \end{aligned}
    \hspace{18pt} , \hspace{18pt}\begin{aligned}
        H &= -\mu \dotP B \\[5pt]
        &= - \gamma S \dotP B
    \end{aligned}
\]

% Larmor Procession
\vspace{20pt} \noindent
\underline{Larmor Precession}\\[10pt]
\(\begin{aligned}
    B &= B_0 \hat{k}\\[10pt]
    H &= - \gamma B_0 S_z \\[5pt]
    &= - \gamma B_0 \left( \begin{matrix} 
            \tfrac{\hbar}{2} & 0\\
            0 & -\tfrac{\hbar}{2}
        \end{matrix} \right) \\[10pt]
\end{aligned} 
\ \ \Rightarrow \ \
\begin{aligned}
    \chi(t) &= \cos(\alpha / 2) \left(\begin{matrix}
            1\\
            0
        \end{matrix}\right) e^{- \frac{i}{\hbar} E_1 t} 
        + \sin(\alpha / 2) \left(\begin{matrix}
            0\\
            1
        \end{matrix}\right) e^{- \frac{i}{\hbar} E_2 t}\\[5pt]
    &= \left(\begin{matrix}
            \cos(\alpha / 2) e^{- \frac{i}{\hbar} E_1 t} \\
            \sin(\alpha / 2) e^{- \frac{i}{\hbar} E_2 t}
        \end{matrix}\right)\\[20pt]
    \left(\begin{matrix}
        \langle S_x \rangle\\[5pt]
        \langle S_y \rangle\\[5pt]
        \langle S_z \rangle
    \end{matrix}\right) &=
    \left(\begin{matrix}
        \frac{\hbar}{2} \sin(\alpha) \cos(\gamma B_0 t)\\[5pt]
        - \frac{\hbar}{2} \sin(\alpha)\sin(\gamma B_0 t)\\[5pt]
        \frac{\hbar}{2} \cos(\alpha)
    \end{matrix}\right) \hspace{18pt} \text{\scriptsize(torque from \(B\) with \(S\) leads to precession)}
\end{aligned}\)

% Stern-Gerlach
\vspace{20pt} \noindent
\underline{Stern-Gerlach}

%-------------------------------------------------------------------------------------------------------------
% Bosons and Fermions
\newpage
\section{Bosons and Fermions}

% Distinguishable Particles
\vspace{5pt} \noindent
Distinguishable Particles: \hspace{18pt}
\(\boxed{ \psi(r_1,r_2) \equiv \psi_a(r_1) \psi_b(r_2) }\)

% Indistinguishable Particles
\vspace{15pt} \noindent
Indistinguishable Particles:
\[
    \boxed{ P_x f(x_1, x_2; \ y_1, y_2; \ ...) \ = \ \pm \ f(x_2, x_1; \ y_1, y_2; \ ...) }
    \hspace{10pt} , \hspace{10pt} 
    \boxed{ 
        \mss{\iint} \hs | \Psi \mss{(x_1, x_2)} |^2 \mss{dx_1 dx_2} = \mss{\iint} \hs\hs \text{Pr} \mss{(x_1, x_2)} \tfrac{dx_1 dx_2}{2} 
    }
\]

\vspace{5pt} \hspace{18pt} \(\begin{aligned}
    & \begin{aligned}
            &\text{Boson:}\\
            &\big( s \in \{ 0,1,2,... \} \big)
        \end{aligned} 
        & \hspace{18pt} \psi_+(r_1,r_2) 
        & \equiv \frac{1}{\sqrt{2}} \Big[ \psi_a(r_1) \psi_b(r_2) + \psi_a(r_2) \psi_b(r_1)  \Big] 
        \\[5pt]
    && \Aboxed{ \psi(r_1,r_2) 
        & = \psi(r_2,r_1) } \hspace{18pt} \rightarrow \hspace{18pt} \boxed{P_i \Psi = \Psi} \hspace{18pt} \text{\scriptsize(symmetric)}
        \\[10pt]
    &\begin{aligned}
            &\text{Fermion:}\\
            &\big( s \in \{ \tfrac{1}{2}, \tfrac{3}{2}, \tfrac{5}{2}, ... \} \big)
        \end{aligned}
        & \hspace{18pt} \psi_-(r_1,r_2) 
        &\equiv \frac{1}{\sqrt{2}} \Big[ \psi_a(r_1) \psi_b(r_2) - \psi_a(r_2) \psi_b(r_1)\Big]\\[5pt]
        && \Aboxed{ \psi(r_1,r_2) 
        &= - \psi(r_2,r_1) }
        \hspace{18pt} \rightarrow \hspace{18pt} \boxed{P_i \Psi = - \Psi} 
        \hspace{18pt} \text{\scriptsize(antisymmetric)}
\end{aligned}\)

% Exchange Forces
\subsection{Exchange Forces: \hspace{18pt} \(\Big\langle (x_1 - x_2)^2 \Big\rangle 
    = \langle x_1^2 \rangle + \langle x_2^2 \rangle 
    - 2 \langle x_1 x_2 \rangle\)}

% Equations
\vspace{5pt}
\hspace{.5cm} \fbox{ \(\begin{aligned}
    &\text{Dist. Part. \ :}&  \big\langle {\scriptstyle (\Delta x)^2} \big\rangle 
        &= \big\langle {\scriptstyle (\Delta x)^2} \big\rangle _d = \langle x^2 \rangle_a + \langle x^2 \rangle_b 
        - 2 \langle x \rangle_a \langle x \rangle_b\\[10pt]
    &\text{Symmetric:}&        \big\langle {\scriptstyle (\Delta x)^2} \big\rangle 
        &= \big\langle {\scriptstyle (\Delta x)^2} \big\rangle _d -
        2 \ \big\Vert \left\langle \psi_b \right| x \left| \psi_a \right\rangle \big\Vert^2 
        \hspace{18pt} \text{\scriptsize(attractive if overlap)}\\[10pt]
    &\text{Antisymmetric:}&      \big\langle {\scriptstyle (\Delta x)^2} \big\rangle 
        &= \big\langle {\scriptstyle (\Delta x)^2} \big\rangle _d +
        2 \ \big\Vert \left\langle \psi_b \right| x \left| \psi_a \right\rangle \big\Vert^2
        \hspace{18pt} \text{\scriptsize(repulsive if overlap)}
\end{aligned}\) }

% Boson/Fermion <x1x2>
\vspace{10pt} \noindent
\(\bullet \ \begin{aligned}[t]
    \langle x_1 x_2 \rangle &= \frac{1}{2} \int \Big[ \psi_a(r_1)^* \psi_b(r_2)^* 
        \pm \psi_b(r_1)^* \psi_a(r_2)^* \Big] x_1 x_2 \Big[ \psi_a(r_1) \psi_b(r_2) 
        \pm \psi_b(r_1) \psi_a(r_2) \Big] dx_1 dx_2\\[5pt]
    &= \ \begin{aligned}[t]
            &   \frac{1}{2} \langle x \rangle_a \langle x \rangle_b +
                \frac{1}{2} \langle x \rangle_b \langle x \rangle_a \\[5pt]
            &\pm \frac{1}{2} \Big\langle \psi_b(r_1) \Big| x_1 \Big| \psi_a(r_1) \Big\rangle 
                \Big\langle \psi_a(r_2) \Big| x_2 \Big| \psi_b(r_2) \Big\rangle \pm \frac{1}{2} 
                \Big\langle \psi_a(r_1) \Big| x_1 \Big| \psi_b(r_1) \Big\rangle
                \Big\langle \psi_b(r_2) \Big| x_2 \Big| \psi_a(r_2) \Big\rangle
        \end{aligned}\\[10pt]
    &= \langle x \rangle_a \langle x \rangle_b \pm
    \big\Vert \left\langle \psi_b \right| x \left| \psi_a \right\rangle \big\Vert^2
\end{aligned}\)

\vspace{15pt} \noindent
Two Electrons: \\[5pt]
\hspace{18pt} \(\psi{\scriptstyle(r_1,r_2)} \chi{\scriptstyle(m_1,m_2)} = \left\{\begin{aligned} 
    \begin{gathered}
        \text{\scriptsize(singlet)}\\
        - \psi{\scriptstyle(r_1,r_2)} \chi{\scriptstyle(m_2,m_1)} 
    \end{gathered}&&
        &\Rightarrow& &\begin{aligned}
            &\chi \ \text{\scriptsize is antisymmetric so} \\
            &\psi \ \text{\scriptsize is symmetric}
        \end{aligned}& 
        &\Rightarrow& &\text{Attractive \scriptsize(ground state)} \\[5pt]
    \begin{gathered}
        \text{\scriptsize(triplet)}\\
        - \psi{\scriptstyle(r_2,r_1)} \chi{\scriptstyle(m_1,m_2)} 
    \end{gathered}&&
        &\Rightarrow& &\begin{aligned}
            &\chi \ \text{\scriptsize is symmetric so} \\
            &\psi \ \text{\scriptsize is antisymmetric}
        \end{aligned}& 
        &\Rightarrow& &\text{Repulsive} \\
\end{aligned}\right.\)

\newpage
\subsection{Statistics}
Sterling's Approx: \hspace{18pt} \(\begin{aligned}
    \log(z!) &\approx z \log(z) - z \hspace{18pt} \hspace{18pt} z \gg 1 \ \ \text{or} \ \ z = 0\\
    \frac{d}{dz} \ \log(z!) &\approx \log(z)
\end{aligned}\)\\[15pt]
Lagrange Multiplier: \hspace{18pt} \(\begin{gathered}
    G(X,\alpha,\beta) = \log(Q{\scriptstyle(X)}) + \alpha f_1(X) + \beta f_2(X)\\[5pt]
    \frac{\partial G}{\partial \alpha}[Q_\text{max}] = 0, \hspace{18pt}
    \frac{\partial G}{\partial \beta}[Q_\text{max}] = 0, \hspace{18pt}
    \frac{\partial G}{\partial N_n}[Q_\text{max}] = 0
\end{gathered}\)

\vspace{10pt} \noindent
\rule[0pt]{1\textwidth}{.5pt}

\[\begin{aligned}
    &\sum_n N_n = N &       &\sum_n N_n E_n = E\\
    f_1(X) &= N - \sum_n N_n = 0 & \hspace{18pt} \hspace{18pt} f_2(X) &= E - \sum_n N_n E_n = 0
\end{aligned}\]

Let there be \(N_n\) particles in the \(E_n\) energy level having \(d_n\) degeneracies, and
\(Q(N_1, N_2, ...)\) be the number of possible configurations for such a state given \(X = (N_1, N_2, ..., N_n)\).

\vspace{25pt} \noindent
\(\begin{aligned}
    & \text{Dist.} 
        &     
        &\left\{ 
            \begin{aligned}
                &\boldsymbol{1.)}\ \begin{aligned}[t]
                        Q(X) &= \prod_n \left( \begin{matrix} 
                                N - N_1 - ... - N_{n-1} \\ 
                                N_n 
                            \end{matrix}\right) d_n^{N_n} 
                            \\
                        &= N! \prod_n \frac{d_n^{N_n} }{N_n!}
                    \end{aligned}
                    \\[10pt]
                &\boldsymbol{2.)}\ \log(Q) = \log(N!) + \sum_n \begin{aligned}[t]
                        N_n \log(d_n) \\[5pt]
                        - \log(N_n!)
                    \end{aligned}
            \end{aligned} 
            \hspace{18pt} \hspace{18pt} 
            \begin{aligned}
                & \boldsymbol{3.)}\ \frac{\partial G}{\partial N_n} \approx 
                    \begin{aligned}
                        &\log(d_n) - \log(N_n) \\[5pt]
                        &- \alpha - \beta E_n
                    \end{aligned} = 0
                    \\[10pt]
                & \boldsymbol{4.)}\ N_n = \frac{d_n}{e^{\beta E_n + \alpha}}
            \end{aligned} 
            \right. 
        \\[20pt]
    & \text{Fermion}
        &   
        & \left\{ 
            \begin{aligned}
                & \boldsymbol{1.)}\ Q(X) = \prod_n 
                    \left( \begin{matrix} 
                        d_n \\ 
                        N_n 
                    \end{matrix}\right)
                    \\[10pt]
                & \boldsymbol{2.)}\ \log(Q) = \sum_n \begin{aligned}[t]
                        & \log(d_n!) - \log(N_n!) \\[5pt]
                        & - \log[(d_n - N_n)!]
                    \end{aligned}
            \end{aligned} 
            \hspace{30pt} 
            \begin{aligned}
                & \boldsymbol{3.)}\ \frac{\partial G}{\partial N_n} \approx \begin{aligned}
                        & - \log(N_n) + \log(d_n - N_n) \\[5pt]
                        & - \alpha - \beta E_n 
                    \end{aligned} = 0
                    \\[10pt] 
                & \boldsymbol{4.)}\ N_n = \frac{d_n}{e^{\beta E_n + \alpha} + 1}
            \end{aligned} 
            \right.
        \\[20pt]
    & \text{Boson}
        &     
        & \left\{ 
            \begin{aligned}
                &\boldsymbol{1.)}\ Q(X) = \prod_n 
                    \left( \begin{matrix} 
                        N_n + d_n - 1 \\ 
                        N_n 
                    \end{matrix}\right)
                    \\[10pt]
                &\boldsymbol{2.)}\ \log(Q) = \sum_n \begin{aligned}[t]
                        &\log[(N_n + d_n - 1)!] \\[5pt]
                        &- \log(N_n!) \\[5pt]
                        &- \log[(d_n - 1)!]
                    \end{aligned}
            \end{aligned} 
            \hspace{25pt} 
            \begin{aligned}
                &\boldsymbol{3.)}\ \frac{\partial G}{\partial N_n} \approx \begin{aligned}
                        &\log(N_n + d_n - 1) - \log(N_n) \\[5pt]
                        &- \alpha - \beta E_n 
                    \end{aligned} = 0
                    \\[10pt]
                &\boldsymbol{4.)}\ N_n = \frac{d_n - 1}{e^{\beta E_n + \alpha} - 1} 
                    \approx \frac{d_n}{e^{\beta E_n + \alpha} - 1}
            \end{aligned} 
            \right.
\end{aligned}\)

%-------------------------------------------------------------------------------------------------------------------
\newpage
Given some substance in thermal equilibrium,
\[\beta = \frac{1}{k_b T} \hspace{18pt} \hspace{18pt} \mu(T) \equiv - \frac{\alpha}{k_b T}\]
where \(\mu\) depends on the situation.

\[\begin{aligned}
    \tfrac{N_n}{d_n}: \hspace{18pt} n(\epsilon) \quad = \quad \begin{cases} 
        \frac{1}{ e^{ (\epsilon - \mu) / k_b T } }       & \hspace{18pt} \text{Maxwell-Boltzmann}\\[10pt]
        \frac{1}{ e^{ (\epsilon - \mu) / k_b T } + 1}    & \hspace{18pt} \text{Fermi-Dirac} \\[10pt]
        \frac{1}{ e^{ (\epsilon - \mu) / k_b T } - 1}    & \hspace{18pt} \text{Bose-Einstein}
    \end{cases}
\end{aligned}\]

%-------------------------------------------------------------------------------------------------------------------
%-------------------------------------------------------------------------------------------------------------------
%-------------------------------------------------------------------------------------------------------------------
%-------------------------------------------------------------------------------------------------------------------
% Perturbation Theory
\newpage
\section{Perturbation Theory}
% Expansion of Terms
\(\begin{aligned}
    H^{(0)} \psi_n & = E_n \psi_n \\
    & \downarrow\\
    H \psi_n' & = E_n' \psi_n'\\[5pt]
    \Big[ H^{(0)} + \lambda H^{(1)} \Big] \Big[ \psi_n + \lambda \psi_n^{(1)} + \lambda^2 \psi_n^{(2)} + ... \Big]
        & = \Big[ E_n + \lambda E_n^{(1)} + \lambda^2 E_n^{(2)} + ... \Big]
        \Big[ \psi_n + \lambda \psi_n^{(1)} + \lambda^2 \psi_n^{(2)} + ... \Big]
        \\[10pt]
    \begin{gathered}
        \cancel{ \lambda^0 H^{(0)} \psi_n }\\[5pt]
        + \ \lambda^1 ( H^{(0)} \psi_n^{(1)} + H^{(1)} \psi_n ) \\[5pt]
        + \ \lambda^2 ( H^{(0)} \psi_n^{(2)} + H^{(1)} \psi_n^{(1)}) \\[5pt]
        + \ ...
    \end{gathered} \ \ & = \ \
        \begin{aligned}
            & \cancel{ \lambda^0 E_n \psi_n }\\[5pt]
            + \ & \lambda^1 ( E_n \psi_n^{(1)} + E_n^{(1)} \psi_n ) \\[5pt]
            + \ & \lambda^2 ( E_n \psi_n^{(2)} + E_n^{(1)} \psi_n^{(1)} + E_n^{(2)} \psi_n ) \\[5pt]
            + \ & ...
        \end{aligned} \hspace{18pt}\hspace{18pt} {\scriptstyle (\lambda = 1)}
\end{aligned}\)


% Non-Degenerate
\subsection{Non-Degenerate Theory}

% Finding 1st order approx, E^1 and psi^1
\vspace{10pt} \noindent
\(
    \underline{ E_n^{(1)} , \ \psi_n^{(1)} :} 
    \begin{aligned}[t]
        E_n \psi_n^{(1)} + E_n^{(1)} \psi_n & = H^{(0)} \psi_n^{(1)} + H^{(1)} \psi_n 
            \\[10pt]
        \langle \psi_m | \ \ \ ( - H^{(1)} + E_n^{(1)} ) | \psi_n \rangle 
            & = \langle \psi_m | \ \ \ (H^{(0)} - E_n) | \psi_n^{(1)} \rangle 
            = \sum c_i^{(1)} (E_i - E_n) \langle \psi_m | \psi_i \rangle
            \\[5pt]
        - \langle \psi_m | H^{(1)} | \psi_n \rangle + E_n^{(1)} \langle \psi_m | \psi_n \rangle & = c_m^{(1)} (E_m - E_n)
    \end{aligned}
\)

\vspace{10pt}
\[ \boxed{ E_n^{(1)} = \langle \psi_n | H^{(1)} | \psi_n \rangle } \hspace{18pt} \hspace{18pt} 
    \boxed{ \psi_n^{(1)} = \sum_{m \neq n} \frac{\langle \psi_m | H^{(1)} | \psi_n \rangle}{E_n - E_m} \psi_m
    + (0) \psi_n } \]

% Finding 2nd order approx, E^2   
\vspace{20pt} \noindent
\(
    \underline{ E_n^{(2)} , \ | n^{(2)} \rangle:} 
    \begin{gathered}[t]
        \begin{aligned}[t]
                \begin{gathered}
                        - \langle m^{(0)} | H^{(1)} | n^{(1)} \rangle 
                            + E_n^{(1)} \langle m^{(0)} | n^{(1)} \rangle 
                            \\[5pt]
                        + E_n^{(2)} \langle m^{(0)} | n^{(0)} \rangle 
                    \end{gathered}
                    \ \ & = \ \ \langle m^{(0)} | H^{(0)} - E_n^{(0)} | n^{(2)} \rangle
                    \\[-10pt]
                & = \ \ c_m^{(2)} (E_m - E_n)
            \end{aligned}
            \\[10pt]
        \boxed{ E_n^{(2)} = \sum_{m \neq n} \frac{ \left| \langle m | H^{(1)} | n \rangle \right|^2 }{E_n - E_m} }
            = \langle n | H^{(1)} | n^{(1)} \rangle     
            \hspace{10pt} , \hspace{10pt} 
            \boxed{ 
                | n^{(2)} \rangle = \sum_{m \neq n} \frac{ 
                    \langle m | H^{(1)} - E_n^{(1)} | n^{(1)} \rangle 
                }{E_n - E_m} \cdot | m \rangle 
            }
    \end{gathered}
\)

% i+1 th order approx
\vspace{30pt}\noindent
\(
    \underline{ E_n^{(i+1)},\ | n^{(i+1)} \rangle: } \hspace{20pt}
    \begin{aligned}[t]
        E_n^{(i+1)} & = \langle n | H^{(1)} | n^{(i)} \rangle\\[5pt]
        | n^{(i+1)} \rangle & = \sum_{m \neq n} \frac{ 
                \langle m | H^{(1)} | n^{(i)} \rangle 
                - \sum_{j=0}^{i} E_n^{(j+1)} \langle m | n^{(i-j)} \rangle 
            }{E_n-E_m}
            \cdot | m \rangle 
    \end{aligned}
\)

%--------------------------------------------------------------------------------------------------------------------------
% Degenerate Perturbation
\newpage
\subsection{Degenerate Perturbation Theory {\scriptsize (see Matrix Operators)}}

% Degenerate State
\vspace{5pt} \noindent 
\(\begin{aligned}
    \Psi &= \sum_i \left( c_i^{(\psi)}{\scriptstyle[\Psi]} \right) \psi_i \\
    &\equiv \sum_i c_i^{(\psi)} \psi_i \\
    &= c_0^{(\psi)} \psi_0 + c_1^{(\psi)} \psi_1 + ...
\end{aligned}\) 
\hspace{18pt} , \hspace{18pt}
\(\forall \psi_i: \hspace{18pt} \begin{aligned}
    &\bullet \ H^{(0)} \psi_i = E_n \psi_i 
        \hspace{18pt} \underline{ \text{\scriptsize(\(\psi_n\) are degenerate eigenfunctions of \(H^{(0)}\))} }\\[5pt]
    &\bullet \ \langle \psi_i | \psi_j \rangle = \delta_{ij} \\[5pt]
    &\bullet \ \langle \psi_i | \hat{Q} | \psi_j \rangle \equiv Q_{ij}    
\end{aligned}\)

\vspace{5pt}
\noindent \hrulefill
\vspace{5pt}

\vspace{5pt} \noindent
\(\begin{aligned}
    E_n \Psi^{(1)} + E^{(1)} \Psi &= H^{(0)} \Psi^{(1)} + H^{(1)} \Psi
        \hspace{18pt} \text{\scriptsize(first order)} 
        \\[10pt]
    \cancel{ E_n \langle \psi_i | \Psi^{(1)} \rangle } + E^{(1)} \langle \psi_i | \Psi \rangle 
        &= \cancel{ \langle H^{(0)} \psi_i | \Psi^{(1)} \rangle } + \langle \psi_i | H^{(1)} | \Psi \rangle
        \\[5pt]
    &= \langle \psi_i | H^{(1)} | c_0 \psi_0 + c_1 \psi_1 + ... \rangle 
        \\[5pt]
    c_i E^{(1)}
        &= c_0 \langle \psi_i | H^{(1)} | \psi_0 \rangle + c_1 \langle \psi_i | H^{(1)} | \psi_1 \rangle + ...
\end{aligned}\) 

\vspace{2pt}
\(
    E^{(1)} 
    \left(\begin{matrix} 
        c_0{\scriptstyle[\Psi]} \\
        c_1{\scriptstyle[\Psi]} \\
        \vdots
    \end{matrix}\right)^{(\psi)}
    = 
    \left(\begin{matrix} 
        H^{(1)}_{0 0} & H^{(1)}_{0 1} & ...\\
        H^{(1)}_{1 0} & H^{(1)}_{1 1} & ...\\
        \vdots & \vdots
    \end{matrix}\right)^{(\psi)}
    \left(\begin{matrix} 
        c_0{\scriptstyle[\Psi]} \\
        c_1{\scriptstyle[\Psi]} \\
        \vdots
    \end{matrix}\right)^{(\psi)} 
    \ \ \Rightarrow \ \ \boxed{
        \begin{gathered}[b]
            \text{\scriptsize(solve for \(E^{(1)}, \vec{c}^{\ (\psi)}{\scriptstyle[\Psi]}\))}\\
            \left\Vert\begin{matrix} 
                H^{(1)}_{a a} - E^{(1)} & H^{(1)}_{a b}               & ...\\
                H^{(1)}_{b a}             & H^{(1)}_{b b} - E^{(1)}   & ...\\
                \vdots  & \vdots
            \end{matrix}\right\Vert
        \end{gathered} = 0 
    } 
\)

\vspace{10pt}\noindent
In general, \\[10pt]
\(\begin{aligned}
    E_i^{(1)} \vec{c}^{\ (\psi)}{\scriptstyle[\Psi_i]} \ \ 
        &= \ \ \overline{H^{(1)}}^{\ (\psi)} \ \vec{c}^{\ (\psi)}{\scriptstyle[\Psi_i]} 
        \hspace{18pt}\hspace{18pt} \text{\scriptsize(\(i\)th eigen-)}
        \\[5pt]
    E_i^{(1)}
        \left(\begin{matrix} 
            |\\
            \vec{c}\ {\scriptstyle[\Psi_i]}\\
            |    
        \end{matrix}\right)^{(\psi)}
        &= \left(\begin{matrix} 
            |   & |     \\
            \vec{c}\ {\scriptstyle[\Psi_i]} & \vec{c}\ {\scriptstyle[\Psi_i]} & ...\\
            |   & |     
        \end{matrix}\right)^{(\psi)}
        \left(\begin{matrix} 
            E_0^{(1)} & 0 & ...\\
            0 & E_1^{(1)} & ...\\
            \vdots & \vdots
        \end{matrix}\right)
        \left(\begin{matrix} 
            -   & \vec{c}^{\ *}{\scriptstyle[\Psi_i]} & -\\
            -   & \vec{c}^{\ *}{\scriptstyle[\Psi_i]} & -\\
                & \vdots    & 
        \end{matrix}\right)^{(\psi)}
        \left(\begin{matrix} 
            |\\
            \vec{c}\ {\scriptstyle[\Psi_i]}\\
            |    
        \end{matrix}\right)^{(\psi)}
\end{aligned}\)

\vspace{10pt} \noindent
Instead of solving the characteristic polynomial, it would be wise to choose a basis \(\{\psi\}\) such that
\(\vec{c}^{\ (\psi)}{\scriptstyle[\Psi_i]} = ( ... 0\ 0\ 1_{(i)}\ 0\ 0\ ...)^T \ \Leftrightarrow \ \Psi_i = \psi_i\),
making \(\overline{H^{(1)}}^{(\psi)}\) diagonal with eigenvalue entries. These are the energy 
eigenvalues, \(E_i^{(1)} = \big( H^{(1)} \big)_{ii}^{(\psi)} 
= \langle \psi_i | H^{(1)} | \psi_i \rangle \), which is just like first-order non-Perturbation energy. 
This also means \(|\psi_i\rangle\) are eigenfunctions of \(H^{(1)}\) (see Matrix Operators).

\vspace{10pt} \noindent
\parbox{.3\textwidth}{
    It is best to find a hermitian operator, \(\hat{A}\), that commutes with \(H^{(0)}\) and \(H^{(1)}\),
    whose eigenvalues within the degenerate basis are unique. The corresponding eigenfunctions will be 
    a basis that makes \(H^{(1)}\) diagonal. This will also make them eigenfunctions of \(H^{(1)}\).
}
\hfill
\begin{minipage}{.66\textwidth}
    \(1.\ A = A^\dagger\)\\[5pt]
    \(2.\ [A,H^{(0)}]=0 \ \rightarrow \ \Big\{ \exists \{\Psi\} \ \Big| \
        (A\Psi_n = a_n \Psi_n) , \ (H^{(0)}\Psi_n = E_n \Psi_n) \Big\}\) \\[5pt]
    \(3.\ \{\psi\} \subset \{\Psi\} \ \ \text{s.t} \ \ \forall \psi_i : \ \left\{
        \begin{aligned}
            &\left( H^{(0)} \psi_i = E_n \psi_i \right) , \ \ \text{\scriptsize\(\leftarrow\) degenerate}\\[3pt]
            &\left( A \psi_i = a_i \psi_i | \right) , \ \
            \left( \forall{\scriptstyle(i \neq j)} \ a_i \neq a_j \right)            
        \end{aligned} \right.
    \)\\[7pt]
    \(4.\ [A,H^{(1)}]=0 \ \Rightarrow \ \begin{aligned}[t]
            0 &= \langle A \psi_i | H^{(1)} | \psi_j \rangle - \langle \psi_i | H^{(1)} | A \psi_j \rangle\\
            0 &= (a_i - a_j) H^{(1)}_{ij}\\
            0 &= H^{(1)}_{ij} \ \ \ \ \checkedbox
        \end{aligned}\)
\end{minipage}

%------------------------------------------------------------------------------------------------------------------------
%
%
%------------------------------------------------------------------------------------------------------------------------
% Apply Perturbation Theory to Hydrogen Atom Corrections
\newpage
\subsection{Hydrogen Energy Corrections}
% Fine Structure Correction
\subsubsection{Fine Structure - \(\alpha^4 mc^2\)}
{\scriptsize The Dirac Equation can derive the total fine structure correction with a \(\alpha^4\) order approx.}

\vspace{15pt}\noindent
% Relativistic
\underline{1. Relativistic}, \ \(\boldsymbol{\hat{p}^4}\)
\begin{align*}
    T & = \sqrt{p^2c^2 + m^2c^4} - mc^2 
        = mc^2 \sqrt{1 + \tfrac{p^2}{m^2c^2}} - mc^2
        \\[5pt]
    & = mc^2 \left[
            \frac{ (\tfrac{1}{2}) }{1!} \left( \frac{p^2}{m^2c^2} \right) 
            + \frac{ (\tfrac{1}{2}) (1 - \tfrac{1}{2}) }{2!} \left( \frac{p^2}{m^2c^2} \right)^2  + ...
        \right]
        \\[5pt]
    & = \frac{p^2}{2m} - \frac{p^4}{8 m^3 c^2} + ...\\
    & \hs \downarrow\\
    H^{(1)}_r & = - \frac{p^4}{8 m^3 c^2} 
        \hspace{18pt} \hspace{18pt}
        \parbox{.5\textwidth}{
            \scriptsize(For some reason \(\hat{p}^4\) needs to be hermitian to use perturbation theory. 
            It only isn't when \(l=0\), while \(\hat{p}^2\) always is hermitian. See Prob. 6.15)    
        }
\end{align*}

\vspace{5pt} \noindent
\(L^2\) and \(L_z\) should commute with \(p^4\) because the perturbation is spherically symmetric, meaning 
\(l\) and \(m_l\) should be conserved (see Operator Evolution). Their eigenvalues are also distinct 
(taking the eigenfunctions of \(nlm_l\) together) within each set of \(n^2\) degeneracies, so their eigenvectors and 
eigenvalues can be used. \(n, l\) and \(m_l\) the ``good'' numbers.

\vspace{10pt}
\(\begin{aligned}
    \langle r^{-1} \rangle &= \tfrac{1}{n^2 a_0^1 }\\[5pt]
    \langle r^{-2} \rangle &= \tfrac{1}{ (l + 1/2) n^3 a_0^2 }
\end{aligned}\)
\hspace{10pt}
\rule[-85pt]{.5pt}{170pt}
\hspace{10pt}
\(\begin{aligned}
    \langle \psi_{nlm_l} | H^{(1)}_r | \psi_{nlm_l} \Big\rangle 
        &= \frac{- 1}{8 m^3 c^2} \langle \psi_{nlm_l} | p^4 | \psi_{nlm_l} \rangle\\[5pt]
    &= \frac{- 1}{8 m^3 c^2} \langle p^2 \psi_{nlm_l} | p^2 | \psi_{nlm_l} \rangle\\[5pt]
    &= \frac{- 1}{8 m^3 c^2} \Big\langle \big[ 2m(E_n-V) \big]^2 \Big\rangle\\[5pt]
    &= \frac{- 4m^2}{8 m^3 c^2} \langle E_n^2 -2E_nV + V^2 \rangle\\[5pt]
    &= - \frac{E_n^2}{2mc^2} \left[ \frac{4n}{l + 1/2} - 3\right]
\end{aligned}\)

% Spin Orbit Coupling
\vspace{20pt}
\noindent
\underline{2. Spin-Orbit Coupling}, \ \(\boldsymbol{S_e \dotP L_e}\)

\vspace{10pt} \noindent 
{\scriptsize In the electron's frame of reference, the proton is spinning around it, creating a \(B\)-field 
affecting its magnetic dipole moment. The non-inertial reference frame requires multiplying by the Thomas 
procession correction, which in this case is \(C_T = g_e-1 = 1/2\). In the lab frame, the moving electron's 
magnetic dipole moment creates an electric dipole moment, which is affected by the proton charge. The latter
is much harder to calculate.}

%-----------------------------------------------------------------------------------------------------------------------
\begin{center}
    \(\begin{aligned}
        H_{so}^{(1)} & = - C_T \ \mu_e \dotP B(L_e)
            \hspace{1cm} \text{\scriptsize(See Electron in Magnetic Field)}
            \\[5pt]
        & = \tfrac{1}{2} \tfrac{qS}{m} \dotP \tfrac{k_\mu}{r^3} \mss{\int} I d\vec{\hs l} \times \vec{\hs r}
            \hspace{15pt} \left( 
                \ \sim\ \tfrac{1}{2} \tfrac{qS}{m} \dotP \tfrac{k_\mu}{r^3} 
                \mss{\int} \tfrac{ m q d\vec{\hs v} \times \vec{\hs r} }{m} 
            \right)
            \\[5pt]
        & = \tfrac{1}{2} \tfrac{qS}{m} \dotP \tfrac{k_\epsilon}{c^2} \tfrac{2\pi}{r} I
            = \tfrac{1}{2} \tfrac{qS}{m} \dotP \tfrac{k_\epsilon}{c^2} \tfrac{2\pi}{r} \tfrac{q (L/mr)}{2\pi r} 
            % = \frac{kqq}{2} \frac{1}{m^2 c^2 r^3} S_e \dotP L_p 
            \\[5pt]
        & = \frac{kqq}{2m} \frac{1}{m c^2} \frac{S \dotP L}{r^3} 
            = \frac{e^2}{8 \pi \epsilon_0 m^2 c^2} \frac{S \dotP L}{r^3} 
    \end{aligned}\)
\end{center}

\vspace{5pt} \noindent
\(S \dotP L\) does not commute with \(L\) or \(S\) (meaning \(m_l\) and \(m_s\) are bad), but %
\([S \dotP L, S^2] = [S \dotP L, L^2] = 0\). The sum of the two, \(J \equiv L + S\), and %
\(J^2\) also commute with the perturbation. They are all conserved, and their unique %
eigenvalues per set of degeneracies - \(l, s{\scriptstyle=1/2}, j, m_j\) - are the ``good'' %
numbers (along with \(n\)).

\vspace{10pt}\noindent
\(\begin{aligned}
    S \dotP L & = \frac{1}{2} \left( J^2 - L^2 - S^2\right)\\[5pt]
    \langle r^{-3} \rangle & = \frac{1}{l(l+1/2)(l+1)n^3a_0^3}\\[5pt]
    & \text{\scriptsize(note: divergent at \(l=0\))}
\end{aligned}\)
\hfill
\vline
\hfill
\(\begin{aligned}
    \langle {\scriptstyle nljm_j} | H^{(1)}_{so} | {\scriptstyle nljm_j} \rangle 
        &= \frac{kqq}{2m} \frac{1}{m c^2} \frac{\hbar^2 [ j(j+1) - l(l+1) - s(s+1) ]}{ 2 l(l+1/2)(l+1) n^3 a_0^3} \\[5pt]
    &= \frac{kqq}{4 m n^4} \frac{\hbar^2 \alpha^3 m^3 c^3}{\hbar^3 m c^2} 
        \frac{n [ j(j+1) - l(l+1) - s(s+1) ]}{l(l+1/2)(l+1)}\\[5pt]
    &= \frac{kqq}{4 \hbar c n^4} \frac{\alpha^3 m^2 c^4}{m c^2} 
        \frac{n [ j(j+1) - l(l+1) - s(s+1) ]}{l(l+1/2)(l+1)}\\[5pt]
    &= \frac{E_n^2}{mc^2} \left\{ \frac{ n \left[ j(j+1) - l(l+1) - 3/4 \right] }{ l (l+1/2) (l+1) } \right\}
\end{aligned}\)

% Darwin Term
\vspace{20pt}\noindent
\underline{3. Darwin Term (correction for \(H_{so}^{(1)}\) when \(l=0\))} skipped

% Total Correction
\vspace{20pt}\noindent
\underline{4. Total Correction}

\vspace{15pt}\noindent
\begin{minipage}{.49\textwidth}
    \(\begin{aligned}
        E^{(1)}_{fs} &= E^{(1)}_r + E^{(1)}_{so} \\[5pt]
        &= - \frac{E_n^2}{2mc^2} \left[ \frac{4n}{l + \tfrac{1}{2}} - 3\right] + \frac{E_n^2}{mc^2} 
            \left\{ \frac{ n \left[ j(j+1) - l(l+1) - 3/4 \right] }{ l (l+1/2) (l+1) } \right\}\\[5pt]
        &= \frac{E_n^2}{2mc^2} \left( 3 - \frac{4n}{j + 1/2} \right) 
            \hspace{20pt} {\scriptstyle(j = l \pm 1/2)}\\
        &\downarrow\\
        E_{nj} &= E_n + E^{(1)}_{fs}\\[5pt]
        &= E_n \left[ 1 
            - \frac{E_n}{2mc^2} \left( \frac{4n}{j + 1/2} - 3 \right) \right]\\[5pt]
        &= -\frac{\alpha^2 mc^2}{2n^2} \left[ 1 
            + \frac{\alpha^2}{n^2} \left( \frac{n}{j + 1/2} - 3/4 \right) \right]
    \end{aligned}\) 
\end{minipage}
\begin{minipage}{.48\textwidth}
    \vspace{3cm}
    Fine structure splits the \(l\) energy degeneracies. However, since \(j = l \pm 1/2\), there are still two \(j\)
    degeneracies if \(n>2\). Overall, the good numbers to use for stationary state
    solutions to the hydrogen atom w/ fine structure correction are \(n, l, s{\scriptstyle=1/2}, j, m_j\). Note, 
    \(J^2, L^2, \text{ and } S^2\) always commute(?)
\end{minipage}


%------------------------------------------------------------------------------------------------------------------------
% Zeeman Effect
\newpage
\subsubsection{Zeeman Effect (Ext. \(B\)-Field)}
\begin{align*}
    H_B^{(1)} &= - (\mu_s + \mu_l) \dotP B_\text{ext}
        \hspace{18pt}\hspace{18pt} \text{\scriptsize(see Electron in Magnetic Field)}\\[5pt]
    &= - \left( \frac{g_e q}{2m}S + \frac{q}{2m}L \right) \dotP B_\text{ext}\\[5pt]
    &= \frac{e}{2m} \left( 2S + L \right) \dotP B_\text{ext}
\end{align*}

% Weak Zeeman
\vspace{5pt} \noindent
\underline{Weak Zeeman (\(B_\text{ext} \ll B_\text{int}\))}

\vspace{-10pt}
\begin{align*}
    H_{WZ}^{(1)} &= \frac{e}{2m} B_\text{ext} \dotP (2S + L)  \\[5pt]
    &= \frac{e}{2m} B_\text{ext} \dotP ( J + S )
\end{align*}

\vspace{5pt}\noindent
Fine structure perturbation dominate the Zeeman perturbation, so the fine structure numbers are the good ones: %
\(n,l,s{\scriptstyle=1/2},j\), and \(m_j\). \(m_l\) and \(m_s\) can't be used for \(\langle L \rangle\) or %
\(\langle S \rangle\), so instead use the fact that %
the ``vector'' \(J = L+S\) is conserved, so a \textbf{time-averaged} \(S\)-component to the \(J\) ``vector'' can be defined as %
\(S_\text{ave} = \frac{S \dotP J}{J^2} J\), where \(S \dotP J = \frac{1}{2} \left( J^2 + S^2 - L^2 \right)\).

\vspace{-10pt}
\begin{align*}
    E^{(1)}_\text{WZ} &= \frac{e}{2m} B_\text{ext} \dotP 
        \langle {\scriptstyle nljm_j} | J + S_\text{ave} | {\scriptstyle nljm_j}\rangle\\[5pt]
    &= \frac{e}{2m} B_\text{ext} \dotP 
        \left\langle J \left( 1 + \frac{S \dotP J}{J^2} \right) \right\rangle\\[5pt]
    &= \frac{e}{2m} B_\text{ext} \dotP \langle J \rangle 
        \left(1 + \frac{ j(j+1) - l(l+1) + 3/4 }{ 2 j (j+1) }\right)\\[5pt]
    &= \frac{e\hbar}{2m} B_\text{ext} m_j
        \left(1 + \frac{ j(j+1) - l(l+1) + 3/4 }{ 2 j (j+1) }\right)
        \hspace{18pt} \hspace{18pt} \text{\scriptsize(let \(B_\text{ext}\) be parallel to the z-axis)}\\[5pt]
    &= \mu_B B_\text{ext} m_j g_j \hspace{18pt} \hspace{18pt} \begin{aligned}
            &{\scriptstyle \mu_B = \text{Bohr magneton} = 5.788 \times 10^{-5} \ \text{ev/T}}\\
            &{\scriptstyle g_j = \text{Lande g-factor}}\\
        \end{aligned}
\end{align*}

% Strong Zeeman
\vspace{10pt}\noindent
\underline{Strong Zeeman (\(B_\text{ext} \gg B_\text{int}\))}\\[5pt]
For a strong magnetic field parallel to the z-axis, \(m_l\) and \(m_s\) are stuck in the same place, 
making them and \(l\) conserved. The external torque, however, means that the total angular momentums, 
\(j\) and \(m_j\) are not. Though unneeded, obviously \(s{\scriptstyle=1/2}\). 

\begin{center} \(\begin{aligned}
    E_{SZ}^{(1)} &= \frac{e}{2m} B_\text{ext} \langle 2S_z + L_z \rangle\\[5pt]
    &= \mu_B B_\text{ext} (2m_s + m_l)
\end{aligned}\) \end{center}

\noindent
The spin-orbit correction must be changed with respect to the new good numbers, \(m_l\) and \(m_s\). 
The relativistic correction uses the same numbers, so it stays the same.

\vspace{15pt}\noindent
\(
    \begin{aligned}[t]
        E_\text{so}^{(1)} &= \frac{e^2}{8 \pi \epsilon_0 m^2 c^2} 
            \left\langle \frac{S_x L_x + S_y L_z + S_z L_z}{r^3} \right\rangle \\[5pt]
        &= \frac{e^2}{8 \pi \epsilon_0 m^2 c^2} \frac{0 + 0 + \hbar^2 m_s m_l}{l (l+1/2) (l+1) n^3 a_0^3}\\[15pt]
        &= \frac{kqq}{2 m^2 c^2} \frac{\hbar^2}{(\hbar / \alpha m c)^3 n^3}\frac{m_s m_l}{l (l+1/2) (l+1)}\\[5pt]
        &= \frac{kqq}{2\hbar c} \frac{\alpha^3 m^2 c^4}{4 mc^2 n^4}\frac{4n m_s m_l}{l (l+1/2) (l+1)}\\[5pt]
        &= \frac{E_n^2}{2mc^2} \frac{4n m_s m_l}{l (l+1/2) (l+1)}\\
    \end{aligned}
    \parbox[t]{18pt}{ \vspace{40pt}
        \( \ \rightarrow\ \)
    } 
    \parbox[t]{9.5cm}{
        \(\begin{aligned}[t]
            E_\text{fs}^{(1)} &= E_\text{so}^{(1)} + E_\text{r}^{(1)}\\[5pt]
            &= \frac{E_n^2}{2mc^2} \frac{4n m_s m_l}{l (l+1/2) (l+1)}
                + \frac{E_n^2}{2mc^2} \left[ 3 - \frac{4n}{l + 1/2} \right]\\[5pt]
            &= \frac{4n E_n^2}{2mc^2} \left[ \frac{m_s m_l}{l (l+1/2) (l+1)} 
                + \frac{3}{4n} - \frac{1}{l + 1/2} \right]\\
            &\hs \downarrow
        \end{aligned}\)

        \(E_{nlm_lm_s} = E_n + E_\text{SZ}^{(1)} + E_\text{fs}^{(1)}\)
    }
\)

% Intermediate Zeeman
\vspace{25pt}\noindent
\underline{Intermediate Zeeman (\(B_\text{ext} \sim B_\text{int}\))}\\[10pt]
There are no good numbers here (see Degenerate Perturbation Theory). The basis is chosen to be 
\(|j \ m_j \rangle = \sum_i C_i |l\ m_l \rangle \otimes |s\ m_s \rangle\) (see 2 Objects w/ Any Spin), as it makes
\(\overline{H^{(1)}}^{(e)}\) easier (instead of using \(l, m_l, m_s\)).

\begin{center}
    \(\begin{aligned}
        &1.)\ \psi_i = |j\ m_{j}\rangle_i&
            &2.)\ \Big( \langle l\ m_l | \langle s\ m_s | \Big)_x 
            \Big( |l\ m_l \rangle |s\ m_s \rangle \Big)_y = \delta_{xy}\\[5pt]
        &3.)\ Q_{rc}^{(\psi)} = \langle \psi_r | \hat{Q} | \psi_c \rangle& \hspace{1.5cm}
            &4.)\ \psi_i \ \ \text{s.t.}\ \ \left\{\begin{aligned}
                &0 \leq l < n\\[5pt]
                &j_{(l\pm)} = l \pm 1/2, \\[5pt]
                & 2l^2 < i \leq 2(l {\scriptstyle+} 1)^2
            \end{aligned}\right.
    \end{aligned}\)
\end{center}

\vspace{10pt}\noindent
\(\begin{aligned}
    \begin{aligned}
        \langle {\scriptstyle jm_j} | H^{(1)}_{fs} | {\scriptstyle jm_j} \rangle 
            &= \frac{E_n^2}{2mc^2} \left( 3 - \frac{4n}{j + 1/2} \right)\\[5pt]
        &\equiv \gamma_n \left( 3 - \frac{4n}{j + 1/2} \right)    
    \end{aligned}\\[15pt]
    \begin{aligned}
        \langle {\scriptstyle jm_j} | H^{(1)}_{IZ} | {\scriptstyle jm_j} \rangle 
        &= \langle {\scriptstyle jm_j} | H^{(1)}_{IZ} 
            \Big( C_i | {\scriptstyle lm_l} \rangle \otimes | {\scriptstyle sm_s} \rangle \Big)\\
        &= \mu_B B_\text{ext} (2m_s + m_l) C_i^2\\
        &\equiv \beta (2m_s + m_l) C_i^2
    \end{aligned}
\end{aligned}\)
\hspace{5pt}
\rule[-75pt]{.5pt}{150pt}
\hspace{5pt}
\(\begin{gathered}
    \begin{aligned}
        &\overline{H^{(1)}}^{(jm_j)} = \overline{H^{(1)}_{fs}}^{(jm_j)} + \overline{H^{(1)}_{IZ}}^{(jm_j)}\\[10pt]
        &\text{See Griffith Prob. 6.25 for example with \(n=2\)}
    \end{aligned}
\end{gathered}\)

% Stark Effect
\newpage
\subsubsection{Stark Effect (Small Ext. \(E\)-Field)}

\(\begin{aligned}
    &\bullet \ H^{(1)} = - p \dotP E = eE \dotP r \hspace{18pt} \hspace{18pt} \text{\scriptsize(small r)}\\[5pt]
    &\bullet \ n=1 \ \rightarrow \ H^{(1)} = 0\\[5pt]
    &\bullet \ n=2 \ \rightarrow \ \begin{cases} 
            H^{(1)} = 0             & \hspace{18pt} m = \pm 1\\[5pt]
            H^{(1)} = k e |E| a_0   & \hspace{18pt} m = 0 
                \hspace{18pt} \hspace{18pt} \text{\scriptsize(\(k\) is some constant)}
        \end{cases}
\end{aligned}\)


% Lamb Shift
\subsubsection{Lamb Shift (quantitized \(E\)-field) - \(\alpha^5 mc^2\) (skipped)}

% Hyperfine
\subsubsection{Hyperfine (Spin-Spin), \(\boldsymbol{S_p \dotP S_e}\) - \(m/m_p \ \alpha^4 mc^2\)}

(Coupling between the electron magnetic moment and the magnetic field from the proton magnetic moment)

\(\begin{aligned}
    &\mu_e = - \frac{g_e e}{2m_e} S_e = - \frac{e}{m_e} S_e, 
        \hspace{1cm} \mu_p = \frac{g_p e}{2m_p} S_p\\[5pt]
    &B(\mu_p) = \frac{\mu_0}{4\pi r^3} [3( \vec{\mu_p} \dotP \hat{r} ) \hat{r} - \vec{\mu_p} ] 
        + \frac{2\mu_0}{3} \vec{\mu_p} \delta^3(r)
\end{aligned}\)
\hspace{10pt}
\rule[-43pt]{.5pt}{90pt}
\hspace{10pt}
\(\begin{aligned}
    H^{(1)}_{hf} &= -\mu_e \dotP B(\mu_p)\\
    &= ...\\
    &\downarrow\\
    E^{(1)}_{hf} &= \left(\frac{e}{m_e} \right) \left( \frac{2\mu_0}{3} \frac{g_p e}{2m_p} \right) 
        \langle S_e \dotP S_p \rangle \big| \psi_{nlm}(0) \big|^2
\end{aligned}\)

\vspace{20pt} \noindent
In the ground state, \(\big| \psi_{100}(0) \big|^2 = 1/(\pi a_0^3)\). \(S_e^2, S_p^2\), and the 
sum \(S=S_e + S_p\) commute with \(S_e \dotP S_p\), so \(s_e, s_p, m_s, s^2\) are the good numbers.
\(S_e\) and \(S_p\) do not, so \(m_{se}\) and \(m_{sp}\) are not good numbers.

\vspace{20pt}
\(\begin{aligned}
    E^{(1)}_{hf} &= \left(\frac{e}{m_e} \right) \left( \frac{2}{3\epsilon_0 c^2} \frac{g_p e}{2m_p} \right) 
        \frac{1}{2 \pi a_0^3} \langle S^2 - S^2_e - S^2_p \rangle \\[5pt]
    &= \frac{g_p e^2}{4 \pi \epsilon_0 c^2 m_p m_e} 
        \frac{4 \alpha^3 m_e^3 c^3 \hbar^2 }{3 \hbar^3} \left[ \frac{s(s+1)}{2} - 3/4 \right]\\[5pt]
    &= \frac{4}{3} g_p \frac{m_e}{m_p} \alpha^4 m_e c^2 \left[ \frac{s(s+1)}{2} - 3/4 \right]\\[5pt]
    &= \frac{4}{3} g_p \frac{m_e}{m_p} \alpha^4 m_e c^2 \cdot 
        \begin{cases}
            \frac{1}{4}     &   \hspace{18pt} s=1 \ \ (\text{triplet})\\
            \frac{-3}{4}    &   \hspace{18pt} s=0 \ \ (\text{singlet})
        \end{cases} \hspace{18pt} \rightarrow \hspace{18pt} \begin{aligned}
            &\Delta E = 5.88 \times 10^{-6} \text{\ eV} \\[5pt]
            &\lambda = 21 \text{\ cm}, \ \ \ \nu = 1420 \text{\ MHz}
        \end{aligned}
\end{aligned}\)

%------------------------------------------------------------------------------------------------------------------------
%
%
%------------------------------------------------------------------------------------------------------------------------
% Transition Amplitude
\newpage
\subsection{Transition Amplitude (See Pictures)}

\noindent
\(
    \begin{aligned}
        H \mss{(t)} & = H^0 + H^1 \mss{(t)}
            \\[-5pt]
        & \hs \downarrow
            \\[-7pt]
        U \mss{(t)} | i^0 \rangle & 
            = \mss{ \sum_n } \ | n^0 \rangle e^{- \tfrac{i}{\hbar} E_n^0 t} \hs \langle n^0 | U_I | i^0 \rangle
            \\
        | \Psi \mss{(t)} \rangle & = \boxed{ \mss{ \sum_n } \ | n^0 \rangle e^{- \tfrac{i}{\hbar} E_n^0 t} \hs d_n \mss{(t)} }
            \\[-5pt]
        & \hs \downarrow
            \\
        0 & = \langle f^0 | i\hbar \tfrac{\partial}{\partial t} - H^0 - H^1 \mss{(t)} | \Psi \mss{(t)} \rangle\\
        & = \mss{ \sum_n } \langle f^0 | 
            \left[ i\hbar \dot{d_n} - H^1 \mss{(t)} \hs d_n \right] 
            | n^0 \rangle
            e^{- \tfrac{i}{\hbar} E_n^0 t} 
            \\[-7pt]
        & \hs \downarrow \\[-7pt]
        \dot{d_f} \mss{(t)} & = \mss{ \sum_n } \hs \tfrac{1}{i\hbar} 
            \langle f^0 | H^1 \mss{(t)} | n^0 \rangle 
            \hs e^{\tfrac{i}{\hbar} (E_f^0 - E_n^0) t} 
            \hs d_n \mss{(t)}
            \\
        & = \mss{ \sum_n } \hs \tfrac{1}{i\hbar} 
            \langle f^0 | H^1 \mss{(t)} | n^0 \rangle 
            \hs e^{i \omega_{fn} t} 
            \hs d_n \mss{(t)}
    \end{aligned}
    \hfill \vline \hfill
    \begin{aligned}
        & d_n \mss{(t)} = d_n \mss{(0)} + \mss{ \int_0^t } \hs \dot{d_n} \hs\hs dt' :
            \\[5pt]
        & \bullet\ d_n \mss{(0)} \hs\hs = \hs\hs \delta_{ni}
            \hspace{20pt} \text{\scriptsize(if \(|d_{n \neq i} \mss{(t)}| \ll 1\))}
            \hspace{15pt} \text{\scriptsize(0\(^\text{th}\) order)}
            \\[5pt]
        & \bullet\ \dot{d_f} \mss{(t)} \hs\hs \approx \hs \tfrac{1}{i\hbar} 
            \langle f^0 | H^1 \mss{(t)} | i^0 \rangle 
            \hs e^{i \omega_{fi} t} 
            \\[5pt]
        & \bullet\ \boxed{ 
                d_n \mss{(t)} \hs\hs \approx \hs \delta_{ni} + \tfrac{1}{i\hbar} 
                \mss{ \int_0^t }
                \langle n^0 | H^1 \mss{(t')} | i^0 \rangle 
                \hs e^{i \omega_{ni} t'} 
                \hs dt'
            }
            \hspace{5pt} \big( \hs \begin{gathered}
                \text{\scriptsize 1\(^\text{st}\)}\\[-9pt]
                \text{\scriptsize order}
            \end{gathered} \hs \big)
            \\[5pt]
        & \bullet\ \begin{aligned}[t]
                \dot{d_f} \mss{(t)} \hs\hs \approx & \
                    \tfrac{1}{i\hbar} 
                    \overline{ H^1_{fi} } \mss{(t)} 
                    \hs e^{i \omega_{fi} t} 
                    \\
                & + \left( \tfrac{1}{i \hbar} \right)^2 
                    \mss{ \int_{t_0}^t } \hs\hs 
                        \mss{ \sum_n } \hs\hs
                        \overline{ H^1_{fn} } \mss{(t)} 
                        \hs e^{i \omega_{fn} t} 
                        \hs \overline{ H^1_{ni} } \mss{(t')} 
                        \hs e^{i \omega_{ni} t'} 
                    \hs dt'
            \end{aligned}
            \\[-5pt]
        & \bullet\ \dots
    \end{aligned}
\)

% Dirac/Interactive Picture Method
\vspace{15pt}\noindent
\underline{Interactive Picture Method}: \\[5pt]
\(
    \begin{aligned}
        % Dirac Picture Propagator
        & U_I \mss{(t,t_0)} = \mathbb{I} 
            + \tfrac{1}{i\hbar} \mss{ \int_{t_0}^t } H_I^1 \mss{(t')} \hs\hs dt'
            + \left( \tfrac{1}{i\hbar} \right)^2 \mss{ \int_{t_0}^t \int_{t_0}^{t'} } 
                H_I^1 \mss{(t')} H_I^1 \mss{(t'')} 
                \hs\hs dt'' dt'
            + \dots
            \\[5pt]
        % Transition amplitude 
        & \bullet\ \begin{aligned}[t]
                \langle f^0 | U_I \mss{(t,t_0)} | i^0 \rangle = &
                    \ \langle f^0 | e^{\tfrac{i}{\hbar} E_f^0 (t-t_0)} \hs U \mss{(t,t_0)} | i^0 \rangle 
                    \\[5pt]
                \equiv\ d_f \mss{(t)} = &\ 
                    \boxed{
                        \delta_{fi} 
                        + \tfrac{1}{i\hbar} 
                        \mss{ \int_{t_0}^t } 
                            \langle f^0 | H^1 \mss{(t')} | i^0 \rangle 
                            \hs e^{i \omega_{fi} (t'-t_0)} 
                        \hs\hs dt' 
                    }
                    \hspace{15pt} \text{\scriptsize(1\(^\text{st}\) order)}
                    \\[5pt]
                & + \left( \tfrac{1}{i\hbar} \right)^2 
                    \mss{ \int_{t_0}^t \int_{t_0}^{t'} \sum_n } 
                        \hs \langle f^0 | H^1 \mss{(t')} | n^0 \rangle
                        \hs e^{i \omega_{fn} (t'-t_0)} 
                        \hs \langle n^0 | H^1 \mss{(t'')} | i^0 \rangle
                        \hs e^{i \omega_{ni} (t''-t_0)} 
                    \hs\hs dt'' dt'
                    + \dots
            \end{aligned}
    \end{aligned}
\)

% Schrodinger Propagator 
\vspace{15pt}\noindent
\underline{Normal Schrodinger Propagator}: \\[5pt]
\(
    \begin{aligned}
        % Schrodinger Picture Propagator
        & \begin{aligned}[t]
                U_S \mss{(t,t_0)} = & \ U^0 \mss{(t,t_0)} + \tfrac{1}{i\hbar} \mss{ \int_{t_0}^t } 
                    U^0 \mss{(t,t_0)} \hs 
                    {U^0}^\dagger \mss{(t',t_0)} \hs H^1 \mss{(t')} \hs U^0 \mss{(t',t_0)} 
                    \hs\hs dt'
                    \\
                & + \left( \tfrac{1}{i\hbar} \right)^2 \mss{ \int_{t_0}^t \int_{t_0}^{t'} } 
                    U^0 \mss{(t,t_0)} \hs 
                    {U^0}^\dagger \mss{(t',t_0)} \hs H^1 \mss{(t')} \hs U^0 \mss{(t',t_0)} 
                    {U^0}^\dagger \mss{(t'',t_0)} \hs H^1 \mss{(t'')} \hs U^0 \mss{(t'',t_0)} 
                    \hs\hs dt'' dt'
                    + \dots
            \end{aligned} 
            \\[5pt]
        % Transition Amplitude
        & \bullet\ \begin{aligned}[t]
                \langle f^0 | U & \mss{(t,t_0)} | i^0 \rangle = \ 
                    \boxed{
                        \delta_{fi} \hs e^{- \tfrac{i}{\hbar} E_f^0 (t-t_0)}
                        + \tfrac{1}{i\hbar} \mss{ \int_{t_0}^t } 
                        e^{- \tfrac{i}{\hbar} E_f^0 (t-t')} 
                        \langle f^0 | H^1 \mss{(t')} | i^0 \rangle 
                        e^{- \tfrac{i}{\hbar} E_i^0 (t'-t_0)} 
                        \hs\hs dt'
                    }
                    \hspace{15pt} \text{\scriptsize(1\(^\text{st}\) order)}
                    \\
                & + \left( \tfrac{1}{i\hbar} \right)^2 \mss{ \int_{t_0}^t \int_{t_0}^{t'} } 
                    \mss{\sum_n} \hs
                    e^{- \tfrac{i}{\hbar} E_f^0 (t-t')}
                    \langle f^0 | H^1 \mss{(t')} | n^0 \rangle 
                    e^{- \tfrac{i}{\hbar} E_n^0 (t'-t'')}
                    \langle n^0 | H^1 \mss{(t'')} | i^0 \rangle 
                    e^{- \tfrac{i}{\hbar} E_i^0 (t''-t_0)}
                    \hs\hs dt'' dt'
                    + \dots
            \end{aligned} 
    \end{aligned} 
\)

%----------------------------------------------------------------------------------------------------------------------
%
%
%----------------------------------------------------------------------------------------------------------------------
% Variation Principle
\newpage
\subsection{Variation Principle - Approx. Ground State Energy}

\(
    \begin{aligned}
        \psi = & \sum c_n \psi_n \ \rightarrow \ E(\psi) > E_0 = E(\psi_0)\\[5pt]
        &\psi \equiv f(b,x), \hspace{18pt} \langle H \rangle 
            = \langle T \rangle + \langle V \rangle
    \end{aligned}
    \hspace{15pt} \Rightarrow \hspace{15pt}
    \begin{aligned}
        &b_\text{min}: \ \frac{d}{db} \langle H \rangle = 0 \\[5pt]
        &E_0 \approx 
            \Big\langle f(b_\text{min}, x) \Big| H \Big| f(b_\text{min}, x) \Big\rangle
    \end{aligned}
\)

% Selection Rules
\subsection{Selection Rules - Orbital Transitions}

\noindent
Electric Dipole Approximation ONLY: \(\lambda_\gamma \gg\) atom length \(\rightarrow \ E, B\) feels homogenously oscillating to the atom

\vspace{10pt}\noindent
\(\psi_{nlm} \rightarrow \psi_{n'l'm'}\):

\vspace{10pt}\noindent
\(\begin{aligned}
    \bullet\ & \Delta m \in \{ -1, \cancelto{\mss{?}}{0},1 \} \\
    & \hspace{5pt} s(\gamma) = 1 \ \rightarrow \ m_s(\gamma) \in \{ -\hbar, \cancelto{\mss{?}}{0}, \hbar \} \\
    & \hspace{5pt} E = E\hat{z} \ \rightarrow \ \Delta m = 0
\end{aligned}\)

\vspace{10pt}\noindent
\(\begin{aligned}
    \bullet\ & \Delta l = \pm 1 \\
    & \hspace{5pt} 1s \leftrightarrow 2p \\
    & \hspace{5pt} \text{Exception:} (2s \rightarrow 1s) \text{ through two-photon emission}
\end{aligned}\)

\vspace{10pt}\noindent
\(\begin{aligned}
    \bullet\ & \Delta j \in \{ -1,0,1 \} \\
    & \text{Exception:} (j=0 \rightarrow j=0) \text{ not allowed}
\end{aligned}\)

% Blackbody Radiation
\section{Blackbody Radiation}

\(\begin{aligned}
    &\bullet \ \text{Power Spectrum}: \ \ I'(\omega) = 
        \frac{\hbar^3 \omega^3}{h^2c^2} \frac{1}{e^{\hbar \omega / k_b T} - 1} \ \left[ \tfrac{I}{\Omega \dotP f} \right]
        &&\hspace{7pt} \text{\scriptsize(\mss{\mu = 0} for photons since photon number isnt conserved)}
        \\[5pt]
    &\bullet \ \text{Stefan-Boltzmann Law}: \ \ I = \frac{dP}{dA} \propto T^4 
        &&\hspace{7pt} \text{!! important !!}
        \\[5pt]
    &\bullet \ \text{Wien's Displacement Law}:\ \ \lambda_\text{max} = \frac{ 2.9 \times 10^{-3} }{T}\ [\text{m}]
        &&\hspace{7pt} \text{\scriptsize(mode of spectrum)}
\end{aligned}\)

%----------------------------------------------------------------------------------------------------------------------------------
%----------------------------------------------------------------------------------------------------------------------------------
%----------------------------------------------------------------------------------------------------------------------------------
%----------------------------------------------------------------------------------------------------------------------------------
% Adiabatic Theorem
\newpage
\section{Adiabatic Theorem - Slow Changing of Potential}

% Explanation
\vspace{5pt} \hspace{2cm} \noindent
\(\begin{aligned}
    t = 0 \ \rightarrow \ 
    \begin{aligned}
        H \mss{(t=0)} & = H^{(0)}\\[5pt]
        H \mss{(0)} | n \rangle & = E_n | n \rangle
    \end{aligned}
\end{aligned}\)
\hspace{30pt}
\(\begin{aligned}
    t = t   \ \rightarrow \ \begin{aligned}
        H & = H^{(0)} \mss{(t)}\\[5pt]
        H \mss{(t)} | n \mss{(t)} \rangle & = E_n \mss{(t)} | n \mss{(t)} \rangle
    \end{aligned}
\end{aligned}\)

\vspace{5pt}\noindent
\hrulefill
\vspace{5pt}

% Proof
\noindent
\(
    \begin{gathered}
        \mss{ \boxed{ \text{Dynamic Phase}:\ \theta_n (t) = - \tfrac{1}{\hbar} \int_0^t E_n \mss{(t')} \hs dt' } }
            \\[7pt]
        \begin{aligned}
                | \Psi_m & \mss{(t)} \rangle
                    \equiv \sum_n | n \mss{(t)} \rangle \hs e^{ i \theta_n (t)} \hs \langle n \mss{(t)} | m \mss{(0)} \rangle
                    \\[5pt]
                & \approx | m \mss{(t)} \rangle \hs e^{ i \theta_m (t)} \hs e^{i \gamma_m (t)}
                \\[5pt]
                & = | m \mss{(t)} \rangle \hs e^{ i \theta_m (t)} \hs e^{\tfrac{i}{\hbar} \int A^m \cdot dR}
            \end{aligned}    
    \end{gathered}
    \hfill
    \vline
    \hfill
    \begin{aligned}
        & \begin{aligned}
                & \sum_n \cancel{ H | n \rangle \hs e^{ i \theta_n } \hs c_n }
                    = i \hbar \sum_n | \dot{n} \rangle \hs e^{ i \theta_n } \hs c_n 
                    + \cancel{ | n \rangle \hs i \dot{\theta}_n e^{ i \theta_n } \hs c_n }
                    + | n \rangle \hs e^{ i \theta_n } \hs \dot{c}_n
                    \\
                & \langle m | \dot{H} | n \rangle + \langle m | H | \dot{n} \rangle 
                    = \cancel{ \langle m | \dot{E}_n | n \rangle } + \langle m | E_n | \dot{n} \rangle
            \end{aligned}
            \\[-3pt]
        & \begin{aligned}
                & \hs \Downarrow\\[-3pt]
                \Aboxed{ \dot{c}_m &
                    = \tfrac{d}{dt} \mss{ \langle m \mss{(t)} | m \mss{(0)} \rangle } 
                    = - \langle m | \dot{m} \rangle \hs c_m 
                    - \mss{ \sum_{n \neq m} } \tfrac{\langle m | \dot{H} | n \rangle}{E_n - E_m}
                    \hs e^{i (\theta_n - \theta_m)} \hs \langle n | \Psi \rangle
                    }
                    \\[-2pt]
                &
                    \hs \approx \hs \begin{gathered}[b]
                        \text{\scriptsize(not trivial)}\\[-4pt]
                        - \langle m \mss{(t)} | \dot{m} \mss{(t)} \rangle \hs c_m
                    \end{gathered}
                    \ \ \Rightarrow\ \ c_m \mss{(t)} \hs\approx\hs c_m \mss{(0)} 
                    \hs e^{\tfrac{i}{\hbar} i \hbar \int \langle m | \dot{m} \rangle dt'}
                    \\[5pt]
                c_n \mss{(t)} & \hs\approx\hs \delta_{nm} \hs e^{i \gamma_m (t)}
                    \hspace{15pt} \mss{ \boxed{
                        \text{Berry Phase}:\ 
                        \gamma_m (t) = i \int_0^t \big\langle m \mss{(t')} \big| \dot{m} \mss{(t')} \big\rangle  \hs dt' 
                        \in \mathbb{R}
                    } }
            \end{aligned}
    \end{aligned}  
\)

\vspace{15pt}\noindent
\begin{minipage}[t]{.58\textwidth}
    % Berry Phase
    \underline{Berry/Geometric Phase}\\[10pt]
    \(\begin{aligned}[t]
        & \gamma_m (t)
            = i \int_0^t \langle m \mss{(t')} | \dot{m} \mss{(t')} \rangle \hs dt'
            = \boxed{ \tfrac{1}{\hbar} \int_{R_i}^{R_f} i\hbar \hs \langle m | \nabla_R \hs m \rangle \cdot dR }
            \\[10pt]
        & \arraycolsep=2pt \begin{array}[t]{c c c c}
            \Rightarrow\ & \displaystyle \tfrac{1}{\hbar} \oint i\hbar \hs \langle m | \nabla_R \hs m \rangle \cdot dR 
                & = 
                & \displaystyle \tfrac{1}{\hbar} \iint \nabla_R \times i\hbar \langle m | \nabla_R \hs m \rangle \cdot da
                \\[5pt]
            \sim & \displaystyle \boxed{ \tfrac{1}{\hbar} \oint A^m \cdot dR }
                & = 
                & \displaystyle \tfrac{1}{\hbar} \iint \nabla_R \times A^m \cdot da
                    = \tfrac{1}{\hbar} \hs \Phi_{B}^m
        \end{array}
    \end{aligned}\)    
\end{minipage}
\hfill\vline\hfill
\begin{minipage}[t]{.40\textwidth}
    % Aharanov-Bohm Effect
    \underline{Aharanov-Bohm Effect}:\\[10pt]
    \(\begin{aligned}[t]
        & i\hbar \tfrac{\partial \Psi}{\partial t} 
            = \mss{ \left[ \frac{(p - qA)^2}{2m} + V + \cancel{q \phi} \right] } \Psi
            \\
        & \Rightarrow\ \begin{aligned}[t]
                \Psi & = e^{\tfrac{i}{\hbar} \int_\mathcal{O}^r q A \cdot dr' } \psi\\
                & = \boxed{ e^{ig} \psi }
            \end{aligned}
            \hspace{5pt} , \hspace{10pt} \breve{E} \psi = \breve{H} \psi
            \\[5pt]
        & \vec{A} = \tfrac{\Phi_B}{2\pi r} \hat{\phi}
            \ \Rightarrow\ \begin{aligned}[t]
                \Delta g_\text{closed} & = \tfrac{q}{\hbar} \oint \tfrac{\Phi_B}{2\pi r} \hat{\phi} \cdot r \hs d\hat{\phi}\\
                & = \tfrac{q \Phi_B}{\hbar} = \gamma_m
            \end{aligned}
    \end{aligned}\)
\end{minipage}
% \vline

% Infinitismal Path Integral
\vspace{15pt}\noindent
\underline{Infinitismal Path Integral}\\[10pt]
\(
    \begin{aligned}
        & \mss{ \begin{aligned}
                R: &\ \text{Slow degree of freedom (nucleus)}\\
                r: &\ \text{Fast degree of freedom (electron)}
            \end{aligned} }
            \\[5pt]
        & \begin{aligned}
                \mathbb{I} & = \mss{ \int dR \sum_{n} | R, n \mss{(R)} \rangle \langle R, n \mss{(R)} | }\\
                & \approx\hs \mss{ \int dR \hs | R, n \mss{(R)} \rangle \langle R, n \mss{(R)} | }
            \end{aligned}
    \end{aligned}   
    \hfill\vline\hfill 
    \begin{aligned}
        \langle \chi \mss{(\epsilon)} | e^{- \tfrac{i}{\hbar} H \epsilon} | \chi \mss{(0)} \rangle &
            = \langle R \mss{(\epsilon)} | e^{- \tfrac{i}{\hbar} H_f \epsilon} | R \mss{(0)} \rangle 
            \langle n \mss{(R(\epsilon))} | e^{- \tfrac{i}{\hbar} H_s \epsilon} | n \mss{(R(0))} \rangle
            \\[5pt]
        \breve{U} \mss{(R_1, \epsilon; R_0, 0)} & 
            = \sqrt{ \tfrac{-im}{2\pi\hbar\epsilon} } \hs e^{\tfrac{i}{\hbar} \mathcal{L}_s \epsilon}
            \hs e^{- \tfrac{i}{\hbar} E_n(R_0) \epsilon}
            \langle n \mss{(R_1)} | n \mss{(R_0)} \rangle
    \end{aligned}
    \hfill\hs
\)

\vspace{5pt}\noindent
\(\begin{aligned}
    \Psi & \mss{(R_1, \epsilon)}
        = \langle \chi \mss{(\epsilon)} | \hat{U} \mss{(\epsilon)} | \Psi \mss{(R_0, 0)} \rangle
        \approx\hs \mss{ \sqrt{ \tfrac{-im}{2\pi\hbar\epsilon} } }
        \int e^{\tfrac{i}{\hbar} \mathcal{L}_s \epsilon} 
        \hs e^{- \tfrac{i}{\hbar} E_n(R_1 + \eta) \epsilon}
        \langle n \mss{(R_1)} | n \mss{(R_1 + \eta)} \rangle
        \Psi \mss{(R_1 + \eta, 0)} 
        \hs\hs d\eta
        \hspace{20pt} \mss{(\eta = R_0 - R_1)}
        \\
    & \approx\hs \mss{ \sqrt{ \tfrac{-im}{2\pi\hbar\epsilon} } }
        \int e^{\tfrac{i}{\hbar} \tfrac{m}{2} \tfrac{ \eta^2 }{\epsilon} } 
        \left[ 1 - \tfrac{i}{\hbar} \epsilon (V_s + E_n) \right]
        \mss{ \langle n \mss{(R_1)} | } \mss{ \left[
            | n \mss{(R_1)} \rangle
            + \eta | \partial n \mss{(R_1)} \rangle
            + \tfrac{\eta^2}{2} | \partial^2 n \mss{(R_1)} \rangle    
        \right] }
        \mss{ \left[
            1 
            + \eta \tfrac{d}{dR}
            + \tfrac{\eta^2}{2} \tfrac{d^2}{dR^2}
        \right] } \Psi \mss{(R_1, 0)}
        \hs\hs d\eta
        \\
    & \approx\hs \mss{ \sqrt{ \tfrac{-im}{2\pi\hbar\epsilon} } }
        \int e^{\tfrac{i}{\hbar} \tfrac{m}{2} \tfrac{ \eta^2 }{\epsilon} } 
        \mss{ \left[
            1 
            - \tfrac{i}{\hbar} \epsilon V \mss{ (R_1, 0) }
            + \cancel{ \eta (...) }
            + \tfrac{\eta^2}{2} \tfrac{d^2}{dR^2}
            + \eta^2 \langle n | \partial n \rangle \tfrac{d}{dR}
            + \tfrac{\eta^2}{2} \langle n | \partial^2 n \rangle 
        \right] } \Psi \mss{(R_1, 0)}
        \hs\hs d\eta
\end{aligned}\)

\vspace{3pt}\noindent
\(
    \arraycolsep=2pt \begin{array}{r l l l}
        \breve{E} | \Psi \rangle & = \hat{H} | \Psi \rangle 
            & : \ 
            & \hat{H} = \tfrac{P_s^2}{2m} + V_s + \hat{H}_f 
            \\[4pt]
        \breve{E} \Psi & = \breve{H} \Psi 
            & : \ 
            & \breve{H} = \tfrac{(P_s - A^n)^2}{2m} + V + \Phi^n 
    \end{array}
    \hspace{5pt} \left| \hspace{10pt} 
        \boxed{
            \begin{aligned}
                A^n & = i\hbar \hs \mss{ \langle n | \partial n \rangle }\\
                \Phi^n & = \tfrac{\hbar^2}{2m} \left[ 
                    \mss{ \langle \partial n | \partial n \rangle - \langle \partial n | n \rangle \langle n | \partial n \rangle }
                    \right]
            \end{aligned} 
        }
    \right.    
    \hspace{11pt}
    \left( \mss{ 
        \begin{gathered}
            \langle n | \partial n \rangle + \langle \partial n | n \rangle = 0 \\
            A^n \text{ is added/subtracted in} 
        \end{gathered} 
    } \right)
\)



%----------------------------------------------------------------------------------------------------------------------------------
%----------------------------------------------------------------------------------------------------------------------------------
%----------------------------------------------------------------------------------------------------------------------------------
%----------------------------------------------------------------------------------------------------------------------------------
\newpage
% Schrodinger Eq. Integral Form
\section{Integral Form}

\hspace{.5cm} \(\begin{aligned}
    \psi(r) &= \psi_0(r) + \int g(r-r_0) V(r_0) \psi(r_0) \ d^3r 
        \hspace{18pt} \hspace{18pt} g(r) = -\frac{m}{2\pi \hbar^2} \frac{e^{ikr}}{r}\\[5pt]
    &= \psi_0 + \int g V \psi{\scriptstyle(r_0)}\\
    &= \psi_0 + \int g V \psi_0 + \int \int g V g V \psi{\scriptstyle(r_0)}\\
    &= \psi_0 + \int g V \psi_0 + \int \int g V g V \psi_0 + \int \int g V g V g V \psi_0 + ...
\end{aligned}\)

%---------------------------------------------------------------
%---------------------------------------------------------------
%---------------------------------------------------------------
%---------------------------------------------------------------
% Klein Gordon
\vspace{20pt}
\section{Klein-Gordon Equation (Spinless Free Particle)}

\begin{gather*}
    (p^2c^2 +m^2c^4)\psi = E^2 \psi\\[5pt]
    (-E^2 + p^2c^2 + m^2c^4)\psi = 0\\[5pt]
    \left[ - (E/c)^2 + p^2 + (mc)^2 \right] \psi = 0\\[5pt]
    \frac{\left[ - (E/c)^2 + p^2 + (mc)^2 \right]}{\hbar^2} \psi = 0\\[5pt]
    \left[ \frac{1}{c^2} \frac{\partial}{\partial t}^2 - \nabla^2 
        + \left( \frac{mc}{\hbar} \right)^2 \right] \psi = 0\\[5pt]
    \boxed{ (- \square^2 + \mu^2) \psi = 0 }
\end{gather*}

%---------------------------------------------------------------------------------------------------------------------------
%---------------------------------------------------------------------------------------------------------------------------
%---------------------------------------------------------------------------------------------------------------------------
%---------------------------------------------------------------------------------------------------------------------------
% Dirac Equation
\newpage
\section{Dirac Equation}

\noindent
\(
    \hfill
    \begin{aligned}[t]
        \mu^2 & = \square^2
            \\[5pt]
        m & = \sqrt{\nabla^2 - \partial_t^2} \\
        & = A \partial_x + B \partial_y + C \partial_z + i D \partial_t \\
        & = i \gamma^\mu \partial_\mu
    \end{aligned}
    \hfill
    \begin{aligned}[t]
        E^2 & = p^2 c^2 + m^2 c^4 = H^2
            \\[5pt]
        \sqrt{p^2 + m^2} & = \alpha \cdot p + \beta m\\
        & = \alpha_1 p_x + \alpha_2 p_y + \alpha_3 p_z + \beta m
    \end{aligned}
    \hfill
\)

\vspace{10pt}\noindent
\begin{gather*}
    \begin{aligned}
            \partial_x^2 \hs\hs + & \hs\hs \partial_y^2 + \partial_z^2 - \tfrac{\partial}{\partial t}^2  = 
                (A \partial_x + B \partial_y + C \partial_z + i D \partial_t)^2\\[5pt]   
            & = \begin{aligned}[t]
                & A^2 \partial_x^2 
                    + B^2 \partial_y^2 
                    + C^2 \partial_z^2 
                    - D^2 \partial_t^2
                    \\
                & + [AB+BA]\partial_x\partial_y 
                    + [AC+CA]\partial_x\partial_z 
                    + [BC+CB]\partial_y\partial_z
                    \\
                & + [AD+DA] i \partial_x\partial_t 
                    + [BD+DB] i \partial_y\partial_t 
                    + [CD+DC] i \partial_z\partial_t
            \end{aligned}
        \end{aligned}
        \\[10pt]
    D=\gamma^0, \hspace{18pt} A = i\gamma^1 = i \beta \alpha_1 , 
        \hspace{18pt} B = i\gamma^2 = i \beta \alpha_2, \hspace{18pt} C = i\gamma^3 = i \beta \alpha_3
        \\[10pt]
    \beta = \mss{ 
            \left(\begin{matrix}
                I_2 & 0\\
                0 & - I_2
            \end{matrix}\right)
        }
        , \hspace{15pt}
        \alpha_i = \mss{
            \left(\begin{matrix}
                0 & \sigma_i\\
                \sigma_i & 0
            \end{matrix}\right)
        }
        \\[10pt]
    \gamma^\mu = \mss{ 
        \left[ 
        \left(\begin{matrix}
            I_2 & 0\\
            0 & - I_2
        \end{matrix}\right) 
        ,  
        \left(\begin{matrix}
            0 & \sigma_x\\
            -\sigma_x & 0
        \end{matrix}\right)
        ,  
        \left(\begin{matrix}
            0 & \sigma_y\\
            -\sigma_y & 0
        \end{matrix}\right)
        , 
        \left(\begin{matrix}
            0 & \sigma_z\\
            -\sigma_z & 0
        \end{matrix}\right)
        \right]
        \hspace{10pt}
        \gamma^5 = 
        \left(\begin{matrix}
            0 & I_2\\
            I_2 & 0
        \end{matrix}\right) 
        }
        \\[10pt]
    \boxed{ 
            \begin{aligned}
                (i\hbar \gamma^\mu \partial_\mu - mc) \psi &= 0\\
                (i \slashed{\partial} - m)\psi &= 0 \hspace{18pt} \hspace{18pt} \text{\scriptsize(natural units)}
            \end{aligned} 
        }
        \hspace{20pt}
        \boxed{
            \begin{aligned}
                & i\hbar \tfrac{\partial}{\partial t} \psi = (c \alpha \cdot p + \beta mc^2) \psi\\
                & i\hbar \tfrac{\partial}{\partial t} \psi = (c \alpha \cdot (p - qA) + \beta mc^2 + q \phi) \psi
            \end{aligned}
        }
\end{gather*}

\vspace{20pt}\noindent
\begin{minipage}[t]{.45\textwidth}
    % E+M Spin and electron g-factor 
    \(
        \begin{aligned}
            i\hbar \tfrac{\partial}{\partial t} \psi & = (c \alpha \cdot (p - qA) + \beta mc^2 + q \phi) \psi\\
            & = (c \alpha \cdot \mathbf{\pi} + \beta mc^2 + q \phi) \psi
        \end{aligned}
    \)

    \vspace{10pt}
    \(
        \begin{aligned}
            \psi \mss{(t)} & = \psi\mss{(p)}\hs e^{i (p \cdot r - E t)} \\
            \phi & = 0 
        \end{aligned}
        \ \Rightarrow\ E \psi = (\alpha \cdot \mathbf{\pi} + \beta m) \psi
    \)

    \vspace{15pt}
    \(
        \mss{
            \left[ \begin{matrix}
                E - m & - \sigma \cdot \pi \\
                - \sigma \cdot \pi & E + m
            \end{matrix} \right]
            \left[ \begin{matrix}
                \psi_+\\
                \psi_-
            \end{matrix} \right]
        }
        = 0
        \ \Leftrightarrow\
        \mss{
            \begin{aligned}
                & (E - m) \psi_+ = (\sigma \cdot \pi) \hs \psi_-\\
                & (E + m) \psi_- = (\sigma \cdot \pi) \hs \psi_+\\        
            \end{aligned}
        }
    \)

    \vspace{15pt}
    \( 
        \begin{aligned}
            (E-m) \psi_\pm & = \tfrac{ (\sigma \cdot \pi)(\sigma \cdot \pi) }{ E + m } \psi_\pm 
                \\[10pt]
            E_s \psi_\pm & \approx \hs \tfrac{ (\sigma \cdot \pi)^2 }{ 2m } \psi_\pm 
                \hspace{15pt} \begin{gathered}
                    \text{\scriptsize(Pauli's Eq.)}\\[-7pt]
                    \text{\scriptsize(\(\sim\) to Schrodinger)}
                \end{gathered}
                \\[5pt]
            \boxed{ \mss{ \begin{gathered}
                    \sigma \cdot A\sigma \cdot B = \\[-3pt]
                    A \cdot B + i \sigma \cdot (A \times B) 
                \end{gathered} } } & 
                = \tfrac{ \sigma \cdot \pi\sigma \cdot \pi }{ 2m } \psi_\pm 
                = \tfrac{ \pi \cdot \pi + i \sigma \cdot (\pi \times \pi) }{ 2m } \psi_\pm 
                \\[5pt]
            & = \left[ \tfrac{ \pi^2 }{ 2m } - \tfrac{ q\hbar }{ 2m } \sigma \cdot B \right] \psi_\pm \\[5pt]
            \boxed{ \mss{(g_e = 2)} } \hspace{10pt} & 
                = \left[ \tfrac{ \pi^2 }{ 2m } - \tfrac{ g_e q}{ 2 m } S \cdot B \right] \psi_\pm
        \end{aligned}
    \)
\end{minipage}
\hfill\vline\hfill
\begin{minipage}[t]{.49\textwidth}
    % Dirac Equation (no charge)
    \(
        i\hbar \tfrac{\partial}{\partial t} \psi = (c \alpha \cdot p + \beta mc^2) \psi
    \)

    \vspace{25pt}
    \(
        \psi \mss{(t)} = \psi\mss{(p)}\hs e^{i (p \cdot r - E t)}
        \ \Rightarrow\ 
        E \psi = (\alpha \cdot p + \beta m) \psi
    \)

    \vspace{25pt}
    \(
        \mss{
            \left[ \begin{matrix}
                E - m & - \sigma \cdot p \\
                - \sigma \cdot p & E + m
            \end{matrix} \right]
            \left[ \begin{matrix}
                \psi_+\\
                \psi_-
            \end{matrix} \right]
        }
        = 0
        \ \Leftrightarrow\
        \mss{
            \begin{aligned}
                & (E - m) \psi_+ = (\sigma \cdot p) \hs \psi_-\\
                & (E + m) \psi_- = (\sigma \cdot p) \hs \psi_+\\        
            \end{aligned}
        }
    \)

    \vspace{15pt}
    \(\boxed{
        p = 0 \ \Rightarrow\ 
        \mss{ \begin{cases}
            \psi = \left[ \arraycolsep=2pt\begin{array}{l}
                    \psi_+ \\
                    0
                \end{array} \right]
                , 
                & E = m 
                \\[9pt]
            \psi = \left[ \arraycolsep=2pt\begin{array}{l}
                    0 \\
                    \psi_-
                \end{array} \right]
                ,
                & E = - m 
        \end{cases} }
    }\)

    \vspace{13pt}
    \(  
        \psi_\pm 
        = \tfrac{(\sigma \cdot p)^2}{E^2 - m^2} \psi_\pm
        = \tfrac{p^2}{E^2 - m^2} \psi_\pm
        \ \Rightarrow\ \boxed{ \mss{ E_\pm = \pm \sqrt{p^2 + m^2} } }
    \)

    \vspace{10pt}
    \(
        \boxed{ \int \Vert \psi_+ \Vert^2 + \Vert \psi_- \Vert^2 \hs d^3r = 1 }
    \)


\end{minipage}

%---------------------------------------------------------------------------------------------------------------------------------
%
%
%
\newpage
% Hydrogen fine structure from Dirac Equation
\noindent
\underline{Hydrogen Fine Structure}\\[5pt]
\(
    \begin{aligned}
        E \psi & = H \psi\\
        E \psi & = (\alpha \cdot p + \beta m + q\phi) \psi
    \end{aligned}
    \ \Rightarrow\ 
    \begin{aligned}
        ( E - m - V ) \psi_+ & = ( \sigma \cdot p ) \psi_-\\
        ( E + m - V ) \psi_- & = ( \sigma \cdot p ) \psi_+
    \end{aligned}            
\)

\vspace{15pt}\noindent
\begin{minipage}[t]{.65\textwidth}
    \(
        \begin{aligned}
            ( E - m - V ) \psi_+ & = ( \sigma \cdot p ) \left( \tfrac{1}{E + m - V } \right) ( \sigma \cdot p ) \psi_+\\
            ( E_s - V ) \psi_+ & = \tfrac{1}{2m} ( \sigma \cdot p ) 
                \left( 1 + \tfrac{E_S - V}{2m} \right)^{-1} 
                ( \sigma \cdot p ) \psi_+
                \\[5pt]                
        \end{aligned}
    \)

    \noindent
    \(
        \begin{aligned}[t]
            & \approx \hs \tfrac{ p^2 }{2m} \psi_+ 
                \hspace{20pt} \text{\scriptsize(1\(^\text{st}\) order, \(v^2\))}
                \\[5pt]
            & \approx \hs \tfrac{ p^2 }{2m} \psi_+
                - \tfrac{\sigma \cdot p}{(2m)^2}
                \left( E_S - V \right)
                ( \sigma \cdot p ) \psi_+
                \hspace{20pt} \text{\scriptsize(2\(^\text{nd}\) order, \(v^4\))}
                \\[-5pt]
            & = \tfrac{ p^2 }{2m} \psi_+
                - \tfrac{\sigma \cdot p}{(2m)^2}
                \Big[
                    ( \sigma \cdot p ) \overbrace{ ( E_S - V ) \psi_+ }^{ \text{1\(^\text{st}\) order} }
                    + \left[ \bcancel{E_S} - V , \hs \sigma \cdot p \right] \psi_+
                \Big] 
                \\[5pt]
            & \approx\hs \left[
                    \tfrac{ p^2 }{2m} 
                    - \tfrac{ p^2 }{(2m)^2} \tfrac{ p^2 }{2m}
                    - \tfrac{ (\sigma \cdot p) (\sigma \cdot \left[p, V\right]) }{ (2m)^2 }
                \right] \psi_+
                \\[5pt]
            E_S \psi_+ & = \left[
                    \tfrac{ p^2 }{2m} 
                    + V
                    - \tfrac{ p^4 }{ 8m^3 }
                    - \tfrac{ i \sigma \cdot ( p \times \left[p, V\right] ) }{ 4m^2 } 
                    - \underbrace{ \tfrac{ p \left[p, V\right] }{ 4m^2 } }
                \right] \psi_+
                = H \psi_+
                \\[-5pt]
            & \hspace{150pt} \text{\scriptsize(isn't Hermitian)}
        \end{aligned}    
    \)

\end{minipage}
\vline\hfill
\(\begin{aligned}[t]
    1 & = \int \Vert \psi_+ \Vert^2 + \Vert \psi_- \Vert^2 \hs d^3r\\
    & = \int \Vert \psi_+ \Vert^2 + \Vert \tfrac{\sigma \cdot p}{E+m-V} \psi_+ \Vert^2 \hs d^3r\\
    & \approx\hs \int \Vert \psi_+ \Vert^2 + \Vert \tfrac{\sigma \cdot p}{2m} \psi_+ \Vert^2 \hs d^3r\\
    & = \int \psi_+^\dagger ( 1 + \tfrac{p^2}{4m^2} ) \psi_+ \hs\hs d^3r\\
    & \approx\hs \left\langle ( 1 + \tfrac{p^2}{8m^2} ) \psi_+ \Big| ( 1 + \tfrac{p^2}{8m^2} ) \psi_+ \right\rangle \\
    & \equiv\hs \langle \psi_S | \psi_S \rangle
\end{aligned}\)

\vspace{10pt}\noindent
\(
    \begin{aligned}
        E_S \Big( 1 + & \tfrac{p^2}{8m^2} \Big)^{-1} \psi_S = H \left( 1 + \tfrac{p^2}{8m^2} \right)^{-1} \psi_S \\
        E_S \psi_S & = \left( 1 + \tfrac{p^2}{8m^2} \right) H \left( 1 + \tfrac{p^2}{8m^2} \right)^{-1} \psi_S\\
        & = \left( H + \tfrac{p^2 H}{8m^2} \right) \left( 1 - \tfrac{p^2}{8m^2} + \mathcal{O}\mss{(p^4)}\right) \psi_S\\
        & \approx\hs \left( H + \left[ \tfrac{p^2}{8m^2} , H \right] \right) \psi_S
            \hs\approx\hs \left( H + \left[ \tfrac{p^2}{8m^2} , V \right] \right) \psi_S
            \hspace{20pt} \text{\scriptsize(2\(^\text{nd}\) order, \(v^4\))}
            \\
        E_S \psi_S & = \left(
                \tfrac{ p^2 }{2m} 
                + V
                - \tfrac{ p^4 }{ 8m^3 }
                - \tfrac{ i \sigma \cdot ( p \times \left[p, V\right] ) }{ 4m^2 } 
                - \tfrac{ p \left[p, V\right] }{ 4m^2 }
                + \tfrac{ \left[ p , V \right] p + p \left[ p , V \right] }{ 8m^2 }
            \right) \psi_S
            \\
        & = \left(
                \tfrac{ p^2 }{2m} 
                + V
                - \tfrac{ p^4 }{ 8m^3 }
                - \tfrac{ i \sigma \cdot ( p \times \left[p, V\right] ) }{ 4m^2 } 
                - \tfrac{ \left[ p , \left[p, V\right] \right] }{ 8m^2 }
            \right) \psi_S
            \\[3pt]
        % The Full Equation Simplified
        & = \hs \boxed{ (H_S + H_\text{rel.} + H_\text{so} + H_\text{darwin}) \psi_S }
            \\[-7pt]
        & = \Big(
                \tfrac{ p^2 }{2m} 
                + V
                - \tfrac{ p^4 }{ 8m^3 }
                - \tfrac{ 1 }{ 4m^2 } \sigma \cdot ( p \times \nabla V )
                + \overbrace{ \tfrac{ 1 }{ 8m^2 } \nabla^2 V }^{\text{Darwin } \Rightarrow}
            \Big) \psi_S
            \\
        & = \left(
                \tfrac{ p^2 }{2m} 
                + V
                - \tfrac{ p^4 }{ 8m^3 }
                - \tfrac{ 1 }{ 2m^2 } S \cdot [ \vec{p} \times \tfrac{qq \vec{r}}{4\pi r^3} ]
                + \tfrac{ 1 }{ 8m^2 } [ qq \delta^3 \mss{(r)} ]
            \right) \psi_S
            \\
        % The Full Equation
        & = \Big(
                \tfrac{ p^2 }{2m} 
                + V
                - \tfrac{ p^4 }{ 8m^3 }
                + \underbrace{ \tfrac{ e^2 }{ 8\pi m^2 } \tfrac{ S \cdot L }{r^3} }_{l \neq 0}
                + \underbrace{ \tfrac{ e^2 }{ 8m^2 } \delta^3 \mss{(r)} }_{l = 0}
            \Big) \psi_S
    \end{aligned}  
    % Explain the Darwin Term
    \begin{aligned}
        \\[2.5cm]
        \overline{V \mss{(r)}} = &\ V \mss{(r)} 
            + \mss{ \sum_i }\hs \overline{ \cancel{ \tfrac{\partial V}{\partial r_i} \delta r_i } }\\
        & + \tfrac{1}{2!} \mss{ \sum_{ij} } \overline{ \tfrac{\partial^2 V}{\partial r_i \partial r_j} \mss{ \delta r_i \delta r_j }} 
            \\[3pt]
        & + \mss{ \mathcal{O} (\delta r^3) }
            \\[5pt]
        = &\ V \mss{(r)} + \underline{ \tfrac{1}{2} (\mss{ \delta r })^2 \nabla^2 V } + .. \\[5pt]
        & \hspace{35pt} \mss{ ( \delta r \sim \tfrac{\hbar}{mc} )}
    \end{aligned}  
\)

% Exact Energy Eigenvalues
\vspace{10pt}
\parbox{2.6cm}{Exact Energy \\
Eigenvalues}: \ \(\displaystyle
    E_{nj} = mc^2 
    \left[ 
        1 + \left( 
            \frac{ \alpha }{ n - (j + 1/2) + \sqrt{ (j+1/2)^2 - \alpha^2 } } 
        \right)^2 
    \right]^{-1/2}
\)

\end{document}
