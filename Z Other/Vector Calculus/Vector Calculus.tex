\documentclass[12pt]{article}
\usepackage[left=.75in, right=.75in, top=1in, bottom = 1in]{geometry} % Resizes the borders
\usepackage{amssymb, amsmath, amsfonts, mathtools, bbm, esint}
\usepackage{array,multirow}
\usepackage{cancel}

\newcommand{\hs}{\hspace{1pt}} % 1pt horizontal space
\newcommand{\hsvec}[1]{\vec{\hs #1}} % 1pt space with a \vec
\newcommand{\nhs}{\hspace{-1pt}} % -1pt horizontal space
\newcommand{\mss}[1]{\text{\scriptsize\(#1\)}} % math scriptsize
\newcommand{\tss}[1]{\text{\scriptsize #1}} % text scriptsize

\newcommand{\checkedbox}{\mbox{\ooalign{$\checkmark$\cr\hidewidth$\square$\hidewidth\cr}}} % checked box
\newcommand{\crossbox}{\mbox{\ooalign{\ding{55}\cr\hidewidth$\square$\hidewidth\cr}}} % cross box

\begin{document}
\setlength{\parindent}{0pt}

%---------------------------------------------------------------------------------------------------------------------------------
%---------------------------------------------------------------------------------------------------------------------------------
%---------------------------------------------------------------------------------------------------------------------------------
%---------------------------------------------------------------------------------------------------------------------------------
\newpage

\(
	\mss{ \begin{gathered}
		\boxed{ \hsvec{\nabla} = \left[ \hsvec{\nabla} \left( r, \theta, \phi \right) \right] \bar{\partial}_\circ }
			\\[5pt]
		d = \left[ dx \ dy \ dz \right] \hsvec{\nabla} = d\hsvec{l}\hs\hs^T \hsvec{\nabla} 
			\\[5pt]
		\begin{aligned}
				d ( r, \theta, \phi ) 
					& = \left[ dx \ dy \ dz \right]
					\hsvec{\nabla} 
					( r, \theta, \phi )
					\\
				\Aboxed{ \partial \bar{l}_\circ\nhs^T & = d\hsvec{l}\hs\hs^T \hsvec{\nabla} (r, \theta, \phi) }
			\end{aligned}
			\\[5pt]
		\begin{aligned}
			\partial \bar{l}_\circ\nhs^T \bar{\partial}_\circ & 
				= d\hsvec{l}\hs\hs^T \left[ \hsvec{\nabla} ( r, \theta, \phi ) \right] \bar{\partial}_\circ
				\\
			\Aboxed{ d = \partial \bar{l}_\circ\nhs^T \bar{\partial}_\circ & = d\hsvec{l}\hs\hs^T \hsvec{\nabla} }
		\end{aligned}
	\end{gathered} } 
	\hfill \vline \hfill
	% Matrix Form for cartesian gradient and d(spherical)
	\begin{aligned}
		% Partial / Partial xyz
		\underline{ \nhs \hsvec{\nabla} } &  
			= 
			\left[\begin{matrix}
				\\[-12pt]
				\tfrac{\partial}{\partial x}\\[5pt]
				\tfrac{\partial}{\partial y}\\[5pt]
				\tfrac{\partial}{\partial z}
			\end{matrix}\right] 
			= 
			\left[ \mss{ \arraycolsep=2pt\begin{matrix}
				| & | & | \\
				\nabla r & \nabla \theta & \nabla \phi \\
				| & | & |
			\end{matrix} } \right]
			\left[\begin{matrix}
				\\[-12pt]
				\tfrac{\partial}{\partial r}\\[5pt]
				\tfrac{\partial}{\partial \theta}\\[5pt]
				\tfrac{\partial}{\partial \phi}
			\end{matrix}\right]
			\\[10pt]
		\underline{ \partial \bar{l}_\circ } & = \nhs
			\left[ { \begin{matrix}
				dr\\
				d\theta\\
				d\phi
			\end{matrix} } \right] 
			\begin{aligned}
				& = \hs\hs [\hsvec{\nabla} (r, \theta, \phi)]^T d\hsvec{l}\\[5pt]
				& = \left[ \mss{ \begin{matrix}
						- \nabla r - \\
						- \nabla \theta - \\
						- \nabla \phi - \\
					\end{matrix} } \right]
					\left[ \mss{ \begin{matrix}
						dx\\
						dy\\
						dz
				\end{matrix} } \right]
			\end{aligned}
	\end{aligned}
	\hfill\vline\hfill
	% \theta Example
	\begin{aligned}
		\theta & = \theta \mss{(x,y,z)} 
			\hspace{10pt} ( \mss{ x^2 + y^2 = z^2 \tan^2{\theta} } )
			\\
		\phi & = \phi \mss{(x,y,z)} 
			\hspace{10pt} ( \mss{ y = x \tan{\phi} } )
			\\[10pt]
		\tfrac{\partial}{\partial x} & = \tfrac{\partial r}{\partial x} \tfrac{\partial}{\partial r}
			+ \tfrac{\partial \theta}{\partial x} \tfrac{\partial}{\partial \theta}
			+ \tfrac{\partial \phi}{\partial x} \tfrac{\partial}{\partial \phi} 
			\\[5pt]
		\tfrac{\partial}{\partial \theta} & = \tfrac{\partial x}{\partial \theta} \tfrac{\partial}{\partial x}
			+ \tfrac{\partial y}{\partial \theta} \tfrac{\partial}{\partial y}
			+ \tfrac{\partial z}{\partial \theta} \tfrac{\partial}{\partial z}
			\\[10pt]
		\mss{ d\theta} & = \mss{ dx \tfrac{\partial \theta}{\partial x} 
			+ dy \tfrac{\partial \theta}{\partial y} 
			+ dz \tfrac{\partial \theta}{\partial z} }
			\\
		\mss{dy_{\vec{r}_\circ}(\vec{r}_\circ \hs\nhs\nhs\nhs\nhs')}|_{t=0} & = \mss{ ( \tfrac{\partial y}{\partial r} dr
			+ \tfrac{\partial y}{\partial \theta} d\theta
			+ \tfrac{\partial y}{\partial \phi} d\phi ) \tfrac{1}{dt}|_{t=0} }
	\end{aligned}
\)

\vspace{15pt}
% Contravector Dot Covector
\(
	\boxed{ \hsvec{b}^{\hs i} \cdot \hsvec{b}_{i} = \delta_{ij} }:\ 
	\begin{aligned}
		& \hsvec{\nabla}{\phi} \cdot \tfrac{\partial \hsvec{r}}{\partial \phi} = 1\\[5pt]
		& \hsvec{\nabla}{\phi} \cdot \tfrac{\partial \hsvec{r}}{\partial \theta} = 0
	\end{aligned}
	\ \Rightarrow\ 
	\left[ \mss{ \begin{matrix}
		- \hsvec{\nabla} r - \\
		- \hsvec{\nabla} \theta - \\
		- \hsvec{\nabla} \phi - 
	\end{matrix} } \right]
	\left[ 
		\tfrac{\partial}{\partial r} \ \tfrac{\partial}{\partial \theta} \ \tfrac{\partial}{\partial \phi}
	\right]
	\hsvec{r}
	= \mathbbm{1}_3
	\ \ \Rightarrow\ \
	\boxed{
		\tfrac{\partial y}{\partial \phi} = 
		\left[ \mss{ 
			0 \ 1 \ 0
		} \right]
		\left[ \mss{ \begin{matrix}
			- \hsvec{\nabla} r - \\
			- \hsvec{\nabla} \theta - \\
			- \hsvec{\nabla} \phi - 
		\end{matrix} } \right]^{-1}
		\left[ \mss{ \begin{matrix}
			0 \\ 0 \\ 1
		\end{matrix} } \right]
		= \tfrac{\partial y}{\partial \phi}^T
	}
\)

\vspace{15pt}
% d operator
\(
	\begin{aligned}
		d & = dx \tfrac{\partial }{\partial x}
			+ dy \tfrac{\partial }{\partial y} 
			+ dz \tfrac{\partial }{\partial z} 
			= d\hsvec{l} \cdot \hsvec{\nabla} 
			&& = \left[ 
				\tfrac{dr}{\lVert \nabla r \rVert} , 
				\tfrac{d\theta}{\lVert \nabla \theta \rVert} ,
				\tfrac{d\phi}{\lVert \nabla \phi \rVert} 
			\right] 
			\left[ 
				\mss{ \lVert \nabla r \rVert } \tfrac{\partial }{\partial r} ,\hs
				\mss{ \lVert \nabla \theta \rVert } \tfrac{\partial }{\partial \theta} ,\hs
				\mss{ \lVert \nabla \phi \rVert } \tfrac{\partial }{\partial \phi} 
			\right]^T
			\\
		& = dr \tfrac{\partial }{\partial r} 
			+ d\theta \tfrac{\partial }{\partial \theta} 
			+ d\phi \tfrac{\partial }{\partial \phi} 
			= \partial \bar{l}_\circ\nhs^T \bar{\partial}_\circ 
			&& = \left[ dr, rd\theta, r\sin\theta d\phi \right]
			\left[ 
				\tfrac{\partial }{\partial r} 
				, \tfrac{1}{r} \tfrac{\partial }{\partial \theta} 
				,\tfrac{1}{r\sin\theta} \tfrac{\partial }{\partial \phi} 
			\right]^T
			\\[3pt]
		& = \tfrac{dr}{\lVert \nabla r \rVert} \mss{ \lVert \nabla r \rVert } \tfrac{\partial }{\partial r} 
			+ \tfrac{d\theta}{\lVert \nabla \theta \rVert} \mss{ \lVert \nabla \theta \rVert } \tfrac{\partial }{\partial \theta} 
			+ \tfrac{d\phi}{\lVert \nabla \phi \rVert} \mss{ \lVert \nabla \phi \rVert } \tfrac{\partial }{\partial \phi}
			&& \dots = d\bar{l}_\circ\nhs^T \bar{\nabla}_\circ = d\hsvec{l}_\circ\nhs^T \hsvec{\nabla}_\circ 
	\end{aligned}
\)

\vspace{20pt}
% Work to Find Line Element and Unit Spherical Vectors
\(
	% Line Element
	\begin{aligned}
		d\hsvec{l} = d\hsvec{r} & 
			= [ dr \tfrac{\partial}{\partial r} 
			+ d\theta \tfrac{\partial}{\partial \theta} 
			+ d\phi \tfrac{\partial}{\partial \phi} ]
			(x,y,z)^T
			\\
		d{(x, y, z)} & 
			= \left[ \mss{ \tfrac{dr}{\lVert \nabla r \rVert} { \lVert \nabla r \rVert } \tfrac{\partial }{\partial r} 
			+ \tfrac{d\theta}{\lVert \nabla \theta \rVert} { \lVert \nabla \theta \rVert } \tfrac{\partial }{\partial \theta} 
			+ \tfrac{d\phi}{\lVert \nabla \phi \rVert} { \lVert \nabla \phi \rVert } \tfrac{\partial }{\partial \phi} 
			} \right] (x,y,z)
			\\[5pt]
		(dx,dy,dz) &
			= dr \hs \hat{r}^T + r \hs d\theta \hs \hat{\theta}^T + r \sin\theta \hs d\phi \hs \hat{\phi}^T
	\end{aligned}
	\hfill\vline\hfill
	% Unit Spherical Vectors
	\begin{aligned}
		\big( \hat{r} , & \hat{\theta} , \hat{\phi} \big) 
			\equiv \underline{ \left( 
				\mss{ \lVert \nabla r \rVert } \tfrac{\partial \hsvec{r}}{\partial r} ,\hs
				\mss{ \lVert \nabla \theta \rVert } \tfrac{\partial \hsvec{r}}{\partial \theta} ,\hs
				\mss{ \lVert \nabla \phi \rVert } \tfrac{\partial \hsvec{r}}{\partial \phi} 
			\right) }
			\\
		& = \left(
				\tfrac{\partial}{\partial r} ,\
				\tfrac{1}{r} \tfrac{\partial}{\partial \theta} ,\
				\tfrac{1}{r \sin{\theta}} \tfrac{\partial}{\partial \phi} 
			\right)
			\otimes (x,y,z)^T
			\\
		& = \bar{\nabla}_\circ^T \otimes \hsvec{r}
	\end{aligned}
\)

\vspace{20pt}
% Line Element and Gradient
\(\begin{aligned}
	% Line Element
	& \boxed{ d\hsvec{l} = (dx, dy, dz) \cdot \big( \hat{x} , \hat{y} , \hat{z} \big) }
		= \boxed{ (dr, rd\theta, r\sin\theta d\theta) \cdot \big( \hat{r} , \hat{\theta} , \hat{\phi} \big) = d\hsvec{l}_\circ }
		= d\bar{l}_\circ\nhs^T \cdot \big( \hat{r} , \hat{\theta} , \hat{\phi} \big)
		\\
	% Gradient
	& \boxed{ 
			\hsvec{\nabla} = (\hat{x}, \hat{y}, \hat{z}) 
			\cdot \left( \tfrac{\partial}{\partial x}, \tfrac{\partial}{\partial y}, \tfrac{\partial}{\partial z} \right) 
		}
		= \boxed{
			(\hat{r}, \hat{\theta}, \hat{\phi}) 
			\cdot \left( 
				\tfrac{\partial}{\partial r}, 
				\tfrac{1}{r} \tfrac{\partial}{\partial \theta}, 
				\tfrac{1}{r\sin\theta} \tfrac{\partial}{\partial \phi}
			\right) 
			= \hsvec{\nabla}_\circ
		}
		= \left(
			\tfrac{\partial \hsvec{r}}{\partial r} ,
			\tfrac{1}{r} \tfrac{\partial \hsvec{r}}{\partial \theta} ,
			\tfrac{1}{r \sin{\theta}} \tfrac{\partial \hsvec{r}}{\partial \phi} 
		\right) \bar{\nabla}_\circ^T
\end{aligned}\)

\vspace{10pt}
% Two ways to write cartesian partial derivatives 
\(\begin{aligned}
	\hsvec{\nabla} & = [ \bar{\nabla}_\circ^T \otimes \hsvec{r} \hs ]  \bar{\nabla}_\circ
		= \left[ \bar{\nabla}_\circ^T \otimes (x,y,z)^T \right]
		\left[ \mss{ \begin{matrix}
			\tfrac{\partial}{\partial r}\\[4pt]
			\tfrac{1}{r} \tfrac{\partial}{\partial \theta}\\[4pt]
			\tfrac{1}{r\sin\theta} \tfrac{\partial}{\partial \phi} 
		\end{matrix} } \right]
		&& \Rightarrow\ 
		\tfrac{\partial}{\partial x} 
		= \underline{ \tfrac{\partial x}{\partial r} } \tfrac{\partial}{\partial r}
		+ \underline{ \mss{ \Vert \nabla \theta \Vert }^2 \tfrac{\partial x}{\partial \theta} } \tfrac{\partial}{\partial \theta}
		+ \underline{ \mss{ \Vert \nabla \phi \Vert }^2 \tfrac{\partial x}{\partial \phi} } \tfrac{\partial}{\partial \phi}
		\\
	& = [ \hsvec{\nabla}(r,\theta,\phi) ] \bar{\partial}_\circ
		= [ \hsvec{\nabla}(r,\theta,\phi) ] 
		\left[ \mss{ \begin{matrix}
			\tfrac{\partial}{\partial r}\\[4pt]
			\tfrac{\partial}{\partial \theta}\\[4pt]
			\tfrac{\partial}{\partial \phi} 
		\end{matrix} } \right]
		&&\Rightarrow\
		\tfrac{\partial}{\partial x} 
		= \underline{ \tfrac{\partial r}{\partial x} } \tfrac{\partial}{\partial r}
		+ \underline{ \tfrac{\partial \theta}{\partial x} } \tfrac{\partial}{\partial \theta}
		+ \underline{ \tfrac{\partial \phi}{\partial x} } \tfrac{\partial}{\partial \phi}
		\ \Rightarrow\ 
		\boxed{ \tfrac{\partial \phi}{\partial y} = \tfrac{\partial y}{\partial \phi} \mss{\Vert \nabla \phi \Vert}^2 }
\end{aligned}\)

\vspace{10pt}
% Unit Vectors Summary
\(\boxed{
	\arraycolsep=3pt
	\begin{array}{r c c c r c r c c c c c c c r}
		& & & & & & & 
			& \text{\scriptsize\underline{contravariant}}_{\hs i} 
			& \multicolumn{3}{c}{ \text{\scriptsize(equal since orthog.)} } 
			& \multicolumn{3}{l}{ \text{\scriptsize\underline{covariant}}^i } 
			\\[6pt]
		\hat{r} & = 
			& (\hat{r}_x, \hat{r}_y, \hat{r}_z) 
			& = 
			& { \tfrac{\hsvec{r}}{r} }
			& = 
			& \tfrac{\partial}{\partial r} \hsvec{r} 
			& = 
			& \tfrac{\partial \hsvec{r}}{\partial r} \Vert \tfrac{\partial \hsvec{r}}{\partial r}  \Vert^{-1} 
			& \ \ \stackrel{\rightarrow}{=}\ \
			& \mss{ \Vert \nabla r \Vert } \tfrac{\partial \hsvec{r}}{\partial r} 
			& \ \stackrel{\leftarrow}{=}\ \ 
			& \tfrac{\nabla r}{\Vert \nabla r \Vert}
			& = 
			& \nabla r
			\\[5pt]
		\hat{\theta} & = 
			& (\hat{\theta}_x, \hat{\theta}_y, \hat{\theta}_z) 
			& = 
			& { \tfrac{\partial \hat{r}}{\partial \theta} }
			& = 
			& \tfrac{1}{r} \tfrac{\partial}{\partial \theta} \hsvec{r}
			& =
			& \tfrac{\partial \hsvec{r}}{\partial\theta} \Vert \tfrac{\partial \hsvec{r}}{\partial\theta} \Vert^{-1} 
			& \ \ \stackrel{\rightarrow}{=}\ \
			& \mss{ \Vert \nabla \theta \Vert } \tfrac{\partial \hsvec{r}}{\partial\theta} 
			& \ \stackrel{\leftarrow}{=}\ \
			& \tfrac{\nabla \theta}{\Vert \nabla \theta \Vert} 
			& = 
			& r \nabla \theta
			\\[5pt]
		\hat{\phi} & = 
			& (\hat{\phi}_x, \hat{\phi}_y, \hat{\phi}_z) 
			& = 
			& { \tfrac{1}{\sin\theta} \tfrac{\partial \hat{r}}{\partial \phi} }
			& = 
			& \tfrac{1}{r \sin\theta} \tfrac{\partial}{\partial \phi} \hsvec{r}
			& =
			& \tfrac{\partial \hsvec{r}}{\partial\phi} \Vert \tfrac{\partial \hsvec{r}}{\partial\phi} \Vert^{-1} 
			& \ \ \stackrel{\rightarrow}{=} \ \
			& \mss{ \Vert \nabla \phi \Vert } \tfrac{\partial \hsvec{r}}{\partial\phi} 
			& \ \stackrel{\leftarrow}{=} \ \
			& \tfrac{\nabla \phi}{\Vert \nabla \phi \Vert}
			& = 
			& \mss{r \sin\theta} \nabla \phi
	\end{array}
}\)

%--------------------------------------------------------------------------------------------------------------------------------------
%
%
%
\newpage
% Fund Theorem
\underline{Fundamental Theorem}: \ \

\vspace{5pt}
\(\arraycolsep=4pt \begin{array}{c c c c l}
	f \circ r\mss{(t_f)} - f \circ r\mss{(t_i)} & = 
		& \mss{\displaystyle \sum \int_{t_i}^{t_f}} \tfrac{dr^i}{dt} \tfrac{\partial f}{\partial r^i} \hs dt 
		& = 
		& \underline{ \mss{\displaystyle \int_{t_i}^{t_f}} \tfrac{dx}{dt} \tfrac{\partial f}{\partial x} \hs dt 
			+ \mss{\displaystyle \int_{t_i}^{t_f} } \tfrac{dy}{dt} \tfrac{\partial f}{\partial y} \hs dt }
		\\[10pt]
	\boxed{ \mss{\int_\gamma} \hsvec{\nabla} f \cdot d\hsvec{r} } & =
		& \mss{\displaystyle \sum \int_{r^i_i}^{r^i_f}} dr^i \hs \tfrac{\partial f}{\partial r^i}
		& = 
		& \underline{ \mss{\displaystyle \int_{x_i}^{x_f} } dx \hs \tfrac{\partial f}{\partial x}
			+ \mss{\displaystyle \int_{y_i}^{y_f} } dy \hs  \tfrac{\partial f}{\partial y} }
		\\[10pt]
	f\mss{(x_f, y_f)} - f\mss{(x_i, y_i)} & = 
		& \underline{ \sum \Delta_{r^i} \tfrac{\partial f}{\partial r^i}|_{c_i} }
		& = 
		& {\scriptstyle(x_f-x_i)} \mss{\displaystyle \int_{x_i}^{x_f} } \tfrac{dx}{x_f-x_i} \tfrac{\partial f}{\partial x}
			+ {\scriptstyle(y_f - y_i)} \mss{\displaystyle \int_{y_i}^{y_f} } \tfrac{dy}{y_f-y_i} \tfrac{\partial f}{\partial y}
		\\[10pt]
	\multicolumn{3}{r}{
			\left( \ \begin{gathered}
				\mss{ x(t, x_f, x_i) = x_i + t(x_f - x_i) }\\
				\mss{ y(t, x_f, x_i) = y_i + t(y_f - y_i) }\\
			\end{gathered} \ \right)
		}
		& = 
		& \underline{ 
			{\scriptstyle(x_f-x_i)} \mss{\displaystyle \int_{0}^{1}} dt \hs \tfrac{\partial f}{\partial x}|_{r(t, x_f, x_i)}
			+ {\scriptstyle(y_f - y_i)} \mss{\displaystyle \int_{0}^{1}} dt \hs 
			\tfrac{\partial f}{\partial y}|_{r(t, x_f, x_i)}
			}
		\\[20pt]
	\text{Partial}:\ \underline{ \tfrac{\partial f}{\partial x} (x_i, y_i) } & = 
			& \multicolumn{3}{l}{
				0 + \underline{ \mss{\displaystyle \int_{0}^{1}} dt \hs \tfrac{\partial f}{\partial x}|_{r_i} }
				+ 0 \mss{\displaystyle \int_{0}^{1}} dt \hs \tfrac{\partial^2 f}{\partial^2 x}|_{r_i}
				+ 0 \mss{\displaystyle \int_{0}^{1}} dt \hs \tfrac{\partial^2 f}{\partial y \partial x}|_{r_i}
			}
\end{array}\)

\vspace{15pt}
% Vectorization and derivative of Matrices
\(\begin{aligned}
	Y^i_{\ j} & = X^i_{\ k}\hs X^k_{\ l}\hs X^l_{\ j}\\
	\tfrac{\partial Y^i_{\ j}}{\partial X^a_{\ b}} & = 
		\delta^i_{\ a} \hs e^b X X e_j + e^i X e_a \hs e^b X e_j + e^i X X e_a \hs \delta^b_{\ j}
		\\[5pt]
	\tfrac{\partial Y}{\partial X^a_{\ b}} & 
		= \mathbbm{1} e_a \hs e^b X X + X e_a \hs e^b X  + X X e_a \hs e^b \mathbbm{1}
		\\
	& = \left[ 
		(XX)^T (\mathbbm{1} e_a \hs e^b)^T + X^T (X e_a \hs e^b)^T + \mathbbm{1}^T (X X e_a \hs e^b)^T
		\right]^T
		\\
	dY & = \left[ \mathbbm{1} e_a \hs e^b X X + X e_a \hs e^b X  + X X e_a \hs e^b \mathbbm{1} \right] dX^a_{\ b}
		\\
	dY & = \mathbbm{1} (dX) XX + X(dX) X + XX(dX)\mathbbm{1} 
		\\[10pt]
	\text{vec} \left( \tfrac{\partial Y}{\partial X^a_{\ b}} \right) &
		= \left[ 
			(XX)^T \otimes \mathbbm{1}  + X^T \otimes X  + \mathbbm{1}^T \otimes (XX) 
		\right] 
		\text{vec}(e_a \hs e^b)
		\\
	\text{vec}(dY) & = \left[ 
			(XX)^T \otimes \mathbbm{1} + X^T \otimes X + \mathbbm{1}^T \otimes (XX)
		\right] 
		\text{vec}(dX)
		= d\text{vec}(Y)
		\\[10pt]
	\tfrac{\partial Y}{\partial X} & \equiv \tfrac{\partial \text{vec(Y)}}{\partial \text{vec(X)}}
		= (XX)^T \otimes \mathbbm{1} + X^T \otimes X + \mathbbm{1}^T \otimes (XX) 
		\\[10pt]
	\text{vec}(D) & = \text{vec}(ABC) = (C^T \otimes A) \text{vec}(B)  \\
	A B C_j & = (C^k_{\ j} A) B_k = (C_j^T \otimes A) \text{vec}(B) = D_j\\
	e^i ABC_j& = (C^k_{\ j} A^i) B_k = (C_j^T \otimes A^i) \text{vec}(B) = D^i_{\ j} \\
\end{aligned}\)

\vspace{15pt}
% Dual Spaces
\(\begin{aligned}
	& \underline{ \text{Dual Space} }:\ \
		V :\ B = 
		\mss{ \arraycolsep=2pt \left[ \begin{matrix}
			| & |\\
			v_1 & v_2 \\
			| & |
		\end{matrix} \right] }
		\ , \ \ 
		\underline{ \text{Hom}(V, \mathbb{R}) = V^* } :\ B^* = 
		{ \arraycolsep=2pt \left[ \begin{matrix}
			- v^1 -\\
			- v^2 -
		\end{matrix} \right] }
		\ \rightarrow\ B^* B = \mathbbm{1}_2
		\\[10pt]
	& \bullet\ \underline{ \begin{gathered}
			\text{Adjoint of}\\
			\text{Linear Maps}
		\end{gathered} }
		\ ,\ F^* 
		\hspace{10pt} \vline \hspace{10pt}
		\begin{gathered}
			F: V \rightarrow W\\
			Fv = w
		\end{gathered}
		\ , \ \
		\begin{gathered}
			F^*: W^* \rightarrow V^*\\
			F^*(a) \in V^*
		\end{gathered}
		\ , \ \ 
		\mss{ \begin{aligned}
			\underline{ F^*(a) \cdot v } & \equiv a \cdot w \\
			& = a^T Fv \\
			\underline{ (F^T a) \cdot v } & = (F^T a)^T v 
		\end{aligned} }
		\ \Rightarrow \ \boxed{F^* = F^T}
		\\[10pt]
	& \bullet\ \begin{gathered}
			\underline{ f: V \otimes W^* \rightarrow\ \text{Hom}(V,W) }\\
			\text{\scriptsize(related to \((V^*)^* = \text{Hom}(V^*, \mathbb{R})\))}
		\end{gathered}
		\ \ \ \vline \ \ 
	 	{ \begin{gathered}
			B: W \rightarrow V \\
			a \in W^* \\
			B = f(v \otimes a)
		\end{gathered} }
		\ , \ \ 
		\mss{ \begin{aligned}
			f(v \otimes a) w & = (v a^T) w \\
			& = v (a \cdot w) \\
			Bw & = (a \cdot w) v 
		\end{aligned} }
		\hspace{20pt}
		\begin{aligned}
			& \ast\ \mss{W = V} \hs \Rightarrow\hs \mss{ Bv = (a\cdot v) v }\\[5pt]
			& \Rightarrow\ \boxed{ f: V \otimes V^* \rightarrow \mathbb{R} }
		\end{aligned}
\end{aligned}\)

%--------------------------------------------------------------------------------------------------------------------------------------
%--------------------------------------------------------------------------------------------------------------------------------------
%--------------------------------------------------------------------------------------------------------------------------------------
%--------------------------------------------------------------------------------------------------------------------------------------
\newpage
% Del
\section{Del}

% Nabla
\(\begin{aligned}[t]
	\nabla F & = 
		\left[ \begin{matrix}
			\hat{r}\\ 
			\hat{\theta}\\
			\hat{\phi} 
		\end{matrix} \right]
		\cdot
		\left[ \mss{ \begin{matrix}
			\\[-12pt]
			\tfrac{\partial}{\partial r}\\[5pt]
			\tfrac{1}{r} \tfrac{\partial}{\partial \theta}\\[5pt]
			\tfrac{1}{r\sin\theta} \tfrac{\partial}{\partial \phi}
		\end{matrix} } \right] 
		F
		=
		\left[ \ \begin{matrix}
			\cos{\phi}\sin{\theta} \hs \hat{x} + \sin{\phi}\sin{\theta} \hs \hat{y} + \cos{\theta} \hs \hat{z} \\[5pt]
			\cos{\phi}\cos{\theta} \hs \hat{x} + \sin{\phi}\cos{\theta} \hs \hat{y} - \sin{\theta} \hs \hat{z} \\[5pt]
			- \sin{\phi} \hs \hat{x} + \cos{\phi} \hs \hat{y}
		\end{matrix} \ \right]
		\cdot
		\left[ \mss{ \begin{matrix}
			\\[-12pt]
			\tfrac{\partial}{\partial r}\\[5pt]
			\tfrac{1}{r} \tfrac{\partial}{\partial \theta}\\[5pt]
			\tfrac{1}{r\sin\theta} \tfrac{\partial}{\partial \phi}
		\end{matrix} } \right] 
		F 
		\\[5pt]
	\left[\begin{matrix}
			\\[-12pt]
			\tfrac{\partial}{\partial x}\\[5pt]
			\tfrac{\partial}{\partial y}\\[5pt]
			\tfrac{\partial}{\partial z}
		\end{matrix}\right] 
		F
		& = \left[
			\mss{ \begin{array}{c}
				\displaystyle
				\cos{\phi}\sin{\theta} \dfrac{\partial}{\partial r} 
				+ \frac{\cos{\phi}\cos{\theta}}{r} \dfrac{\partial}{\partial \theta} 
				- \frac{\sin{\phi}}{ r \sin{\theta} } \dfrac{\partial}{\partial \phi} 
				\\[10pt]
				\displaystyle
				\sin{\phi}\sin{\theta} \dfrac{\partial}{\partial r} 
				+ \frac{\sin{\phi}\cos{\theta}}{r} \dfrac{\partial}{\partial \theta} 
				+ \frac{\cos{\phi}}{ r \sin{\theta} } \dfrac{\partial}{\partial \phi}
				\\[10pt]
				\displaystyle
				\cos{\theta} \dfrac{\partial}{\partial r} 
				- \frac{\sin{\theta}}{ r } \dfrac{\partial}{\partial \theta} 
				\\[10pt]
			\end{array} }
		\right]
		F 		
		\ = 
		\left[\mss{ \begin{array}{c}
			\displaystyle
			\dfrac{\partial r}{\partial x} \dfrac{\partial}{\partial r} 
			+ \dfrac{\partial \theta}{\partial x} \dfrac{\partial}{\partial \theta} 
			+ \dfrac{\partial \phi}{\partial x} \dfrac{\partial}{\partial \phi} 
			\\[10pt]
			\displaystyle
			\dfrac{\partial r}{\partial y} \dfrac{\partial}{\partial r} 
			+ \dfrac{\partial \theta}{\partial y} \dfrac{\partial}{\partial \theta} 
			+ \dfrac{\partial \phi}{\partial y} \dfrac{\partial}{\partial \phi} 
			\\[10pt]
			\displaystyle
			\dfrac{\partial r}{\partial z} \dfrac{\partial}{\partial r} 
			+ \dfrac{\partial \theta}{\partial z} \dfrac{\partial}{\partial \theta} 
			+ \dfrac{\partial \phi}{\partial z} \dfrac{\partial}{\partial \phi} 
			\\[10pt]
		\end{array} }\right] 
		F
		\ =
		\left[\begin{matrix}
			\hat{x}\\
			\hat{y}\\
			\hat{z}
		\end{matrix}\right]
		\cdot
		\left[\begin{matrix}
			\\[-12pt]
			\tfrac{\partial}{\partial x}\\[5pt]
			\tfrac{\partial}{\partial y}\\[5pt]
			\tfrac{\partial}{\partial z}
		\end{matrix}\right] 
		F
\end{aligned}\)

\vspace{15pt}
\(
	\begin{aligned}
		% Divergence
		\nabla \cdot \vec{A} \ & 
			= \ \tfrac{1}{r} \tfrac{1}{r \sin{\theta}} 
			\mss{ \left\langle 
				\dfrac{\partial}{\partial r} , \ 
				\dfrac{\partial}{\partial \theta} , \ 
				\dfrac{\partial}{\partial \phi} 
			\right\rangle }
			\ \cdot \ [ r \cdot r \sin{\theta} ]
			\mss{ \left\langle A_r , \ \frac{1}{r} A_\theta , \ \frac{1}{ r \sin{\theta} } A_\phi \right\rangle }
			\\[15pt]
		% Curl
		\nabla \times \vec{A} \ & 
			= \ \dfrac{1}{r} \dfrac{1}{ r \sin{\theta} } 
			\mss{ \begin{Vmatrix}
				\hat{r} 					 & r \hat{\theta} 			    & r \sin{\theta} \hat{\phi}\\[10pt]
				\dfrac{\partial}{\partial r} & \dfrac{\partial}{\partial \theta} & \dfrac{\partial}{\partial \phi}\\[10pt]
				A_r							 & r A_\theta				    & r \sin{\theta} A_\phi
			\end{Vmatrix} }
			= 
			\mss{ \begin{Vmatrix}
				\tfrac{\partial \hsvec{r}}{\partial r}  
					& \tfrac{\partial \hsvec{r}}{\partial \theta} 
					& \tfrac{\partial \hsvec{r}}{\partial \phi} 
					\\[10pt]
				\dfrac{\partial}{\partial r} & \dfrac{\partial}{\partial \theta} & \dfrac{\partial}{\partial \phi}\\[10pt]
				A_r							 & r A_\theta				    & r \sin{\theta} A_\phi
			\end{Vmatrix} }
	\end{aligned}
\)
\hfill \vline \hfill
\(
	% Hodge Star and Exterior Derivative
	\begin{aligned}
		\nabla f & = df\\
		A \cdot B & = {}^* \left[ (A_i \hs dx^i) \wedge {}^* (B_j \hs dx^j) \right]\\
		\nabla \cdot A & = {}^* d {}^* (A_i\hs dx^i)\\
		\nabla \times A & = 2{}^* d (A_i\hs dx^i)\\
	\end{aligned}
\)

\vspace{15pt}\noindent
% Double Cross Product
% A x (B x C)
\(
	\begin{array}[t]{r c l c l}
		% [A x (B x C)]^T
		[ \vec{A} \times (\vec{B} \times \vec{C}) ]_i & =
			& \vec{A} \cdot B_i \vec{C} - \vec{A} \cdot \vec{B} C_i 
			\\[5pt]
		[ \vec{A} \times (\vec{B} \times \vec{C}) ]^T & =
			& \vec{A}_r * \vec{B}_r \vec{C}_c - \vec{A}_r * \vec{B}_c \vec{C}_r
			\\[10pt]
		\vec{A} \times (\vec{B} \times \vec{C}) & = 
			& \boxed{ (A_r B_c) C_c - (A_r \ast B_c) C_c }
			\\[5pt]
		& = 
			& \underline{ (A \odot B) C - (A \cdot B) C}
			\\[5pt]
		\text{\scriptsize(\(A, B\) commute)} & = 
			& B (A \cdot C) - (A \cdot B) C 
			\\[5pt]
		\vec{\nabla} \times (\vec{\nabla} \times \vec{C}) & = 
			& \nabla (\nabla \cdot C) - (\nabla \cdot \nabla) C
			\\[5pt]
		[ \vec{A} \times (\vec{\nabla} \times \vec{C}) ]^T & =
			& \vec{A}_r ( \vec{\nabla}_r \vec{C}_c ) - ( \vec{A} \cdot \vec{\nabla} ) \vec{C}^T 
			\\[5pt]
		[ \vec{\nabla} \times (\vec{B} \times \vec{C}) ]_i & =
			& \vec{\nabla} \cdot B_i \vec{C} - \vec{\nabla} \cdot \vec{B} C_i 
			& =
			& \vec{C} \cdot \vec{\nabla} B_i + \vec{\nabla} \cdot \vec{C} B_i 
				- \vec{\nabla} \cdot \vec{B} C_i - \vec{B} \cdot \vec{\nabla} C_i 
			\\[5pt]
		[ \vec{\nabla} \times (\vec{B} \times \vec{C}) ]^T & =
			& \vec{\nabla}_r * (\vec{B}_r \vec{C}_c) - \vec{\nabla}_r * (\vec{B}_c \vec{C}_r)
			& =
			&  \mss{ \vec{C}_r * \vec{\nabla}_c \vec{B}_r + \vec{\nabla}_r * \vec{C}_c \vec{B}_r 
				- \vec{\nabla}_r * \vec{B}_c \vec{C}_r - \vec{B}_r * \vec{\nabla}_c \vec{C}_r }
	\end{array}
	\hspace{-220pt}
	% A x (B x C)
	\begin{aligned}[t]
		& \mss{ \vec{A} \times ( \vec{B} \times \vec{C} ) }
			= \mss{ \left[ \arraycolsep=0pt \begin{matrix}
				0 & -A_3 & A_2\\
				A_3 & 0 & -A_1\\
				-A_2 & A_1 & 0
			\end{matrix} \right] }
			\mss{ \left[ \arraycolsep=0pt \begin{matrix}
				0 & -B_3 & B_2\\
				B_3 & 0 & -B_1\\
				-B_2 & B_1 & 0
			\end{matrix} \right] }		
			\mss{ \left[ \begin{matrix}
				C_1\\ 
				C_2\\ 
				C_3
			\end{matrix} \right] }
			\\[5pt]
		& \hspace{5pt} = \mss{ \left[ \arraycolsep=2pt \begin{matrix}
			A_1 B_1 & A_2 B_1 & A_3 B_1\\
			A_1 B_2 & A_2 B_2 & A_3 B_2\\
			A_1 B_3 & A_2 B_3 & A_3 B_3 
		\end{matrix} \right] }
		\vec{C}
		- (A \cdot B) \mathbbm{1}_3 \hs \vec{C}
			\\[5pt]
		& \hspace{5pt} = (A_r B_c) C_c - (A_r \ast B_c) C_c \\[5pt]
		& \hspace{5pt} = \mss{ \underline{ (A^T {_r\otimes_c} B) C - (A^T B) C } }
			= \mss{ \underline{ (A \odot B) C - (A \cdot B) C } } 
	\end{aligned}
\)

\vspace{15pt}\noindent
% (A x B) x C
\(
	\begin{aligned}
		% (A x B) x C
		[ (\vec{A} \times \vec{B}) \times \vec{C} \hs\hs ]_i & = \vec{A} B_i \cdot \vec{C} - A_i \vec{B} \cdot \vec{C} 
			\\[5pt]
		(\vec{A} \times \vec{B}) \times \vec{C} & = \boxed{ (\vec{A}_r \vec{B}_c) \vec{C}_c - \vec{A}_c \vec{B}_r * \vec{C}_c } \\
		& = \underline{ (A^T {_r\otimes_c} B) C - (A B^T) C } \\[5pt]
		& = \underline{ (A \odot B) C - A (B \cdot C) } \\[5pt]
		\text{\scriptsize(\(B, C\) commute)} & = (A \cdot C) B - A (B \cdot C) 
			\\[5pt]
		(\vec{A} \times \vec{\nabla}) \times \vec{C} & 
			= ( \vec{A}_r \hsvec{\nabla}_c ) \vec{C}_c - \vec{A}_c ( \vec{\nabla}_r \cdot \vec{C}_c )
			\\[5pt]
		(\vec{\nabla} \times \vec{B}) \times \vec{C} & = ( \vec{\nabla}_r \hsvec{B}_c ) \vec{C} - ( \vec{\nabla}_c \vec{B}_r ) \vec{C} 
	\end{aligned}
	\hfill \vline \hfill
	\begin{aligned}
		% [(A x B) x C]^T
		& \mss{ [ (\vec{A} \times \vec{B}) \times \vec{C} \hs\hs ]^T } = \mss{ \left[ A_1\ A_2\ A_3 \right] }
			\mss{ \left[ \arraycolsep=0pt \begin{matrix}
				0 & -B_3 & B_2\\
				B_3 & 0 & -B_1\\
				-B_2 & B_1 & 0
			\end{matrix} \right] }
			\mss{ \left[ \arraycolsep=0pt \begin{matrix}
				0 & -C_3 & C_2\\
				C_3 & 0 & -C_1\\
				-C_2 & C_1 & 0
			\end{matrix} \right] }
			\\[5pt]
		& \hspace{5pt} = A^T 
			\mss{ \left[ \arraycolsep=2pt \begin{matrix}
				B_1 C_1 & B_2 C_1 & B_3 C_1\\
				B_1 C_2 & B_2 C_2 & B_3 C_2\\
				B_1 C_3 & B_2 C_3 & B_3 C_3 
			\end{matrix} \right] }
			- A^T \mathbbm{1}_3 \hs (B \cdot C)
			\\[5pt]
		& \hspace{5pt} = A_r (B_r C_c) - A_r (B_r \ast C_c)\\[5pt]
		& \hspace{5pt} = A^T (B \odot C) - A^T (B \cdot C)
	\end{aligned}
\)

%--------------------------------------------------------------------------------------------------------------------------------------
%
%
%
\newpage
Orthogonal Coord. Change:\\[5pt]
\(
	\begin{aligned}[t]
		\hsvec{v} = v^i \frac{\partial}{\partial x^i} & = v'^{\hs i} \frac{\partial}{\partial x'^{\hs i}} \\[5pt]
		v^i e_i & = v'^{\hs i} e'_{\hs i} \\
	\end{aligned}
	\hspace{25pt}
	\begin{aligned}[t]
		e^n & = e^n_{\ i} \hs dx^i\\
		e_n & = e_n^{\ i} \hs \tfrac{\partial}{\partial x^i}\\
	\end{aligned}
	\hspace{20pt}
	\begin{gathered}[t]
		\hsvec{v} = v^i \frac{\partial}{\partial x^i}
			= v^i \delta_i^{k}\frac{\partial}{\partial x^k}
			= \left(v^i \frac{\partial x'^{\hs j}}{\partial x^i}\right) 
			\left(\frac{\partial x^k}{\partial x'^{\hs j}} \frac{\partial}{\partial x^k}\right)
			\equiv v'^{\hs j} e'_j
			\\
		\bullet\ e'_j = \frac{\partial x^i}{\partial x'^{\hs j}} e_i
			\hspace{20pt} \bullet\ v'^{\hs j} = \frac{\partial x'^{\hs j}}{\partial x^i} v^i
			\hspace{5pt} , \hspace{5pt} dx'^{\hs j} = \frac{\partial x'^{\hs j}}{\partial x^i} dx^i
			\\
	\end{gathered}
\)

% \vspace{15pt}
% \(
% 	% Delta
% 	\begin{aligned}
% 		\delta^i_{\ j} e_i^{\ a} e^j_{\ a}
% 	\end{aligned}
% \)

\vspace{15pt}
\(
	% Metric
	\begin{aligned}
		g'_{ij} & = \frac{\partial x^m}{\partial y_i} \frac{\partial x^n}{\partial y_j} g_{mn}\\
		g'^{\hs ij} & = \frac{\partial y_i}{\partial x^m} \frac{\partial y_j}{\partial x^n} g^{mn}
	\end{aligned}
	\hspace{20pt}
	\begin{aligned}
		\delta_i^k & = g_{ij} g^{jk} =  \delta^i_k = g^{ij} g_{jk} \\[5pt]
		\tfrac{\partial x^k}{\partial y^i} \tfrac{\partial y^i}{\partial x^i} & 
			= \tfrac{\partial x^m}{\partial y_i} \tfrac{\partial x^n}{\partial y_j} \eta_{mn}
			\cdot \tfrac{\partial y^j}{\partial x^p} \tfrac{\partial y^k}{\partial x^q} \eta^{pq}
			\\
		& = \tfrac{\partial x^m}{\partial y_i} \eta_{mn}
			\cdot \tfrac{\partial y^k}{\partial x^q} \eta^{qn}
			\\
		& = \tfrac{\partial x^m}{\partial y_i} \eta_{mn}
			\cdot \tfrac{\partial y^k}{\partial x^m} \eta^{mn}
	\end{aligned}
\)

%--------------------------------------------------------------------------------------------------------------------------------------
%--------------------------------------------------------------------------------------------------------------------------------------
%--------------------------------------------------------------------------------------------------------------------------------------
%--------------------------------------------------------------------------------------------------------------------------------------
\newpage
% Frenet
\section{Frenet Equations}

% Frenet in derivatives of position
\(
    \begin{aligned}
		\mss{ a \cdot (b \times c) } & = \mss{ (a \times b) \cdot c) } \\[5pt]
		\mss{ a \times (b \times c) } & = \mss{ (c \cdot a) b - (b \cdot a) c } \\
		\mss{ (a \times b) \times c } & = \mss{ b (c \cdot a) - a (c \cdot b)  } \\[5pt]
		\mss{ (a \times b) \cdot (c \times d) } & = \mss{ a \cdot b \times (c \times d) }
			\\
		= \mss{ \left| \hs
				\left[ \begin{matrix}
					a \hs \cdot \\ b \hs \cdot 
				\end{matrix} \right]
				\left[ c \ d \right] \hs
			\hs \right| }
			& = \mss{ \left| \ \begin{matrix}
                \\[-11pt]
                \hsvec{a} \cdot \hsvec{c} & \hsvec{a} \cdot \hsvec{d} \\[5pt]
                \hsvec{b} \cdot \hsvec{c} & \hsvec{b} \cdot \hsvec{d}
                \\[-12pt]
                \hs
            \end{matrix} \ \right| }
            \\[5pt]
        \Aboxed{ \tfrac{dt}{ds} & = \tfrac{1}{v} }
    \end{aligned}
    \hfill
    \vline
    \hfill
    \begin{aligned}
        T & = \hat{v} = \tfrac{\hsvec{v}}{v} \\
        \tfrac{dT}{dt} & = \tfrac{ (\hsvec{v} \cdot \hsvec{v}) \hsvec{a} - (\hsvec{v} \cdot \hsvec{a}) \hsvec{v} }{v^3} 
            = \tfrac{ \hsvec{v} \times (\hsvec{a} \times \hsvec{v}) }{v^3}
            = \tfrac{ (\hsvec{v} \times \hsvec{a}) \times \hsvec{v} }{v^3}
            \\
        \lVert \tfrac{dT}{dt} \rVert & = \tfrac{ \sqrt{ v^2 a^2 - (\hsvec{v} \cdot \hsvec{a})^2 } }{v^2}
            = \tfrac{\lVert \hsvec{a} \times \hsvec{v} \rVert }{v^2}
            \hspace{5pt} , \hspace{10pt} \tfrac{dT}{ds} = k \hat{N}
            \\
        \hat{N} & = \tfrac{T'}{\lVert T' \rVert} 
            = \tfrac{ (\nhs\hsvec{v} \times \hsvec{a}) \times \hsvec{v} }{\lVert \hsvec{v} \times \hsvec{a} \rVert v} 
            = \hat{B} \times \hat{v}
            \\
        \hat{B} & = \tfrac{\hsvec{v} \times \hsvec{a}}{\lVert \hsvec{v} \times \hsvec{a} \rVert}
            = \widehat{v \times a}
            = \hat{v} \times \hat{N}
            \hspace{15pt} \underline{ \mss{(\hat{B} \cdot \vec{v} = 0)} }
            \\
        \tfrac{d\hat{B}}{dt} & = \tfrac{\hsvec{v} \times \dot{\hsvec{a}}}{\Vert \hsvec{v} \times \hsvec{a} \Vert }
            - \left[ \tfrac{\hsvec{v} \times \dot{\hsvec{a}}}{\Vert \hsvec{v} \times \hsvec{a} \Vert } \cdot \hat{B} \right] \hat{B}
            \hspace{5pt} , \hspace{5pt} \tfrac{dB}{ds} = \tau \hat{N}
			\\
		\tau & = \hat{N} \cdot \tfrac{d\hat{B}}{ds} 
			= \tfrac{ \hat{B} \cdot \dot{\hsvec{a}} }{ \Vert \nhs\hsvec{v} \times \hsvec{a} \Vert }
			= \tfrac{ (\nhs\hsvec{v} \times \hsvec{a}) \cdot \dot{\hsvec{a}} }{ \Vert \nhs\hsvec{v} \times \hsvec{a} \Vert^2 }
    \end{aligned}
    \hfill
    \boxed{
        \begin{aligned}
            \hsvec{a} & = a_T \hat{T} + a_N \hat{N} \\[5pt]
            a_T & = \hsvec{a} \cdot \hat{v} = \tfrac{dv}{dt} \\[5pt]
            a_N & = \tfrac{\Vert \hsvec{a} \times \hsvec{v} \Vert}{v} = \Vert \hsvec{a} \times \hat{v} \Vert \\[5pt]
            a^2 & = a_T^2 + a_N^2 = \Vert \tfrac{d\hsvec{v}}{dt} \Vert^2
        \end{aligned}
    }
\)

\vspace{15pt}\noindent
% Frenet trihedron - when curve is parametrized by arc length
\underline{Frenet Trihedron for Regular \textit{Parametrized} Curves}\\
\(
    \begin{aligned}
        & \text{\scriptsize Differentiable (in this book)}:\ C^\infty\\
        & \text{\scriptsize No singular pts. Order 0 (Regular)}:\ \hsvec{v}(t) \neq 0\\
        & \mss{ \bullet\ \Vert \hsvec{v}(t) \Vert = c \rightarrow 1
            \ \Rightarrow\ 
            \int_s \Vert \hsvec{v}(t) \Vert \hs dt = t = \Delta s 
            }
            \\
        & \hspace{10pt} \mss{ \rightarrow s:\ \hsvec{x}(t) = \hsvec{x}(s) } \\
        & \mss{ \bullet\ \tfrac{1}{2}\tfrac{d}{dt}(\hsvec{v} \cdot \hsvec{v}) = \boxed{ \hsvec{v} \cdot \hsvec{a} = 0 } } \\
        & \text{\scriptsize No singular pts. Order 1}:\ \hsvec{a}(t) \neq 0\\
        & \mss{ \bullet\ \text{Curvature, } k \neq 0 \text{ (see right)} }
			\hspace{10pt} \mss{ \bullet\ \text{Vertex, } k' = 0 }
    \end{aligned}
    \hfill\vline\hfill
    \begin{aligned}
        & 1 = \Vert \hsvec{t} \Vert = \Vert \hsvec{n} \Vert = \Vert \hsvec{b} \Vert
            \ ,\ \ 
            0 = \hsvec{t} \cdot \hsvec{n} = \hsvec{n} \cdot \hsvec{b} = \hsvec{b} \cdot \hsvec{t}
            \\
        & \bullet\ \hsvec{v}(s) = \hsvec{t}(s) \hspace{20pt} \boxed{ \mss{ (t = n \times b) } } \\
        & \bullet\ \hsvec{a}(s) = \boxed{ \hsvec{t'}(s) = k\mss{(s)} \hsvec{n}\mss{(s)} }
            ,\ k(s) \geq 0
            \hspace{10pt} \begin{gathered}
                \text{\scriptsize(can be L or R-handed)}\\[-8pt]
                \text{\scriptsize(can be neg. if in \(\mathbb{R}^2\))}
            \end{gathered}
            \\
        & \ast\ k(s) > 0 \text{ for well defined curve with \(\hat{n}\)}\\
        & \bullet\ \boxed{ \hsvec{b} = \hsvec{t} \times \hsvec{n} }
            ,\ \ \tfrac{d}{dt} (\hsvec{b} \cdot \hsvec{b}) = \hsvec{b} \cdot \hsvec{b'} = 0
            ,\ \ \ast\ \boxed{ \hsvec{b'}(s) = \tau\mss{(s)} \hsvec{n}\mss{(s)} }
            \\
        & \bullet\ \boxed{ \hsvec{n} = \hsvec{b} \times \hsvec{t} } 
            ,\ \ \ast\ \boxed{ \hsvec{n'}(s) = -k \hsvec{t} - \tau \hsvec{b} }
            ,\ \ \ast\ \mss{ \text{t-n pl.} = \text{osculating pl.} }
    \end{aligned}
\)

% More info on derivatives of frenet
\vspace{10pt} \noindent
\(\begin{aligned}
    & \bullet\ t''(s) = k' n - k^2 t - k\tau b 
        \hspace{20pt} \bullet\ b''(s) = \tau' n - \tau kt - \tau^2 b
        \hspace{20pt} \bullet\ n''(s) = -k't -\tau'b - (k^2 + \tau^2) n
        \\
    & \bullet\ |\tau| = \Vert b' \Vert 
        \hspace{20pt}
        \bullet\ \tau = -\tfrac{ ( t \times t' ) \cdot t'' }{k^2} = -\tfrac{ {t} \cdot ( {t'} \times t'' )}{\Vert t' \Vert^2} 
        \hspace{20pt} 
        \bullet\ k = \Vert t' \Vert = \tfrac{ ( b \times b' ) \cdot b'' }{\tau^2} 
        = \tfrac{ {b} \cdot ( {b'} \times b'' )}{\Vert b' \Vert^2}
        \\
    & \bullet\ n \Rightarrow k,\tau:
        \hspace{20pt}
        \ast\ \Vert n' \Vert^2 = k^2 + \tau^2
        \hspace{20pt}
        \ast\ \tfrac{ ( n \times n' ) \cdot n'' }{\Vert n' \Vert^2} = \tfrac{k' \tau - k \tau'}{k^2 + \tau^2}
        = \tfrac{ \tfrac{d}{ds} (k / \tau) }{(k/\tau)^2 + 1} = \tfrac{d}{ds} \arctan(k/\tau)
\end{aligned}\)

\vspace{15pt}\noindent
\(\begin{aligned}[t]
	% Indicatrix
    & \underline{ \text{Indicatrix of Tangents},\ \hsvec{t}(\theta\mss{(s)}) }:\\[5pt]
    & \bullet\ \hsvec{t}(\theta\mss{(s)}) = (\cos\theta, \sin\theta) = (x'\mss{(s)}, y'\mss{(s)})\\
    & \bullet\ \hsvec{t'}(\theta) = \underline{ \theta'(s)} (-\sin\theta, \cos\theta) = \underline{k(s)} \hsvec{n}\\
	& \bullet\ \theta\mss{(s)} = \arctan(y'/x')\\
	& \bullet\ \mss{\int_0^l} \hs k\mss{(s)} \hs\hs ds = \theta(s)\Big|^l_0 = 2\pi I_\text{rot. index}\\
	& \bullet\ k\mss{(s)} = \lim_{s \rightarrow 0} \tfrac{r\theta(s)}{s} \big|_{r=1}
		\hspace{15pt} \text{\scriptsize(See Gaussian \(K\))}
\end{aligned}\)
\hspace{20pt}
\(\begin{aligned}[t]
	% Local Canonical Form
    & \text{\underbar{Local Canonical Form at \(t=0\)}}:\\[5pt]
    & \bullet\ (\hat{t},\hat{n},\hat{b}) = (\hat{x},\hat{y},\hat{z})\\
    & \bullet\ \hsvec{r}(s) - \hsvec{r}(0) 
        \approx ( s - \tfrac{k^2 s^3}{6},\ \tfrac{k}{2} s^2 + \tfrac{k' s^3}{6},\ \tfrac{-k\tau}{6} s^3 )
        \\
    & \bullet\ \tau < 0 \ \Rightarrow\ \tfrac{dz}{ds} > 0
		\\[5pt]
	% Isoperimetric Ineq.
	& \text{\underline{Isoperimetric Inequality}}:\ 0 \leq l^2 - 4\pi A
		\\[5pt]
	% 4-Vertex Theor.
	& \text{\underline{Four-Vertex Theorem}}:\ \text{\scriptsize A simple closed curve has \(\geq\) 4 vertices}
\end{aligned}\)

% Cauchy-Crofton Formula
\vspace{15pt}\noindent
\(\begin{aligned}[t]
	& \text{\underline{Cauchy-Crofton Formula (measure of number of times lines intersect a curve)}}:\\[5pt]
	& \bullet\ \text{Tangent line at } (\rho, \theta) :\ x\cos\theta + y\sin\theta = \rho
		\hspace{15pt} \bullet\ \text{Curve } c:\ y=0,\ x \in (-l/2, l/2)
		\hspace{5pt} , \hspace{10pt} C = \mss{\sum}\hs c_i
		\\
	& \bullet\ \mss{\int}\ \text{\scriptsize Lines that cross } c 
		= \mss{ \int_0^{2\pi} \int_0^{|\cos\theta| l/2} } d\rho \hs d\theta = 2l
		\ \Rightarrow\
		\mss{ \int_0^{2\pi} \int_0^\infty } n_C \ d\rho \hs d\theta = 2l
\end{aligned}\)

%--------------------------------------------------------------------------------------------------------------------------------------
%--------------------------------------------------------------------------------------------------------------------------------------
%--------------------------------------------------------------------------------------------------------------------------------------
%--------------------------------------------------------------------------------------------------------------------------------------
\newpage
% Differential
\section{Jacobian/Differential, \( dF_{\alpha(0)} : \underline{ \mathbb{R}^n \rightarrow \mathbb{R}^m } \) }

\(\begin{aligned}
	& \bullet\ \boxed{ \alpha(0) = \beta(0) } \ \Rightarrow\ \underline{F(t=0)} 
		= F \circ \alpha \big|_{t=0} = F \circ \beta \big|_{t=0} 
		\\[5pt]
	& \bullet\ \boxed{ \alpha'(0) = \beta'(0) } \ \Rightarrow \tfrac{\partial x}{\partial \alpha_i} \big|_{t=0} 
		= \tfrac{\partial x}{\partial \beta_i} \big|_{t=0} 
		\cdot \bcancel{ \tfrac{d \beta_i / dt}{d \alpha_i / dt} \big|_{t=0} }
		\ \Rightarrow\ \nhs\nhs \boxed{ dF_{\alpha(0)}(\alpha'\mss{(0)}) = dF_{\beta(0)}(\beta'\mss{(0)}) }
		\hspace{5pt} \underline{ \text{\scriptsize(doesn't depend on \(\alpha\))} }
		\\[5pt]
	& \ast\ F = (f_0, f_1, \dots, f_m) 
		\ \Rightarrow\ \underline{ dF_{\alpha(0)} (\alpha'\mss{(0)}) } \hs \equiv\hs \tfrac{d}{dt}(F \circ \alpha) \big|_{t=0}
		= \left[ 
			\mss{ \arraycolsep=3pt \begin{matrix}
				\tfrac{\partial f_0}{\partial \alpha_0} & | & \dots \\[7pt]
				\tfrac{\partial f_1}{\partial \alpha_0} & F_{\alpha_1} & \dots \\
				\vdots & |
			\end{matrix} }
		\right]_{t=0}
		\left[ 
			\mss{ \begin{matrix}
				\tfrac{d \alpha_0}{dt}\\[7pt]
				\tfrac{d \alpha_1}{dt}\\
				\vdots
			\end{matrix} }
		\right]_{t=0}
		\hspace{-10pt} = \boxed{ J_F\mss{(0)} \cdot \alpha'\mss{(0)} }
		\\[-5pt]
	& \ast\ 
		\begin{gathered}
			\text{\scriptsize Surface Tangent}:\\[-5pt]
			\text{\scriptsize(see below)}
		\end{gathered}
		\
		\begin{aligned}
			q & = \gamma\mss{(t=0)} = (u\mss{(0)}, v\mss{(0)}) = X^{-1} \circ \alpha\mss{(0)}\\
			X\mss{(q)} & = X \circ \gamma \mss{(0)} = \alpha\mss{(0)} \in S
				\ \Rightarrow \ dX_{q}(\gamma'\mss{(0)}) = \alpha'\mss{(0)}
		\end{aligned}
		\\[10pt]
	& \bullet d(G \circ F)_p = dG_{F(p)} \circ dF_p
		\hspace{15pt} \bullet\ \begin{aligned}[t]
			& \underline{ \text{\scriptsize Regular \textbf{Value}, \(F(p)\)} }:\ 
				\text{\scriptsize onto}\ dF_{\forall p} / \text{\scriptsize Full Rank} 
				\\[-3pt]
			& \mss{\ast\ F: R^n \rightarrow R \ \Rightarrow\ dF_p \neq 0}
		\end{aligned}
		\hspace{15pt} \bullet\ \underline{ \text{\scriptsize Critical \textbf{Point}, \(p\)} }:\ !\text{\scriptsize onto}\ dF_p
\end{aligned}\)

\vspace{0pt}\noindent
% F is a Homeomorphism
\(
	\mss{ \begin{gathered}
		{F\ \text{is a}}\\[-2pt]
		\underline{\text{Homeomorphism}}\\[-2pt]
		\text{onto image}\ F(X)
	\end{gathered} }:\ 
	\begin{aligned}
		& \bullet\ F \ \text{\scriptsize is bijective between \(X\ \&\ F(X)\) }\\
		& \bullet\ F \ \text{\scriptsize is cont.}
			\hspace{10pt} \bullet\ F^{-1} \ \text{\scriptsize is cont.}
	\end{aligned}
\)
\hfill
% F is a Diffeomorphism
\(
	\mss{ \begin{gathered}
		{F\ \text{is a}}\\[-2pt]
		\text{\underline{Diffeomorphism}}\\[-2pt]
		\text{onto image}\ F(X)
	\end{gathered} }:\ 
	\begin{aligned}
		& \bullet\ F \in C^{\infty} \ \ \text{\scriptsize(cont. part. deri. of all orders)}\\
		& \bullet\ F^{-1} \in C^\infty \hspace{10pt} \bullet\ F \text{\scriptsize\ is a bijection}
			\\[-5pt]
		& \mss{ \Rightarrow\ F^{-1} \circ F = 1 \Rightarrow dF^{-1} \hs dF = 1 } 
			\hspace{5pt} \text{\scriptsize(\(\exists\) left-inv)}
			\\[-5pt]
		& \mss{ \Rightarrow\ F \circ F^{-1} = 1 \Rightarrow dF \hs dF^{-1} = 1 }
			\hspace{5pt} \text{\scriptsize(\(\exists\) right-inv)}
			\\
	\end{aligned}
\)

% Inverse Function Theorem
\vspace{0pt}\noindent
\(
	\begin{gathered}
		\text{\underline{Inverse Function}}\\
		\text{\underline{Theorem} (IFT)} 
	\end{gathered}:\ 
	\begin{aligned}
		& \bullet\ F: \underline{ \mss{ \mathbb{R}^n \rightarrow \mathbb{R}^n } } ,\ F \in C^{\infty} \\
		& \bullet\ \exists dF_p^{-1} \ \ \text{(\scriptsize sq. matrix \(dF_p\) is an isomorphism/non-zero det.)} 
	\end{aligned}
	\ \Rightarrow\ \exists F^{-1} \in C^\infty \hspace{10pt} \text{\scriptsize(locally at F(p))}
\)

%-----------------------------------------------------------------------------------------------------------------------------------
%-----------------------------------------------------------------------------------------------------------------------------------
\section{Surfaces, \(S:\ 
	X\mss{(q)} = X\mss{(u,v)} = \bigl( x\mss{(u,v)}, y\mss{(u,v)}, z\mss{(u,v)} \bigr) 
	= p \in S \subset \mathbb{R}^3
\)}

% Surfaces
\noindent
\(\begin{aligned}
	% Regular Parametrized Surface
	& \underline{ \text{Regular Parametrized Surface} } \\
	& - \forall p \in S,\ \underline{\exists X \in C^\infty} 
		,\ X: V_q\ \text{\scriptsize(neighborhood of q)} \rightarrow V_p \cap S
		\hspace{15pt} \text{\scriptsize(diff. parametrizations are possible, btw)}
		\\
	& - dX_q \ \ \text{\scriptsize is one-to-one = (maybe non sq.) matrix col. are lin. ind. 
		= any 2x2 \(\big| \text{sub-}J_X \big|\) \(\neq 0\)}
		\ \Rightarrow\ \exists(\text{\scriptsize tangent at all points})
		\\[5pt]
	% Regular Surface
	& \underline{ \text{Regular Surface ({\scriptsize is reg. param. surface})} } \\
	& - \begin{gathered}
			X \ \text{\scriptsize is a homeo. in } V_q\\[-3pt]
			( \text{\scriptsize or}\ X \ \text{\scriptsize is one-to-one} )
		\end{gathered}
		\ \rightarrow\ 
		\begin{gathered}
			\underline{ X^{-1} \in C^0 } \hspace{5pt} \text{\scriptsize(is cont.)}\\[-5pt]
			\mss{ \forall p \in S,\ X^{-1}(V_p) = V_q }
		\end{gathered}
		\ \Rightarrow\ \text{\scriptsize \(\exists\) no self-intersections; cont. = } 
		\begin{gathered}
			\text{\scriptsize doesn't depend on parametrization}\\[-7pt]
			\text{\scriptsize(see coor. change below)}
		\end{gathered} 
\end{aligned}\)

% Coordinate Change between Two Parametrizations
\vspace{10pt}\noindent
\(\begin{aligned}
	& \bullet\ \text{\small Coordinate Change, \(h\), between Two Param. %
		\underline{is a Diffeomorphism (need for diff. func. on \(S\))}}: 
		\\[5pt]
	& \ast\ X^{-1} \text{\scriptsize\ is a homeomorphism} 
		\ \rightarrow\ \underline{ h = X^{-1} \circ Y \text{\scriptsize\ is is a homeomorphism from \(Y\) to \(X\)} }
		\ \Rightarrow\Rightarrow\ \underline{ h^{-1} \ \text{\scriptsize is a homomorphism} }
		\\[5pt]
	& \ast\ p \in S \ ,\ \ p = Y\mss{(\epsilon, \eta)} = X\mss{(u, v)} = \bigl( x\mss{(u,v)}, y\mss{(u,v)}, z\mss{(u,v)} \bigr) 
		\ , \ \ \tfrac{\partial(x,y)}{\partial(u,v)} \neq 0 
		\hspace{10pt} \text{\scriptsize(can change axes to make this true)}
		% \hspace{10pt} \text{\scriptsize(see Impl. Func. Theo. below)}
		\\
	& \hspace{13pt} F\mss{(u,v,t)} = \bigl( x\mss{(u,v)}, y\mss{(u,v)}, z\mss{(u,v)} + t \bigr) 
		: \ F \mss{(u,v,t)},\ X\mss{(u,v)} \in C^\infty
		\ , \ \ \exists dF^{-1} \ \stackrel{(IFT)}{\Rightarrow}\ F^{-1} \in C^\infty
		\\[3pt]
	& \hspace{13pt} \mss{(u,v)} = \mss{X^{-1} \circ Y} \mss{(\epsilon, \eta)} = h \mss{(\epsilon, \eta)} 
		\stackrel{\sim}{=} ( \mss{F^{-1} \circ Y} ) \mss{(\epsilon,\eta)} 
		\ \Rightarrow\ \underline{ h \in C^\infty } 
		\ \Rightarrow\Rightarrow\ \underline{ h^{-1} \in C^\infty } 
		\hspace{10pt} \text{\scriptsize(same for \(Y^{-1} \circ X\))}
		\\[2pt]
	& \ast \text{\scriptsize\ Needed that \(X^{-1} \in\) 
		\textbf{\(C^0\) on a [3D] neigh. for every point} \( [\forall p \in S,\ X^{-1}(V_p) = V_q \stackrel{\sim}{=} F^{-1}(V_p)] \)
		, to avoid \( (t \neq 0,\ F^{-1} \circ Y \neq h) \)
		}
		\\[2pt]
	& \ast \text{\scriptsize\ Ex:}\ \ \mss{ 
			\begin{gathered}
				\gamma(t) = (\cos t, \sin 2t) \\[-2pt]
				\gamma(\mathbb{R}) = \alpha(I_1) = \beta(I_2)\\[-2pt]
				{(\infty \text{ - graph {\scriptsize \underline{not reg.}}})}
			\end{gathered}
			\ ,\ \arraycolsep=1pt \begin{array}{c c c c c}
				I_1 & = 
					& (-\tfrac{\pi}{2} , \tfrac{3\pi}{2}) 
					& = 
					& (-\tfrac{\pi}{2} , \tfrac{\pi}{2}) \cup \hs\hs\tfrac{\pi}{2}\hs\hs \cup (\tfrac{\pi}{2} , \tfrac{3\pi}{2})
					\\[2pt]
				I_2 & = 
					& (\tfrac{\pi}{2} , \tfrac{5\pi}{2}) 
					& = 
					& \underline{ (\tfrac{3\pi}{2} , \tfrac{5\pi}{2}) \cup \tfrac{3\pi}{2} \cup (\tfrac{\pi}{2} , \tfrac{3\pi}{2}) }
			\end{array}
			\ \Rightarrow\ 
			\begin{aligned}
				& \beta^{-1} \text{ is 1:1 but not cont. for any \textbf{[2D] neigh.} of } (0,0)\\[-2pt]
				& F^{-1}(x,y) = (t', u) \ \neq\ \beta^{-1}(x,y) \stackrel{\sim}{=} (t,0) \ \text{ near } (0,0) \\[-1pt]
				& \underline{\beta^{-1} \circ \alpha (I_1)} \text{ is 1:1 but not cont., so not diffeo.}
			\end{aligned}
		}
		\\
	&
\end{aligned}\) 

%--------------------------------------------------------------------------------------------------------------------------------------
%
%
%
\newpage

% Theorems about Reg Surfaces
\vspace{15pt}\noindent
\(\begin{aligned}	
	& \bullet\ \underline{ f \in C^\infty } 
		\ \Rightarrow\ \boxed{ \big( \hsvec{x}, f\mss{(\hsvec{x})} \big) \ \text{\scriptsize is a reg. surf.} }
		\\[10pt]
	% Regular Value Theorem
	& \bullet\ 
		\begin{aligned}
			& f : \mss{ \mathbb{R}^n \rightarrow \mathbb{R} }\\
			& f \mss{ (X) } = c \\[-7pt]
			& \underline{ \text{\scriptsize is a reg. val.} }
		\end{aligned}
		\ , \ \ \begin{aligned}
			& f \in C^\infty \\
			& F\mss{(X)} = ( \mss{ x_1,\hs ...\ , x_{n-1} , f(X) } )\\
			& \exists dF_p^{-1}
		\end{aligned}
		\ \ \stackrel{\text{(IFT)}}{\Rightarrow}\ \
		\begin{aligned}
			& \exists F^{-1} \in C^\infty \\
			& F^{-1}( \mss{ f_1,\hs ...\ , f_{n-1} , f(\nhs\hsvec{x}) } ) = X
		\end{aligned}
		\ , \ \ 
		\begin{aligned}
			& x_n\ \mss{ = f^{-1}_n }: \mss{ \mathbb{R}^n \rightarrow \mathbb{R} }\\
			& \underline{ x_n\ \mss{ = f^{-1}_n } \in C^\infty }
		\end{aligned}
		\\[5pt]
	& \hspace{10pt} \rightarrow\ 
		\begin{aligned}
			x_n & = \mss{ {f}^{-1}_n }( \mss{ x_1,\hs ...\ , x_{n-1} , f(\nhs\hsvec{x}) = c } )\\
			& = \underline{ \mss{ {f'}^{-1}_n } ( \mss{ x_1,\hs ...\ , x_{n-1} } ) }
		\end{aligned}
		\ \Rightarrow \ 
		\begin{aligned}
			& S = \underline{ ( \mss{ x_1,\hs ...\ , x_{n-1} , {f'}^{-1}_n } ) }
				\ \text{\scriptsize where}\ \mss{ f(\nhs\hsvec{x}) = c } 
				\\
			& S \ ==\ \text{\scriptsize Surface}\ f^{-1}{(c)}
		\end{aligned}
		\Rightarrow \boxed{ \begin{gathered}
			\underline{ \text{\scriptsize Regular Value Theorem} }\\
			\text{\scriptsize Surface}\ f^{-1}{(c)}\ \text{\scriptsize is reg.} 
		\end{gathered} }
		\\[5pt]
	% Implicit Function Theorem
	& \bullet\ \underline{ \tfrac{\partial(x,y)}{\partial(u,v)} \neq 0 }
		\ \Rightarrow\ { \pi_{\text{proj.}} \circ X } \mss{(u,v)} 
		\equiv  ( x\mss{(u,v)}, y\mss{(u,v)}, \bcancel{ z\mss{(x,y)} } )
		\stackrel{(IFT)}{\Rightarrow} \ (\mss{\pi \circ X})^{-1}\mss{( x, y )}
		= (u\mss{(x,y)}, v\mss{(x,y)})
		\\
		% \begin{aligned}
		% 	\text{\scriptsize(For coord. change, \(\gamma(t)\) isn't 1:1)} \hspace{10pt} 
		% 	& ( \text{\scriptsize\& } X \ \text{\scriptsize is \underline{one-to-one}} )
		% \end{aligned}
		% ,\ 
	& \hspace{10pt} 
		\begin{aligned}
			% Implicit Function Theorem
			& \ast\ X\mss{(u,v)} = ( x\mss{(u,v)}, y\mss{(u,v)}, \underline{ z\mss{(u,v)} } ) 
				\ \Rightarrow \ z( u(\mss{x,y}), v(\mss{x,y})) 
				= z \circ (\mss{\pi \circ X})^{-1}\mss{(x,y)} 
				= \boxed{ \begin{gathered}
					\underline{ \text{\scriptsize Implicit Func. Theor.} } \\[-5pt]
					\text{\scriptsize(locally orientable)}\\[-3pt]
					 f\mss{(x,y)} = z \ \mss{\in C^\infty} 
				\end{gathered} }
				\\[-3pt]
			% 1:1 -> Continuous
			& \ast\ \begin{gathered}
					\underline{ \text{\scriptsize Know } S \text{\scriptsize\ is reg. sur.} }\\[-2pt]
					X \text{\scriptsize\ is param?}
				\end{gathered}
				,\ \begin{gathered}
					X \in C^\infty\\[-2pt]
					dX_q \text{\scriptsize\ is 1:1} 
				\end{gathered}
				,\ \underline{ X \text{\scriptsize\ is 1:1} }
				\ \Rightarrow \ \underline{ (\mss{\pi \circ X})^{-1} \circ \pi } \circ X \mss{(u,v)} 
				= \underline{X^{-1}} \circ X \mss{(u,v)}
				\ \Rightarrow\ \boxed{X^{-1} \in C^0}
		\end{aligned}
\end{aligned}\)

\vspace{10pt}\noindent
\(
	% Surface Tangents
	\bullet\ \begin{gathered}
		\underline{ \text{Surface} }\\
		\underline{ \text{Tangent} }
	\end{gathered} 
	:\ \
	\begin{aligned}
		q & = \gamma\mss{(t=0)} = (u\mss{(0)}, v\mss{(0)}) = X^{-1} \circ \alpha\mss{(0)}\\
		X\mss{(q)} & = X \circ \gamma \mss{(0)} = \alpha\mss{(0)} \in S
			\ \Rightarrow \ dX_{q}(\gamma'\mss{(0)}) = \alpha'\mss{(0)}
			= \tfrac{\partial X}{\partial u}\mss{(q)} \hs u'\mss{(0)} + \tfrac{\partial X}{\partial v}\mss{(q)} \hs v'\mss{(0)}
	\end{aligned}
\)

\vspace{5pt}
\(\begin{aligned}
	% First Form
	& \textbf{{1st Fund. Form}}:\ \left< \alpha'\mss{(0)} , \alpha'\mss{(0)} \right> 
		= \mss{
			\left[ u' \ v' \right]
			\left[ \begin{matrix}
				X_u\\
				X_v
			\end{matrix} \right]
			\left[ X_u \ X_v \right]
			\left[ \begin{matrix}
				u'\\
				v'
			\end{matrix} \right]
		}
		= \begin{gathered}
			\mss{ \lVert X_u \rVert^2 (u')^2 + 2 \left< X_u, X_v \right> u'v'  + \lVert X_v \rVert^2 (v')^2 }\\[-2pt]
			\boxed{ \mss{ E (u')^2 + 2 F u'v'  + G (v')^2 } }
		\end{gathered}
		\\[-5pt]
	& \boxed{ 
			\begin{aligned}
				% X_u 
				\bullet\ X_u = 
					\alpha'( \begin{aligned}
						\scriptstyle u & \scriptstyle = t\\[-10pt]
						\scriptstyle v & \scriptstyle = v_0
					\end{aligned} )
					\\[-3pt]
				% X_v
				\bullet\ X_v = \alpha'( \begin{aligned}
						\scriptstyle u & \scriptstyle = u_0\\[-10pt]
						\scriptstyle v & \scriptstyle = t
					\end{aligned} )
			\end{aligned}
		}
		\hspace{5pt}
		% Line Element
		\bullet\ \begin{gathered}
			\underline{ \text{Line} }\\[-3pt]
			\underline{ \text{Element} }
		\end{gathered}
		: 
		\begin{aligned}
			ds & = \lVert \alpha'\mss{(t)} \rVert dt\\
			ds^2 & = \mss{c_i c_j g_{ij} dx^i dx^j}
		\end{aligned}
		\hspace{5pt}
		% Area Element
		\bullet\ \begin{gathered}
			\underline{ \text{Area} }\\[-3pt]
			\underline{ \text{Element} }
		\end{gathered}
		: \begin{aligned}
			\\[-2pt]
			& dA = { \lVert X_u \times X_v \rVert } du\hs dv \\[-1pt]
			& \hspace{2pt} = \begin{gathered}[t]
					\sqrt{ EG - F^2 }\\[-3pt]
					\mss{ ( \sqrt{1 - \cos^2} ) }
				\end{gathered} 
				\mss{ \hs du\hs dv }
				= \begin{gathered}[t]
					\mss{ \sqrt{\det(g_{ij})} \hs dx^1 ...\hs dx^n } \\[-5pt]
					\underline{ \text{\scriptsize(Volume too!!!)} }\\ 
				\end{gathered}
		\end{aligned}
\end{aligned}\)

%------------------------------------------------------------------------------------------------------------------------------------

\vspace{-2pt}\noindent
% Regular Curves
\(\ast\ \begin{aligned}[t]
	& \underline{ \text{Regular Curves, \(C \in R^3\) (instead of Regular Parametrized Curves)} }\\[5pt]
	& \bullet\ \forall p \in C
		,\ \exists \alpha \in C^\infty
		,\ \alpha: I_t \ \text{\scriptsize(neighborhood of \(t\))} \subset R 
		\rightarrow V_p \cap C \ \text{\scriptsize(neighborhood of \(p\))}
		\\
	& \bullet\ \forall t \in I\ ,\ \ d\alpha_t \ \text{\scriptsize is 1:1}
		\hspace{20pt} \bullet\ \alpha \text{\scriptsize\ is a homeo. in } I_t
		\\
	& \ast\ \text{\scriptsize Change of param. are homeomorphisms} 
		\ \Rightarrow\ \text{\scriptsize Properties like arc length, curvature, torsion, etc. aren't param. dependent}
\end{aligned}\)

% Coordinate Curves
\vspace{10pt}\noindent
\(
	\ast\ \underline{\text{Coordinate Curves}}:\ 
	\alpha\mss{(t)} = X \circ \gamma\mss{(t)} \ \big|\ 
	\gamma \in \{ ( u\mss{(t)}, v_0 ) ,\ ( u_0, v\mss{(t)} ) \}
	\hspace{15pt} \text{\scriptsize(maps of parallels and meridians)}
\) 

%------------------------------------------------------------------------------------------------------------------------------------

% Function on S
\vspace{15pt}\noindent
\(\begin{aligned}
	& \underline{ \text{Function, \(f: S \subset \mathbb{R}^n \rightarrow \mathbb{R}\)} }\\[5pt]
	& \bullet\ \big( \forall p \in S,\ \underline{ f(p) \neq 0 } \big)
		\ \Rightarrow\ \big( \forall p \in S,\ \underline{ f(p) > 0 } \big) 
		\text{ or } \big( \forall p \in S,\ \underline{ f(p) < 0 } \big)
		\\
	& \bullet\ \underline{ \textit{Differentiable on } S }:\ f \circ X \in C^\infty
		\hspace{10pt} \text{\scriptsize(doesn't depend on param./coord. change)}
		\\
	& \bullet\ \text{E.g., } X^{-1}\mss{(p)},\ \hsvec{v} \cdot p,\ |p-p_0|^2 
		\ \Rightarrow\ \boxed{X^{-1} \in C^\infty} \ , \ \ \boxed{U \ \text{\scriptsize is diffeo. to } X(U)}
\end{aligned}\)

% Function from S_1 to S_2
\vspace{10pt}\noindent
\(\begin{aligned}
	& \underline{ \text{Function, \(\phi: S_1 \rightarrow S_2\) is a Diffeomorphism from \(S_1\) to \(S_2\)} }
		\hspace{15pt} \bullet\ \boxed{ d\phi_p: T_{p}(S_1) \rightarrow T_{\phi p}(S_2) }
		\\[5pt]
	& \bullet\ \underline{ \textit{Differentiable}}:\ X_2^{-1} \circ (\phi \circ X_1) \in C^\infty
		\hspace{10pt} \text{\scriptsize(doesn't depend on param./coord. change)}
		\\
	% ?????
	& \bullet\ \underline{ \text{Differential Map} }: \ \ 
		\beta'(0) = d\phi_p(w) = d\phi_p \hs \alpha'\mss{(0)} = d\phi_p\hs dX_q (u'\mss{(0)}, v'\mss{(0)})^T 
		\hspace{25pt} \text{(p.85???)}
		\\
	% p.86 Inverse Function Theorem
	& \bullet\ \underline{ \text{Inverse Function Theorem} }:\ \phi \in C^\infty ,\ \exists d\phi_p^{-1} 
		\ \Rightarrow\ \phi^{-1} \in C^\infty \hspace{10pt} \text{\scriptsize(Diffeomorphism from \(S_1 \rightarrow S_2\)??????)}
\end{aligned}\)


%-------------------------------------------------------------------------------------------------------------------------------------



%----------------------------------------------------------------------------------------------------------------------------------
%
%
%
\newpage
% Gauss Map/Normals
\section{Gauss Map (Normals),\ \( N(p) = \tfrac{X_u \times X_v}{| X_u \times X_v |} 
= \tfrac{X_u \times X_v}{EG-F^2} : S \rightarrow S^2 \)}

\noindent
\(\begin{aligned}
	% Derivative of Gauss Map
	& N'\mss{(p)} = \underline{ dN_p \hs \alpha'\mss{(0)} }
		= \Big[ \begin{gathered}
			\mss{(dN_p)}\\[-6pt]
			\mss{ N_x \hs\hs N_y \hs\hs N_z }
		\end{gathered} \Big]
		\Big[ \begin{gathered}
			\mss{(dX_q)}\\[-6pt]
			\mss{ X_u \hs\hs X_v } ]
		\end{gathered}
		\mss{ \left[ \begin{gathered}
			u'\\[-2pt]
			v'	
		\end{gathered}\right] }
		\equiv 
		\Big[ \mss{ N_u \ N_v } ] 
		\mss{ \left[ \begin{gathered}
			u'\\[-2pt]
			v'	
		\end{gathered}\right] }
		\stackrel{\text{\scriptsize(see below)}}{=}
		\mss{ \Big[ b_1\ \ b_2 = aX_u + bX_v \Big] }
		\mss{ \Big[ (dN_p) \Big] }
		\mss{ \left[ \begin{gathered}
			c_1 = a u' + bu'\\[-2pt]
			c_2	
		\end{gathered}\right] }
		\\[3pt]
	% Orientation
	& \bullet\ \boxed{ S = f^{-1}(c) \ \Leftrightarrow\ { \textit{Orientated}} \hs\hs } 
		= \underline{ 
			\text{\scriptsize normals \(N(p)\) are in same dir. \((\pm 1)\)} 
			}
		= \boxed{ \exists \tfrac{\partial(\hat{u},\hat{v})}{\partial(u,v)} > 0 \text{ over all } S }
\end{aligned}\)

\vspace{10pt}\noindent
\(\begin{aligned}
	% Second Fundamental Form
	& {\textbf{2nd Fund. [Quadratic] Form}}:\ 
		\left< -dN_p(\alpha'\mss{(0)}), \alpha'\mss{(0)} \right> 
		= \underline{\left< \alpha'\mss{(0)}, -dN_p(\alpha'\mss{(0)}) \right>}
		\hspace{15pt} \text{\scriptsize(self-adjoint=orthog. eig)}
		\\[3pt]
	%% Normal Curvature = Second Form
	& \ast\ { \mss{ \left< N\mss{(s)}, \alpha'\mss{(s)}\right> = 0 } }
		\ \Rightarrow
		\mss{ \begin{aligned}
			& \boxed{ \left< N\mss{(s=0)}, \alpha''\mss{(0)} \right> }\\
			& = -\left< N'\mss{(0)}, \alpha'\mss{(0)} \right>
		\end{aligned} }
		= \begin{gathered}
			\text{\scriptsize(depends on \(\alpha'(0)\))}\\[-2pt]
			\underline{ - \left< dN_p \hs \alpha'\mss{(0)}, \alpha'\mss{(0)} \right> }
		\end{gathered}
		= \begin{gathered}
			\text{\scriptsize(\textbf{Normal Curvature} of \(\alpha\) at \(p\))}\\[-2pt]
			\boxed{ \left< N, kn \right> \mss{(p)} \equiv k_n\mss{(p)} }
		\end{gathered}
		= \ \begin{gathered}
			\text{\scriptsize\(k\) of \(\alpha\) from a}\\[-9pt]
			\text{\scriptsize\underline{normal (cross)}}\\[-6pt]
			\text{\scriptsize\underline{section} of \(S\)}
		\end{gathered}
		\\[3pt]
	%% Second Form in terms of efg
	& \ast\ \begin{aligned}[t]
			& \hspace{16pt} \underline{ \mss{ \left< -dN_p \hs \alpha', \alpha' \right> } } = \mss{ 
					-[ N_u \ N_v ] 
					\left[ \begin{gathered}
						u'\\[-5pt]
						v'
					\end{gathered} \right]
					[ u' \ v' ]
					\left[ \begin{gathered}
						X_u\\[-5pt]
						X_v
					\end{gathered} \right] 
				}
				= \mss{ \underline{ 
					\Big( \underbrace{ -\left< N_u, X_u \right> }_{e} , 
					\underbrace{ -\left< N_u , X_v \right> - \left< N_v , X_u \right> }_{2f = 2 \left< N_u , X_v \right> } , 
					\underbrace{ -\left< N_v , X_v \right> }_{g} \Big) 
					\cdot \Big( (u')^2, u'v' ,(v')^2 \Big)
				} }
				\\[-10pt]
			& \boxed{ \begin{aligned}
					k_n\mss{(p, \alpha')} & = \mss{ e (u')^2 + 2f u'v' + g (v')^2 }\\[-5pt]
					\text{\scriptsize(locally, \(\leq 2\) sol.)} & = \mss{ (Au' + Bv')(Cu' + Dv') }
				\end{aligned}
			}
		\end{aligned}
		\\[-2pt]
	% Tangent Plane	Principal Directions/Eigenbasis
	& \bullet\ \begin{gathered}
			\text{\scriptsize(\underline{Prin. dir.} at \(p\))}\\[-7pt]
			\text{\scriptsize\underline{Eigenbasis}}
		\end{gathered} 
		: \exists e_1, e_2 
		\hs\hs \big| \hs\hs \text{\scriptsize span\((e_1, e_2)\)} 
		= T_p(S) \hs \ni \hs 
		\begin{gathered}[b]
			\text{\scriptsize(see below)}\\[-5pt]
			\underline{ -dN_p(c_1e_1 + c_2e_2) = k_1 c_1 e_1 + k_2 c_1 e_2 }
		\end{gathered}
		\hspace{15pt} \begin{gathered}[b]
			\text{\scriptsize(\underline{Prin. curv.} at \(p\))}\\[-5pt]
			\text{\scriptsize(eigenvalues, \(k_1 \geq k_2\))}
		\end{gathered}
		\\[3pt]
	%% Euler's Formula for 2nd Form
	& \ast\ \underline{\text{Euler's Formula (for 2nd Form)}}:\ 
		\underline{ \left< -dN_p \hs \hsvec{t} , \hsvec{t} = \mss{e_1 \cos\theta + e_2 \sin\theta} \right> }
		= \boxed{ 
			\begin{aligned}[t]
				k_1 \cos^2\theta + k_2 \sin^2\theta & = k_n\mss{(p,\theta)} \\[-6pt]
				% \mss{e (u')^2 + 2f u'v' + g (v')^2} & =
			\end{aligned}
		}
		\\[5pt]
	% Gaussian Curvature, K
	& \bullet\ \begin{gathered}
			\textbf{{Gaussian Curvature}}:\\[-3pt]
			( \text{\scriptsize \(2n\) dim. !\(\Delta\) det. w/ orien. flip} )
		\end{gathered} 
		\ \ \boxed{ \begin{aligned}
			K\mss{(p)}\ & \mss{ = \det(dN_p) }\\[-6pt]
			& \mss{ = (-k_1)(-k_2) }
		\end{aligned} }
		\hspace{15pt}
		% Mean Curvature
		\bullet\ \text{\underline{Mean Curvature}}:\ \boxed{ H\mss{(p)} = \tfrac{-\text{Tr}(dN_p)}{2} = \tfrac{k_1 + k_2}{2} }
		\\[3pt]
	%& Quadradic Type for K
	& \ast\ \text{\scriptsize Planar: \(dN_p = 0\), Ellip.\(\rightarrow K > 0\), Para.\(\rightarrow K = 0\), ...} 
		\hspace{15pt} 
		\ast\ \mss{ K>0 \Rightarrow \exists V_p: p+T_p(S) \text{ !div.}\hs\hs V_p} 
		\ , \ \mss{ K<0 \Rightarrow \forall V_p: p+T_p(S) \text{ div.}\hs\hs V_p} 
		\\[3pt]
	% Interpretation of Gaussian Curvature
	& \ast\ \mss{ \begin{aligned}
			\text{(2D)} \hspace{7pt} | Tv \times Tw | & = |v \times w| k_1k_2 \\[-3pt]
			| dN_p X_u \times dN_p X_v | & = |X_u \times X_v| \cdot K \\[-3pt]
			\Aboxed{ A(N(R)) & = \iint_R K d\sigma }
		\end{aligned} }
		\ \Rightarrow\ K_{\neq 0} 
		= 
		\begin{gathered}
			\underline{ 
				\mss{ \lim_{\int dudv \rightarrow 0} \int | dN_p X_u \times dN_p X_v | \hs du dv } \ /\ \mss{ \int dudv } 
				}
				\\[-2pt]
			\mss{ \lim_{\int dudv \rightarrow 0} \int | X_u \times X_v | \hs du dv } \ /\ \mss{ \int dudv }
		\end{gathered}
		= \begin{gathered}
			\boxed{ \mss{ \lim_{A(R) \rightarrow 0} } \tfrac{ A(N(R)) }{ A(R) } = K } \\
			( \text{\scriptsize See Indi. of Tan. for \(k\)} )
		\end{gathered}
		\\
	% dN in basis of X_u, X_v
	& \bullet\ \begin{aligned}[t]
			& \underline{N_u, N_v \in T_p(S)} \ \Rightarrow\ dN_p \hs \alpha'\mss{(0)}
				= 
				\underline{ [ \mss{ N_u \ N_v } ] }
				\mss{ \left[ \begin{gathered}
					u'\\[-2pt]
					v'	
				\end{gathered}\right] }
				\equiv 
				\underline{
					[ \mss{X_u \ X_v} ] 
					\Big[ dN \Big] 
				}
				\mss{ \left[ \begin{gathered}
					u'\\[-2pt]
					v'	
				\end{gathered}\right] }
				=
				% [ \mss{X_u \ X_v} ] 
				% \nhs \Big[ \mss{e_1 \ e_2} \Big] \nhs
				\mss{ \left[ \begin{gathered}
					(X_u,X_v) \cdot e_1\\[-2pt]
					(X_u,X_v) \cdot e_2
				\end{gathered}\right] }^T
				\mss{ \left[ \arraycolsep=0pt \begin{matrix}
					-k_1 & 0\\
					0 & -k_2	
				\end{matrix} \right] } 	
				% \mss{ \left[ \begin{gathered}
				% 	-e_1-\\[-2pt]
				% 	-e_2-	
				% \end{gathered}\right] }
				\mss{ \left[ \begin{gathered}
					c_1 = e_1 \cdot (u',v') \\[-2pt]
					c_2
				\end{gathered}\right] }
				\\
			%% continuation
			& \begin{gathered}
					\text{\scriptsize{\underline{General Basis} for \(N_u, N_v\)}}\\[-5pt]
					% \text{\scriptsize{(Weingarten Eq.)}}
				\end{gathered}:\ 
				\mss{ \left[ \arraycolsep=3pt \begin{matrix}
					X_u \cdot N_u = -e & X_u \cdot N_v = -f\\
					X_v \cdot N_u = -f & X_v \cdot N_v = -g
				\end{matrix} \right] }
				= \mss{ \left[ \arraycolsep=2pt\begin{matrix}
					X_u^2 = E & X_u \cdot X_v = F\\
					X_v \cdot X_u = F & X_v^2 = G
				\end{matrix} \right] } 
				\Big[ dN \Big]
				\hspace{15pt} \underline{ \mss{ \begin{aligned}
					\left< N, X_{ij} \right> & = - \left< N_i, X_j \right> \\
					& = - \left< N_j, X_i \right>
				\end{aligned} } }
		\end{aligned}
		\\
	%% dN entries with efg,EFG
	& \ast\ \begin{gathered}
			\text{\scriptsize{(Weingarten Eq.)}}\\[-3pt]
			\boxed{ 
				\big[ dN \big] = \tfrac{-1}{EG-F^2}
				\mss{ 
					\left[ \arraycolsep=2pt\begin{matrix}
						G & -F\\
						-F & E
					\end{matrix} \right]
					\left[ \arraycolsep=2pt\begin{matrix}
						e & f\\
						f & g
					\end{matrix} \right]
				}
			}
		\end{gathered}
		\hspace{15pt}
		% Gaussian + Mean Curvature
		\ast\ \begin{gathered}
		\mss{ (k^2 - 2Hk + K = 0) } \\[-3pt]
		\boxed{ k_\pm = \mss{ H \pm \sqrt{H^2 - K} } } 
		\end{gathered} : \
		\boxed{ K = \tfrac{ eg - f^2 }{ EG - F^2 } }
		\hspace{5pt} , \hspace{5pt} 
		\boxed{ H = \tfrac{1}{2} \tfrac{ e G - 2fF + g E }{ EG - F^2 } }
\end{aligned}\)

\vspace{10pt}\noindent
% Umbilical Point
\(
	\underline{\text{Umbilical Point}}:\ \begin{gathered}
		p \in S \ \big|\ k_1 = k_2 \ \Rightarrow\ H^2 = K \\[-5pt]
		\text{\scriptsize(only spheres \& planes have all umb. pts.)}
	\end{gathered}
\)
\hspace{20pt}
%$ Asymptotic Directions
\( \underline{\text{Asymptotic Direction}}:\ k_n \mss{(p,\theta)} = 0 \)

\vspace{10pt}\noindent
\(\begin{aligned}
	% Conjugate Directions
	& \underline{\text{Conjugate Directions}}:\ 
		(\theta, \phi) \ \text{\scriptsize from \(e_1\)} \hs\hs \big| \hs\hs 
		\left< -dN_p\hs \hat{t}_1\mss{(\theta)} , \hat{t}_2 \mss{(\phi)}\right> 
		\equiv \boxed{ 0 = k_1 \cos\theta \cos\phi + k_2 \sin\theta \sin\phi }
\end{aligned}\)

\vspace{10pt}\noindent
\(\begin{aligned}
	% Dupin Indicatrix
	& \text{\underline{Dupin Indicatrix}}:\ 
		\left< -dN_p\hs\hat{t}, \hat{t} \hs \hs \right> = \pm \tfrac{1}{\rho^2} = k_n
		\ \Rightarrow\ 
		\begin{aligned}[t]
			\left< -dN_p(\rho\hat{t}\hs\hs), (\rho\hat{t}\hs\hs) \right> & 
				= k_1 \hs \underline{ \rho^2 \cos^2\theta } + k_2\hs \underline{ |k_n|^{-1} \sin^2\theta}
				\\
			\underline{\text{\scriptsize Conic Graph \( (\xi,\eta) \)}} & 
				= \boxed{ \underline{\xi^2}/k_1^{-1} + \underline{\eta^2}/k_2^{-1} = \pm 1 }
		\end{aligned}
		\\[-15pt]
	%% Asymptotic Direction with Dupin Indicatrix for Elliptic Curvature (There's none)
	& \bullet\ K > 0 \ \Rightarrow\ \forall \theta,\ k_n\mss{(\theta)} \neq 0,\ \mss{(\xi, \eta) = \text{ellipse}}
		\\[5pt]
	%% Asymptotic Direction with Dupin Indicatrix for Hyperbolic Curvature
	& \bullet\ K < 0 \ \Rightarrow\ \exists \theta_{\underline{1,2}} \ \big|\ 
		k_n\mss{(\theta)} = 0 = k_1 \cos^2\theta + k_2 \sin^2\theta,\
		\text{\scriptsize \( (\xi, \eta) = \) hyperbola, \(\theta_{\underline{1,2}}\) are asymptotes}
		\\[5pt]
	%% Conjugate Direction with Dupin Indicatrix
	& \bullet\ \text{Conj. Dir.}\ (\phi_1, \phi_2):\ \phi_{{2,1}}
		= \arctan \tfrac{-k_1 \cos\phi_{1,2}}{k_2 \sin\phi_{1,2}}
		= \arctan \tfrac{d\eta}{d\xi}\big|_{\theta = \phi_{1,2}}
\end{aligned}\)

%---------------------------------------------------------------------------------------------------------------------------------
%
%
%
\newpage
\noindent
\(\begin{aligned}
	% Line of Curvature
	& \underline{\text{Line of Curvature}}:\ \alpha\mss{(t)} \ \big|\ 
		N'\mss{(t)} = \underline{ dN_p \hs \alpha'\mss{(t)} } = \underline{ \lambda\mss{(t)} \hs \alpha'\mss{(t)} }
		\hspace{15pt} \text{\scriptsize(curve s.t. tangent is always in a princ. dir.)}
		\\[3pt]
	% Determinant Condition for Line of Curvature
	& \bullet\ [ \mss{ u'\ v' } ]
		\mss{ \left[ \begin{matrix}
			0 & 1\\
			-1 & 0
		\end{matrix} \right] }
		\Big[ {dN} \Big]
		\mss{ \left[ \begin{matrix}
			u'\\
			v'
		\end{matrix} \right] }
		= [ \mss{ -v'\ u' } ] \lambda\mss{(t)} 
		\mss{ \left[ \begin{matrix}
			u'\\
			v'
		\end{matrix} \right] }
		= 0
		\ \stackrel{\text{(expand)}}{\Rightarrow} \ 
		\boxed{
			\left\vert \mss{ \arraycolsep=2pt \begin{matrix}
				(v')^2 & -u'v' & (u')^2\\
				e & f=0 & g \\
				E & F=0 & G
			\end{matrix} } \right\vert = 0
		}
		\hspace{15pt}
		\mss{ ( X_u \cdot X_v = 0 \Rightarrow \boxed{ F=f=0 } ) }
		\\[3pt]
	%% Asymptotic Curves
	& \ast\ \begin{gathered}
			\underline{ \text{Asymp. Curve} }
		\end{gathered} :\ 
		\begin{aligned}
			& \alpha\mss{(t)} \hs\hs \big|\hs\hs 
				\mss{ 
					\lambda(t) = k_n \mss{(p,\theta)} 
					= \boxed{
						k_1 \cos^2\theta + k_2 \sin^2\theta = e (u')^2 + 2f u'v' + g (v')^2 = (Au' + Bv')(Cu' + Dv') = 0
					} 
				}
				\\
			& \mss{ 
				(ef-g),\ \underline{K < 0} 
				\ \Rightarrow\ \underline{ 0 = (Au' + Bv')(Au' + Dv') } :\ A^2 = e,\ A(B+D) = f,\ BD = g 
				\ \Rightarrow\ \boxed{ \exists \alpha_1, \alpha_2 }
				}
		\end{aligned}
		\\[3pt]
	& \ast\ \underline{K < 0,\ e=g=0} 
		\ \Leftrightarrow\ \boxed{ \alpha \circ (c, v\mss{(t)}) } \wedge \boxed{ \alpha \circ (u\mss{(t)}, c) }
		\ \ \text{\scriptsize are asympt. curves}
\end{aligned}\)

\vspace{15pt}\noindent
\(\begin{aligned}
	% Surface of Revolution
	& \text{\underline{Surface of Revolution}}:\ X\mss{(u,v)} 
		= \big( \rho\mss{(v)} \cos u ,\hs \rho\mss{(v)} \sin u,\hs z\mss{(v)} \big)
		\hspace{5pt} \big| \hspace{5pt} \alpha_u\mss{(v)} = f\big( z\mss{(v)} ,\ \rho\mss{(v)} \big)
		\hspace{5pt} , \hspace{5pt} \Vert \tfrac{d \alpha_u}{dv} \Vert =^* 1
		\\[5pt]
	% First Fundamental Form
	& \bullet\ \left< \alpha', \alpha' \right> 
		= \begin{gathered}[t]
			\underline{ \big[ \rho^2,\ 0,\ (\rho')^2 + (z')^2 =^* 1 \big] } \\
			\mss{(\rho' \rho'' + z' z'' = 0)}^*
		\end{gathered}
		\left[ \hs \mss{ \begin{matrix}
			(u')^2 \\
			2u'v'\\
			(v')^2
		\end{matrix} } \hs \right]
		\hspace{15pt}
		% Second Fundamental Form
		\bullet\ \left< N, \alpha'' \right> 
		= \underline{ \big[ -\rho z',\ 0,\ \rho'' z' - \rho' z'' \hs \big] } 
		\left[ \hs \mss{ \begin{matrix}
			(u')^2 \\
			2u'v'\\
			(v')^2
		\end{matrix} } \hs \right]
		\\
	% Eigenvalues and Curvature
	& \bullet\ \boxed{ k_1 = \tfrac{e}{E} = - \tfrac{z'}{\rho} }
		\ , \ \ \boxed{ k_2 = \tfrac{g}{G} = \rho'' z' - \rho' z'' }
		\ , \ \ \boxed{ K = - \tfrac{z' ( \rho'' z' - \rho' z'' ) }{\rho} =^* - \tfrac{\rho''}{\rho} }
\end{aligned}\)


\vspace{15pt}\noindent
\(\begin{aligned}
	% Graph of a Differential Function
	& \underline{ \text{Graph of a Differentiable Function} }:\ X\mss{(u,v)} = \big( u,v, z\mss{(u,v)} \big) 
		\hspace{20pt}
		% Normal
		\bullet\ N\mss{(p)} = \tfrac{(- z_u, -z_v, 1)}{\sqrt{z_u^2 + z_v^2 + 1}}
		\\
	% First Fundamental Form
	& \bullet\ \left< \alpha', \alpha' \right> 
		= \underline{ \big[ 1 + z_u^2,\ z_u z_v,\ 1 + z_v^2 \big] } 
		\left[ \hs \mss{ \begin{matrix}
			(u')^2 \\
			2u'v'\\
			(v')^2
		\end{matrix} } \hs \right]
		\hspace{15pt}
		% Second Fundamental Form
		\bullet\ \left< N, \alpha'' \right> 
		= \underline{ \tfrac{1}{\sqrt{z_y^2 + z_v^2 + 1}} \big[ z_{uu},\ z_{uv},\ z_{vv} \big] } 
		\left[ \hs \mss{ \begin{matrix}
			(u')^2 \\
			2u'v'\\
			(v')^2
		\end{matrix} } \hs \right]
		\\[5pt]
	% Local Form Hessian
	& \bullet\ z\mss{(0,0)} = p \ , \ \ N(p) = \mss{(0,0,1)} 
		\ \Rightarrow\ \underline{\text{Hessian}}:\ 
		k_n\mss{(p)} = \underline{ \big[ z_{xx},\ z_{xy},\ z_{yy} \big] } 
		\left[ \hs \mss{ \begin{matrix}
			x^2 \\
			2xy\\
			y^2
		\end{matrix} } \hs \right] 
		\ , \ \hsvec{v} = \mss{(x,y)} 
		\\[3pt]
	%% Hessian to Dupin Indicatrix
	& \ast\ \hsvec{v} = \mss{xe_1 + ye_2} 
		\ \Rightarrow\ \begin{gathered}[t]
			z\mss{(x,y)} - z\mss{(0,0)}\\[-5pt]
			\text{\scriptsize(\(p\) is non-planer!!)}
		\end{gathered} 
		= \tfrac{1}{2!} k_n\mss{(p)} + \mathcal{O}(r^3)
		\hs \approx\hs 
		\mss{ \begin{aligned}[t]
			\tfrac{1}{2}(z_{xx} x^2 + z_{yy} y^2) & = \epsilon\\
			k_1x^2 + k_2 y^2 & = 2\epsilon
		\end{aligned} }
		\rightarrow
		\mss{ \begin{gathered}[t]
			k_1\xi^2 + k_2 \eta^2 = \pm 1\\[-2pt]
			\boxed{ \text{\scriptsize(Dupin Indicatrix)} }
		\end{gathered} }
\end{aligned}\)

\vspace{10pt}\noindent
\(
	\begin{aligned}
		% Vector Field : w
		& \text{\underline{{\scriptsize(Diff., Tangent)} Vector Field over \(S\)}}:\ 
			\boxed{ w\mss{(p)} = a\mss{(u,v)} X_u + b\mss{(u,v)} X_v }
			\hspace{15pt} \big( \text{e.g.}\ \underline{ \gamma\mss{(t)} \rightarrow w_\gamma\mss{(p)} = u' X_u + v' X_v } \hs \big)
			\\[5pt]
		% Trajectory of w
		& \textit{\underline{Trajectory of \(w\)}}:\ 
			\alpha\mss{(t)} \subset S \ \big| \ \boxed{ \alpha\mss{(0)} = p,\ \alpha'\mss{(t)} = w( \alpha\mss{(t)} ) }
			% \hspace{20pt}
			\\
		% Local Flow of w
		& \textit{\underline{(Local) Flow of \(w\)}}:\ 
			\alpha\mss{(p,t)} \equiv \alpha_p\mss{(t)} \ \big| \ 
			\boxed{ \alpha_p\mss{(0)} = p }
			,\ \boxed{ \alpha_p'\mss{(t)} = w( \alpha_p\mss{(t)} ) }
			\ \Rightarrow\ \boxed{ \alpha_p\mss{(t)} = p + ( a_0^1\mss{(t)}, a_0^2\mss{(t)}, a_0^3\mss{(t)} )}
			\\[5pt]
		%% exists some local inverse that relates to the first integral of w
		& \bullet\ \mss{ \begin{aligned}
				& \boxed{ w\mss{(p_0)} \neq 0 }\\
				& w\mss{(p_0)} \cdot \hat{x} = |w|\\
				& \tilde{\alpha}_{p_0}\mss{(t)} = \alpha_{p_0}\mss{(t)} \big|_{x = x_0}
			\end{aligned} }
			\ \Rightarrow\
			\begin{aligned}
				& d\alpha_{p_0} = [ \mss{ \mathbbm{1}_3\ w(\alpha) } ]\\
				& d\tilde{\alpha}_{p_0} = [ \mss{ \mathbbm{1}_3\ w(\alpha) } ] 
					\left[ \mss{ \begin{matrix}
							0\\
						\mathbbm{1}_3
					\end{matrix} } \right]
					= \Big[ \mss{e_2\ e_3\ w(\alpha)} \Big]
			\end{aligned}			
			\ \Rightarrow\ 
			\mss{ \begin{aligned}
				& \det( d\tilde{\alpha}_{p_0} ) = w\mss{(p_0)} \neq 0 \\
				& \boxed{ \exists \tilde{\alpha}^{-1} :\ V_{\alpha(p_0)} \subset S \rightarrow V_{p_0} \big|_{x = x_0} } 
					\hspace{5pt} \text{(IFT)} 
					\\ 
				& \forall p \in \alpha_{p_0}(t)
					,\ \underline{g\mss{(p)}} \equiv \pi_t \circ \tilde{\alpha}^{-1}_{p_0} \mss{(p)} = p_0 
			\end{aligned} }
			\\[10pt]
		% First Integral of w
		& \textit{\underline{(Local) First Integral of \(w\)}}:\ 
			f\mss{(p)} \ \big|\ \forall p \in \alpha_{p_0}\mss{(t)}
			,\ \boxed{ f\mss{(p)} = c},\ \boxed{ df_{p} \neq 0 }
			\hspace{15pt} \big( \ \begin{gathered}
				\text{\scriptsize \(f(p) = \) arcdist\(( p_0, \underline{g(p)} )\)}\\[-8pt]
				\text{\scriptsize along \(S|_{x=x_0}\)}
			\end{gathered} \ \big)
			\\
		%% if w != 0, there always exists a first integral function
		& \bullet\ \boxed{ w\mss{(p_0)} } \neq 0 \hspace{5pt} \text{\scriptsize(see above)}
			\ \Rightarrow\ \big( \exists V_{p_0} \subset S \big) 
			\hs\big(\hs \forall p \in V_{p_0},\ \underline{ \exists f\mss{(p)} } \hs\big)
			\\[-10pt]
		%% there exists coordinate curves for the trajectories of two lin. ind. w_1 and w_2 
		& \bullet\ \mss{ w_1\mss{(p_0)} }\ \underline{ \neq }\ \mss{ Aw_2\mss{(p_0)}  } 
			,\ \phi\mss{(p_0)} 
			= \left[ \mss{ \begin{matrix}
				f_1\mss{(p_0)} = u_0\\
				f_2\mss{(p_0)} = v_0
			\end{matrix} } \right]
			\ \Rightarrow\ \big[ d\phi_p \big] 
			\big[ \mss{ w_1\mss{(p_0)}\ w_2\mss{(p_0)} } \big]
			= \left[ \mss{ \arraycolsep=2pt \begin{matrix}
				0 & a\\
				b & 0
			\end{matrix} } \right]
			\hs\hs \underline{ \neq }\ 0
			\ \stackrel{ \text{\scriptsize(IFT)} }{\Rightarrow} \ 
			\mss{ \begin{gathered}
				\exists \underline{X} : \underline{ \phi^{-1} }\mss{(u_0,v_0)} = p_0\\
				\underline{\ast}\ \boxed{ \begin{aligned}
					& X\mss{(u_0, v)} \in \alpha_1\\ 
					& X\mss{(u, v_0)} \in \alpha_2
				\end{aligned} }
			\end{gathered} }
			\\[-10pt]
		& \bullet\ w_1 \equiv X_u,\ w_2 \equiv -\tfrac{X_u \cdot X_v}{X_u \cdot X_u} X_u + X_v 
			\ \Rightarrow\ \boxed{ 
				\exists (w_1,w_2) 
				\big( w_2\mss{(p_0)} \cdot w_2\mss{(p_0)} = 0 ,\ \underline{ \exists \ast } \big)
			}
			\\[5pt]
		& \bullet\ \underline{K<0} \ \Rightarrow\ \underline{ \exists \alpha_1,\alpha_2 (k_n = 0) }
			\ \rightarrow\ \underline{ \exists (w_1, w_2) (\exists \ast) }
			\hspace{15pt}
			\ast\ k_1 \neq k_2,\ \exists (\alpha_1,\alpha_2)_{k_n} 
			\ \rightarrow\ \underline{ \exists (w_1, w_2) (\exists \ast) }
	\end{aligned}
\)

%----------------------------------------------------------------------------------------------------------------------------------
%
%
%
\newpage
\(\begin{aligned}
	% Direction/Ray Field
	& \underline{\textit{Direction/Ray/Line Field}}:\ \boxed{ r_w = c_{\neq0} \big( b\mss{(u,v)}, -a\mss{(u,v)} \big) }
		\ \rightarrow\ \tfrac{y'}{x'} = \tfrac{-a}{b}
		\\
	% Orthogonal Ray Field
	& \underline{\textit{Orthogonal Field to \(r\)}}:\ \boxed{ 
		\overline{r}_w \equiv r_{\overline{w}} : \ 
		\overline{w} \cdot w = (\overline{a} X_u + \overline{b} X_v ) \cdot (aX_u + bX_v) = 0 
		}
		\\[5pt]
	%% example of orthog ray field
	& \text{\scriptsize E.g.}:\ 
		\begin{gathered}
			X\mss{(q)} = (u,v,u^2 - v^2)\\
			\gamma\mss{(t)} : \mss{u^2 - v^2 = c} \rightarrow \tfrac{v'}{u'} = \tfrac{-u}{v}
		\end{gathered} 
		\ \Rightarrow\ 
		\begin{aligned}
			& w_\gamma = u'\mss{(t)}X_u + v'\mss{(t)}X_v \hs\stackrel{\rightarrow}{=}\hs v X_u - u X_v\\
			& \overline{w}_\gamma \cdot w_\gamma = \overline{a} v - \overline{b} u 
				= \underline{ u'\mss{(\hs \overline{t} \hs)} v - v'\mss{(\hs \overline{t} \hs)} u = 0 }
		\end{aligned}
		\ \Rightarrow\
		\begin{aligned}
			& \overline{\gamma}\mss{(\hs \overline{t} \hs)} :
				\underline{ u\mss{(\hs \overline{t} \hs)} v\mss{(\hs \overline{t} \hs)} = c }
				\\
			& X_c = \mss{ (u, \tfrac{c}{u}, u^2 - \tfrac{c^2}{u^2}) }
		\end{aligned}
\end{aligned}\)

% Skipped Ruled/Minimal surfaces Sec 3.5

%--------------------------------------------------------------------------------------------------------------------------------
%--------------------------------------------------------------------------------------------------------------------------------
\noindent
% Intrinsic Geometry of Surfaces
\section{Intrinsic Surface Geometry}

% Surface Trihedron
\noindent
\(
	% Crhistoffel Symbols 
	\begin{gathered}
		\underline{ \text{Christoffel Symbols, \(\Gamma\)} }\\[-5pt]
		\text{\scriptsize(For Surf. Trihe., \(X_u, X_v, N\))}
			\\[5pt]
		\mss{ 
			\begin{gathered}
				(R)\\
				\left[ \begin{matrix}
						\Gamma_{11}^1 & \Gamma_{11}^2 & e \\[2pt]
						\Gamma_{12}^1 & \Gamma_{21}^2 & f \\[2pt]
						\Gamma_{22}^1 & \Gamma_{22}^2 & g 
					\end{matrix} \right]
					\\
				\left[ \begin{matrix}
						a_{11} & a_{21} & 0\\
						a_{12} & a_{22} & 0\\
					\end{matrix} \right]
			\end{gathered}
			\begin{gathered}
				(c)\\
				\left[ \begin{matrix}
						X_u\\
						X_v\\
						N
					\end{matrix} \right] 
			\end{gathered}
			}
			=
			\mss{ \begin{gathered}
				(Rc)\\
				\left[ \begin{matrix}
						X_{uu}\\
						X_{uv}\\
						X_{vv}
					\end{matrix} \right]
					\\
				\left[ \begin{matrix}
						N_{u}\\
						N_{v}
					\end{matrix} \right]
			\end{gathered} }
	\end{gathered}
	\hfill
	\begin{aligned}
		% Weingarten Eq.		
		\underline{ \big[ dN^T \ 0 \big] } & = 
			\underline{ \mss{ 
				\tfrac{-1}{EG-F^2} 
				\left[ \arraycolsep=2pt \begin{matrix}
					e & f\\
					f & g
				\end{matrix} \right]
				\left[ \arraycolsep=2pt \begin{matrix}
					G & -F & 0\\
					-F & E & 0
				\end{matrix} \right] 
			} }
			\hspace{15pt}
			\underline{ \text{\scriptsize(Weingarten Eq.)} }
			\\[5pt]
		% Christoffel Symbols
		(Rcc^T) \hspace{5pt} \mss{
			\left[ \arraycolsep=2pt \begin{matrix}
				\Gamma_{11}^1 & \Gamma_{11}^2 & e\\[2pt]
				\Gamma_{21}^1 & \Gamma_{12}^2 & f\\[2pt]
				\Gamma_{22}^1 & \Gamma_{22}^2 & g
			\end{matrix} \right]
			\left[ \arraycolsep=2pt \begin{matrix}
				E & F & 0\\
				F & G & 0\\
				0 & 0 & 1
			\end{matrix} \right]
			}
			& =
			\mss{ \left[ \arraycolsep=4pt \begin{matrix}
				X_{uu} \cdot X_u & X_{uu} \cdot X_v & e\\
				X_{uv} \cdot X_u & X_{vu} \cdot X_v & f\\
				X_{vv} \cdot X_u & X_{vv} \cdot X_v & g
			\end{matrix} \right] }
			= 
			\mss{ \left[ \arraycolsep=0pt\begin{matrix}
				\tfrac{1}{2} E_u & F_u - \tfrac{1}{2} E_v  \\[2pt]
				\tfrac{1}{2} E_v & \tfrac{1}{2} G_u \\[2pt]
				F_v - \tfrac{1}{2} G_u  & \tfrac{1}{2} G_v 
			\end{matrix} 
			\hspace{2pt} \right| \hspace{-2pt}
			\left. \begin{matrix}
				e\\[2pt]
				f\\[2pt]
				g	
			\end{matrix}\right] }
			\\
		\begin{gathered}
				(\Gamma)\\[-2pt]
				\mss{ \boxed{\Gamma = f(E,F,G)} }
			\end{gathered}
			\hspace{10pt} 
			\underline{ \mss{ \left[ \begin{matrix}
				\Gamma_{11}^1 & \Gamma_{11}^2 \\[2pt]
				\Gamma_{12}^1 & \Gamma_{21}^2 \\[2pt]
				\Gamma_{22}^1 & \Gamma_{22}^2 
			\end{matrix} \right] } }
			& = 
			\underline{ \mss{ 
				\left[ \arraycolsep=0pt\begin{matrix}
					\tfrac{1}{2} E_u & F_u - \tfrac{1}{2} E_v \\[2pt]
					\tfrac{1}{2} E_v & \tfrac{1}{2} G_u \\[2pt]
					F_v - \tfrac{1}{2} G_u  & \tfrac{1}{2} G_v
				\end{matrix} \right] 
				\left[ \arraycolsep=1pt \begin{matrix}
					G & -F\\
					-F & E
				\end{matrix} \right] 
				\tfrac{1}{EG-F^2} 
			} }
	\end{aligned}
\)

\vspace{15pt}\noindent
% Gauss Compatibility Equations
\underline{Gauss and Mainardi-Codazzi Compatibility Equations}:\\[10pt]
\(
	\begin{gathered}
		\partial_u \mss{ \left[ \begin{matrix}
				X_{uv}\\
				X_{vv}\\
				N_v
			\end{matrix} \right] }
			=
			\mss{ \partial_v \mss{ \left[ \begin{matrix}
				X_{uu}\\
				X_{vu}\\
				N_u
			\end{matrix} \right] }
			\ \rightarrow\
			\partial_u \mss{ \left[ \begin{matrix}
				-r_2-\\
				-r_3-\\
				-r_5-
			\end{matrix} \right] } Rc
			=
			\partial_v \mss{ \left[ \begin{matrix}
				-r_1-\\
				-r_2-\\
				-r_4-
			\end{matrix} \right] } Rc 
			\ \rightarrow\
			\mss{ \left[ \begin{matrix}
				-r_2-\\
				-r_3-\\
				-r_5-
			\end{matrix} \right] } 
			\left[
				\partial_u R
				+ 
				R 
				\mss{ \left[ \begin{matrix}
					-r_1-\\
					-r_2-\\
					-r_4-
				\end{matrix} \right] } 
				R
			\right]
			c
			=
			\mss{ \left[ \begin{matrix}
				-r_1-\\
				-r_2-\\
				-r_4-
			\end{matrix} \right] }
			\left[ 
				\partial_v R
				+ 
				R 
				\mss{ \left[ \begin{matrix}
					-r_2-\\
					-r_3-\\
					-r_5-
				\end{matrix} \right] }
				R 
			\right] 
			c
			}
			\\
		\boxed{ 
			\partial_u 
			\mss{ \left[ \begin{matrix}
				-r_2-\\
				-r_3-\\
				-r_5-
			\end{matrix} \right] } 
			R
			-
			\partial_v
			\mss{ \left[ \begin{matrix}
				-r_1-\\
				-r_2-\\
				-r_4-
			\end{matrix} \right] }
			R
			= 
			\mss{ \left[ \begin{matrix}
				-r_1-\\
				-r_2-\\
				-r_4-
			\end{matrix} \right] }
			R 
			\mss{ \left[ \begin{matrix}
				-r_2-\\
				-r_3-\\
				-r_5-
			\end{matrix} \right] }
			R 
			-
			\mss{ \left[ \begin{matrix}
				-r_2-\\
				-r_3-\\
				-r_5-
			\end{matrix} \right] }
			R 
			\mss{ \left[ \begin{matrix}
				-r_1-\\
				-r_2-\\
				-r_4-
			\end{matrix} \right] }
			R 
			}
			\hspace{15pt} \left(\hs \mss{ \begin{gathered}
				\underline{\text{lin. ind.}}\\[-3pt]
				Ac - Bc = 0 \\[-3pt]
				\rightarrow A = B
			\end{gathered} } \hs\right)
			\\
		A = \mss{ \left[ \begin{matrix}
				\partial_u R_2 - \partial_v R_1 \\[2pt]
				\partial_u R_3 - \partial_v R_2\\[2pt]
				\partial_u R_5 - \partial_v R_4 
			\end{matrix} \right] }
			= 
			\mss{ \left[ \arraycolsep=2pt \begin{matrix}
				\Gamma_{11}^1 & \Gamma_{11}^2 & e \\[2pt]
				\Gamma_{21}^1 & \Gamma_{21}^2 & f \\[2pt]
				a_{11} & a_{12} & 0
			\end{matrix}\right] } 
			\mss{ \left[ \arraycolsep=2pt \begin{matrix}
				\Gamma_{12}^1 & \Gamma_{12}^2 & f \\[2pt]
				\Gamma_{22}^1 & \Gamma_{22}^2 & g \\[2pt]
				a_{12} & a_{22} & 0
			\end{matrix} \right] } 
			-
			\mss{ \left[ \arraycolsep=2pt \begin{matrix}
				\Gamma_{21}^1 & \Gamma_{12}^2 & f \\[2pt]
				\Gamma_{22}^1 & \Gamma_{22}^2 & g \\[2pt]
				a_{12} & a_{22} & 0
			\end{matrix} \right] }
			\mss{ \left[ \arraycolsep=2pt \begin{matrix}
				\Gamma_{11}^1 & \Gamma_{11}^2 & e \\[2pt]
				\Gamma_{12}^1 & \Gamma_{21}^2 & f \\[2pt]
				a_{11} & a_{21} & 0
			\end{matrix} \right] } 
			\hspace{15pt}
			\mss{ \begin{aligned}
				& \ast A_{11}: ea_{12} - fa_{11} = \boxed{ FK = f_{11}(\Gamma) }\\
				& \ast \left[ \arraycolsep=2pt \begin{matrix}
						A_{11} & A_{12}\\
						A_{21} & A_{22}
					\end{matrix} \right]:
					f(\Gamma) = 
					\left[ \arraycolsep=2pt \begin{matrix}
						FK & -EK\\
						GK & -FK
					\end{matrix} \right]
			\end{aligned} }
	\end{gathered}
\)

\vspace{10pt}\noindent
% cont.
\(\begin{aligned}
	\ast\ F = 0 \ \Rightarrow\ GK = \mss{ fa_{12} - ga_{11} } & = -\tfrac{1}{2} \tfrac{ G_{uu} E - E_u G_u + E_{vv} E - E_v^2 }{E^2} 
		+ \tfrac{G_u^2 + G_v E_v}{4GE} - \tfrac{E_v^2 + E_u G_u}{4E^2}
		\\
	\Aboxed{ 
		K|_{F=0} & = -\tfrac{1}{2\sqrt{EG}} \left[ \left( \tfrac{E_v}{\sqrt{EG}} \right)_v 
			+ \left( \tfrac{G_u}{\sqrt{EG}} \right)_u  \right]
		}
\end{aligned}\)
	% 	\\ 
	% & \left[ \begin{matrix}
	% 	\Gamma_{11}^1 & \Gamma_{11}^2 & e \\[2pt]
	% 	\Gamma_{12}^1 & \Gamma_{21}^2 & f \\[2pt]
	% 	\Gamma_{22}^1 & \Gamma_{22}^2 & g 
	% \end{matrix} \right]
	% \\
	% & \left[ 
	% 	\hs dN^T \ \ \begin{gathered}
	% 		0\\[-3pt]
	% 		0
	% 	\end{gathered} 
	% \right]

\vspace{15pt}\noindent
\(\begin{aligned}
	% Isometry
	& \text{\underline{Isometric} Map}\ \ \text{\scriptsize\underline{Diffeo.}}\ \phi: S \rightarrow \overline{S}:\ 
		\ \mss{ \begin{aligned}
			\forall p & \in S \\[-4pt]
			\forall v & \in T_p(S)
		\end{aligned} }
		\ ,\ \boxed{ \Vert \beta'\Vert^2_{\phi(p)} = \Vert d\phi_p \hs v \Vert^2 
		= \left< v, v \right>_p = \Vert v \Vert_p^2 = \Vert \alpha' \Vert^2_p }
		\\
	%% Local and Global Isometry
	& \bullet\ \underline{\text{\scriptsize Local Isom.}}:\ \forall p, V_p \in S,\ \exists \phi_p: S \rightarrow \overline{S} 
		\hspace{15pt} \text{\scriptsize(Local + Diffeo. \(\phi\ =\)\ \underline{Global Isom.})}
		\\
	%% X \circ \overline{X}^{-1} is a local isom if the coeff. of 1st fund form is the same
	& \bullet\ \mss{ \big( X : U \rightarrow S \big) } :
		\big( \mss{ \overline{X} : U \rightarrow \overline{S} }
		,\ \underline{ E = \overline{E} ,\ F = \overline{F} ,\ G = \overline{G} }
		,\ \boxed{ \phi = \overline{X} \circ X^{-1} } \big)
		\hs\hs \stackrel{\Leftarrow}{\hs\hs \Rightarrow} \ \Vert d\phi_p \hs v \Vert^2 = \left< v, v \right>_p
		\\
	%% Converse of above
	& \bullet\ \mss{ \big( X : U \rightarrow S \big) } :
		\Vert d\phi_p \hs v \Vert^2 = \left< v, v \right>_p
		\hs\hs \stackrel{\Leftarrow}{\hs\hs \Rightarrow} \
		\big( \mss{ \overline{X} : U \rightarrow \overline{S} }
		,\ \boxed{ \overline{X} = \phi \circ X }
		,\ \underline{ E = \overline{E} ,\ F = \overline{F} ,\ G = \overline{G} } \big)
		\\[5pt]
\end{aligned}\)

\vspace{5pt}\noindent
\(\begin{aligned}		
	% Conformal Map
	& \text{\underline{Conformal Map}}\ \ \text{\scriptsize Diffeo.}\ \phi: S \rightarrow \overline{S}:\ 
		\ \forall v \in T_p(S)
		,\ \boxed{ \Vert d\phi_p \hs v \Vert^2 = \lambda^2\mss{(p)} \left< v, v \right>_p }
		\hspace{15pt} \begin{gathered}
			\text{\scriptsize(preserve \(\cos\theta\))}\\[-6pt]
			( \mss{\forall p,\ \lambda \neq 0,\ \exists\lambda'} )
		\end{gathered}
		\\
	%% Local and Global Isometry
	& \bullet\ \underline{\text{\scriptsize Locally Conf.}}:\ \forall p, V_p \in S,\ \exists \phi_p: S \rightarrow \overline{S} 
		\hspace{15pt} \text{\scriptsize(Locally + Diffeo. \(\phi\ =\)\ \underline{Globally Conf.})}
		\hspace{10pt} \big( \hs \begin{gathered}
			\text{\scriptsize hard: all reg. surf. are loc.}\\[-9pt]
			\text{\scriptsize conf. as \(\exists\)``isothermal'' map}
		\end{gathered} \hs \big)
		\\
	%% X \circ \overline{X}^{-1} is locally conf. if the coeff. of 1st fund form is the same
	& \bullet\ \mss{ \big( X : U \rightarrow S,\ \overline{X} : U \rightarrow \overline{S} \big) }
		,\ \big( \underline{ E = \lambda^2 \overline{E} ,\ F = \lambda^2 \overline{F} ,\ G = \lambda^2 \overline{G} }
		,\ \boxed{ \phi = X \circ \overline{X}^{-1} } \big)
		\ \Rightarrow\ \Vert d\phi_p \hs v \Vert^2 = \lambda^2 \left< v, v \right>_p
		\\[5pt]
\end{aligned}\)

%----------------------------------------------------------------------------------------------------------------------------------
%
%
%----------------------------------------------------------------------------------------------------------------------------------
% Covariant Derivative of tangent vector field

\newpage
\subsection{Covariant Derivative of \(w\): 
	\( 
		\boxed{ \tfrac{Dw}{dt}_{\nhs \nhs\alpha'} \equiv \tfrac{dw}{dt}_{\parallel T_p(S)} }
		=
		\mss{
			\begin{gathered}[b]
				( \alpha' = X_u u' + X_v v' )\\[-3pt]
				\underline{ 
						\left[ 
							\mss{[a', b']}
							+ \mss{[u', v']}
							\mss{ \left[ \begin{matrix}
								a\Gamma_1 + b\Gamma_2\\
								a\Gamma_2 + b\Gamma_3
							\end{matrix} \right] }
						\right]
						\mss{ \left[ \begin{matrix}
							X_u\\
							X_v
						\end{matrix} \right] }
					}
			\end{gathered}
		}
	\)
}

\(\begin{aligned}
	% Full dw/dt
	& \ast\ \tfrac{dw}{dt}_{\alpha'} = \mss{ a'X_u + b'X_v + a(X_{uu}u' + X_{uv}v') + b(X_{vu}u' + X_{vv}v') }
		= [a', b', 0] \hsvec{c} 
		+ [u', v']
		\mss{ \left[ \begin{matrix}
			aR_1 + bR_2\\
			aR_2 + bR_3
		\end{matrix} \right] }
		\hsvec{c}
		\\
	% Parallel Vector Field
	& \bullet\ \text{\underline{``Parallel''}}\ w:\ \forall p \in \alpha\mss{(t)},\ \boxed{ \tfrac{Dw}{dt} = 0 }
		\hspace{10pt} \big( \hs\hs \begin{gathered}
			\text{\scriptsize !truly parallel in \(R^3\),}\\[-9pt]
			\text{\scriptsize e.g., sphere: \(a_\parallel = 0\)}
		\end{gathered} \hs\hs \big)
		\hspace{15pt}
		% Always exists one
		\bullet\ \forall \alpha\mss{(t)} \in S, \ \underline{ \exists! } \big\{ w\mss{(t)} \hs\big|\hs\hs \tfrac{Dw}{dt} = 0 \big\}
		\\[5pt]
	% Component of acceleration parallel to plane of velocity
	& \bullet\ \tfrac{D\alpha'}{dt} = (\alpha'' = kn)_{\parallel T_{\alpha}\mss{(S)}}
		\hspace{10pt}
		% Geodesics
		\ast\ \boxed{ 
			\text{{Geodesic}}:\ \alpha_g\mss{(t)} 
			\hs \big|\hs \tfrac{D\alpha_g'}{dt} = 0 = \tfrac{D\alpha_g'}{ds}
			\ \Rightarrow\ n_{\parallel T_p(S)} = 0
		}
		\hspace{10pt} \boxed{ \begin{aligned}
			& \text{\scriptsize\(\ast(n\) is perp. to \(T_p(S)\))}\\[-8pt]
			& \mss{ \ast ( \tfrac{d}{dt} \alpha \cdot \alpha = 0 ) }
		\end{aligned} }
		\\
	% Geodesic Equation
	& \ast\ \boxed{ 
			{\text{Geosedic Eq. (GEq)}}:\ 
			\left[ 
				\mss{[u'', v'']}
				+ \mss{[u', v']}
				\mss{ \left[ \begin{matrix}
					u'\Gamma_1 + v'\Gamma_2\\
					u'\Gamma_2 + v'\Gamma_3
				\end{matrix} \right] }
			\right]
			= \mss{ \left[ 0\ 0 \right] }
		}
		\hspace{10pt}
		%
		\ast\ \boxed{ 
			\mss{ \begin{aligned}
				& E=1, F=0\\[-2pt]
				& F=0, G=1
			\end{aligned} }
			\ \stackrel{ \text{\scriptsize(not only)} }{\rightarrow} \
			\mss{ \begin{aligned}
				\alpha_g( u'' = 0, v' = 0 )\\
				\alpha_g( u' = 0, v'' = 0 )\\
			\end{aligned} }
		}
\end{aligned}\)

\vspace{10pt}\noindent
\(\begin{aligned}
	% Algebraic Value of Dw/dt
	& \textbf{{Algebraic Value} of \( 
			\tfrac{Dw}{ds} \hspace{5pt} 
			\big( \hs\hs \underline{ \mss{ \begin{aligned}
				w \cdot w & = 1\\[-5pt]
				w \cdot w' & = 0 
			\end{aligned} } } \hs\hs \big)
		\)}:\ 
		\boxed{ 
			\big[ \tfrac{Dw}{ds} \big] \equiv \lambda\mss{(s)} 
			= \left< \tfrac{dw}{ds} , N \times w \right> 
			\hs\equiv\hs w' \cdot \overline{w} 
		}
		\ \Rightarrow\ 
		\underline{ \tfrac{Dw}{ds} = \lambda \overline{w} }
		\\
	% Geodesic Curvature
	& \boxed{ \textbf{{Geodesic Curvature}}:\ k_g = \left[ \tfrac{D\alpha'}{ds} \right] }
		\hspace{10pt} \big( \hs \begin{gathered}
			\text{\scriptsize geodesics}\\[-9pt]
			\text{\scriptsize have \(k_g = 0\)}
		\end{gathered} \hs \big)
		\hspace{5pt} \Rightarrow\ \boxed{ k^2 = k_n^2 + k_g^2 
		\ ==\ {\Vert t' \Vert^2} = \mss{ \left< N, kn \right>^2 } + k_g^2 }
\end{aligned}\)

\vspace{5pt}
\(\begin{aligned}		
	% Two Tangent Vector Fields
	& \underline{\text{Two Fields}}:\ \boxed{ w = v \cos\theta + \overline{v} \sin\theta }
		\hspace{10pt} \big( \hs\hs \begin{gathered}
			\text{\scriptsize unit circle}\\[-9pt]
			\text{\scriptsize from \(\hat{x}\)}
		\end{gathered} \hs\hs \big)
		\ \Rightarrow\ w' \cdot \overline{w} = \theta' + v' \cdot \overline{v} 
		= \boxed{ \tfrac{d\theta_{wv}}{ds} + \big[ \tfrac{Dv}{ds} \big] = \big[ \tfrac{Dw}{ds} \big] }
		\\
	& \ast\ 0 = \big[ \tfrac{Dv}{ds} \big] = \big[ \tfrac{Dw}{ds} \big] 
		\Rightarrow \boxed{\theta = \theta_0}
		\hspace{15pt} \ast\ 0 = \big[ \tfrac{Dv}{ds} \big]
		\Rightarrow \big[ \tfrac{Dw}{ds} \big] 
		= \boxed{ \big[ \tfrac{D\alpha'}{ds} \big] = k_g = \tfrac{d\theta_{\alpha'\parallel}}{ds} }
		\hspace{10pt} \big( \hs\hs \begin{gathered}
			\text{\scriptsize clockwise unit}\\[-9pt]
			\text{\scriptsize circle from \(\hat{y}\)}
		\end{gathered} \hs\hs \big)			
\end{aligned}\)

\vspace{10pt}\noindent
\(\begin{aligned}
	% Orthogonal Coordinates
	& \begin{gathered}
			\underline{\text{Orthog. Coord.}}:\ \mss{ F = 0 = e_1 \cdot e_2 }\\
			\mss{ 
				\underline{v} = \underline{\alpha'}\mss{(s)} 
				= e_1\mss{(s)} = \tfrac{X_u}{\sqrt{E}},\ N \times e_1 = e_2 = \tfrac{X_v}{\sqrt{G}} 
			}
		\end{gathered}
		\ \Rightarrow
		\begin{aligned}
			& \mss{ \left< \tfrac{\partial e_1}{\partial s}, e_2 \right> }
				= \tfrac{du}{ds} \mss{ \left< \tfrac{\partial e_1}{\partial u}, e_2 \right> } 
				+ v' \mss{ \left< \tfrac{X_{uv}}{\sqrt{E}} , \tfrac{X_v}{\sqrt{G}} \right> }
				= \boxed{ \tfrac{ - u' E_v + v' G_u }{ 2\sqrt{EG} } = \big[ \tfrac{De_1}{ds} \big] }
				\\
			& \boxed{ \big[ \tfrac{Dw}{ds} \big] = \tfrac{ v' G_u - u' E_v }{ 2\sqrt{EG} } + \tfrac{d\theta_{we_1}}{ds} }
				\ \Rightarrow\ \boxed{ \theta_{\parallel e_1} 
				= \theta_0 + \mss{\int} \tfrac{ u' E_v- v' G_u }{ 2\sqrt{EG} } \hs ds }
		\end{aligned}
		\\
	& \ast\ \begin{aligned}
			\cos\theta_{\alpha'e_1} & = \left< \underline{w}, e_1 \right> 
				= \mss{ \left< \underline{\alpha'}, e_1 \right> } 
				= \sqrt{E} u'
				\\
			\sin\theta_{\alpha'e_1} & = \mss{ \left< X_u u' + X_v v', \tfrac{X_v}{\sqrt{G}} \right> } = \sqrt{G} v'
		\end{aligned}
		\ \Rightarrow\ \begin{aligned}
				\big[ \tfrac{Da'}{ds} \big] & =
					\underline{ - \tfrac{E_v}{2E\sqrt{G}} } \cos\theta  
					+ \underline{ \tfrac{G_u}{2G\sqrt{E}} } \sin\theta 
					+ \theta'
				\\
			\Aboxed{ 
				k_g & = \underline{ (k_g)_1 } \cos\theta_{\alpha'e_1} 
					+ \underline{ (k_g)_2 } \sin\theta_{\alpha'e_1} 
					+ \tfrac{d\theta_{\alpha'e_1} }{ds} 
					= \tfrac{d\theta_{\alpha'\parallel}}{ds}
				}
		\end{aligned}
\end{aligned}\)

\vspace{15pt}\noindent
% Surface of Revolutions
\(\begin{aligned}
	& \underline{ \text{{Surface of Revolutions Geodesics}} }:\ X\mss{(u,v)} 
		= \big( \rho\mss{(v)} \cos u ,\hs \rho\mss{(v)} \sin u,\hs z\mss{(v)} \big)
		\\
	& \Gamma = \mss{ \left[ \arraycolsep=2pt \begin{matrix}
			0 & - \tfrac{\rho\rho'}{(\rho')^2 + (z')^2}\\[3pt]
			\tfrac{\rho\rho'}{\rho^2} & 0 \\[3pt]
			0 & \tfrac{\rho'\rho" + z'z"}{(\rho')^2 + (z')^2}
		\end{matrix} \right] }
		\rightarrow
		\left[ 
			\mss{[u'', v'']}
			+ \mss{[u', v']}
			\mss{ \left[ \arraycolsep=2pt \begin{matrix}
				\tfrac{v'\rho\rho'}{\rho^2} & - \tfrac{u'\rho\rho'}{(\rho')^2 + (z')^2}\\[3pt]
				\tfrac{u'\rho\rho'}{\rho^2} & v'\tfrac{\rho'\rho" + z'z"}{(\rho')^2 + (z')^2}
			\end{matrix} \right] }
		\right]
		\Rightarrow\ 
		\boxed{ \begin{aligned}
			0 & = \tfrac{1}{2} \tfrac{d}{ds} 
				\begin{gathered}[b]
					\text{\scriptsize\underline{(see Clairaut's Relation)}} \\
					\left[ \underline{ \rho^2 u' = \rho \sqrt{E} u' = \rho \cos\theta_{\alpha'X_u} } \right] 
				\end{gathered}
				\\[5pt]
			0 & = \tfrac{1}{2}\tfrac{d}{ds} \left[ ( \tfrac{d\rho}{dv}^2 + \tfrac{dz}{dv}^2 )(v')^2 \right] 
				+ \rho\rho' (u')^2 
		\end{aligned} \hspace{-9pt} }
		\\[5pt]
	% Coordinate Curves Geodesics
	& \bullet\ \begin{aligned}
			\alpha_u & : \underline{ \gamma\mss{(t)} = (u, v\mss{(t)}) }
				\ \rightarrow\ \Vert \tfrac{d\alpha_u}{ds} \Vert = 1
				\ \rightarrow\ \tfrac{d}{ds} \mss{ \left[ \left( \tfrac{dz}{dv}^2 + \tfrac{d\rho}{dv}^2 \right) (v')^2 \right] } = 0
				\ \Rightarrow\ \boxed{ \left[ \tfrac{D\alpha_u}{ds} \right] = 0 } 
				\\
			\alpha_v & : \underline{ \gamma\mss{(t)} = (u\mss{(t)}, v) } 
				\ \rightarrow\ \boxed{ \tfrac{du}{ds} = c_{\neq0},\ \tfrac{d\rho}{dv} = 0
				\ \Rightarrow \left[ \tfrac{D\alpha_v}{ds} \right] = 0 }
				\hspace{10pt} \text{\scriptsize e.g., \(\underline{ \tfrac{d\rho/dv}{dz/dv} = 0 }\)}
		\end{aligned}
		\\[5pt]
	% Clairaut's Relation Geodesics
	& \bullet \textit{\underline{Clairaut's Relation}}:\ \boxed{ \rho \cos\theta_{\alpha'X_u} = \rho^2 \tfrac{du}{ds} = c}
		\hspace{15pt} \left( \mss{ X_u \cdot \hat{z} = 0 } \right)
		\\
	& \ast \Vert \alpha' \Vert^2 = 1 
		= \rho^2 \tfrac{du}{ds}^2 + \left[ \tfrac{d\rho}{dv}^2 + \tfrac{dz}{dv}^2 \right] \tfrac{dv}{ds}^2 
		\ \rightarrow\ \underline{ \mss{ c \neq 0 \ \Leftrightarrow\ \theta_{\alpha'X_u} \neq \tfrac{\pi}{2} } }
		\ , \ \boxed{ \tfrac{du}{dv} = \tfrac{c}{f} \sqrt{ \tfrac{ (df/dv)^2 + (dg/dv)^2 }{f^2 - c^2} } }
\end{aligned}\)

%---------------------------------------------------------------------------------------------------------------------------------
%
%
%---------------------------------------------------------------------------------------------------------------------------------
% Gauss Bonnet
\newpage

\subsection{Gauss-Bonnet Theorem}

\(
	\begin{aligned}[t]
		% Exterior Angle
		& \underline{ \text{External Angle at \(t_i\)} } :\\ 
		& \boxed{ 
				\theta_E\mss{(t_i)} = \mss{ \lim_{\epsilon \rightarrow 0} } 
				\ \Delta \theta \big|^{ \alpha'\mss{(t_i + \epsilon)} }_{ \alpha'\mss{(t_i - \epsilon)} } 
				% \ \theta( \alpha'\mss{(t_i + \epsilon)} ) - \theta( \alpha'\mss{(t_i - \epsilon)} ) 
			}
			\\[2pt]
		% Interior Angle
		& \bullet\ \mss{ \underline{ \text{Interior Angle at \(t_i\)} } :\ \boxed{ \theta_I(t_i) = \pi - \theta_E(t_i) } }
	\end{aligned}
	\hspace{35pt}
	% Turning Theorem
	\begin{aligned}[t]	
		& \text{\underline{Turning Theorem along \(\alpha_\text{closed}\) {\scriptsize(going all around makes \(\pm 2\pi\))}}}:\\
		& \boxed{
				\mss{\bigcirc \hspace{-10pt} \sum_i^{Ver}} \
				\theta_{\alpha'e_1}\mss{(t_{i+1})} - \theta_{\alpha'e_1}\mss{(t_{i})} + \theta_E\mss{(t_i)}
				= \pm 2\pi
			}
	\end{aligned}
\)

\vspace{15pt}
% Gauss-Bonnet Theorem
\(
	\begin{aligned}[t]
		\mss{\oint} & \big[ \tfrac{Dw}{ds} \big] \hs ds 
			= \mss{\oint} \tfrac{ - u' E_v + v' G_u }{ 2\sqrt{EG} } + \tfrac{d\theta_{we_1}}{ds} \hs ds
			\\
		& = \mss{\iint} \left( \tfrac{ G_u }{ 2\sqrt{EG} } \right)_u 
			+ \left( \tfrac{ E_v }{ 2\sqrt{EG} } \right)_v du dv 
			+ \Delta \theta_{we_1}
			\\
		& = - \mss{\iint} (K \sqrt{EG}) \hs dudv + \Delta \theta_{we_1}
			\\
		\Aboxed{ 
			\mss{\oint} & \big[ \tfrac{Dw}{ds} \big] \hs ds 
			= - \mss{\iint} K d\sigma + \Delta \theta_{we_1} 
			% = - \mss{\iint} K d\sigma \pm 2\pi - \mss{\sum_i} \theta_E\mss{(t_i)}
			}
	\end{aligned}
	\hfill \vline \hfill
	\begin{aligned}[t]
		% parallel transport change of angle over closed loop is area integral of gaussian curvature
		& \ast\ \boxed{ \Delta \theta_{\parallel \hs \bcancel{e_1}} = \mss{\iint} K \hs d\sigma } 
			\rightarrow \underline{ 
				\mss{ \lim_{R \rightarrow p} } \hs\hs \tfrac{\Delta \theta}{A(R)} 
				= K(p)
				= \mss{ \lim_{R \rightarrow p} } \hs\hs \tfrac{ A(N(R)) }{ A(R) }  
			}
			\\[10pt]
		% Local Gauss Bonnet
		& {\textbf{(Local) Gauss-Bonnet Theorem}}: 
			\\
		& \ \boxed{ 
				\begin{gathered}[b]
					( \mss{ \Delta \theta_{\alpha'\parallel} } )\\
					\mss{\oint} \hs k_g \hs ds 
				\end{gathered}
				+ 
				\begin{gathered}[b]
					( \mss{ \Delta \theta_{\parallel0} } )\\
					\mss{\iint} K\hs d\sigma
				\end{gathered}
				+ \mss{\sum_i^{Ver}} \hs \theta_E\mss{(t_i)} 
				= \pm 2\pi 
			}
			\hspace{7pt} 
			\mss{
				\begin{gathered}
					\\[-28pt]
					\underline{\text{e.g., no vertices}} \\
					{ \Delta \theta_{'} + (2\pi - \Delta \theta_{'}) + 0 }\\
					\underline{\text{e.g., w/ vertices}} \\
					{ \Delta \theta_{'} + (\tfrac{3\pi}{2} - \Delta \theta_{'}) + \tfrac{\pi}{2} }
				\end{gathered}
			}
	\end{aligned}
\)

\vspace{10pt}\noindent
% Global Gauss Bonnet
\(\begin{aligned}
	& \textbf{Global Gauss-Bonnet Theorem}:\ \boxed{
			\mss{\oint_{\partial R}} \hspace{-3pt} k_g \hs ds 
			+ \mss{\iint_R} K\hs d\sigma
			+ \mss{\sum_i^{Ver}} \hs \theta_E\mss{(t_i)} 
			= 2\pi \chi(R) = 2\pi(V - E + F)
		}
		\\[5pt]
	& \begin{aligned}
			% Sphere has X = 2
			& \bullet\ \mss{ R \sim S^2 :\ \underline{ \chi(R) = 2 } }\\[-3pt]
			% Cylinder has X = 0
			& \bullet\ \mss{ R \sim \text{Cylinder} :\ \underline{ \chi(R) = 0 } }
		\end{aligned}
		\hspace{15pt} 
		% Simple closed curve has X = 1
		\bullet\ \mss{ \begin{gathered}
			\text{Simple Region}\\[-2pt]
			R \sim S_{\neq S^2}	
		\end{gathered} } :\ 
		\underline{ \chi(R) = 1 }
		\hspace{5pt} \text{\scriptsize(needs \(>0\) vertex; circle edge begins/ends at one point/vertex)}
		\\[5pt]
	% Compact Connected Surface: If there is no boundary, K_g integral is 0
	& \bullet\ { \mss{ \begin{gathered}
			\text{Compact, connected } S\\[-2pt]
			R \sim \mss{\oiint_{\partial \tau}},\ \nexists \partial R 
		\end{gathered} } } :\ 
		\boxed{ 
			\begin{gathered}
				\text{\scriptsize(\underline{one} \(n>0\))} \hspace{5pt} \mss{ n \in \{1,0,-1,-2,\dots\} }\\
				\mss{\iint_R} K\hs d\sigma =  2\pi \chi(S) = 2\pi(2n)
			\end{gathered}
		}
		\hspace{15pt}
		% Genus
		\begin{gathered}
			\ast\ \boxed{ \text{\underline{Genus}}:\ g \equiv \tfrac{2 - \chi}{2} } \\[-3pt]
			\text{\scriptsize(\(S^2\) w/ \(g\) torus holes \(\sim S\))}
		\end{gathered}
		\hspace{10pt}
		% Compact positive curvature ~ Sphere
		\ast\ \underline{K > 0} \ \mss{ \begin{aligned}
			& \Rightarrow \chi(S) = 2 \\
			& \Rightarrow \boxed{ S \sim S^2 }
		\end{aligned} }
\end{aligned}\)

\vspace{10pt}\noindent
\(\begin{aligned}
	% Two Closed Geodesic Boundary
	& R = \text{\underline{Bounded by [any] Two Geodesics}}:\ 
		\mss{\iint_R} K\hs d\sigma
		+ \theta_E\mss{(t_1)} + \theta_E\mss{(t_2)} 
		= 2\pi \chi(R)
		\\[5pt]
	% On K <= 0, can't intersect more than once
	& \bullet\ K \leq 0 :\ 
		\cancel{ \mss{ \begin{gathered}
			\text{intersect \(2\times\)}\\[-3pt]
			(R \sim S_{\neq S^2})
		\end{gathered} } }
		\ \rightarrow\ \mss{ \begin{aligned}
			2\pi \chi(R) & = \theta_E\mss{(t_1)} + \theta_E\mss{(t_2)} + {\iint_R} K\hs d\sigma \\
			2\pi & = (< \pi) + (< \pi) + (< 0)
		\end{aligned} }
		\ \Rightarrow\ \underline{ \begin{gathered}
			\text{\scriptsize any two geo.}\\[-9pt]
			\text{\scriptsize intersect \(\leq 1\times\)}
		\end{gathered} }
		\\	
	% On Cylinder, must intersect
	& \bullet\ K < 0 ,\ S \sim \text{\scriptsize Cylinder}:\ 
		\cancel{ \mss{ \begin{gathered}
			\text{two [closed] geo.}\\[-3pt]
			\text{intersect}\ 0\times\\[-3pt]
			(R \sim \text{cylinder})
		\end{gathered} } }
		\ \rightarrow\ 0 = 2\pi \chi(R) = \mss{\iint_R} K\hs d\sigma < 0 
		\ \Rightarrow\ \underline{ \text{\scriptsize \(\nexists\) Two CLOSED geo.} }
		\\	
	% On Sphere, must intersect
	& \bullet\ K > 0 \ \Leftrightarrow\ S \sim S^2:\ 
		\cancel{ \mss{ \begin{gathered}
			\text{two [closed] geo.}\\[-3pt]
			\text{intersect}\ 0\times\\[-3pt]
			(R \sim \text{cylinder})
		\end{gathered} } }
		\ \rightarrow\ 0 = 2\pi \chi(R) = \mss{\iint_R} K\hs d\sigma > 0 
		\ \Rightarrow\ \underline{ \begin{gathered}
			\text{\scriptsize two closed geo.}\\[-9pt]
			\text{\scriptsize intersect \(\geq 1\times\)}
		\end{gathered} }
		\\
	% 
\end{aligned}\)

\vspace{10pt}\noindent
% Geodesic Triangle
\(\begin{aligned}
	& \begin{gathered}
		\mss{ R = \text{Three Geodesics} } \\
		\underline{ \text{Geodesic Triangle},\ T } :
		\end{gathered} 
		\ 
		\begin{aligned}
			\mss{ \mss{\iint_R} K\hs d\sigma + \mss{\sum_{i=1}^3} \hs \theta_E\mss{(t_i)} } & = \mss{ 2\pi = 2\pi \chi(R) }
				\\
			\Aboxed{ \mss{\iint_R} K\hs d\sigma + \pi & = \mss{\sum_{i=1}^3} \hs \theta_I\mss{(t_i)} }	
		\end{aligned}
		\hspace{15pt}
		\bullet\ \boxed{ \mss{\iint_R} K\hs d\sigma = \mss{\sum_{i=1}^3} \hs \theta_I\mss{(t_i)} - \pi = A(N(T)) }
\end{aligned}\)

\vspace{10pt}\noindent
\(\begin{aligned}
	& \begin{aligned}
			% Index of Differential Vector Space v on S, at p
			& \underline{ \text{[Diff.] Vector Space, \(v\), on } S } :\ 
				\begin{gathered}
					p_i \in R \hs |\ v\mss{(p_i)} = 0\\[-4pt]
					\mss{ \partial R = \alpha,\  v\mss{(t)} = v\circ \alpha\mss{(t)} }
				\end{gathered}
				\rightarrow 
				\\
			& \mss{ \oint_0^l } \tfrac{ d\theta_{v(0) e_1} }{ds} \hs ds  
				= \mss{ \theta_{v(l) e_1} - \theta_{v(0) e_1} }
				= \boxed{ \Delta \theta \big|^{v(l)}_{v(0)} \equiv 2\pi I_{p_i} }
				\hspace{10pt} 
				\underline{ \mss{ \begin{gathered}
					v\mss{(p_i)} \neq 0 \\[-3pt]
					\rightarrow I_{p_i} = 0\\
				\end{gathered} } }
		\end{aligned}
		\ \
		\bullet\ \text{\scriptsize Compact}\ S,\ 
		\begin{gathered}
			\text{\scriptsize\underline{Poincare's Theorem}}\\
			\boxed{ \tfrac{1}{2\pi} \mss{\oiint} K \hs d\sigma = \mss{\sum_i} I_{p_i} = \chi(S)}
		\end{gathered}
\end{aligned}\)

%--------------------------------------------------------------------------------------------------------------------------------
%
%
%
\newpage

% Exponential Map
\subsection{Exponential Map, \( 
\mss{ \begin{aligned}
	\exp_p:\ & v \in T_p(S) \ \rightarrow\ \exp_p(v) \in S \\
	\equiv X:\ & q \in S_1 \ \rightarrow\ p' \in S_2 
\end{aligned} } 
\) }
\(
	\mss{ \begin{gathered}
		\left[ \tfrac{D\alpha'}{dt} \right] = 0 \\
		\tfrac{d}{dt} \Vert \alpha'\mss{(t)} \Vert = 0\\
		\alpha'\mss{(0)} = v\\
		\exists \alpha\mss{(s = |v|t)}
	\end{gathered} }
	\ , \ \
	\mss{ \begin{aligned}
		& |t| < \epsilon : \ \alpha\mss{(t)} = \alpha\circ\mss{(\lambda \overline{t})} 
			= \overline{\alpha}\mss{(\overline{t} = \tfrac{t}{\lambda})} \ : |\overline{t}| < \tfrac{\epsilon}{\lambda}
			\\
		& \alpha'\mss{(t)} = \tfrac{1}{\lambda} \overline{\alpha}'\mss{(\tfrac{t}{\lambda})}
			\ \Rightarrow\ 
			\lambda v = \lambda \alpha'\mss{(0)} = \overline{\alpha}'\mss{(0)} = \overline{v}
			\\
		& |t| < \epsilon : \ \gamma\mss{(t, v)} = \gamma\mss{(\overline{t}, \overline{v})} 
			= \gamma\mss{( \tfrac{t}{\lambda} , v\lambda )} \ : |\overline{t}| < \tfrac{\epsilon}{\lambda}
			\\
		& \nhs\nhs\nhs \tfrac{s}{|v|} < \epsilon : \ \gamma\mss{(\tfrac{s}{|v|}, v)}
			= \gamma\mss{( \tfrac{s}{\lambda|v|} , v\lambda )} 
			\\
		& \rightarrow\ \underline{
			( |\overline{v}| = \lambda |v| = s < \epsilon|v| )
			\ \Leftrightarrow\ ( \overline{t} = 1 < \tfrac{\epsilon}{\lambda} )  
			}
	\end{aligned} }
	\ ,\ \
	\boxed{
		\begin{aligned}
			\exp_p(0) & = \alpha_g{\scriptstyle(s=0)} = \gamma_v\mss{(0)} = p = \gamma_0\mss{(t)} \\
			\exp_p(v) & = \alpha_g\mss{(|v|)} = \underline{ \gamma_v\mss{(1)} } = e^{ (1) \tfrac{d}{dt} } \gamma_v\mss{(0)}
		\end{aligned}
	}
\)

\vspace{15pt}\noindent
% if \gamma is diff then exp_p is
\(
	\bullet\ \underline{ \text{IF}^* } \ \ 
	\begin{gathered}
		\gamma \mss{(t,v)} \in C^\infty \\[-6pt]
		\underline{ \mss{ \forall\ \text{directions}\ v } }
	\end{gathered}
	\hs \Big|\
	\mss{ \begin{aligned}
		|t| & < \epsilon_t\\[-3pt]
		|v| & < \epsilon_v
	\end{aligned} }
	\ \Rightarrow\ 
	\gamma\mss{(t',v')} \in C^\infty \hs \Big|\ 
	\mss{ \begin{aligned}
		\vert t' = \tfrac{2t}{\epsilon_t} \vert & < 2 \\[-3pt]
		\vert v' = \tfrac{\epsilon_t v}{2} \vert & < \tfrac{\epsilon_t \epsilon_v}{2} = \epsilon
	\end{aligned} }
	\ \Rightarrow\ 
	{ \begin{gathered}
		\mss{ ( \text{GEq unique.+exist. theor. used} )^* }\\[-5pt]
		\boxed{ \gamma\mss{(1,v')} = \exp_p \circ \hs v' \in C^\infty }
	\end{gathered} }
\)

\vspace{15pt}\noindent
% diffeomorphism for exp_p
\(\ast\ \begin{aligned}
	& \mss{ \begin{aligned}
			q\mss{(t)} & = v_0 t = (u_0 e_1 + v_0 e_2)t\\[-3pt]
			\alpha\mss{(t)} & = \exp_p \circ \hs\hs q\mss{(t)} \in \alpha_g
		\end{aligned} }
		\ ,\ \ \underline{ \tfrac{d\alpha}{dt} \big|_{t=0} }
		= \tfrac{d}{dt} \exp_p \mss{(1, v_0 t)} \big|_{t=0}
		= \tfrac{d}{dt} \gamma \mss{(t, v_0)} \big|_{t=0}
		= \underline{ v_0 }
		\\
	& \mss{ \begin{aligned}
			\bar{q} \mss{(t)} & = \bar{v}_0 t,\ {\scriptstyle(\bar{v}_0 \cdot v_0 = 0)}  \\[-3pt]
			\bar{\alpha} \mss{(t)} & = \exp_p \circ \hs\hs \bar{q}\mss{(t)}
		\end{aligned} }
		\ , \ \ \underline{ \tfrac{d\bar{\alpha}}{dt} \big|_{t=0} = \bar{v}_0 }
		\ \Rightarrow\ 
		\boxed{ 
			\begin{gathered}
				\mss{ (X_u\ X_v)_{q=0} = [e_1\ e_2] \neq \mathbbm{1}_3 } \\[-1pt]
				\left[ d( \exp_p )_{v=0} \right] q'\mss{(0)} = v_0 \\[-1pt]
				dX q'\mss{(0, v_0)} = q'\mss{(0, v_0)} = v_0
			\end{gathered}
		} 
		\ \stackrel{(\text{IFT})}{\Rightarrow} \ 
		\boxed{ 
			\begin{gathered}
				\exists\ \text{\scriptsize Diffeo}\ [ \exp_p \mss{(v)} ]^{-1} \\[-3pt]
				\underline{ \text{\scriptsize near \(q=v=0\)} }
			\end{gathered}
			\in C^\infty 
		}
\end{aligned}\)

\vspace{15pt}
\(\begin{aligned}
	% Normal Neighborhood
	& \underline{ \text{Normal Neighborhood, \(V_p\)} } :\ \text{Diffeo.}\ \exp_p(V_q) = V_p\\
	% Normal Coordinates
	& \underline{ \text{Normal Coordinates} } :\ w = u e_1 + v e_2 \in T_p(S)
		\hspace{10pt}
		\mss{ \begin{aligned}
			& \bullet\ w(t) = w_0 t,\ \alpha(t) = \exp_p \circ\hs w\mss{(t)} \in \alpha_g
				\hspace{5pt} \underline{ \text{\scriptsize(radial geo.)} }
				\\
			& \bullet\ \boxed{dX_{q=p}}: X_u = e_1,\ X_v = e_2 \ \rightarrow\ \underline{E|_p = G|_p =1,\ F|_p = 0}
		\end{aligned} }
		\\[5pt]
	% Polar coordinates
	& \underline{ \text{Geod. Polar Coordinates} } :\ 
		w = \hsvec{\rho} \mss{(0 < \rho,\ \underline{0 < \theta < 2\pi} )} \in T_p(S)
		\hspace{10pt}
		\mss{ \bullet\ \text{Diffeo.} \rightarrow \theta \in (0,2\pi);\ L \equiv \exp_p(w:\hs \theta = 0) }
		\\[10pt]
	& \bullet\ w:\hs (\rho\mss{(s)}, \theta) = (s, \theta_0) 
		\ \Rightarrow\ 
		\begin{aligned}
			% E = 1 for polar
			& \ast\ \Vert \alpha'|_{\theta=\theta_0}^{\rho=s} \Vert^2 = \Vert \alpha'\mss{(s)} \Vert^2 
				\ \stackrel{\rightarrow}{=}\ \boxed{ E = 1 }
				\\[5pt]
			% F_\rho = 0 for polar
			& \ast \begin{aligned}
					\text{(GEq)} &:\ (u')^2 \Gamma_{11} = (\rho')^2 \Gamma_{11} = [0\ 0]\\
					(\Gamma) &:\ \Gamma_{11}^2 = \tfrac{1}{2[EG-F^2]} \mss{ [E_u \ 2F_u - E_v] \cdot [-F\ E] }
				\end{aligned} 
				\ \rightarrow\ \underline{ F_\rho = 0 } 
		\end{aligned}
		\\[5pt]
	%% F = 0 for polar
	& \bullet\ \lim_{\rho \rightarrow 0} \left[ F\mss{(\rho,\theta)} = \mss{X_u \cdot X_v} \right]
		= \lim_{\rho \rightarrow 0} \tfrac{d \alpha}{ds}|_{\theta=\theta_0}^{\rho=s} 
		\cdot \lim_{\rho \rightarrow 0} \bcancel{ \tfrac{d \alpha}{d\phi}|_{\theta=\phi}^{\rho=\rho_0} }
		= 0,\ \big( F_\rho = 0 \big)
		\ \Rightarrow\ \forall \rho,\ \boxed{ F = 0 }
		\\[5pt]
	%%% Gauss's Lemma
	& \ast\ \underline{ \text{Gauss' Lemma}:\ F = 0 \ \leftrightarrow\ \text{\scriptsize radial geod. orthog. to geod. circles} }
		\\[5pt]
	%% \sqrt{G} = \rho for polar
	& \bullet\ \Vert X_u \times X_v \Vert = \Vert \overline{X_u} \times \overline{X_v} \Vert 
		\ \Rightarrow\ \underline{ \mss{ \sqrt{EF-G^2} }|_{\rho\theta} }
		= \underline{ 
			\mss{ \sqrt{\overline{E} \overline{F} - \overline{G}^2} }|_{uv,\hs\hs \rho=0} 
			\big\Vert \mss{ \arraycolsep=2pt \begin{matrix}
				\cos\theta & -\rho \sin\theta \\
				\sin\theta & \rho \cos\theta
			\end{matrix} } \big\Vert
		}
		\ \Rightarrow\ \boxed{ \lim_{\rho\rightarrow0} \sqrt{G} = \rho }
		\\[5pt]
	% Not exact way
	& \ast\ w = \mss{ \left[ X_u\ X_v \right] }
		\mss{ \left[ \begin{matrix}
			u'\\
			v'
		\end{matrix} \right] }
		= \underline{
			\mss{ \left[ X_u\ X_v \right] }
			\mss{ \left[ \arraycolsep=2pt \begin{matrix}
				u_\rho & u_\theta \\
				v_\rho & v_\theta
			\end{matrix} \right] }
		}
		\mss{ \left[ \begin{matrix}
			\rho'\\
			\theta'
		\end{matrix} \right] }
		= \underline{ \mss{ \left[ X_\rho\ X_\theta \right] } }
		\mss{ \left[ \begin{matrix}
			\rho'\\
			\theta'
		\end{matrix} \right] }
		\hspace{10pt} \text{\scriptsize(only useful so far for \(dX_{q=p}\), so above is better)}
\end{aligned}\)

\vspace{10pt}
\(\begin{aligned}
	% Equation for G and K
	& \bullet\ F=0,\ E=1 \ \rightarrow\ \boxed{ \sqrt{G}_{\rho\rho} + K\sqrt{G} = 0 }
		\hspace{15pt}
		%% another relation for d^3(G) and K
		\ast\ \mss{ \lim_{\rho\rightarrow0} \sqrt{G}_{\rho\rho\rho} + K_\rho \sqrt{G} + K\sqrt{G}_\rho } =
		\boxed{ 0 = \lim_{\rho\rightarrow0}\sqrt{G}_{\rho\rho\rho} + K\mss{(p)} }
		\\[5pt]
	%% Taylor Expansion of G at rho=0
	& \sqrt{G}\mss{(\rho,\theta)} = \cancel{ \sqrt{G}\mss{(0,\theta)} }
		+ \sqrt{G}_\rho \mss{(0,\theta)} \rho 
		+ \cancel{ \sqrt{G}_{\rho\rho} \mss{(0,\theta)} } \tfrac{\rho^2}{2!}
		+ \sqrt{G}_{\rho\rho\rho} \mss{(0,\theta)} \tfrac{\rho^3}{3!}
		+ R \rho^4
		= \rho - K\mss{(p)} \tfrac{\rho^3}{3!} + R \rho^4
		\\
	%% Arclength of Exponential map of circle
	& \boxed{ L\mss{(\rho)} } = \nhs\nhs \lim_{\rho,\hs \epsilon\rightarrow0} \oint_{0+\epsilon}^{2\pi-\epsilon} 
		\hspace{-16pt} \mss{ \sqrt{\cancel{Ed\rho^2} + G \hs d\theta^2 } }
		= \lim_{\rho\rightarrow0} 2\pi\rho - K\mss{(p)} \tfrac{2\pi \rho^3}{3!} 
		\ \Rightarrow\ \boxed{
			K\mss{(p)} = \lim_{\rho,\hs \epsilon\rightarrow0} \tfrac{3!}{2\pi}
			\tfrac{2\pi\rho - L}{\rho^3}
		}
		\hspace{5pt} \big( \hs \begin{gathered}
			\text{\scriptsize arclength of \(\bigcirc \in T_p(S)\)}\\[-7pt]
			\text{\scriptsize - arclength of \(\bigcirc \in S\)}
		\end{gathered} \hs \big)
		\\
	%% Area of Exponential map of circle
	& \boxed{ A\mss{(\rho)} } 
		= \nhs\nhs \lim_{\rho,\hs \epsilon\rightarrow0} 
		\int_{0+\epsilon}^{2\pi-\epsilon} \hspace{-7pt} \int_0^\rho \hspace{-5pt} \mss{ \sqrt{EG-F^2} } \hs dA
		= \lim_{\rho\rightarrow0} \tfrac{2\pi\rho^2}{2} - K\mss{(p)} \tfrac{2\pi \rho^4}{4!} 
		\ \Rightarrow\ \boxed{
			K\mss{(p)} = \lim_{\rho,\hs \epsilon\rightarrow0} \tfrac{4!}{2\pi}
			\tfrac{\pi\rho^2 - A}{\rho^4}
		}
		\hspace{5pt} \big( \hs \begin{gathered}
			\text{\scriptsize area of \(\bigcirc \in T_p(S)\)}\\[-7pt]
			\text{\scriptsize - area of \(\bigcirc \in S\)}
		\end{gathered} \hs \big)
\end{aligned}\)

%---------------------------------------------------------------------------------------------------------------------------------
%
%
%
\newpage

\(\begin{aligned}[t]
	& \ast\ K = 0:\ \underline{ \sqrt{G} = \rho }
		\hspace{20pt} 
		\ast\ K > 0:\ \sqrt{G} = \underline{ \tfrac{1}{\sqrt{K}} \sin( \sqrt{K}\rho ) }
		\ \rightarrow\ \sqrt{G}_{\rho\rho} < 0
		\ \Rightarrow\ \underline{ \tfrac{d^2}{d\rho^2} L\mss{(\rho)} \big|^{\theta_1}_{\theta_0} < 0 }
		\\[5pt]
	& \ast\ K < 0:\ \underline{ \sqrt{G} = \tfrac{1}{\sqrt{-K}} \sinh( \sqrt{-K}\rho ) }
		\ \rightarrow\ \sqrt{G}_{\rho\rho} > 0
		\ \Rightarrow\ \underline{ \tfrac{d^2}{d\rho^2} L\mss{(\rho)} \big|^{\theta_1}_{\theta_0} > 0 }
		\\[5pt]
	& \ast\ \psi:\ 
		\mss{ \begin{aligned}
			\psi(e_i \in T_p(S)) & = \overline{e}_i \in T_{\hs\overline{p}}(\overline{S})\\
			d\psi(e_i \in T_p(S)) & = \overline{e}_i \in T_{\hs\overline{p}}(\overline{S})
		\end{aligned} }
		\ \rightarrow\ \Psi:\ 
		\mss{ \begin{aligned}
			& \Psi = \exp_{\hs \overline{p}} \circ\ \psi \circ \exp_{p}^{-1}\\
			& [d\Psi] e_i = \overline{e}_i
		\end{aligned} }
		\ \ \vline \ \
		\boxed{
			\begin{gathered}
				\mss{ K_0 = K(V_p) = K(V_{\hs\overline{p}}) }\\[-5pt]
				\text{\scriptsize Minding's Theorem}\\[-9pt]
				\text{\scriptsize const. \(K \rightarrow\) isometry}
			\end{gathered}
			\ \Rightarrow\ 
			\mss{ \begin{aligned}
				& (E,F,G) = (\overline{E}, \overline{F}, \overline{G})\\
				& \Vert \overline{w} \Vert = \Vert d\Psi(w) \Vert = \Vert w \Vert
			\end{aligned} }
		}
\end{aligned}\)

\vspace{10pt}
\(\begin{aligned}
	& \ast dl_\alpha = \sqrt{E d\rho^2 + G d\theta^2} 
		\ \rightarrow\ l_\alpha\mss{(\epsilon)} 
		= \int_\epsilon^{t - \epsilon} \hspace{-8pt} \mss{ \sqrt{ (\rho')^2 + G \cdot (\theta')^2} } \hs dt 
		\ \geq\ l_\gamma - 2\epsilon = \int_\epsilon^{t - \epsilon} \hspace{-8pt} \mss{ \sqrt{ (\rho')^2 } } \hs dt 
\end{aligned}\)

%------------------------------------------------------------------------------------------------------------------------------------
%------------------------------------------------------------------------------------------------------------------------------------
%------------------------------------------------------------------------------------------------------------------------------------
%------------------------------------------------------------------------------------------------------------------------------------
\newpage
% Abstract Surfaces
\section{Abstract Surface/Riemannian Maniford, \(M\)}

% Definition of Abstract Surface/Riemannian Manifold
\(\begin{aligned}
	1.\ & \bigcup_\alpha X_\alpha (U_\alpha) = M
		\hspace{15pt}
		\text{open}\ U \subset R^n
		\hspace{30pt}
		3.\ \left\{ U_\alpha, X_\alpha \right\} \ \text{is maximal rel. to 1. and 2.}
		\\
	2.\ & \forall (\alpha,\beta),\ W = X_\alpha(U_\alpha) \cap X_\beta(U_\beta) \neq \varnothing
		\ \Rightarrow\ 
		\begin{aligned}
			& \bullet\ \text{open}\ X^{-1}_{\alpha}(W), X^{-1}_{\alpha}(W) \subset R^n\\
			& \bullet\ X^{-1}_\beta \circ X_\alpha,\ X^{-1}_\alpha \circ X_\beta \in C^{\infty}
		\end{aligned}
		\\[5pt]
	& \bullet\ \text{Manifold},\ M = \text{Hausdorff Space w/ Complete Atlas} \\
	& \bullet\ \text{Sub[space]manifold},\ M \ \stackrel{\text{[top] subspace}}{\hookrightarrow}\ N
		;\ C^\infty \ni \text{\scriptsize Immer}\ \iota: \text{\scriptsize Mani}\ M \rightarrow \text{\scriptsize Mani}\ N
		\hspace{15pt} \text{\scriptsize(Inclusion map)}
		\\
\end{aligned}\)

\vspace{10pt}
% Coordinate System
\( 
	\underline{ \textbf{Coordinate System} }:\ \forall p \in \exists V_{p_0} \subset M,\ 
	\xi(p) = (x^1, x^2, \dots x^m)\mss{(p)} \in R^m 
\)

\vspace{10pt}
% Tangent Vectors
\(
	\underline{ \begin{gathered}
		{ \text{Tangent Vector} }\\[-4pt]
		{ \text{at } \alpha\mss{(0)} \in M }
	\end{gathered} } 
	:
	\arraycolsep=2pt \begin{array}{c c c c l}
		\mss{f \circ \alpha(t)} & \rightarrow 
			& \tfrac{d(f \circ \alpha)}{dt} \big|_{t=0} 
			& = 
			& \big[ 
				\tfrac{\partial f}{ \partial {r^1} },\hs \tfrac{\partial f}{ \partial {r^2} },\hs \tfrac{\partial f}{ \partial {r^3} } 
				\big] 
				\alpha'\mss{(0)}
				\hspace{5pt} , \hspace{5pt} f \in \mathcal{F}(M) 
				\hspace{10pt} ( \text{\scriptsize all diff. func. on } M )
			\\[5pt]
		\mss{f \circ X \circ q(t)} & \rightarrow
			& \tfrac{d(f \circ X \circ q)}{dt} \big|_{t=0} 
			& = 
			& \big[ 
				\tfrac{\partial f}{ \partial {r^1} },\hs \tfrac{\partial f}{ \partial {r^2} },\hs \tfrac{\partial f}{ \partial {r^3} } 
				\big]
				\big[  \vec{ \tfrac{\partial X}{ \partial {u^1} } } ,\hs \vec{ \tfrac{\partial X}{ \partial {u^2} } } \big] q'\mss{(0)}
				= Df DX Dq
			\\[5pt]
		\mss{F(u^1, u^2) \circ q(t)} & \rightarrow
			& \tfrac{d(F\circ q)}{dt}\big|_{t=0} 
			& = 
			& \boxed{ \big( q_1' \tfrac{\partial}{\partial u^1} \big|_0 + q_2' \tfrac{\partial}{\partial u^2} \big|_0 \big) 
			(f \circ \xi^{-1}) }
			\\
	\end{array}
\)\\[5pt]
\(
	% Tangent Space
	\begin{gathered}
		T_{p_0}(M) = \{ v_{p_0}: \mathcal{F}(M) \rightarrow \mathbb{R} \} \\[-3pt]
		\mss{(\text{Is a Derivation})}\\
	\end{gathered}
	\hspace{15pt}
	% Req. Properties
	\begin{aligned}
		% Linear
		& \bullet\ v_{p_0}(af + bg) = a \underline{v_{p_0}(f)} + b \underline{v_{p_0}(g)} \in \mathbb{R}\\
		% Leibnizian
		& \bullet\ v_{p_0}(fg) = v_{p_0}(f) \underline{g(p_0)} + v_{p_0}(g) \underline{f(p_0)}
			\hspace{15pt} f,g : M \rightarrow \underline{ \mathbb{R} }
			\\
		& \ast\ c v_{p_0}(1) = c v_{p_0}(1*1) = 2c v_{p_0}(1) = \boxed{0 = v_{p_0}(c)}\\
	\end{aligned}
\)\\[10pt]
\(	
	% Basis Vectors
	\begin{gathered}
		\text{Basis Vectors}:\\
		\text{\scriptsize \(\scriptstyle q(t, q_1, q_0) \ =\ q_0 + t [q_1 - q_0] \)}\\[-8pt]
		\mss{\scriptstyle p(t, q_1, q_0) \ =\ \xi^{-1} \circ \hs q(t, q_1, q_0) }\\
	\end{gathered}
	\ \
	\bullet\ \boxed{ \tfrac{\partial f}{\partial x^i}\mss{(p)} \equiv \tfrac{\partial (f \circ \xi^{-1})}{\partial u^i} \mss{(\xi p)} }
	= \tfrac{\partial F}{\partial u^i} \mss{(q)}
	\ \Rightarrow\ T_{p_0}(M) \ni \boxed{ \partial_i \big|_p \equiv 
		\tfrac{\partial}{\partial x^i}\big|_p : \mathcal{F}(M) \rightarrow \mathbb{R}
	}
\)\\[5pt]
\(
	\begin{aligned}[t]
		\bullet\ \mss{F\big|_{q_0}^{q_1} = F\circ q\big|_0^1} & = \mss{\sum} {\scriptstyle[q^i_1 - q^i_0]} 
			\mss{\int_0^1} \tfrac{\partial F}{\partial u^i} \circ q{\scriptstyle(t,q_1,q_0)} \hs dt
			\stackrel{\text{\scriptsize(no need)}}{=} 
			\mss{\sum \int_0^1} \tfrac{\partial F}{\partial u^i} \circ q{\scriptstyle(t,q_1,q_0)} \tfrac{dq^i}{dt} \hs dt
			= \mss{\int_0^1} DF\hs Dq\hs dt 
			= \underline{ \mss{\int_0^1} \tfrac{d(F \circ q)}{dt} \hs dt }
			\\
		\mss{ F \circ q_1 - F(q_0) } & = \mss{\sum} {\scriptstyle[x^i p_1 - x^i p_0]} \mss{\int_0^1} 
			\tfrac{\partial (f \circ \xi^{-1})}{\partial u^i} \circ \xi p{\scriptstyle(t,p_1,p_0)} \hs dt
			\\
		\mss{ f \circ p_1 - f(p_0) } & = \mss{\sum} {\scriptstyle[x^i p_1 - x^i p_0]} \mss{\int_0^1} 
			\tfrac{\partial f}{\partial x^i} \circ p{\scriptstyle(t,p_1,p_0)} \hs dt
			\\
	\end{aligned}
\)\\[2pt]
\(
	% Partial Derivative at zero/initial
	\begin{aligned}
		\ast\ \text{\scriptsize(no need?)} \hspace{10pt}
			\tfrac{\partial F}{\partial u^j} \big|_{q_0} - \cancel{ \tfrac{\partial F(q_0)}{\partial u^j} } 
			& = \cancel{ \tfrac{\partial u^j}{\partial u^j} }
			\mss{\int_0^1} \tfrac{\partial F}{\partial u^j} \circ q{\scriptstyle(t,q_0,q_0)} \hs dt
			+ \cancel{\mss{\sum_i} {\scriptstyle[q^i_0 - q^i_0]}
			\mss{\int_0^1 \bcancel{\sum_k}} \bcancel{\tfrac{\partial u^{k}}{\partial u^j}} t
			\tfrac{\partial^2 F}{\partial u^{\bcancel{k}j} \partial u^i} \circ q{\scriptstyle(t,u,q_0)} \hs dt \big|_{q_0}}
			\\
		\tfrac{\partial f}{\partial x^j} \big|_{p_0} - \cancel{ \tfrac{\partial f(p_0)}{\partial x^j} } 
			& = \mss{\int_0^1} \tfrac{\partial f}{\partial x^i} \circ p{\scriptstyle(t,p_0,p_0)} \hs dt 
			\\
	\end{aligned}
\)\\
\(
	% v decomposed by basis vectors
	\begin{aligned}
		\ast\ v_{p_0} f\mss{(p)} - \cancel{ v_{p_0} f\mss{(p_0)} } & = 
			\mss{\sum} 
			v_{p_0} {\scriptstyle[x^i(p) - \cancel{ x^i(p_0) }]} 
			\cdot \mss{\int_0^1} \tfrac{\partial f}{\partial x^i} \big|_{p(t,p_0,p_0)} \hs dt
			+ \bcancel{ {\scriptstyle[x^i(p_0) - x^i(p_0)]} 
			\cdot v_{p_0} \mss{\int_0^1} \tfrac{\partial f}{\partial x^i} \big|_{p(t,p,p_0)} \hs dt }
			\\ 
		& = v_{p_0} {\scriptstyle[x^i(p)]} \cdot \mss{\sum} \tfrac{\partial f}{\partial x^i} \big|_{p_0}
			\ \Rightarrow\ \boxed{ v_{p_0} = v_{p_0} \left( \sum x^i \tfrac{\partial}{\partial x^i}\big|_{p_0} \right) 
			= \sum v_{p_0}(x^i) \hs \partial_i \big|_{p_0} }
			\\
	\end{aligned}
\)\\[4pt]
\( 	
	% Linear Independant
	\ast\ \sum c^i \partial_i \big|_{p_0} = 0 = 0 \cdot x^j = \sum c^i \partial_i \big|_{p_0} x^j = c^i \delta_{ij} = c^j
	\hspace{15pt} \boxed{ \text{(Lin. Ind.)} }
\)\\[5pt]
\(
	% Change of basis
	\bullet\ \text{Change of Basis}:\ 
	v_p = \sum v_p(x^i) \hs \tfrac{\partial}{\partial x^i}\big|_p = \sum v_p(y^j) \hs \tfrac{\partial}{\partial y^j}\big|_p 
	\ \leftrightarrow\ v_p(y^j) = \sum v_p(x^i) \hs \tfrac{\partial y^j}{\partial x^i} \big|_p
\)

%---------------------------------------------------------------------------------------------------------------------------------
%---------------------------------------------------------------------------------------------------------------------------------
\newpage
\(
	\begin{aligned}
		% Manifold Mapping
		& \underline{ \text{Manifold Mapping} },\ \phi : M \rightarrow N\\
		% Vector at Mapping
		& \underline{ \text{Vector at Mapping},\ v_\phi \in T_{\phi p}(N) }: \ \ 
			\boxed{ v_\phi(g) \equiv v(g \circ \phi) }
			\\
	\end{aligned}
	\hspace{15pt}
	% Leibnizian
	\begin{aligned}
		\bullet\ v_\phi(g_1 g_2) & = v( \mss{(g_1 \circ \phi)(g_2 \circ \phi)} ) \\
		& = v_\phi \mss{(g_1)} g_2\mss{(\phi p)} + v_\phi\mss{(g_2)} g_1\mss{(\phi p)}\\
	\end{aligned}
\)

\vspace{5pt}
% Differential Map
\(
	\underline{ \text{Differential Map},\ d\phi_p: T_p(M) \rightarrow T_{\phi p}(N) }: \ \ 
	\boxed{ v(g \circ \phi) \equiv v_\phi(g) = [ d\phi_p \hs v ](g) }
	\hspace{15pt} 
	\mss{ \begin{aligned}
		& v \in T_p(M)\\
		& g \in \mathcal{F}(N)\\
	\end{aligned} }
	,\ \tfrac{\partial g}{\partial y^i} = \tfrac{\partial G}{\partial v^i}
\)\\[5pt]
\(\displaystyle
	\underline{ q' \nhs\nhs \cdot \nhs\nhs\nhs \hsvec{\nabla}_{\nhs\nhs u} \big|_p (\mss{g \circ \phi \circ \xi^{-1}}) }
	% = \tfrac{ d(g \circ \phi \circ \xi^{-1} \circ q )}{dt}\big|_{0} 
	= \tfrac{ d(g \circ \eta^{-1} \circ \eta \circ \phi \circ \xi^{-1} \circ q )}{dt}\big|_{0} 
	= \tfrac{ d(G \circ \Phi \circ q )}{dt}\big|_{0} 
	= \mss{ [\hsvec{\nabla}^T_{\nhs\nhs v} \hspace{-3pt} G] [\hsvec{\nabla}^T_{\nhs\nhs u} \hspace{-3pt} \Phi] \hs q'|_0 } 
	= \underline{ \mss{
		\left[ [q'|_0^T \hsvec{\nabla}_{\nhs\nhs u} \Phi^T] \hsvec{\nabla}_{\nhs\nhs v} \right] 
		\nhs\nhs G 
	} }
	= \underline{ \mss{
		\left[ [\hsvec{\nabla}_{\nhs\nhs u}^T \hspace{-3pt} \Phi \hs v(\hsvec{x})] \nhs\cdot\nhs\nhs \hsvec{\nabla}_{\nhs\nhs v} \right] 
		\nhs\nhs G 
	} }
\)\\[7pt]
\(
	% Change of Basis
	\begin{aligned}
		% basis vector
		& \bullet\ T_{\phi p}(N) \ni d\phi_p \tfrac{\partial}{\partial x^i} \big|_p 
			= \sum_j \left( d\phi_p \tfrac{\partial y^j}{\partial x^i} \big|_p \right) \tfrac{\partial}{\partial y^j} \big|_{\phi p}
			= \sum_j \tfrac{\partial (y^j \circ \phi)}{\partial x^i}\big|_p \tfrac{\partial}{\partial y^j} \big|_{\phi p}
			\ \Leftarrow\ \eta(\phi p \in N^n ) = y \in \mathbb{R}^n
			\\
		% general vector
		& \bullet\ T_{\phi p}(N) \ni d\phi_p v = d\phi_p \sum_i v(x^i) \tfrac{\partial}{\partial x^i} \big|_p
			= \underline{ \mss{\sum_{i,j}} v(x^i) \tfrac{\partial (y^j \circ \phi)}{\partial x^i}\big|_p 
			\tfrac{\partial}{\partial y^j} \big|_{\phi p} }
			\\
	\end{aligned}
\)\\[5pt]
\(
	% Compounded Mapping Functions 
	\bullet\ \mss{ \begin{aligned}
		& \psi : N \rightarrow P\\
		& \phi : M \rightarrow N\\ 
	\end{aligned}
	\ , \ 
	\begin{aligned}
		& h \in \mathcal{F}(P)\\
		& v \in T_p(M)\\
	\end{aligned} }
	\ ,
	\begin{aligned}
		[ d(\psi \circ \phi)_p \hs v ] (h) & = v_{\psi \circ \phi} (h) = v(\mss{h \circ \psi \circ \phi}) = v_\phi \mss{(h \circ \psi)}
			= d\phi_p \hs v \mss{(h \circ \psi)} = d\psi_{\phi p}\hs d\phi_p \hs v \mss{(h)}
			\\
		\Aboxed{ d(\psi \circ \phi)_p \hs v & = d\psi_{\phi p}\hs d\phi_p \hs v }
	\end{aligned}
\)\\[10pt]
\(
	% Inverse Function Theorem
	\bullet\ \underline{ \text{Inverse Function Theorem} }: \ \ 
	\boxed{ \text{Linear Iso.}\ d\phi_p\ \text{at}\ p 
	\ \Leftrightarrow\ \text{[Local] Diffeo.}\ \phi: \exists \mathcal{V}_{p} \rightarrow \phi(\mathcal{V}_p) }
\)

\vspace{15pt}
\(
	% Curves
	\begin{aligned}
		& \underline{ \text{Curve}\ \alpha(t) = \xi^{-1} \circ q(t) : \mathbb{R} \rightarrow M}
			\\[5pt]
		& \bullet\ T_{\alpha(t)}(M) \ni \alpha'(t) = d\alpha \hs \tfrac{\partial}{\partial u} \big|_t
			= \sum \alpha' \big|_t (x^i) \tfrac{\partial}{\partial u^i} \big|_{\alpha(t)}
			= \sum \tfrac{\partial(x^i \circ \alpha)}{\partial u} \big|_t \tfrac{\partial}{\partial u^i} \big|_{\alpha(t)}
			\\
		& \bullet\ \alpha'\mss{(t)} f = d\alpha \hs \tfrac{\partial f}{\partial u} \big|_t 
			= \tfrac{\partial(f \circ \alpha)}{\partial u} \big|_t 
			\\
	\end{aligned}
\)

\vspace{15pt}
\(
	% Vector Field
	\begin{gathered}
		\underline{ \text{Vector Field},\ V \in \mathcal{X}(M): p \in M \rightarrow V_p \in T(M) }: \ \ \\[-3pt]
		( \text{\scriptsize \(\mathcal{X}(M) =\) Module over [commutative, unital] Ring \(\mathcal{F}(M)\)} ) \\
	\end{gathered}
	\boxed{ \arraycolsep=2pt \begin{array}{c c c c c}
		V_p f & = & V_p(f) & = & (Vf)_p = (\overline{V}f)_p\\
		\mss{ \mathbb{R} } && \mss{ V_p \in T_p(M) } && \mss{ Vf \in \mathcal{F}(M) },\ 
			\mss{\overline{V} : \mathcal{F}(M) \rightarrow \mathcal{F}(M) }
			\\
	\end{array} }
\)\\[7pt]
\(
	\begin{aligned}[t]
		% Module over Ring of Functions
		& \ast\ (fV)_p = f\mss{(p)} V_p\\
		& \ast\ (V + W)|_p = V_p + W_p \\
	\end{aligned}
	\hspace{15pt}
	\begin{aligned}[t]
		% Basis
		& \bullet\ \underline{ \text{[Free] Module Basis},\ \partial_i }:\ \partial_i(p) = \partial_i \big|_p \in T_p(M)\\
		% General
		& \bullet\ V_p = \mss{\sum }\hs V_p x^i \hs \partial_i \big|_p
			= \left( \underline{ \mss{\sum }\hs [V x^i] \hs \partial_i } \right)_p
			= (\underline{V})_p
	\end{aligned}
\)\\
\(
	\begin{aligned}
		% vector field leads to derivation
		& \bullet\ \underline{ \text{\scriptsize Vector Field}\ V 
			 \Leftrightarrow\hs \overline{V} \ \text{\scriptsize is a derivation on \(\mathcal{F}(M)\)}
			}
			\\[3pt]
		% commutator
		& \bullet\ \underline{\text{Commutator}}:\ 
			\arraycolsep=2pt \begin{array}{c c c}
				\text{\scriptsize Derivation}\ [\overline{V}, \overline{W}] & \Leftrightarrow 
					& \text{\scriptsize Vector Field}\ [{V}, {W}]_p 
					\\
				\mss{ \big( [\overline{V}, \overline{W}]f \big)_p = \big( \overline{V}(Wf) - \overline{W}(Vf) \big)_p  } & 
					& \mss{ [{V}, {W}]_p f = V_p(\overline{W}f) - W_p(Vf) }
			\end{array}
			\hspace{15pt}
			%
			\ast\ \left[ \partial_i, \partial_j \right] = 0
			\\[3pt]
		% identity
		& \ast\ \begin{aligned}[t]
				[f\overline{V}, g\overline{W}]h & = \underline{ fV(g(Wh)) } - gW(f(Vh)) \\ 
				& = \mss{ \underline{ fgV(Wh) } - gfW(Vh) + \underline{ f(Vg)(Wh) } - g(Wf)(Vh) }
					= fg[{V}, {W}]h + f(Vg)(Wh) - g(Wf)(Vh)
					\\
			\end{aligned}
			\\[5pt]
		% \phi-Related Maps
		& \bullet\ \underline{ \text{\(\phi\)-Related Maps, \(X \sim_\phi Y\)}} :\ \ 
			\mss{\forall p \in M},\ d\phi_p(X_p) = Y_{\phi p}
			\ \Leftrightarrow\ \mss{\forall g \in \mathcal{F}(N)},\ \overline{X}(g \circ \phi) = (\overline{Y}g) \circ \phi
			\\[3pt]
		& \ast\ \text{Transferred Vec. Field of \(X\)}:\ \ (Y) = (d\phi X)\ \Rightarrow\ 
			\boxed{ (d\phi X)g = \overline{X}(g \circ \phi) \circ \phi^{-1} \in \mathcal{F}(N) }
			\\[3pt]
		& \ast\ X_1 \sim_\phi Y_1,\ X_2 \sim_\phi Y_2 \ \Rightarrow\ \underline{ [X_1, X_2] \sim_\phi [Y_1, Y_2] }\\ 
	\end{aligned}
\)

%-----------------------------------------------------------------------------------------------------------------------------
%
%
%
\newpage
\(
	\begin{aligned}
		% Covector/Cotangent/Dual Space
		& \underline{ \text{Cotangent Space} \ T^*_p(M) } \ \ni \ 
			\underline{ \text{Covector} \ v^*_p: T_p(M) \rightarrow \mathbb{R} }
			\\[5pt]
		% One-Form
		& \begin{gathered}
				\underline{ \text{One-Form} \ \theta \in \mathcal{X}^*(M)}\\[-3pt]
				\text{\scriptsize(Covec. Field \(=\) Module Over \(\mathcal{F}(M)\))}
			\end{gathered}
			: 
			p \in M \rightarrow \theta_p \in \bigcup_{p \in M} T^*_p(M) : \ 
			\boxed{ \arraycolsep=2pt \begin{array}{c c c c c}
				\theta_p \hs V_p & = & \theta_p(V_p) & = & (\theta V)_p = (\overline{\theta} {V})_p\\
				\mss{ \mathbb{R} } & & \mss{ \theta_p: T_p(M) \rightarrow \mathbb{R} } 
					& & \mss{ \overline{\theta}: \mathcal{X}(M) \rightarrow \mathcal{F}(M) }
					\\
			\end{array} }
			\\[5pt]
		% Differential of a function
		& \bullet\ \underline{ \text{Differential}, d : \mathcal{F}(M) \rightarrow \mathcal{X}^*(M)}\\
		& \hspace{15pt} df \in \mathcal{X}^*(M):\ 
			\arraycolsep=2pt \begin{array}[t]{c l l}
				\tfrac{\partial (f \circ \xi^{-1} \circ q) }{\partial t}\big|_0 
					& = \tfrac{\partial F}{\partial u}^T q'(0) 
					= \mss{\sum} q_i'\mss{(0)} \hs \partial_i \big|_0 F 
					& \displaystyle = \big( \left[ \mss{\sum} v(x^i) \hs \partial_i \right] \nhs\nhs F \big)|_0
					\\[5pt]
				& = df_p V_p = \underline{ (\overline{df} \hs V)_p } & = \underline{ (\overline{V}f)_p }
			\end{array}
			\\[5pt]
		% Basis
		& \bullet\ \underline{ \text{[Free] Module Basis},\ dx^i }:\ dx^i (\partial_j) = \partial_j(x_i) = \delta_{ij} 
			\\[5pt]
		% General
		& \bullet\ \theta_p = \mss{\sum} \theta_p(\partial_i|_p) \hs dx^i_p
			= \mss{\sum} \partial_i|_p \theta_p \hs dx^i_p
			\ \Rightarrow\ df = \mss{\sum} \tfrac{\partial f}{\partial x^i} \hs dx^i
			\\
	\end{aligned}
\)

\vspace{15pt}
% Tangent Bundle
\(
	\underline{ \text{Tangent Bundle} }:\
	\hspace{-35pt}
	\begin{aligned}[t]
		T(M) & = \left\{ (p,w) \right\}
			\hspace{5pt} , \hspace{5pt}
			p \in M,\ w \in T_p(M) 
			\\
		y_\alpha\mss{(u_{1,\alpha}, ..., x_1, ...)} & = \left\{ \big( 
				X_\alpha \mss{ ( u_{1,\alpha}, u_{2,\alpha} , ...) },\ 
				x_1 \tfrac{\partial X_\alpha}{\partial u_{1,\alpha}} + x_2 \tfrac{\partial X_\alpha}{\partial u_{2,\alpha}} + ... 
			\big) \right\}
			\hspace{5pt} , \hspace{5pt}
			x_i \in R
			\\
		T(M) & = \bigcup_\alpha y_\alpha( U_\alpha \times R^n )
	\end{aligned}
\)

\vspace{10pt}
\(
	\begin{aligned}[t]
		% Hypersurface
		& \underline{ \text{Hypersurface} ,\ P \stackrel{\text{subman}}{\hookrightarrow} M }:\ \dim{M} = \dim{P} + 1
			\\[5pt]
		% Level Hypersurface
		& \bullet\ \underline{ \text{Level Hypersurface}:\ f^{-1}(q) }\\
		& \hspace{10pt} \ast\ f: M \rightarrow N = \mathbb{R}^1\\
		& \hspace{10pt} \ast\ \forall p,\ q = f(p),\ df_{p} \neq 0\\
	\end{aligned}
	\hfill
	\begin{aligned}[t]
		% Regular Value
		& \underline{ \text{Regular Value},\ q \in N }:\ \forall p \big( \phi\mss{(p)} = q \big)
			\big( \text{\scriptsize onto}\ d\phi_{p} \big)
			\\
		& \bullet\ \phi^{-1}(q) \stackrel{\text{subman}}{\hookrightarrow} M\\
		& \bullet\ \dim{M} = \dim{N} + \dim{\phi^{-1}(q)}
			\\[5pt]
		% Submersion
		& \underline{ \text{Submersion},\ \phi }:\ \forall p \big( p \in M \big) \big( \text{\scriptsize onto}\ d\phi_p \big)
	\end{aligned}
\)


%-----------------------------------------------------------------------------------------------------------------------------------

\vspace{10pt}
\(
	% Immersion
	\begin{aligned}[t]
		& \underline{ \text{Immersion [Map] , \(\phi\)} } :\\
		& \begin{aligned}
				\phi: &\ M \rightarrow N\\[-2pt]
				d\phi_p: &\ T_p(M) \rightarrow T_{\phi p}(N) \ \text{is 1-1}
			\end{aligned}
	\end{aligned}
	\hfill
	% Isometric Immersion
	\begin{aligned}[t]
		& \underline{ \text{Isometric Immersion} }: \\
		& \begin{gathered}
				\left< d\phi_p(v), d\phi_p(w) \right>_{\phi(p)} =	\left< v, w \right>_{p} \\
				\mss{ \text{Euclid. Metric on } R^n } = \mss{ \text{Riem. Metric on } S }
			\end{gathered}
	\end{aligned}
	\hfill
	% Embedding
	\begin{aligned}[t]
		& \underline{ \text{Smooth Embedding} }: \\
		& \text{Homeo.} + \text{Immers.}\\[-3pt]
		& \mss{\bullet\ \text{Immer}\ \phi: M \rightarrow N }\\[-5pt]
		& \mss{\bullet\ \text{Homeo}\ \overline{\phi}: M \rightarrow \phi(\mathcal{M}) \subseteq N}\\
	\end{aligned}
\)

\vspace{5pt}
\(
	\left. \bullet\ \mss{ \begin{aligned}
		& \text{Immer}\ \phi: M \rightarrow N \\
		& \text{Diffeo}\ \overline{\phi}: M \rightarrow \phi({M}) \\
	\end{aligned} }
	\right\} \mss{\Rightarrow\ \text{Induc}\ \iota = \phi \circ \overline{\phi}^{-1}: \phi(M) \rightarrow N
	\Rightarrow\ \text{Subman}\ {\phi}(M)}
	\hspace{20pt}
	\begin{aligned}
		& \bullet\ P \stackrel{\text{subman}}{\hookrightarrow} N \ \Rightarrow\ \text{\scriptsize Immer}\ \iota: P \rightarrow N
	\end{aligned}
\)

\vspace{10pt}
% Immersed Manifold Subset = Immersed Submanifold
\(
	\underline{ \begin{gathered}
		\text{Immersed Submanifold}\\[-5pt]
		\text{\scriptsize(Immersed Manifold Subset)}\\
	\end{gathered} }
	: \ \ 
	\text{\scriptsize Mani}\ P \subset \text{\scriptsize Mani}\ N 
	,\ \text{\scriptsize Immer}\ \iota: P \rightarrow N
\)

\vspace{10pt}
% Examples (Skipped)
* \underline{Examples (skipped, p.430)}:\ \ Hyperbolic Geom., Flat Torus, \(P^2\), Klein Bottle


%-----------------------------------------------------------------------------------------------------------------------------
%
%
%
\newpage
% Riemannian Metric
\(
	\begin{gathered}
		\underline{ \text{Riemannian Metric} }\ \text{for}\\[-4pt]
		\underline{ \text{Geometric Surface} }\\[-5pt]
		\text{\scriptsize(Riem. mani. of dim. n)}
	\end{gathered}
	\ , \
	\begin{gathered}
		\left< \ \ , \ \ \right>_p\\
		\forall T_p(M)	
	\end{gathered}
	\ :\ 
	\begin{aligned}
		& \bullet\ g_{ij} = \left< \tfrac{\partial }{\partial u_i}, \tfrac{\partial }{\partial u_j} \right>_p
			\ \rightarrow\ \mss{ E = g_{11} ,\ F = g_{12} = g_{21} ,\ G = g_{22} }
			\\
		& \bullet\ \Vert w = \mss{ \sum_i u_i' \tfrac{\partial }{\partial u_i} } \Vert_p^2 
			= \mss{ \sum_i } \hs g_{ii} (u_i')^2 + \mss{ \sum_{i \neq j} } \hs 2 g_{ij} u_i' u_j'
			\hspace{15pt} \text{\scriptsize(+ check cond. 2)}
	\end{aligned}
\)

\vspace{15pt}
% Covariant Derivative
\(
	\underline{ 
		\begin{gathered}
			{ \text{Covariant Derivative of} }\\[-4pt]
			{ \text{\scriptsize vec. field \(v\) rela. to vec. field \(w\)} }
		\end{gathered}
		\ ,\ D_w(v) 
	} 
	:\ \begin{aligned}
		& \bullet\ D_{f\mss{(M)} u + g\mss{(M)} w}(v) = f D_u(v) + g D_w(v)\\
		& \bullet\ D_{v} \big( f\mss{(M)} u + g\mss{(M)} w \big) 
			= f D_v(u) + \tfrac{\partial f}{\partial v} u
			+ g D_v(w) + \tfrac{\partial g}{\partial v} w
 			\\
		& \bullet\ \tfrac{\partial f}{\partial v} = \tfrac{d(f \circ \alpha)}{dt} \big|_0 \ , \ \alpha'(0) = v
	\end{aligned}
\)

\vspace{15pt}
% Christoffel Symbols
\(
	\underline{ 
		\begin{gathered}
			\text{Christoffel Symbols} \\[-5pt]
			\text{\scriptsize(for covar. der., not \(X_{u_i u_j}\))} 
		\end{gathered}
		\ , \ \Gamma^k_{ij} 
	}
	:\ 
	\bullet\ D_{X_i} X_j = \begin{gathered} \mss{ \displaystyle \sum_{k=1}^n } \end{gathered} \ \Gamma^k_{ij} X_k
	\ \Rightarrow\hs\hs 
	\begin{aligned}[t]
		\mss{ \left[ \begin{matrix}
				D_{X_1} X_1\\
				D_{X_1} X_2\\
				\vdots
			\end{matrix} \right] }
			\mss{ \left[ X_1 \ X_2 \ \dots \right] }
			& = 
			\mss{ \left[ \hs \arraycolsep=2pt \begin{matrix}
				\Gamma^1_{11} & \Gamma^2_{11} & \dots\\[2pt]
				\Gamma^1_{12} & \Gamma^2_{12} & \dots\\[-2pt]
				\vdots & \vdots
			\end{matrix} \right] }
			\mss{ \left[ \begin{matrix}
				X_1 \\[2pt]
				X_2 \\[-2pt]
				\vdots
			\end{matrix} \right] }
			\mss{ \left[ X_1 \ X_2 \ \dots \right] }
			\\
		& = 
			\mss{ \left[ \hs \arraycolsep=2pt \begin{matrix}
				\Gamma^1_{11} & \Gamma^2_{11} & \dots\\[2pt]
				\Gamma^1_{12} & \Gamma^2_{12} & \dots\\[-2pt]
				\vdots & \vdots
			\end{matrix} \right] }
			\mss{ \left[ \hs \arraycolsep=2pt \begin{matrix}
				g_{11} & g_{12} & \dots\\[2pt]
				g_{12} & g_{22} & \dots\\[-2pt]
				\vdots & \vdots
			\end{matrix} \right] }
			\\
	\end{aligned}
\)

% continue. christoffel
\vspace{-35pt}
\(	
	\begin{aligned}
		& \mss{ 
				\bullet\ D_{X_i} X_j = D_{X_j} X_i \ \leftrightarrow\ \Gamma^k_{ij} = \Gamma^k_{ji}
				\hspace{10pt}
				\bullet\ \tfrac{\partial}{\partial u_i} \left< X_j, X_k \right> 
				= \left< D_{X_i} X_j, X_k \right> + \left< X_j, D_{X_i} X_k \right>
			}
			\\
		& \Rightarrow\ \ast\ \boxed{
				\tfrac{1}{2}
				\left( \tfrac{\partial}{\partial u_i} g_{jk} 
				+ \tfrac{\partial}{\partial u_j} g_{ik} 
				- \tfrac{\partial}{\partial u_k} g_{ij} \right)
				= \mss{ \left< D_{X_i} X_j, X_k \right> }
				= \hs\hs \begin{gathered} \mss{\sum_n} \end{gathered} \hs\hs \Gamma^n_{ij} \hs g_{nk}
			}
			\\
		& \Rightarrow\ \ast\ 
			\underline{
				\mss{ \left[ \hs \arraycolsep=2pt \begin{matrix}
					\Gamma^1_{11} & \Gamma^2_{11} & \dots\\[2pt]
					\Gamma^1_{12} & \Gamma^2_{12} & \dots\\[-2pt]
					\vdots & \vdots
				\end{matrix} \right] }
				=
				\mss{ \left[ \begin{matrix}
					D_{X_1} X_1\\
					D_{X_1} X_2\\
					\vdots
				\end{matrix} \right] }
				\mss{ \left[ X_1 \ X_2 \ \dots \right] }
				\begin{gathered}[b]
					\mss{(g_{ij}^{-1} = g^{ij})} \\[-5pt]
					\mss{ \left[ \hs \arraycolsep=2pt \begin{matrix}
							g_{11} & g_{12} & \dots\\[2pt]
							g_{12} & g_{22} & \dots\\[-2pt]
							\vdots & \vdots
						\end{matrix} \right]^{-1} }
				\end{gathered}
			}
			\ \leftrightarrow\ 
			\boxed{ \begin{aligned}
				\Gamma_{ij}^k & = \mss{\sum_n} \hs \mss{ \left< D_{X_i} X_j, X_n \right> } \hs g^{nk}\\[-3pt]
				& = \tfrac{1}{2} \mss{\sum_n} \hs g^{nk} 
					\left[ \tfrac{\partial g_{jn}}{\partial u_i}  
					+ \tfrac{\partial g_{in}}{\partial u_j} 
					- \tfrac{\partial g_{ij}}{\partial u_n} \right]
			\end{aligned} }
	\end{aligned}
\)

\vspace{15pt}
% Sectional Curvature
\(
	\underline{ 
		\begin{gathered}
			\text{Sectional Curvature}\\[-3pt]
			\text{of \(M\) at \(p\) along \(\sigma\)}
		\end{gathered}
		,\ 
		K \Big( \mss{ \begin{aligned}
			p & \in M, \\[-4pt]
			\sigma & \in T_p(M) 
		\end{aligned} } \Big)
	}
	:\ 
	K(p, \sigma) = K_p(S) \ \big| \
	S \in \bigcup \hs \mss{ ( \text{geode. at} \ p \ \text{tangent to} \ \sigma ) }
	\hspace{10pt} \big( \mss{ \begin{gathered}
		\text{no else}\\[-5pt]
		\text{given}
	\end{gathered} } \big)
\)

%----------------------------------------------------------------------------------------------------------------------------------

\vspace{15pt}
% Metric Variable Change
\underline{Variable Change Inner Product is Isometric to Original Inner Product}

\vspace{10pt}
\(
	\begin{aligned}[t]
		w & = \sum_{i=1} \tfrac{du^i}{dt} \tfrac{\partial}{\partial u^i} 
			\equiv \left[\tfrac{du}{dt}\right]^i \left[\tfrac{\partial}{\partial u}\right]_i 
			\\[5pt]	
		\Vert w \Vert^2 & = \sum_{i,j} \tfrac{du^i}{dt} \tfrac{du^j}{dt} 
			\left< \tfrac{\partial}{\partial u^i} , \tfrac{\partial}{\partial u^i} \right>
			\\
		& \equiv \boxed{ \left[\tfrac{du}{dt}\right]^i \left[\tfrac{du}{dt}\right]^j g_{ij} }
	\end{aligned}
	\hspace{10pt} \vline \hspace{10pt}
	\begin{aligned}[t]
		\overline{w} & = \sum_{n=1} \tfrac{dx^n}{dt} \tfrac{\partial}{\partial x^n} 
			\equiv \left[\tfrac{dx}{dt}\right]^n \left[\tfrac{\partial}{\partial x}\right]_n 
			\\
		\left[\tfrac{dx}{dt}\right]^n & = \sum_{i=1} \tfrac{\partial x^n}{\partial u^i} \tfrac{du^i}{dt}
			\equiv \left[ \tfrac{\partial x}{\partial u} \right]^n_{\ i} \left[\tfrac{du}{dt}\right]^i 
			\\
		\left[\tfrac{\partial}{\partial x}\right]_n & = \sum_{i=1} \tfrac{\partial u^i}{\partial x^n} \tfrac{\partial}{\partial u^i} 
			\equiv \left[ \tfrac{\partial u}{\partial x} \right]^i_{\ n} \left[\tfrac{\partial}{\partial u}\right]_i 
	\end{aligned}
	\hspace{10pt} \vline \hspace{10pt}
	\begin{aligned}[t]
		\delta_{ij} & = \left[ \tfrac{\partial x}{\partial u}^{-1} \right]^i_{\ n} \left[ \tfrac{\partial x}{\partial u} \right]^n_{\ j}
			\\
		\delta^i_{\ j} & = \left[ \tfrac{\partial u}{\partial x} \right]^i_{\ n} \left[ \tfrac{\partial x}{\partial u} \right]^n_{\ j} 
	\end{aligned}
\)

\vspace{15pt}
\(
	\begin{aligned}
		\Vert \overline{w} \Vert^2 = \sum_{n,m} \tfrac{dx^n}{dt} \tfrac{dx^m}{dt} 
			\left< \tfrac{\partial}{\partial x^n} , \tfrac{\partial}{\partial x^m} \right>
			& \equiv \left[\tfrac{dx}{dt}\right]^n \cdot \left[\tfrac{dx}{dt}\right]^m \cdot h_{nm}
			\\[-5pt]
		& = 
			\left[ \tfrac{\partial x}{\partial u} \right]^n_{\ k} \left[\tfrac{du}{dt}\right]^k 
			\cdot \left[ \tfrac{\partial x}{\partial u} \right]^m_{\ \ l} \left[\tfrac{du}{dt}\right]^l
			\cdot \left[ \tfrac{\partial u}{\partial x} \right]^i_{\ n} \left[ \tfrac{\partial u}{\partial x} \right]^j_{\ m} \ g_{ij}
			\\[5pt]
		& = 
			\left[\tfrac{du}{dt}\right]^k 
			\left[ \tfrac{\partial u}{\partial x} \right]^i_{\ n} 
			\left[ \tfrac{\partial x}{\partial u} \right]^n_{\ k} 
			\cdot \left[\tfrac{du}{dt}\right]^l
			\left[ \tfrac{\partial u}{\partial x} \right]^j_{\ m} 
			\left[ \tfrac{\partial x}{\partial u} \right]^m_{\ \ l}
			g_{ij} 
			\\[5pt]
		\boxed{ \left[\tfrac{du}{dt}\right]^i \left[\tfrac{du}{dt}\right]^j g_{ij} } 
			& = \left[\tfrac{du}{dt}\right]^k \delta^i_{\ k} \cdot \left[\tfrac{du}{dt}\right]^l \delta^j_{\ l} \ g_{ij}
	\end{aligned}
\)

%--------------------------------------------------------------------------------------------------------------------------------
%--------------------------------------------------------------------------------------------------------------------------------
%--------------------------------------------------------------------------------------------------------------------------------
%--------------------------------------------------------------------------------------------------------------------------------

\newpage
\underline{Simplify the Following into one Fraction or Better (do the work on some paper):}

\vspace{15pt}
\(
	\begin{aligned} 
		1.)\ & (abc)^2\\[5pt]
		2.)\ & (ac^{3}b^{-1})^2\\[5pt]
		3.)\ & \left( \tfrac{ac^3}{b} \right)^{-2} \\[5pt]
		3.)\ & \frac{ \frac{a}{b} }{ c } \\[5pt]
	\end{aligned}
	\hfill
	\begin{aligned} 
		1.)\ & a / b\\[5pt]
		2.)\ & 1 / a \cdot b \\[5pt]
		3.)\ & a^{-1}b^2 \\[5pt]
		2.)\ & 1 / (a / b) \\[5pt]
		2.)\ & 1 / a / b \\[5pt]
	\end{aligned}
	\hfill
	\begin{aligned} 
		2.)\ & \frac{a}{ \frac{c}{d} } \\[5pt]
		3.)\ & \frac{ \frac{a}{b} }{ c } \\[5pt]
		1.)\ & \tfrac{ a/b }{ c/d } \\[5pt]
		1.)\ & 1 / b^{-m}
	\end{aligned}
	\hfill \hs
\)

\vspace{15pt}
\(
	\begin{aligned} 
		1.)\ & \frac{b^n b^m}{b^{m+n}}\\[5pt]
		1.)\ & b^n \frac{ b^{-n+m} }{b^{-3}}\\[5pt]
		3.)\ & \left( \tfrac{ac^3}{b} \right)^{-2} \\[5pt]
		3.)\ & \frac{ \frac{a}{b} }{ c } \\[5pt]
	\end{aligned}
	\hfill
	\begin{aligned} 
		1.)\ & a / b\\[5pt]
		2.)\ & 1 / a \cdot b \\[5pt]
		3.)\ & a^{-1}b^2 \\[5pt]
		2.)\ & 1 / (a / b) \\[5pt]
		2.)\ & 1 / a / b \\[5pt]
	\end{aligned}
	\hfill
	\begin{aligned} 
		2.)\ & \frac{a}{ \frac{c}{d} } \\[5pt]
		3.)\ & \frac{ \frac{a}{b} }{ c } \\[5pt]
		1.)\ & \tfrac{ a/b }{ c/d } \\[5pt]
	\end{aligned}
	\hfill \hs
\)

\end{document}