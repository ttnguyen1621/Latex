\documentclass[12pt]{article}
\usepackage[left=.75in, right=.75in, top=1in, bottom = 1in]{geometry}
\usepackage{amssymb, amsmath, amsfonts, mathtools, bbm}
\usepackage{array,multirow}
\usepackage{cancel}

\newcommand{\hs}{\hspace{1pt}} % 1pt horizontal space
\newcommand{\hsvec}[1]{\vec{\hs #1}} % 1pt space with a \vec
\newcommand{\nhs}{\hspace{-1pt}} % -1pt horizontal space
\newcommand{\mss}[1]{\text{\scriptsize\(#1\)}} % math scriptsize
\newcommand{\tss}[1]{\text{\scriptsize #1}} % text scriptsize

\newcommand{\checkedbox}{\mbox{\ooalign{$\checkmark$\cr\hidewidth$\square$\hidewidth\cr}}} % checked box
\newcommand{\crossbox}{\mbox{\ooalign{\ding{55}\cr\hidewidth$\square$\hidewidth\cr}}} % cross box

\begin{document}
\setlength{\parindent}{0pt}

%---------------------------------------------------------------------------------------------------------------------------------
%---------------------------------------------------------------------------------------------------------------------------------
%---------------------------------------------------------------------------------------------------------------------------------
%---------------------------------------------------------------------------------------------------------------------------------
\newpage

\(
	\mss{ \begin{gathered}
		\boxed{ \hsvec{\nabla} = \left[ \hsvec{\nabla} \left( r, \theta, \phi \right) \right] \bar{\partial}_\circ }
			\\[5pt]
		d = \left[ dx \ dy \ dz \right] \hsvec{\nabla} = d\hsvec{l}\hs\hs^T \hsvec{\nabla} 
			\\[5pt]
		\begin{aligned}
				d ( r, \theta, \phi ) 
					& = \left[ dx \ dy \ dz \right]
					\hsvec{\nabla} 
					( r, \theta, \phi )
					\\
				\Aboxed{ \partial \bar{l}_\circ\nhs^T & = d\hsvec{l}\hs\hs^T \hsvec{\nabla} (r, \theta, \phi) }
			\end{aligned}
			\\[5pt]
		\begin{aligned}
			\partial \bar{l}_\circ\nhs^T \bar{\partial}_\circ & 
				= d\hsvec{l}\hs\hs^T \left[ \hsvec{\nabla} ( r, \theta, \phi ) \right] \bar{\partial}_\circ
				\\
			\Aboxed{ d = \partial \bar{l}_\circ\nhs^T \bar{\partial}_\circ & = d\hsvec{l}\hs\hs^T \hsvec{\nabla} }
		\end{aligned}
	\end{gathered} } 
	\hfill \vline \hfill
	% Matrix Form for cartesian gradient and d(spherical)
	\begin{aligned}
		% Partial / Partial xyz
		\underline{ \nhs \hsvec{\nabla} } &  
			= 
			\left[\begin{matrix}
				\\[-12pt]
				\tfrac{\partial}{\partial x}\\[5pt]
				\tfrac{\partial}{\partial y}\\[5pt]
				\tfrac{\partial}{\partial z}
			\end{matrix}\right] 
			= 
			\left[ \mss{ \arraycolsep=2pt\begin{matrix}
				| & | & | \\
				\nabla r & \nabla \theta & \nabla \phi \\
				| & | & |
			\end{matrix} } \right]
			\left[\begin{matrix}
				\\[-12pt]
				\tfrac{\partial}{\partial r}\\[5pt]
				\tfrac{\partial}{\partial \theta}\\[5pt]
				\tfrac{\partial}{\partial \phi}
			\end{matrix}\right]
			\\[10pt]
		\underline{ \partial \bar{l}_\circ } & = \nhs
			\left[ { \begin{matrix}
				dr\\
				d\theta\\
				d\phi
			\end{matrix} } \right] 
			\begin{aligned}
				& = \hs\hs [\hsvec{\nabla} (r, \theta, \phi)]^T d\hsvec{l}\\[5pt]
				& = \left[ \mss{ \begin{matrix}
						- \nabla r - \\
						- \nabla \theta - \\
						- \nabla \phi - \\
					\end{matrix} } \right]
					\left[ \mss{ \begin{matrix}
						dx\\
						dy\\
						dz
				\end{matrix} } \right]
			\end{aligned}
	\end{aligned}
	\hfill\vline\hfill
	% \theta Example
	\begin{aligned}
		\theta & = \theta \mss{(x,y,z)} 
			\hspace{10pt} ( \mss{ x^2 + y^2 = z^2 \tan^2{\theta} } )
			\\
		\phi & = \phi \mss{(x,y,z)} 
			\hspace{10pt} ( \mss{ y = x \tan{\phi} } )
			\\[10pt]
		\tfrac{\partial}{\partial x} & = \tfrac{\partial r}{\partial x} \tfrac{\partial}{\partial r}
			+ \tfrac{\partial \theta}{\partial x} \tfrac{\partial}{\partial \theta}
			+ \tfrac{\partial \phi}{\partial x} \tfrac{\partial}{\partial \phi} 
			\\[5pt]
		\tfrac{\partial}{\partial \theta} & = \tfrac{\partial x}{\partial \theta} \tfrac{\partial}{\partial x}
			+ \tfrac{\partial y}{\partial \theta} \tfrac{\partial}{\partial y}
			+ \tfrac{\partial z}{\partial \theta} \tfrac{\partial}{\partial z}
			\\[10pt]
		\mss{ d\theta} & = \mss{ dx \tfrac{\partial \theta}{\partial x} 
			+ dy \tfrac{\partial \theta}{\partial y} 
			+ dz \tfrac{\partial \theta}{\partial z} }
			\\
		\mss{dy_{\vec{r}_\circ}(\vec{r}_\circ \hs\nhs\nhs\nhs\nhs')}|_{t=0} & = \mss{ ( \tfrac{\partial y}{\partial r} dr
			+ \tfrac{\partial y}{\partial \theta} d\theta
			+ \tfrac{\partial y}{\partial \phi} d\phi ) \tfrac{1}{dt}|_{t=0} }
	\end{aligned}
\)

\vspace{15pt}
% Contravector Dot Covector
\(
	\boxed{ \hsvec{b}^{\hs i} \cdot \hsvec{b}_{i} = \delta_{ij} }:\ 
	\begin{aligned}
		& \hsvec{\nabla}{\phi} \cdot \tfrac{\partial \hsvec{r}}{\partial \phi} = 1\\[5pt]
		& \hsvec{\nabla}{\phi} \cdot \tfrac{\partial \hsvec{r}}{\partial \theta} = 0
	\end{aligned}
	\ \Rightarrow\ 
	\left[ \mss{ \begin{matrix}
		- \hsvec{\nabla} r - \\
		- \hsvec{\nabla} \theta - \\
		- \hsvec{\nabla} \phi - 
	\end{matrix} } \right]
	\left[ 
		\tfrac{\partial}{\partial r} \ \tfrac{\partial}{\partial \theta} \ \tfrac{\partial}{\partial \phi}
	\right]
	\hsvec{r}
	= \mathbbm{1}_3
	\ \ \Rightarrow\ \
	\boxed{
		\tfrac{\partial y}{\partial \phi} = 
		\left[ \mss{ 
			0 \ 1 \ 0
		} \right]
		\left[ \mss{ \begin{matrix}
			- \hsvec{\nabla} r - \\
			- \hsvec{\nabla} \theta - \\
			- \hsvec{\nabla} \phi - 
		\end{matrix} } \right]^{-1}
		\left[ \mss{ \begin{matrix}
			0 \\ 0 \\ 1
		\end{matrix} } \right]
		= \tfrac{\partial y}{\partial \phi}^T
	}
\)

\vspace{15pt}
% d operator
\(
	\begin{aligned}
		d & = dx \tfrac{\partial }{\partial x}
			+ dy \tfrac{\partial }{\partial y} 
			+ dz \tfrac{\partial }{\partial z} 
			= d\hsvec{l} \cdot \hsvec{\nabla} 
			&& = \left[ 
				\tfrac{dr}{\lVert \nabla r \rVert} , 
				\tfrac{d\theta}{\lVert \nabla \theta \rVert} ,
				\tfrac{d\phi}{\lVert \nabla \phi \rVert} 
			\right] 
			\left[ 
				\mss{ \lVert \nabla r \rVert } \tfrac{\partial }{\partial r} ,\hs
				\mss{ \lVert \nabla \theta \rVert } \tfrac{\partial }{\partial \theta} ,\hs
				\mss{ \lVert \nabla \phi \rVert } \tfrac{\partial }{\partial \phi} 
			\right]^T
			\\
		& = dr \tfrac{\partial }{\partial r} 
			+ d\theta \tfrac{\partial }{\partial \theta} 
			+ d\phi \tfrac{\partial }{\partial \phi} 
			= \partial \bar{l}_\circ\nhs^T \bar{\partial}_\circ 
			&& = \left[ dr, rd\theta, r\sin\theta d\phi \right]
			\left[ 
				\tfrac{\partial }{\partial r} 
				, \tfrac{1}{r} \tfrac{\partial }{\partial \theta} 
				,\tfrac{1}{r\sin\theta} \tfrac{\partial }{\partial \phi} 
			\right]^T
			\\[3pt]
		& = \tfrac{dr}{\lVert \nabla r \rVert} \mss{ \lVert \nabla r \rVert } \tfrac{\partial }{\partial r} 
			+ \tfrac{d\theta}{\lVert \nabla \theta \rVert} \mss{ \lVert \nabla \theta \rVert } \tfrac{\partial }{\partial \theta} 
			+ \tfrac{d\phi}{\lVert \nabla \phi \rVert} \mss{ \lVert \nabla \phi \rVert } \tfrac{\partial }{\partial \phi}
			&& \dots = d\bar{l}_\circ\nhs^T \bar{\nabla}_\circ = d\hsvec{l}_\circ\nhs^T \hsvec{\nabla}_\circ 
	\end{aligned}
\)

\vspace{20pt}
% Work to Find Line Element and Unit Spherical Vectors
\(
	% Line Element
	\begin{aligned}
		d\hsvec{l} = d\hsvec{r} & 
			= [ dr \tfrac{\partial}{\partial r} 
			+ d\theta \tfrac{\partial}{\partial \theta} 
			+ d\phi \tfrac{\partial}{\partial \phi} ]
			(x,y,z)^T
			\\
		d{(x, y, z)} & 
			= \left[ \mss{ \tfrac{dr}{\lVert \nabla r \rVert} { \lVert \nabla r \rVert } \tfrac{\partial }{\partial r} 
			+ \tfrac{d\theta}{\lVert \nabla \theta \rVert} { \lVert \nabla \theta \rVert } \tfrac{\partial }{\partial \theta} 
			+ \tfrac{d\phi}{\lVert \nabla \phi \rVert} { \lVert \nabla \phi \rVert } \tfrac{\partial }{\partial \phi} 
			} \right] (x,y,z)
			\\[5pt]
		(dx,dy,dz) &
			= dr \hs \hat{r}^T + r \hs d\theta \hs \hat{\theta}^T + r \sin\theta \hs d\phi \hs \hat{\phi}^T
	\end{aligned}
	\hfill\vline\hfill
	% Unit Spherical Vectors
	\begin{aligned}
		\big( \hat{r} , & \hat{\theta} , \hat{\phi} \big) 
			\equiv \underline{ \left( 
				\mss{ \lVert \nabla r \rVert } \tfrac{\partial \hsvec{r}}{\partial r} ,\hs
				\mss{ \lVert \nabla \theta \rVert } \tfrac{\partial \hsvec{r}}{\partial \theta} ,\hs
				\mss{ \lVert \nabla \phi \rVert } \tfrac{\partial \hsvec{r}}{\partial \phi} 
			\right) }
			\\
		& = \left(
				\tfrac{\partial}{\partial r} ,\
				\tfrac{1}{r} \tfrac{\partial}{\partial \theta} ,\
				\tfrac{1}{r \sin{\theta}} \tfrac{\partial}{\partial \phi} 
			\right)
			\otimes (x,y,z)^T
			\\
		& = \bar{\nabla}_\circ^T \otimes \hsvec{r}
	\end{aligned}
\)

\vspace{20pt}
% Line Element and Gradient
\(\begin{aligned}
	% Line Element
	& \boxed{ d\hsvec{l} = (dx, dy, dz) \cdot \big( \hat{x} , \hat{y} , \hat{z} \big) }
		= \boxed{ (dr, rd\theta, r\sin\theta d\theta) \cdot \big( \hat{r} , \hat{\theta} , \hat{\phi} \big) = d\hsvec{l}_\circ }
		= d\bar{l}_\circ\nhs^T \cdot \big( \hat{r} , \hat{\theta} , \hat{\phi} \big)
		\\
	% Gradient
	& \boxed{ 
			\hsvec{\nabla} = (\hat{x}, \hat{y}, \hat{z}) 
			\cdot \left( \tfrac{\partial}{\partial x}, \tfrac{\partial}{\partial y}, \tfrac{\partial}{\partial z} \right) 
		}
		= \boxed{
			(\hat{r}, \hat{\theta}, \hat{\phi}) 
			\cdot \left( 
				\tfrac{\partial}{\partial r}, 
				\tfrac{1}{r} \tfrac{\partial}{\partial \theta}, 
				\tfrac{1}{r\sin\theta} \tfrac{\partial}{\partial \phi}
			\right) 
			= \hsvec{\nabla}_\circ
		}
		= \left(
			\tfrac{\partial \hsvec{r}}{\partial r} ,
			\tfrac{1}{r} \tfrac{\partial \hsvec{r}}{\partial \theta} ,
			\tfrac{1}{r \sin{\theta}} \tfrac{\partial \hsvec{r}}{\partial \phi} 
		\right) \bar{\nabla}_\circ^T
\end{aligned}\)

\vspace{10pt}
% Two ways to write cartesian partial derivatives 
\(\begin{aligned}
	\hsvec{\nabla} & = [ \bar{\nabla}_\circ^T \otimes \hsvec{r} \hs ]  \bar{\nabla}_\circ
		= \left[ \bar{\nabla}_\circ^T \otimes (x,y,z)^T \right]
		\left[ \mss{ \begin{matrix}
			\tfrac{\partial}{\partial r}\\[4pt]
			\tfrac{1}{r} \tfrac{\partial}{\partial \theta}\\[4pt]
			\tfrac{1}{r\sin\theta} \tfrac{\partial}{\partial \phi} 
		\end{matrix} } \right]
		&& \Rightarrow\ 
		\tfrac{\partial}{\partial x} 
		= \underline{ \tfrac{\partial x}{\partial r} } \tfrac{\partial}{\partial r}
		+ \underline{ \mss{ \Vert \nabla \theta \Vert }^2 \tfrac{\partial x}{\partial \theta} } \tfrac{\partial}{\partial \theta}
		+ \underline{ \mss{ \Vert \nabla \phi \Vert }^2 \tfrac{\partial x}{\partial \phi} } \tfrac{\partial}{\partial \phi}
		\\
	& = [ \hsvec{\nabla}(r,\theta,\phi) ] \bar{\partial}_\circ
		= [ \hsvec{\nabla}(r,\theta,\phi) ] 
		\left[ \mss{ \begin{matrix}
			\tfrac{\partial}{\partial r}\\[4pt]
			\tfrac{\partial}{\partial \theta}\\[4pt]
			\tfrac{\partial}{\partial \phi} 
		\end{matrix} } \right]
		&&\Rightarrow\
		\tfrac{\partial}{\partial x} 
		= \underline{ \tfrac{\partial r}{\partial x} } \tfrac{\partial}{\partial r}
		+ \underline{ \tfrac{\partial \theta}{\partial x} } \tfrac{\partial}{\partial \theta}
		+ \underline{ \tfrac{\partial \phi}{\partial x} } \tfrac{\partial}{\partial \phi}
		\ \Rightarrow\ 
		\boxed{ \tfrac{\partial \phi}{\partial y} = \tfrac{\partial y}{\partial \phi} \mss{\Vert \nabla \phi \Vert}^2 }
\end{aligned}\)

\vspace{10pt}
% Unit Vectors Summary
\(\boxed{
	\arraycolsep=3pt
	\begin{array}{r c c c r c r c c c c c c c r}
		& & & & & & & 
			& \text{\scriptsize\underline{contravariant}}_{\hs i} 
			& \multicolumn{3}{c}{ \text{\scriptsize(equal since orthog.)} } 
			& \multicolumn{3}{l}{ \text{\scriptsize\underline{covariant}}^i } 
			\\[6pt]
		\hat{r} & = 
			& (\hat{r}_x, \hat{r}_y, \hat{r}_z) 
			& = 
			& { \tfrac{\hsvec{r}}{r} }
			& = 
			& \tfrac{\partial}{\partial r} \hsvec{r} 
			& = 
			& \tfrac{\partial \hsvec{r}}{\partial r} \Vert \tfrac{\partial \hsvec{r}}{\partial r}  \Vert^{-1} 
			& \ \ \stackrel{\rightarrow}{=}\ \
			& \mss{ \Vert \nabla r \Vert } \tfrac{\partial \hsvec{r}}{\partial r} 
			& \ \stackrel{\leftarrow}{=}\ \ 
			& \tfrac{\nabla r}{\Vert \nabla r \Vert}
			& = 
			& \nabla r
			\\[5pt]
		\hat{\theta} & = 
			& (\hat{\theta}_x, \hat{\theta}_y, \hat{\theta}_z) 
			& = 
			& { \tfrac{\partial \hat{r}}{\partial \theta} }
			& = 
			& \tfrac{1}{r} \tfrac{\partial}{\partial \theta} \hsvec{r}
			& =
			& \tfrac{\partial \hsvec{r}}{\partial\theta} \Vert \tfrac{\partial \hsvec{r}}{\partial\theta} \Vert^{-1} 
			& \ \ \stackrel{\rightarrow}{=}\ \
			& \mss{ \Vert \nabla \theta \Vert } \tfrac{\partial \hsvec{r}}{\partial\theta} 
			& \ \stackrel{\leftarrow}{=}\ \
			& \tfrac{\nabla \theta}{\Vert \nabla \theta \Vert} 
			& = 
			& r \nabla \theta
			\\[5pt]
		\hat{\phi} & = 
			& (\hat{\phi}_x, \hat{\phi}_y, \hat{\phi}_z) 
			& = 
			& { \tfrac{1}{\sin\theta} \tfrac{\partial \hat{r}}{\partial \phi} }
			& = 
			& \tfrac{1}{r \sin\theta} \tfrac{\partial}{\partial \phi} \hsvec{r}
			& =
			& \tfrac{\partial \hsvec{r}}{\partial\phi} \Vert \tfrac{\partial \hsvec{r}}{\partial\phi} \Vert^{-1} 
			& \ \ \stackrel{\rightarrow}{=} \ \
			& \mss{ \Vert \nabla \phi \Vert } \tfrac{\partial \hsvec{r}}{\partial\phi} 
			& \ \stackrel{\leftarrow}{=} \ \
			& \tfrac{\nabla \phi}{\Vert \nabla \phi \Vert}
			& = 
			& \mss{r \sin\theta} \nabla \phi
	\end{array}
}\)

%--------------------------------------------------------------------------------------------------------------------------------------
%--------------------------------------------------------------------------------------------------------------------------------------
%--------------------------------------------------------------------------------------------------------------------------------------
%--------------------------------------------------------------------------------------------------------------------------------------
\newpage
% Frenet
\section{Frenet Equations}

% Frenet in derivatives of position
\(
    \begin{aligned}
		\mss{ a \cdot (b \times c) } & = \mss{ (a \times b) \cdot c) } \\[5pt]
		\mss{ a \times (b \times c) } & = \mss{ (c \cdot a) b - (b \cdot a) c } \\
		\mss{ (a \times b) \times c } & = \mss{ b (c \cdot a) - a (c \cdot b)  } \\[5pt]
		\mss{ (a \times b) \cdot (c \times d) } & = \mss{ a \cdot b \times (c \times d) }
			\\
		= \mss{ \left| \hs
				\left[ \begin{matrix}
					a \hs \cdot \\ b \hs \cdot 
				\end{matrix} \right]
				\left[ c \ d \right] \hs
			\hs \right| }
			& = \mss{ \left| \ \begin{matrix}
                \\[-11pt]
                \hsvec{a} \cdot \hsvec{c} & \hsvec{a} \cdot \hsvec{d} \\[5pt]
                \hsvec{b} \cdot \hsvec{c} & \hsvec{b} \cdot \hsvec{d}
                \\[-12pt]
                \hs
            \end{matrix} \ \right| }
            \\[5pt]
        \Aboxed{ \tfrac{dt}{ds} & = \tfrac{1}{v} }
    \end{aligned}
    \hfill
    \vline
    \hfill
    \begin{aligned}
        T & = \hat{v} = \tfrac{\hsvec{v}}{v} \\
        \tfrac{dT}{dt} & = \tfrac{ (\hsvec{v} \cdot \hsvec{v}) \hsvec{a} - (\hsvec{v} \cdot \hsvec{a}) \hsvec{v} }{v^3} 
            = \tfrac{ \hsvec{v} \times (\hsvec{a} \times \hsvec{v}) }{v^3}
            = \tfrac{ (\hsvec{v} \times \hsvec{a}) \times \hsvec{v} }{v^3}
            \\
        \lVert \tfrac{dT}{dt} \rVert & = \tfrac{ \sqrt{ v^2 a^2 - (\hsvec{v} \cdot \hsvec{a})^2 } }{v^2}
            = \tfrac{\lVert \hsvec{a} \times \hsvec{v} \rVert }{v^2}
            \hspace{5pt} , \hspace{10pt} \tfrac{dT}{ds} = k \hat{N}
            \\
        \hat{N} & = \tfrac{T'}{\lVert T' \rVert} 
            = \tfrac{ (\nhs\hsvec{v} \times \hsvec{a}) \times \hsvec{v} }{\lVert \hsvec{v} \times \hsvec{a} \rVert v} 
            = \hat{B} \times \hat{v}
            \\
        \hat{B} & = \tfrac{\hsvec{v} \times \hsvec{a}}{\lVert \hsvec{v} \times \hsvec{a} \rVert}
            = \widehat{v \times a}
            = \hat{v} \times \hat{N}
            \hspace{15pt} \underline{ \mss{(\hat{B} \cdot \vec{v} = 0)} }
            \\
        \tfrac{d\hat{B}}{dt} & = \tfrac{\hsvec{v} \times \dot{\hsvec{a}}}{\Vert \hsvec{v} \times \hsvec{a} \Vert }
            - \left[ \tfrac{\hsvec{v} \times \dot{\hsvec{a}}}{\Vert \hsvec{v} \times \hsvec{a} \Vert } \cdot \hat{B} \right] \hat{B}
            \hspace{5pt} , \hspace{5pt} \tfrac{dB}{ds} = \tau \hat{N}
			\\
		\tau & = \hat{N} \cdot \tfrac{d\hat{B}}{ds} 
			= \tfrac{ \hat{B} \cdot \dot{\hsvec{a}} }{ \Vert \nhs\hsvec{v} \times \hsvec{a} \Vert }
			= \tfrac{ (\nhs\hsvec{v} \times \hsvec{a}) \cdot \dot{\hsvec{a}} }{ \Vert \nhs\hsvec{v} \times \hsvec{a} \Vert^2 }
    \end{aligned}
    \hfill
    \boxed{
        \begin{aligned}
            \hsvec{a} & = a_T \hat{T} + a_N \hat{N} \\[5pt]
            a_T & = \hsvec{a} \cdot \hat{v} = \tfrac{dv}{dt} \\[5pt]
            a_N & = \tfrac{\Vert \hsvec{a} \times \hsvec{v} \Vert}{v} = \Vert \hsvec{a} \times \hat{v} \Vert \\[5pt]
            a^2 & = a_T^2 + a_N^2 = \Vert \tfrac{d\hsvec{v}}{dt} \Vert^2
        \end{aligned}
    }
\)

\vspace{15pt}\noindent
% Frenet trihedron - when curve is parametrized by arc length
\underline{Frenet Trihedron}\\
\(
    \begin{aligned}
        & \text{\scriptsize Differentiable (in this book)}:\ C^\infty\\
        & \text{\scriptsize No singular pts. Order 0 (Regular)}:\ \hsvec{v}(t) \neq 0\\
        & \mss{ \bullet\ \Vert \hsvec{v}(t) \Vert = c \rightarrow 1
            \ \Rightarrow\ 
            \int_s \Vert \hsvec{v}(t) \Vert \hs dt = t = \Delta s 
            }
            \\
        & \hspace{10pt} \mss{ \rightarrow s:\ \hsvec{x}(t) = \hsvec{x}(s) } \\
        & \mss{ \bullet\ \tfrac{1}{2}\tfrac{d}{dt}(\hsvec{v} \cdot \hsvec{v}) = \boxed{ \hsvec{v} \cdot \hsvec{a} = 0 } } \\
        & \text{\scriptsize No singular pts. Order 1}:\ \hsvec{a}(t) \neq 0\\
        & \mss{ \bullet\ \text{Curvature, } k \neq 0 \text{ (see right)} }
			\hspace{10pt} \mss{ \bullet\ \text{Vertex, } k' = 0 }
    \end{aligned}
    \hfill\vline\hfill
    \begin{aligned}
        & 1 = \Vert \hsvec{t} \Vert = \Vert \hsvec{n} \Vert = \Vert \hsvec{b} \Vert
            \ ,\ \ 
            0 = \hsvec{t} \cdot \hsvec{n} = \hsvec{n} \cdot \hsvec{b} = \hsvec{b} \cdot \hsvec{t}
            \\
        & \bullet\ \hsvec{v}(s) = \hsvec{t}(s) \hspace{20pt} \boxed{ \mss{ (t = n \times b) } } \\
        & \bullet\ \hsvec{a}(s) = \boxed{ \hsvec{t'}(s) = k\mss{(s)} \hsvec{n}\mss{(s)} }
            ,\ k(s) \geq 0
            \hspace{10pt} \begin{gathered}
                \text{\scriptsize(can be L or R-handed)}\\[-8pt]
                \text{\scriptsize(can be neg. if in \(\mathbb{R}^2\))}
            \end{gathered}
            \\
        & \ast\ k(s) > 0 \text{ for well defined curve with \(\hat{n}\)}\\
        & \bullet\ \boxed{ \hsvec{b} = \hsvec{t} \times \hsvec{n} }
            ,\ \ \tfrac{d}{dt} (\hsvec{b} \cdot \hsvec{b}) = \hsvec{b} \cdot \hsvec{b'} = 0
            ,\ \ \ast\ \boxed{ \hsvec{b'}(s) = \tau\mss{(s)} \hsvec{n}\mss{(s)} }
            \\
        & \bullet\ \boxed{ \hsvec{n} = \hsvec{b} \times \hsvec{t} } 
            ,\ \ \ast\ \boxed{ \hsvec{n'}(s) = -k \hsvec{t} - \tau \hsvec{b} }
            ,\ \ \ast\ \mss{ \text{t-n pl.} = \text{osculating pl.} }
    \end{aligned}
\)

% More info on derivatives of frenet
\vspace{10pt} \noindent
\(\begin{aligned}
    & \bullet\ t''(s) = k' n - k^2 t - k\tau b 
        \hspace{20pt} \bullet\ b''(s) = \tau' n - \tau kt - \tau^2 b
        \hspace{20pt} \bullet\ n''(s) = -k't -\tau'b - (k^2 + \tau^2) n
        \\
    & \bullet\ |\tau| = \Vert b' \Vert 
        \hspace{20pt}
        \bullet\ \tau = -\tfrac{ ( t \times t' ) \cdot t'' }{k^2} = -\tfrac{ {t} \cdot ( {t'} \times t'' )}{\Vert t' \Vert^2} 
        \hspace{20pt} 
        \bullet\ k = \Vert t' \Vert = \tfrac{ ( b \times b' ) \cdot b'' }{\tau^2} 
        = \tfrac{ {b} \cdot ( {b'} \times b'' )}{\Vert b' \Vert^2}
        \\
    & \bullet\ n \Rightarrow k,\tau:
        \hspace{20pt}
        \ast\ \Vert n' \Vert^2 = k^2 + \tau^2
        \hspace{20pt}
        \ast\ \tfrac{ ( n \times n' ) \cdot n'' }{\Vert n' \Vert^2} = \tfrac{k' \tau - k \tau'}{k^2 + \tau^2}
        = \tfrac{ \tfrac{d}{ds} (k / \tau) }{(k/\tau)^2 + 1} = \tfrac{d}{ds} \arctan(k/\tau)
\end{aligned}\)

% Indicatrix
\vspace{15pt}\noindent
\(\begin{aligned}[t]
    & \text{\underbar{Indicatrix [of Tangents]}}:\\[5pt]
    & \bullet\ \hsvec{t}(\theta\mss{(s)}) = (\cos\theta, \sin\theta) = (x'\mss{(s)}, y'\mss{(s)})\\
    & \bullet\ \hsvec{t'}(\theta) = \underline{ \theta'(s)} (-\sin\theta, \cos\theta) = \underline{k(s)} \hsvec{n}\\
	& \bullet\ \theta\mss{(s)} = \arctan(y'/x')\\
	& \bullet\ \mss{\int_0^l} \hs k\mss{(s)} \hs\hs ds = \theta(s)\Big|^l_0 = 2\pi I_\text{rot. index}
\end{aligned}\)
% Local Canonical Form
\hspace{20pt}
\(\begin{aligned}[t]
    & \text{\underbar{Local Canonical Form at \(t=0\)}}:\\[5pt]
    & \bullet\ (\hat{t},\hat{n},\hat{b}) = (\hat{x},\hat{y},\hat{z})\\
    & \bullet\ \hsvec{r}(s) - \hsvec{r}(0) 
        \approx ( s - \tfrac{k^2 s^3}{6},\ \tfrac{k}{2} s^2 + \tfrac{k' s^3}{6},\ \tfrac{-k\tau}{6} s^3 )
        \\
    & \bullet\ \tau < 0 \ \Rightarrow\ \tfrac{dz}{ds} > 0
\end{aligned}\)

% Isoperimetric Ineq.
\vspace{15pt}\noindent
\(\begin{aligned}[t]
	& \text{\underline{Isoperimetric Inequality}}:\ 0 \leq l^2 - 4\pi A
\end{aligned}\)
% 4-Vertex Theor.
\hspace{20pt}\noindent
\(\begin{aligned}[t]
	& \text{\underline{Four-Vertex Theorem}}:\ \text{\scriptsize A simple closed curve has \(\geq\) 4 vertices}
\end{aligned}\)

% Cauchy-Crofton Formula
\vspace{15pt}\noindent
\(\begin{aligned}[t]
	& \text{\underline{Cauchy-Crofton Formula (measure of number of times lines intersect a curve)}}:\\[5pt]
	& \bullet\ \text{Tangent line at } (\rho, \theta) :\ x\cos\theta + y\sin\theta = \rho
		\hspace{15pt} \bullet\ \text{Curve } c:\ y=0,\ x \in (-l/2, l/2)
		\hspace{5pt} , \hspace{10pt} C = \mss{\sum}\hs c_i
		\\
	& \bullet\ \mss{\int}\ \text{\scriptsize Lines that cross } c 
		= \mss{ \int_0^{2\pi} \int_0^{|\cos\theta| l/2} } d\rho \hs d\theta = 2l
		\ \Rightarrow\
		\mss{ \int_0^{2\pi} \int_0^\infty } n_C \ d\rho \hs d\theta = 2l
\end{aligned}\)

%--------------------------------------------------------------------------------------------------------------------------------------
%--------------------------------------------------------------------------------------------------------------------------------------
%--------------------------------------------------------------------------------------------------------------------------------------
%--------------------------------------------------------------------------------------------------------------------------------------
\newpage
% Differential
\section{Jacobian/Differential, \(dF_{\alpha(0)} : \mathbb{R}^n \rightarrow \mathbb{R}^m\)}

\(\begin{aligned}
	& \bullet\ \boxed{ \alpha(0) = \beta(0) } \ \Rightarrow\ \underline{F(t=0)} 
		= F \circ \alpha \big|_{t=0} = F \circ \beta \big|_{t=0} 
		\\[5pt]
	& \bullet\ \boxed{ \alpha'(0) = \beta'(0) } \ \Rightarrow \tfrac{\partial x}{\partial \alpha_i} \big|_{t=0} 
		= \tfrac{\partial x}{\partial \beta_i} \big|_{t=0} 
		\cdot \bcancel{ \tfrac{d \beta_i / dt}{d \alpha_i / dt} \big|_{t=0} }
		\ \Rightarrow\ \nhs\nhs \boxed{ dF_{\alpha(0)}(\alpha'\mss{(0)}) = dF_{\beta(0)}(\beta'\mss{(0)}) }
		\hspace{5pt} \underline{ \text{\scriptsize(doesn't depend on \(\alpha\))} }
		\\[5pt]
	& \ast\ F = (f_0, f_1, \dots, f_m) 
		\ \Rightarrow\ \underline{ dF_{\alpha(0)} (\alpha'\mss{(0)}) } \hs \equiv\hs \tfrac{d}{dt}(F \circ \alpha) \big|_{t=0}
		= \left[ 
			\mss{ \arraycolsep=3pt \begin{matrix}
				\tfrac{\partial f_0}{\partial \alpha_0} & \tfrac{\partial f_0}{\partial \alpha_1} & \dots \\[7pt]
				\tfrac{\partial f_1}{\partial \alpha_0} & \tfrac{\partial f_1}{\partial \alpha_1} & \dots \\
				\vdots
			\end{matrix} }
		\right]_{t=0}
		\left[ 
			\mss{ \begin{matrix}
				\tfrac{d \alpha_0}{dt}\\[7pt]
				\tfrac{d \alpha_1}{dt}\\
				\vdots
			\end{matrix} }
		\right]_{t=0}
		\hspace{-10pt} = \boxed{ J_F\mss{(0)} \cdot \alpha'\mss{(0)} }
		\\[5pt]
	& \bullet d(G \circ F)_p = dG_{F(p)} \circ dF_p
		\hspace{15pt} \bullet\ \underline{ \text{\scriptsize Regular Value, \(p\)} }:\ dF_p \neq 0
		\hspace{15pt} \bullet\ \underline{ \text{\scriptsize Critical Value, \(p\)} }:\ dF_p = 0
\end{aligned}\)

\vspace{15pt}\noindent
% F is a Homeomorphism
\(
	\mss{ \begin{gathered}
		{F\ \text{is a}}\\[-2pt]
		\underline{\text{Homeomorphism}}\\[-2pt]
		\text{onto image}\ F(X)
	\end{gathered} }:\ 
	\begin{aligned}
		& \bullet\ F \ \text{\scriptsize is bijective between \(X\ \&\ F(X)\) }\\
		& \bullet\ F \ \text{\scriptsize is cont.}
			\hspace{10pt} \bullet\ F^{-1} \ \text{\scriptsize is cont.}
	\end{aligned}
\)
\hfill
% F is a Diffeomorphism
\(
	\mss{ \begin{gathered}
		{F\ \text{is a}}\\[-2pt]
		\text{\underline{Diffeomorphism}}\\[-2pt]
		\text{onto image}\ F(X)
	\end{gathered} }:\ 
	\begin{aligned}
		& \bullet\ F \in C^{\infty} \ \ \text{\scriptsize(cont. part. deri. of all orders)}\\
		& \bullet\ F^{-1} \in C^\infty
	\end{aligned}
\)

% Inverse Function Theorem
\vspace{15pt}\noindent
\(
	\begin{gathered}
		\text{\underline{Inverse Function}}\\
		\text{\underline{Theorem} (IVT)} 
	\end{gathered}:\ 
	\begin{aligned}
		& \bullet\ F \in C^{\infty} \\
		& \bullet\ \exists dF_p^{-1} \ \ \text{(\scriptsize matrix \(dF_p\) is an isomorphism)} 
	\end{aligned}
	\ \Rightarrow\ \exists F^{-1} \in C^\infty	
\)

% Regular Surfaces
\vspace{15pt}\noindent
\(\begin{aligned}
	& \underline{\text{Regular Surface},\ S \in \mathbb{R}^3}: \\[5pt]
	& - \forall p \in S,\ \underline{\exists F \in C^\infty} 
		,\ F: V_q\ \text{\scriptsize(neighborhood of q)} \rightarrow V_p \cap S
		\hspace{15pt} \text{\scriptsize(diff. parametrizations are possible, btw)}
		\\
	& - \begin{gathered}
			F \ \text{\scriptsize is a homeomorphism}\\[-3pt]
			( \text{\scriptsize or}\ F \ \text{\scriptsize is one-to-one} )
		\end{gathered}
		\ \rightarrow\ \exists F^{-1} \in C^\infty
		\ \Rightarrow\ \text{\scriptsize \(\exists\) no self-intersections; cont. = doesn't depend on parametrization}
		\\
	& - dF_p \ \ \text{\scriptsize is one-to-one = columns are lin. ind. = any 2x2 \(\big| \text{sub-}J_F \big|\) \(\neq 0\)}
		\ \Rightarrow\ \exists(\text{\scriptsize tangent at all points})
\end{aligned}\)

% Info about Reg Surfaces
\vspace{5pt}\noindent
\(\begin{aligned}	
	& \bullet\ \underline{ f \in C^\infty } 
		\ \Rightarrow\ \boxed{ \big( \hsvec{x}, f\mss{(\hsvec{x})} \big) \ \text{\scriptsize is a reg. surf.} }
		\\[10pt]
	& \bullet\ 
		\begin{aligned}
			& f : \mss{ \mathbb{R}^n \rightarrow \mathbb{R} }\\
			& f \mss{ (\hsvec{x}) } = c \\[-7pt]
			& \underline{ \text{\scriptsize is a reg. val.} }
		\end{aligned}
		\ , \ \ \begin{aligned}
			& f \in C^\infty \\
			& F\mss{(\nhs\hsvec{x})} = ( \mss{ x_1,\hs ...\ , x_{n-1} , f(\nhs\hsvec{x}) } )\\
			& \exists dF_p^{-1}
		\end{aligned}
		\ \ \stackrel{\text{(IVT)}}{\Rightarrow}\ \
		\begin{aligned}
			& \exists F^{-1} \in C^\infty \\
			& F^{-1}( \mss{ f_1,\hs ...\ , f_{n-1} , f(\nhs\hsvec{x}) } ) = \hsvec{x}
		\end{aligned}
		\ , \ \ 
		\begin{aligned}
			& x_n\ \mss{ = f^{-1}_n }: \mss{ \mathbb{R}^n \rightarrow \mathbb{R} }\\
			& \underline{ x_n\ \mss{ = f^{-1}_n } \in C^\infty }
		\end{aligned}
		\\[5pt]
	& \hspace{10pt} \rightarrow\ 
		\begin{aligned}
			x_n & = \mss{ {f}^{-1}_n }( \mss{ x_1,\hs ...\ , x_{n-1} , f(\nhs\hsvec{x}) = c } )\\
			& = \underline{ \mss{ {f'}^{-1}_n } ( \mss{ x_1,\hs ...\ , x_{n-1} } ) }
		\end{aligned}
		\ \Rightarrow \ 
		\begin{aligned}
			& S = \underline{ ( \mss{ x_1,\hs ...\ , x_{n-1} , {f'}^{-1}_n } ) }
				\ \text{\scriptsize where}\ \mss{ f(\nhs\hsvec{x}) = c } 
				\\
			& S \ ==\ \text{\scriptsize Surface}\ f^{-1}{(c)}
		\end{aligned}
		\Rightarrow \boxed{ \text{\scriptsize Surface}\ f^{-1}{(c)}\ \text{\scriptsize is reg.} }
		\\[10pt]
	& \bullet\ f: S \subset \mathbb{R}^n \rightarrow \mathbb{R},\ \forall p \in S,\ \underline{ f(p) \neq 0 }
		\ \Rightarrow\ \forall p \in S,\ \underline{ f(p) > 0 \text{ or } f(p) < 0 }
		\\[10pt]
	& \hs\hs \begin{aligned}
			\bullet\ \hs & F\mss{(u,v)} = ( x\mss{(u,v)}, y\mss{(u,v)}, \underline{ z\mss{(u,v)} } ) \\
			& ( \text{\scriptsize\& } F \ \text{\scriptsize is one-to-one} )
		\end{aligned}
		,\ \underline{ \tfrac{\partial(x,y)}{\partial(u,v)} \neq 0 }
		\ \Rightarrow\ \pi_{\text{proj.}} \circ F \mss{(u,v)} \equiv  ( x\mss{(u,v)}, y\mss{(u,v)} )
		\\
	& \hspace{10pt} \stackrel{(IVT)}{\Rightarrow} \ (\pi \circ F)^{-1}\mss{( x, y )}
		= (u\mss{(x,y)}, v\mss{(x,y)})
		\ \ \begin{aligned}
			& \mss{\bullet \nhs \Rightarrow}\ z( u(\mss{x,y}), v(\mss{x,y})) 
				= z \circ (\pi \circ F)^{-1}\mss{(x,y)} = \boxed{ f\mss{(x,y)} = z \ \mss{\in C^\infty} } 
				\\
			& \mss{\text{\&} \nhs \Rightarrow}\ \underline{ (\pi \circ F)^{-1} \circ \pi } \circ F \mss{(u,v)} 
				= \underline{F^{-1}} \circ F \mss{(u,v)}
				\ \Rightarrow\ \boxed{F^{-1} \in C^\infty}
		\end{aligned}
\end{aligned}\)

%--------------------------------------------------------------------------------------------------------------------------------------
%--------------------------------------------------------------------------------------------------------------------------------------
%--------------------------------------------------------------------------------------------------------------------------------------
%--------------------------------------------------------------------------------------------------------------------------------------
\newpage
% Del
\section{Del}

\parbox[t]{.67\textwidth}{
	% Nabla
	\(\begin{aligned}[t]
		\nabla F & = \hs \left( 
				\hat{r} \dfrac{\partial}{\partial r} 
				+ \hat{\theta} \frac{1}{r} \dfrac{\partial}{\partial \theta} 
				+ \hat{\phi} \frac{1}{ r \sin{\theta} } \dfrac{\partial}{\partial \phi} 
			\right)
			F 
			\\[5pt]
		& = 
			\left[ \begin{matrix}
				\hat{r}\\ 
				\hat{\theta}\\
				\hat{\phi} 
			\end{matrix} \right]
			\cdot
			\left[ \mss{ \begin{matrix}
				\\[-12pt]
				\tfrac{\partial}{\partial r}\\[5pt]
				\tfrac{1}{r} \tfrac{\partial}{\partial \theta}\\[5pt]
				\tfrac{1}{r\sin\theta} \tfrac{\partial}{\partial \phi}
			\end{matrix} } \right] 
			F
			=
			\left[ \ \begin{matrix}
				\cos{\phi}\sin{\theta} \hs \hat{x} + \sin{\phi}\sin{\theta} \hs \hat{y} + \cos{\theta} \hs \hat{z} \\[5pt]
				\cos{\phi}\cos{\theta} \hs \hat{x} + \sin{\phi}\cos{\theta} \hs \hat{y} - \sin{\theta} \hs \hat{z} \\[5pt]
				- \sin{\phi} \hs \hat{x} + \cos{\phi} \hs \hat{y}
			\end{matrix} \ \right]
			\cdot
			\left[ \mss{ \begin{matrix}
				\\[-12pt]
				\tfrac{\partial}{\partial r}\\[5pt]
				\tfrac{1}{r} \tfrac{\partial}{\partial \theta}\\[5pt]
				\tfrac{1}{r\sin\theta} \tfrac{\partial}{\partial \phi}
			\end{matrix} } \right] 
			F 
			\\[5pt]
		\left[\begin{matrix}
				\\[-12pt]
				\tfrac{\partial}{\partial x}\\[5pt]
				\tfrac{\partial}{\partial y}\\[5pt]
				\tfrac{\partial}{\partial z}
			\end{matrix}\right] 
			F
			& = \left[
				\mss{ \begin{array}{c}
					\displaystyle
					\cos{\phi}\sin{\theta} \dfrac{\partial}{\partial r} 
					+ \frac{\cos{\phi}\cos{\theta}}{r} \dfrac{\partial}{\partial \theta} 
					- \frac{\sin{\phi}}{ r \sin{\theta} } \dfrac{\partial}{\partial \phi} 
					\\[10pt]
					\displaystyle
					\sin{\phi}\sin{\theta} \dfrac{\partial}{\partial r} 
					+ \frac{\sin{\phi}\cos{\theta}}{r} \dfrac{\partial}{\partial \theta} 
					+ \frac{\cos{\phi}}{ r \sin{\theta} } \dfrac{\partial}{\partial \phi}
					\\[10pt]
					\displaystyle
					\cos{\theta} \dfrac{\partial}{\partial r} 
					- \frac{\sin{\theta}}{ r } \dfrac{\partial}{\partial \theta} 
					\\[10pt]
				\end{array} }
			\right]
			F 		
			\ = 
			\left[\mss{ \begin{array}{c}
				\displaystyle
				\dfrac{\partial r}{\partial x} \dfrac{\partial}{\partial r} 
				+ \dfrac{\partial \theta}{\partial x} \dfrac{\partial}{\partial \theta} 
				+ \dfrac{\partial \phi}{\partial x} \dfrac{\partial}{\partial \phi} 
				\\[10pt]
				\displaystyle
				\dfrac{\partial r}{\partial y} \dfrac{\partial}{\partial r} 
				+ \dfrac{\partial \theta}{\partial y} \dfrac{\partial}{\partial \theta} 
				+ \dfrac{\partial \phi}{\partial y} \dfrac{\partial}{\partial \phi} 
				\\[10pt]
				\displaystyle
				\dfrac{\partial r}{\partial z} \dfrac{\partial}{\partial r} 
				+ \dfrac{\partial \theta}{\partial z} \dfrac{\partial}{\partial \theta} 
				+ \dfrac{\partial \phi}{\partial z} \dfrac{\partial}{\partial \phi} 
				\\[10pt]
			\end{array} }\right] 
			F
			\ =
			\left[\begin{matrix}
				\hat{x}\\
				\hat{y}\\
				\hat{z}
			\end{matrix}\right]
			\cdot
			\left[\begin{matrix}
				\\[-12pt]
				\tfrac{\partial}{\partial x}\\[5pt]
				\tfrac{\partial}{\partial y}\\[5pt]
				\tfrac{\partial}{\partial z}
			\end{matrix}\right] 
			F
	\end{aligned}\)

	\vspace{15pt}
	\(\begin{aligned}[t]
		% Divergence
		\nabla \cdot \vec{A} \ & 
			= \ \frac{1}{r} \frac{1}{r \sin{\theta}} 
			\left\langle 
				\dfrac{\partial}{\partial r} , \ 
				\dfrac{\partial}{\partial \theta} , \ 
				\dfrac{\partial}{\partial \phi} 
			\right\rangle 
			\ \cdot \ [ r \cdot r \sin{\theta} ]
			\left\langle A_r , \ \frac{1}{r} A_\theta , \ \frac{1}{ r \sin{\theta} } A_\phi \right\rangle
			\\[15pt]
		% Curl
		\nabla \times \vec{A} \ & 
			= \ \dfrac{1}{r} \dfrac{1}{ r \sin{\theta} } 
			\begin{Vmatrix}
				\hat{r} 					 & r \hat{\theta} 			    & r \sin{\theta} \hat{\phi}\\[10pt]
				\dfrac{\partial}{\partial r} & \dfrac{\partial}{\partial \theta} & \dfrac{\partial}{\partial \phi}\\[10pt]
				A_r							 & r A_\theta				    & r \sin{\theta} A_\phi
			\end{Vmatrix}
			= 
			\begin{Vmatrix}
				\tfrac{\partial \hsvec{r}}{\partial r}  
					& \tfrac{\partial \hsvec{r}}{\partial \theta} 
					& \tfrac{\partial \hsvec{r}}{\partial \phi} 
					\\[10pt]
				\dfrac{\partial}{\partial r} & \dfrac{\partial}{\partial \theta} & \dfrac{\partial}{\partial \phi}\\[10pt]
				A_r							 & r A_\theta				    & r \sin{\theta} A_\phi
			\end{Vmatrix}
	\end{aligned}\)

	\vspace{20pt}
	% Double Cross Product
	\(\begin{array}{r c l c l}
		\relax[ \vec{A} \times (\vec{B} \times \vec{C}) ]_i 
			&=& \vec{A} \cdot ( B_i \vec{C} ) - \vec{A} \cdot ( \vec{B} C_i ) 
			\\[10pt]
		\vec{A} \times (\vec{B} \times \vec{C}) 
			&=& \boxed{ ( \vec{A} \cdot ( \vec{B} \otimes \vec{C} )^T )^T - ( \vec{A} \cdot \vec{B} \otimes \vec{C} )^T }
			&=& ( \vec{A} \otimes \vec{B} ) \cdot \vec{C} - ( \vec{A} \cdot \vec{B} \otimes \vec{C} )^T 
			\\[5pt]
		&=& ( A^T ( BC^T )^T )^T - ( A^T BC^T )^T
			&=& (AB^T)^T C - ( A^T BC^T )^T
	\end{array}\)
}

\end{document}