\documentclass[12pt]{article}
\usepackage[left=.75in, right=.75in, top=1in, bottom = 1in]{geometry}
\usepackage{amssymb, amsmath, amsfonts, mathtools, bbm}
\usepackage{array,multirow}
\usepackage{cancel}

\newcommand{\hs}{\hspace{1pt}} % 1pt horizontal space
\newcommand{\hsvec}[1]{\vec{\hs #1}} % 1pt space with a \vec
\newcommand{\nhs}{\hspace{-1pt}} % -1pt horizontal space
\newcommand{\mss}[1]{\text{\scriptsize\(#1\)}} % math scriptsize
\newcommand{\tss}[1]{\text{\scriptsize #1}} % text scriptsize

\newcommand{\checkedbox}{\mbox{\ooalign{$\checkmark$\cr\hidewidth$\square$\hidewidth\cr}}} % checked box
\newcommand{\crossbox}{\mbox{\ooalign{\ding{55}\cr\hidewidth$\square$\hidewidth\cr}}} % cross box

\begin{document}
\setlength{\parindent}{0pt}

%---------------------------------------------------------------------------------------------------------------------------------
%---------------------------------------------------------------------------------------------------------------------------------
%---------------------------------------------------------------------------------------------------------------------------------
%---------------------------------------------------------------------------------------------------------------------------------
\newpage

\(
	\mss{ \begin{gathered}
		\boxed{ \hsvec{\nabla} = \left[ \hsvec{\nabla} \left( r, \theta, \phi \right) \right] \bar{\partial}_\circ }
			\\[5pt]
		d = \left[ dx \ dy \ dz \right] \hsvec{\nabla} = d\hsvec{l}\hs\hs^T \hsvec{\nabla} 
			\\[5pt]
		\begin{aligned}
				d ( r, \theta, \phi ) 
					& = \left[ dx \ dy \ dz \right]
					\hsvec{\nabla} 
					( r, \theta, \phi )
					\\
				\Aboxed{ \partial \bar{l}_\circ\nhs^T & = d\hsvec{l}\hs\hs^T \hsvec{\nabla} (r, \theta, \phi) }
			\end{aligned}
			\\[5pt]
		\begin{aligned}
			\partial \bar{l}_\circ\nhs^T \bar{\partial}_\circ & 
				= d\hsvec{l}\hs\hs^T \left[ \hsvec{\nabla} ( r, \theta, \phi ) \right] \bar{\partial}_\circ
				\\
			\Aboxed{ d = \partial \bar{l}_\circ\nhs^T \bar{\partial}_\circ & = d\hsvec{l}\hs\hs^T \hsvec{\nabla} }
		\end{aligned}
	\end{gathered} } 
	\hfill \vline \hfill
	% Matrix Form for cartesian gradient and d(spherical)
	\begin{aligned}
		% Partial / Partial xyz
		\underline{ \nhs \hsvec{\nabla} } &  
			= 
			\left[\begin{matrix}
				\\[-12pt]
				\tfrac{\partial}{\partial x}\\[5pt]
				\tfrac{\partial}{\partial y}\\[5pt]
				\tfrac{\partial}{\partial z}
			\end{matrix}\right] 
			= 
			\left[ \mss{ \arraycolsep=2pt\begin{matrix}
				| & | & | \\
				\nabla r & \nabla \theta & \nabla \phi \\
				| & | & |
			\end{matrix} } \right]
			\left[\begin{matrix}
				\\[-12pt]
				\tfrac{\partial}{\partial r}\\[5pt]
				\tfrac{\partial}{\partial \theta}\\[5pt]
				\tfrac{\partial}{\partial \phi}
			\end{matrix}\right]
			\\[10pt]
		\underline{ \partial \bar{l}_\circ } & = \nhs
			\left[ { \begin{matrix}
				dr\\
				d\theta\\
				d\phi
			\end{matrix} } \right] 
			\begin{aligned}
				& = \hs\hs [\hsvec{\nabla} (r, \theta, \phi)]^T d\hsvec{l}\\[5pt]
				& = \left[ \mss{ \begin{matrix}
						- \nabla r - \\
						- \nabla \theta - \\
						- \nabla \phi - \\
					\end{matrix} } \right]
					\left[ \mss{ \begin{matrix}
						dx\\
						dy\\
						dz
				\end{matrix} } \right]
			\end{aligned}
	\end{aligned}
	\hfill\vline\hfill
	% \theta Example
	\begin{aligned}
		\theta & = \theta \mss{(x,y,z)} 
			\hspace{10pt} ( \mss{ x^2 + y^2 = z^2 \tan^2{\theta} } )
			\\
		\phi & = \phi \mss{(x,y,z)} 
			\hspace{10pt} ( \mss{ y = x \tan{\phi} } )
			\\[10pt]
		\tfrac{\partial}{\partial x} & = \tfrac{\partial r}{\partial x} \tfrac{\partial}{\partial r}
			+ \tfrac{\partial \theta}{\partial x} \tfrac{\partial}{\partial \theta}
			+ \tfrac{\partial \phi}{\partial x} \tfrac{\partial}{\partial \phi} 
			\\[5pt]
		\tfrac{\partial}{\partial \theta} & = \tfrac{\partial x}{\partial \theta} \tfrac{\partial}{\partial x}
			+ \tfrac{\partial y}{\partial \theta} \tfrac{\partial}{\partial y}
			+ \tfrac{\partial z}{\partial \theta} \tfrac{\partial}{\partial z}
			\\[10pt]
		\mss{ d\theta} & = \mss{ dx \tfrac{\partial \theta}{\partial x} 
			+ dy \tfrac{\partial \theta}{\partial y} 
			+ dz \tfrac{\partial \theta}{\partial z} }
			\\
		\mss{dy_{\vec{r}_\circ}(\vec{r}_\circ \hs\nhs\nhs\nhs\nhs')}|_{t=0} & = \mss{ ( \tfrac{\partial y}{\partial r} dr
			+ \tfrac{\partial y}{\partial \theta} d\theta
			+ \tfrac{\partial y}{\partial \phi} d\phi ) \tfrac{1}{dt}|_{t=0} }
	\end{aligned}
\)

\vspace{15pt}
% Contravector Dot Covector
\(
	\boxed{ \hsvec{b}^{\hs i} \cdot \hsvec{b}_{i} = \delta_{ij} }:\ 
	\begin{aligned}
		& \hsvec{\nabla}{\phi} \cdot \tfrac{\partial \hsvec{r}}{\partial \phi} = 1\\[5pt]
		& \hsvec{\nabla}{\phi} \cdot \tfrac{\partial \hsvec{r}}{\partial \theta} = 0
	\end{aligned}
	\ \Rightarrow\ 
	\left[ \mss{ \begin{matrix}
		- \hsvec{\nabla} r - \\
		- \hsvec{\nabla} \theta - \\
		- \hsvec{\nabla} \phi - 
	\end{matrix} } \right]
	\left[ 
		\tfrac{\partial}{\partial r} \ \tfrac{\partial}{\partial \theta} \ \tfrac{\partial}{\partial \phi}
	\right]
	\hsvec{r}
	= \mathbbm{1}_3
	\ \ \Rightarrow\ \
	\boxed{
		\tfrac{\partial y}{\partial \phi} = 
		\left[ \mss{ 
			0 \ 1 \ 0
		} \right]
		\left[ \mss{ \begin{matrix}
			- \hsvec{\nabla} r - \\
			- \hsvec{\nabla} \theta - \\
			- \hsvec{\nabla} \phi - 
		\end{matrix} } \right]^{-1}
		\left[ \mss{ \begin{matrix}
			0 \\ 0 \\ 1
		\end{matrix} } \right]
		= \tfrac{\partial y}{\partial \phi}^T
	}
\)

\vspace{15pt}
% d operator
\(
	\begin{aligned}
		d & = dx \tfrac{\partial }{\partial x}
			+ dy \tfrac{\partial }{\partial y} 
			+ dz \tfrac{\partial }{\partial z} 
			= d\hsvec{l} \cdot \hsvec{\nabla} 
			&& = \left[ 
				\tfrac{dr}{\lVert \nabla r \rVert} , 
				\tfrac{d\theta}{\lVert \nabla \theta \rVert} ,
				\tfrac{d\phi}{\lVert \nabla \phi \rVert} 
			\right] 
			\left[ 
				\mss{ \lVert \nabla r \rVert } \tfrac{\partial }{\partial r} ,\hs
				\mss{ \lVert \nabla \theta \rVert } \tfrac{\partial }{\partial \theta} ,\hs
				\mss{ \lVert \nabla \phi \rVert } \tfrac{\partial }{\partial \phi} 
			\right]^T
			\\
		& = dr \tfrac{\partial }{\partial r} 
			+ d\theta \tfrac{\partial }{\partial \theta} 
			+ d\phi \tfrac{\partial }{\partial \phi} 
			= \partial \bar{l}_\circ\nhs^T \bar{\partial}_\circ 
			&& = \left[ dr, rd\theta, r\sin\theta d\phi \right]
			\left[ 
				\tfrac{\partial }{\partial r} 
				, \tfrac{1}{r} \tfrac{\partial }{\partial \theta} 
				,\tfrac{1}{r\sin\theta} \tfrac{\partial }{\partial \phi} 
			\right]^T
			\\[3pt]
		& = \tfrac{dr}{\lVert \nabla r \rVert} \mss{ \lVert \nabla r \rVert } \tfrac{\partial }{\partial r} 
			+ \tfrac{d\theta}{\lVert \nabla \theta \rVert} \mss{ \lVert \nabla \theta \rVert } \tfrac{\partial }{\partial \theta} 
			+ \tfrac{d\phi}{\lVert \nabla \phi \rVert} \mss{ \lVert \nabla \phi \rVert } \tfrac{\partial }{\partial \phi}
			&& \dots = d\bar{l}_\circ\nhs^T \bar{\nabla}_\circ = d\hsvec{l}_\circ\nhs^T \hsvec{\nabla}_\circ 
	\end{aligned}
\)

\vspace{20pt}
% Work to Find Line Element and Unit Spherical Vectors
\(
	% Line Element
	\begin{aligned}
		d\hsvec{l} = d\hsvec{r} & 
			= [ dr \tfrac{\partial}{\partial r} 
			+ d\theta \tfrac{\partial}{\partial \theta} 
			+ d\phi \tfrac{\partial}{\partial \phi} ]
			(x,y,z)^T
			\\
		d{(x, y, z)} & 
			= \left[ \mss{ \tfrac{dr}{\lVert \nabla r \rVert} { \lVert \nabla r \rVert } \tfrac{\partial }{\partial r} 
			+ \tfrac{d\theta}{\lVert \nabla \theta \rVert} { \lVert \nabla \theta \rVert } \tfrac{\partial }{\partial \theta} 
			+ \tfrac{d\phi}{\lVert \nabla \phi \rVert} { \lVert \nabla \phi \rVert } \tfrac{\partial }{\partial \phi} 
			} \right] (x,y,z)
			\\[5pt]
		(dx,dy,dz) &
			= dr \hs \hat{r}^T + r \hs d\theta \hs \hat{\theta}^T + r \sin\theta \hs d\phi \hs \hat{\phi}^T
	\end{aligned}
	\hfill\vline\hfill
	% Unit Spherical Vectors
	\begin{aligned}
		\big( \hat{r} , & \hat{\theta} , \hat{\phi} \big) 
			\equiv \underline{ \left( 
				\mss{ \lVert \nabla r \rVert } \tfrac{\partial \hsvec{r}}{\partial r} ,\hs
				\mss{ \lVert \nabla \theta \rVert } \tfrac{\partial \hsvec{r}}{\partial \theta} ,\hs
				\mss{ \lVert \nabla \phi \rVert } \tfrac{\partial \hsvec{r}}{\partial \phi} 
			\right) }
			\\
		& = \left(
				\tfrac{\partial}{\partial r} ,\
				\tfrac{1}{r} \tfrac{\partial}{\partial \theta} ,\
				\tfrac{1}{r \sin{\theta}} \tfrac{\partial}{\partial \phi} 
			\right)
			\otimes (x,y,z)^T
			\\
		& = \bar{\nabla}_\circ^T \otimes \hsvec{r}
	\end{aligned}
\)

\vspace{20pt}
% Line Element and Gradient
\(\begin{aligned}
	% Line Element
	& \boxed{ d\hsvec{l} = (dx, dy, dz) \cdot \big( \hat{x} , \hat{y} , \hat{z} \big) }
		= \boxed{ (dr, rd\theta, r\sin\theta d\theta) \cdot \big( \hat{r} , \hat{\theta} , \hat{\phi} \big) = d\hsvec{l}_\circ }
		= d\bar{l}_\circ\nhs^T \cdot \big( \hat{r} , \hat{\theta} , \hat{\phi} \big)
		\\
	% Gradient
	& \boxed{ 
			\hsvec{\nabla} = (\hat{x}, \hat{y}, \hat{z}) 
			\cdot \left( \tfrac{\partial}{\partial x}, \tfrac{\partial}{\partial y}, \tfrac{\partial}{\partial z} \right) 
		}
		= \boxed{
			(\hat{r}, \hat{\theta}, \hat{\phi}) 
			\cdot \left( 
				\tfrac{\partial}{\partial r}, 
				\tfrac{1}{r} \tfrac{\partial}{\partial \theta}, 
				\tfrac{1}{r\sin\theta} \tfrac{\partial}{\partial \phi}
			\right) 
			= \hsvec{\nabla}_\circ
		}
		= \left(
			\tfrac{\partial \hsvec{r}}{\partial r} ,
			\tfrac{1}{r} \tfrac{\partial \hsvec{r}}{\partial \theta} ,
			\tfrac{1}{r \sin{\theta}} \tfrac{\partial \hsvec{r}}{\partial \phi} 
		\right) \bar{\nabla}_\circ^T
\end{aligned}\)

\vspace{10pt}
% Two ways to write cartesian partial derivatives 
\(\begin{aligned}
	\hsvec{\nabla} & = [ \bar{\nabla}_\circ^T \otimes \hsvec{r} \hs ]  \bar{\nabla}_\circ
		= \left[ \bar{\nabla}_\circ^T \otimes (x,y,z)^T \right]
		\left[ \mss{ \begin{matrix}
			\tfrac{\partial}{\partial r}\\[4pt]
			\tfrac{1}{r} \tfrac{\partial}{\partial \theta}\\[4pt]
			\tfrac{1}{r\sin\theta} \tfrac{\partial}{\partial \phi} 
		\end{matrix} } \right]
		&& \Rightarrow\ 
		\tfrac{\partial}{\partial x} 
		= \underline{ \tfrac{\partial x}{\partial r} } \tfrac{\partial}{\partial r}
		+ \underline{ \mss{ \Vert \nabla \theta \Vert }^2 \tfrac{\partial x}{\partial \theta} } \tfrac{\partial}{\partial \theta}
		+ \underline{ \mss{ \Vert \nabla \phi \Vert }^2 \tfrac{\partial x}{\partial \phi} } \tfrac{\partial}{\partial \phi}
		\\
	& = [ \hsvec{\nabla}(r,\theta,\phi) ] \bar{\partial}_\circ
		= [ \hsvec{\nabla}(r,\theta,\phi) ] 
		\left[ \mss{ \begin{matrix}
			\tfrac{\partial}{\partial r}\\[4pt]
			\tfrac{\partial}{\partial \theta}\\[4pt]
			\tfrac{\partial}{\partial \phi} 
		\end{matrix} } \right]
		&&\Rightarrow\
		\tfrac{\partial}{\partial x} 
		= \underline{ \tfrac{\partial r}{\partial x} } \tfrac{\partial}{\partial r}
		+ \underline{ \tfrac{\partial \theta}{\partial x} } \tfrac{\partial}{\partial \theta}
		+ \underline{ \tfrac{\partial \phi}{\partial x} } \tfrac{\partial}{\partial \phi}
		\ \Rightarrow\ 
		\boxed{ \tfrac{\partial \phi}{\partial y} = \tfrac{\partial y}{\partial \phi} \mss{\Vert \nabla \phi \Vert}^2 }
\end{aligned}\)

\vspace{10pt}
% Unit Vectors Summary
\(\boxed{
	\arraycolsep=3pt
	\begin{array}{r c c c r c r c c c c c c c r}
		& & & & & & & 
			& \text{\scriptsize\underline{contravariant}}_{\hs i} 
			& \multicolumn{3}{c}{ \text{\scriptsize(equal since orthog.)} } 
			& \multicolumn{3}{l}{ \text{\scriptsize\underline{covariant}}^i } 
			\\[6pt]
		\hat{r} & = 
			& (\hat{r}_x, \hat{r}_y, \hat{r}_z) 
			& = 
			& { \tfrac{\hsvec{r}}{r} }
			& = 
			& \tfrac{\partial}{\partial r} \hsvec{r} 
			& = 
			& \tfrac{\partial \hsvec{r}}{\partial r} \Vert \tfrac{\partial \hsvec{r}}{\partial r}  \Vert^{-1} 
			& \ \ \stackrel{\rightarrow}{=}\ \
			& \mss{ \Vert \nabla r \Vert } \tfrac{\partial \hsvec{r}}{\partial r} 
			& \ \stackrel{\leftarrow}{=}\ \ 
			& \tfrac{\nabla r}{\Vert \nabla r \Vert}
			& = 
			& \nabla r
			\\[5pt]
		\hat{\theta} & = 
			& (\hat{\theta}_x, \hat{\theta}_y, \hat{\theta}_z) 
			& = 
			& { \tfrac{\partial \hat{r}}{\partial \theta} }
			& = 
			& \tfrac{1}{r} \tfrac{\partial}{\partial \theta} \hsvec{r}
			& =
			& \tfrac{\partial \hsvec{r}}{\partial\theta} \Vert \tfrac{\partial \hsvec{r}}{\partial\theta} \Vert^{-1} 
			& \ \ \stackrel{\rightarrow}{=}\ \
			& \mss{ \Vert \nabla \theta \Vert } \tfrac{\partial \hsvec{r}}{\partial\theta} 
			& \ \stackrel{\leftarrow}{=}\ \
			& \tfrac{\nabla \theta}{\Vert \nabla \theta \Vert} 
			& = 
			& r \nabla \theta
			\\[5pt]
		\hat{\phi} & = 
			& (\hat{\phi}_x, \hat{\phi}_y, \hat{\phi}_z) 
			& = 
			& { \tfrac{1}{\sin\theta} \tfrac{\partial \hat{r}}{\partial \phi} }
			& = 
			& \tfrac{1}{r \sin\theta} \tfrac{\partial}{\partial \phi} \hsvec{r}
			& =
			& \tfrac{\partial \hsvec{r}}{\partial\phi} \Vert \tfrac{\partial \hsvec{r}}{\partial\phi} \Vert^{-1} 
			& \ \ \stackrel{\rightarrow}{=} \ \
			& \mss{ \Vert \nabla \phi \Vert } \tfrac{\partial \hsvec{r}}{\partial\phi} 
			& \ \stackrel{\leftarrow}{=} \ \
			& \tfrac{\nabla \phi}{\Vert \nabla \phi \Vert}
			& = 
			& \mss{r \sin\theta} \nabla \phi
	\end{array}
}\)

%--------------------------------------------------------------------------------------------------------------------------------------
%--------------------------------------------------------------------------------------------------------------------------------------
%--------------------------------------------------------------------------------------------------------------------------------------
%--------------------------------------------------------------------------------------------------------------------------------------
\newpage
% Del
\section{Del}

\parbox[t]{.67\textwidth}{
	% Nabla
	\(\begin{aligned}[t]
		\nabla F & = \hs \left( 
				\hat{r} \dfrac{\partial}{\partial r} 
				+ \hat{\theta} \frac{1}{r} \dfrac{\partial}{\partial \theta} 
				+ \hat{\phi} \frac{1}{ r \sin{\theta} } \dfrac{\partial}{\partial \phi} 
			\right)
			F 
			\\[5pt]
		& = 
			\left[ \begin{matrix}
				\hat{r}\\ 
				\hat{\theta}\\
				\hat{\phi} 
			\end{matrix} \right]
			\cdot
			\left[ \mss{ \begin{matrix}
				\\[-12pt]
				\tfrac{\partial}{\partial r}\\[5pt]
				\tfrac{1}{r} \tfrac{\partial}{\partial \theta}\\[5pt]
				\tfrac{1}{r\sin\theta} \tfrac{\partial}{\partial \phi}
			\end{matrix} } \right] 
			F
			=
			\left[ \ \begin{matrix}
				\cos{\phi}\sin{\theta} \hs \hat{x} + \sin{\phi}\sin{\theta} \hs \hat{y} + \cos{\theta} \hs \hat{z} \\[5pt]
				\cos{\phi}\cos{\theta} \hs \hat{x} + \sin{\phi}\cos{\theta} \hs \hat{y} - \sin{\theta} \hs \hat{z} \\[5pt]
				- \sin{\phi} \hs \hat{x} + \cos{\phi} \hs \hat{y}
			\end{matrix} \ \right]
			\cdot
			\left[ \mss{ \begin{matrix}
				\\[-12pt]
				\tfrac{\partial}{\partial r}\\[5pt]
				\tfrac{1}{r} \tfrac{\partial}{\partial \theta}\\[5pt]
				\tfrac{1}{r\sin\theta} \tfrac{\partial}{\partial \phi}
			\end{matrix} } \right] 
			F 
			\\[5pt]
		\left[\begin{matrix}
				\\[-12pt]
				\tfrac{\partial}{\partial x}\\[5pt]
				\tfrac{\partial}{\partial y}\\[5pt]
				\tfrac{\partial}{\partial z}
			\end{matrix}\right] 
			F
			& = \left[
				\mss{ \begin{array}{c}
					\displaystyle
					\cos{\phi}\sin{\theta} \dfrac{\partial}{\partial r} 
					+ \frac{\cos{\phi}\cos{\theta}}{r} \dfrac{\partial}{\partial \theta} 
					- \frac{\sin{\phi}}{ r \sin{\theta} } \dfrac{\partial}{\partial \phi} 
					\\[10pt]
					\displaystyle
					\sin{\phi}\sin{\theta} \dfrac{\partial}{\partial r} 
					+ \frac{\sin{\phi}\cos{\theta}}{r} \dfrac{\partial}{\partial \theta} 
					+ \frac{\cos{\phi}}{ r \sin{\theta} } \dfrac{\partial}{\partial \phi}
					\\[10pt]
					\displaystyle
					\cos{\theta} \dfrac{\partial}{\partial r} 
					- \frac{\sin{\theta}}{ r } \dfrac{\partial}{\partial \theta} 
					\\[10pt]
				\end{array} }
			\right]
			F 		
			\ = 
			\left[\mss{ \begin{array}{c}
				\displaystyle
				\dfrac{\partial r}{\partial x} \dfrac{\partial}{\partial r} 
				+ \dfrac{\partial \theta}{\partial x} \dfrac{\partial}{\partial \theta} 
				+ \dfrac{\partial \phi}{\partial x} \dfrac{\partial}{\partial \phi} 
				\\[10pt]
				\displaystyle
				\dfrac{\partial r}{\partial y} \dfrac{\partial}{\partial r} 
				+ \dfrac{\partial \theta}{\partial y} \dfrac{\partial}{\partial \theta} 
				+ \dfrac{\partial \phi}{\partial y} \dfrac{\partial}{\partial \phi} 
				\\[10pt]
				\displaystyle
				\dfrac{\partial r}{\partial z} \dfrac{\partial}{\partial r} 
				+ \dfrac{\partial \theta}{\partial z} \dfrac{\partial}{\partial \theta} 
				+ \dfrac{\partial \phi}{\partial z} \dfrac{\partial}{\partial \phi} 
				\\[10pt]
			\end{array} }\right] 
			F
			\ =
			\left[\begin{matrix}
				\hat{x}\\
				\hat{y}\\
				\hat{z}
			\end{matrix}\right]
			\cdot
			\left[\begin{matrix}
				\\[-12pt]
				\tfrac{\partial}{\partial x}\\[5pt]
				\tfrac{\partial}{\partial y}\\[5pt]
				\tfrac{\partial}{\partial z}
			\end{matrix}\right] 
			F
	\end{aligned}\)

	\vspace{15pt}
	\(\begin{aligned}[t]
		% Divergence
		\nabla \cdot \vec{A} \ & 
			= \ \frac{1}{r} \frac{1}{r \sin{\theta}} 
			\left\langle 
				\dfrac{\partial}{\partial r} , \ 
				\dfrac{\partial}{\partial \theta} , \ 
				\dfrac{\partial}{\partial \phi} 
			\right\rangle 
			\ \cdot \ [ r \cdot r \sin{\theta} ]
			\left\langle A_r , \ \frac{1}{r} A_\theta , \ \frac{1}{ r \sin{\theta} } A_\phi \right\rangle
			\\[15pt]
		% Curl
		\nabla \times \vec{A} \ & 
			= \ \dfrac{1}{r} \dfrac{1}{ r \sin{\theta} } 
			\begin{Vmatrix}
				\hat{r} 					 & r \hat{\theta} 			    & r \sin{\theta} \hat{\phi}\\[10pt]
				\dfrac{\partial}{\partial r} & \dfrac{\partial}{\partial \theta} & \dfrac{\partial}{\partial \phi}\\[10pt]
				A_r							 & r A_\theta				    & r \sin{\theta} A_\phi
			\end{Vmatrix}
			= 
			\begin{Vmatrix}
				\tfrac{\partial \hsvec{r}}{\partial r}  
					& \tfrac{\partial \hsvec{r}}{\partial \theta} 
					& \tfrac{\partial \hsvec{r}}{\partial \phi} 
					\\[10pt]
				\dfrac{\partial}{\partial r} & \dfrac{\partial}{\partial \theta} & \dfrac{\partial}{\partial \phi}\\[10pt]
				A_r							 & r A_\theta				    & r \sin{\theta} A_\phi
			\end{Vmatrix}
	\end{aligned}\)

	\vspace{20pt}
	% Double Cross Product
	\(\begin{array}{r c l c l}
		% A x (B x C)
		[ \vec{A} \times (\vec{B} \times \vec{C}) ]_i & =
			& \vec{A} \cdot B_i \vec{C} - \vec{A} \cdot \vec{B} C_i 
			\\[5pt]
		[ \vec{A} \times (\vec{B} \times \vec{C}) ]^T & =
			& \vec{A}_r * \vec{B}_r \vec{C}_c - \vec{A}_r * \vec{B}_c \vec{C}_r
			% & =
			% & \boxed{ \vec{A} \cdot ( \vec{B} \otimes \vec{C} )^T - \vec{A} \cdot \vec{B} \otimes \vec{C} }
			\\[10pt]
		\vec{A} \times (\vec{B} \times \vec{C}) & = 
			& B (A \cdot C) - (A \cdot B) C
			\\[5pt]
		\vec{\nabla} \times (\vec{\nabla} \times \vec{C}) & = 
			& \nabla (\nabla \cdot C) - (\nabla \cdot \nabla) C
			\\[5pt]
		[ \vec{A} \times (\vec{\nabla} \times \vec{C}) ]^T & =
			& \vec{A}_r ( \vec{\nabla}_r \vec{C}_c ) - ( \vec{A} \cdot \vec{\nabla} ) \vec{C}^T 
			\\[5pt]
		[ \vec{\nabla} \times (\vec{B} \times \vec{C}) ]_i & =
			& \vec{\nabla} \cdot B_i \vec{C} - \vec{\nabla} \cdot \vec{B} C_i 
			& =
			& \vec{C} \cdot \vec{\nabla} B_i + \vec{\nabla} \cdot \vec{C} B_i 
				- \vec{\nabla} \cdot \vec{B} C_i - \vec{B} \cdot \vec{\nabla} C_i 
			\\[5pt]
		[ \vec{\nabla} \times (\vec{B} \times \vec{C}) ]^T & =
			& \vec{\nabla}_r * (\vec{B}_r \vec{C}_c) - \vec{\nabla}_r * (\vec{B}_c \vec{C}_r)
			& =
			&  \mss{ \vec{C}_r * \vec{\nabla}_c \vec{B}_r + \vec{\nabla}_r * \vec{C}_c \vec{B}_r 
				- \vec{\nabla}_r * \vec{B}_c \vec{C}_r - \vec{B}_r * \vec{\nabla}_c \vec{C}_r }
			\\[20pt]
		% (A x B) x C
		[ (\vec{A} \times \vec{B}) \times \vec{C} \hs\hs ]_i & =
			& \vec{A} B_i \cdot \vec{C} - A_i \vec{B} \cdot \vec{C}
			\\[5pt]
		(\vec{A} \times \vec{B}) \times \vec{C} \hs\hs & =
			& \vec{A}_r \vec{B}_c * \vec{C}_c - \vec{A}_c \vec{B}_r * \vec{C}_c
			\\[5pt]
		(\vec{A} \times \vec{B}) \times \vec{C} & = 
			& (A \cdot C) B - A (B \cdot C)
			\\[5pt]
		(\vec{A} \times \vec{\nabla}) \times \vec{C} & =
			& ( \vec{A}_r \hsvec{\nabla}_c ) \vec{C}_c - \vec{A}_c ( \vec{\nabla}_r \cdot \vec{C}_c )
			\\[5pt]
		(\vec{\nabla} \times \vec{B}) \times \vec{C} & =
			& ( \vec{\nabla}_r \hsvec{B}_c ) \vec{C} - ( \vec{\nabla}_c \vec{B}_r ) \vec{C} 
	\end{array}\)
}

%--------------------------------------------------------------------------------------------------------------------------------------
%--------------------------------------------------------------------------------------------------------------------------------------
%--------------------------------------------------------------------------------------------------------------------------------------
%--------------------------------------------------------------------------------------------------------------------------------------
\newpage
% Frenet
\section{Frenet Equations}

% Frenet in derivatives of position
\(
    \begin{aligned}
		\mss{ a \cdot (b \times c) } & = \mss{ (a \times b) \cdot c) } \\[5pt]
		\mss{ a \times (b \times c) } & = \mss{ (c \cdot a) b - (b \cdot a) c } \\
		\mss{ (a \times b) \times c } & = \mss{ b (c \cdot a) - a (c \cdot b)  } \\[5pt]
		\mss{ (a \times b) \cdot (c \times d) } & = \mss{ a \cdot b \times (c \times d) }
			\\
		= \mss{ \left| \hs
				\left[ \begin{matrix}
					a \hs \cdot \\ b \hs \cdot 
				\end{matrix} \right]
				\left[ c \ d \right] \hs
			\hs \right| }
			& = \mss{ \left| \ \begin{matrix}
                \\[-11pt]
                \hsvec{a} \cdot \hsvec{c} & \hsvec{a} \cdot \hsvec{d} \\[5pt]
                \hsvec{b} \cdot \hsvec{c} & \hsvec{b} \cdot \hsvec{d}
                \\[-12pt]
                \hs
            \end{matrix} \ \right| }
            \\[5pt]
        \Aboxed{ \tfrac{dt}{ds} & = \tfrac{1}{v} }
    \end{aligned}
    \hfill
    \vline
    \hfill
    \begin{aligned}
        T & = \hat{v} = \tfrac{\hsvec{v}}{v} \\
        \tfrac{dT}{dt} & = \tfrac{ (\hsvec{v} \cdot \hsvec{v}) \hsvec{a} - (\hsvec{v} \cdot \hsvec{a}) \hsvec{v} }{v^3} 
            = \tfrac{ \hsvec{v} \times (\hsvec{a} \times \hsvec{v}) }{v^3}
            = \tfrac{ (\hsvec{v} \times \hsvec{a}) \times \hsvec{v} }{v^3}
            \\
        \lVert \tfrac{dT}{dt} \rVert & = \tfrac{ \sqrt{ v^2 a^2 - (\hsvec{v} \cdot \hsvec{a})^2 } }{v^2}
            = \tfrac{\lVert \hsvec{a} \times \hsvec{v} \rVert }{v^2}
            \hspace{5pt} , \hspace{10pt} \tfrac{dT}{ds} = k \hat{N}
            \\
        \hat{N} & = \tfrac{T'}{\lVert T' \rVert} 
            = \tfrac{ (\nhs\hsvec{v} \times \hsvec{a}) \times \hsvec{v} }{\lVert \hsvec{v} \times \hsvec{a} \rVert v} 
            = \hat{B} \times \hat{v}
            \\
        \hat{B} & = \tfrac{\hsvec{v} \times \hsvec{a}}{\lVert \hsvec{v} \times \hsvec{a} \rVert}
            = \widehat{v \times a}
            = \hat{v} \times \hat{N}
            \hspace{15pt} \underline{ \mss{(\hat{B} \cdot \vec{v} = 0)} }
            \\
        \tfrac{d\hat{B}}{dt} & = \tfrac{\hsvec{v} \times \dot{\hsvec{a}}}{\Vert \hsvec{v} \times \hsvec{a} \Vert }
            - \left[ \tfrac{\hsvec{v} \times \dot{\hsvec{a}}}{\Vert \hsvec{v} \times \hsvec{a} \Vert } \cdot \hat{B} \right] \hat{B}
            \hspace{5pt} , \hspace{5pt} \tfrac{dB}{ds} = \tau \hat{N}
			\\
		\tau & = \hat{N} \cdot \tfrac{d\hat{B}}{ds} 
			= \tfrac{ \hat{B} \cdot \dot{\hsvec{a}} }{ \Vert \nhs\hsvec{v} \times \hsvec{a} \Vert }
			= \tfrac{ (\nhs\hsvec{v} \times \hsvec{a}) \cdot \dot{\hsvec{a}} }{ \Vert \nhs\hsvec{v} \times \hsvec{a} \Vert^2 }
    \end{aligned}
    \hfill
    \boxed{
        \begin{aligned}
            \hsvec{a} & = a_T \hat{T} + a_N \hat{N} \\[5pt]
            a_T & = \hsvec{a} \cdot \hat{v} = \tfrac{dv}{dt} \\[5pt]
            a_N & = \tfrac{\Vert \hsvec{a} \times \hsvec{v} \Vert}{v} = \Vert \hsvec{a} \times \hat{v} \Vert \\[5pt]
            a^2 & = a_T^2 + a_N^2 = \Vert \tfrac{d\hsvec{v}}{dt} \Vert^2
        \end{aligned}
    }
\)

\vspace{15pt}\noindent
% Frenet trihedron - when curve is parametrized by arc length
\underline{Frenet Trihedron for Regular \textit{Parametrized} Curves}\\
\(
    \begin{aligned}
        & \text{\scriptsize Differentiable (in this book)}:\ C^\infty\\
        & \text{\scriptsize No singular pts. Order 0 (Regular)}:\ \hsvec{v}(t) \neq 0\\
        & \mss{ \bullet\ \Vert \hsvec{v}(t) \Vert = c \rightarrow 1
            \ \Rightarrow\ 
            \int_s \Vert \hsvec{v}(t) \Vert \hs dt = t = \Delta s 
            }
            \\
        & \hspace{10pt} \mss{ \rightarrow s:\ \hsvec{x}(t) = \hsvec{x}(s) } \\
        & \mss{ \bullet\ \tfrac{1}{2}\tfrac{d}{dt}(\hsvec{v} \cdot \hsvec{v}) = \boxed{ \hsvec{v} \cdot \hsvec{a} = 0 } } \\
        & \text{\scriptsize No singular pts. Order 1}:\ \hsvec{a}(t) \neq 0\\
        & \mss{ \bullet\ \text{Curvature, } k \neq 0 \text{ (see right)} }
			\hspace{10pt} \mss{ \bullet\ \text{Vertex, } k' = 0 }
    \end{aligned}
    \hfill\vline\hfill
    \begin{aligned}
        & 1 = \Vert \hsvec{t} \Vert = \Vert \hsvec{n} \Vert = \Vert \hsvec{b} \Vert
            \ ,\ \ 
            0 = \hsvec{t} \cdot \hsvec{n} = \hsvec{n} \cdot \hsvec{b} = \hsvec{b} \cdot \hsvec{t}
            \\
        & \bullet\ \hsvec{v}(s) = \hsvec{t}(s) \hspace{20pt} \boxed{ \mss{ (t = n \times b) } } \\
        & \bullet\ \hsvec{a}(s) = \boxed{ \hsvec{t'}(s) = k\mss{(s)} \hsvec{n}\mss{(s)} }
            ,\ k(s) \geq 0
            \hspace{10pt} \begin{gathered}
                \text{\scriptsize(can be L or R-handed)}\\[-8pt]
                \text{\scriptsize(can be neg. if in \(\mathbb{R}^2\))}
            \end{gathered}
            \\
        & \ast\ k(s) > 0 \text{ for well defined curve with \(\hat{n}\)}\\
        & \bullet\ \boxed{ \hsvec{b} = \hsvec{t} \times \hsvec{n} }
            ,\ \ \tfrac{d}{dt} (\hsvec{b} \cdot \hsvec{b}) = \hsvec{b} \cdot \hsvec{b'} = 0
            ,\ \ \ast\ \boxed{ \hsvec{b'}(s) = \tau\mss{(s)} \hsvec{n}\mss{(s)} }
            \\
        & \bullet\ \boxed{ \hsvec{n} = \hsvec{b} \times \hsvec{t} } 
            ,\ \ \ast\ \boxed{ \hsvec{n'}(s) = -k \hsvec{t} - \tau \hsvec{b} }
            ,\ \ \ast\ \mss{ \text{t-n pl.} = \text{osculating pl.} }
    \end{aligned}
\)

% More info on derivatives of frenet
\vspace{10pt} \noindent
\(\begin{aligned}
    & \bullet\ t''(s) = k' n - k^2 t - k\tau b 
        \hspace{20pt} \bullet\ b''(s) = \tau' n - \tau kt - \tau^2 b
        \hspace{20pt} \bullet\ n''(s) = -k't -\tau'b - (k^2 + \tau^2) n
        \\
    & \bullet\ |\tau| = \Vert b' \Vert 
        \hspace{20pt}
        \bullet\ \tau = -\tfrac{ ( t \times t' ) \cdot t'' }{k^2} = -\tfrac{ {t} \cdot ( {t'} \times t'' )}{\Vert t' \Vert^2} 
        \hspace{20pt} 
        \bullet\ k = \Vert t' \Vert = \tfrac{ ( b \times b' ) \cdot b'' }{\tau^2} 
        = \tfrac{ {b} \cdot ( {b'} \times b'' )}{\Vert b' \Vert^2}
        \\
    & \bullet\ n \Rightarrow k,\tau:
        \hspace{20pt}
        \ast\ \Vert n' \Vert^2 = k^2 + \tau^2
        \hspace{20pt}
        \ast\ \tfrac{ ( n \times n' ) \cdot n'' }{\Vert n' \Vert^2} = \tfrac{k' \tau - k \tau'}{k^2 + \tau^2}
        = \tfrac{ \tfrac{d}{ds} (k / \tau) }{(k/\tau)^2 + 1} = \tfrac{d}{ds} \arctan(k/\tau)
\end{aligned}\)

\vspace{15pt}\noindent
\(\begin{aligned}[t]
	% Indicatrix
    & \underline{ \text{Indicatrix of Tangents},\ \hsvec{t}(\theta\mss{(s)}) }:\\[5pt]
    & \bullet\ \hsvec{t}(\theta\mss{(s)}) = (\cos\theta, \sin\theta) = (x'\mss{(s)}, y'\mss{(s)})\\
    & \bullet\ \hsvec{t'}(\theta) = \underline{ \theta'(s)} (-\sin\theta, \cos\theta) = \underline{k(s)} \hsvec{n}\\
	& \bullet\ \theta\mss{(s)} = \arctan(y'/x')\\
	& \bullet\ \mss{\int_0^l} \hs k\mss{(s)} \hs\hs ds = \theta(s)\Big|^l_0 = 2\pi I_\text{rot. index}\\
	& \bullet\ k\mss{(s)} = \lim_{s \rightarrow 0} \tfrac{r\theta(s)}{s} \big|_{r=1}
		\hspace{15pt} \text{\scriptsize(See Gaussian \(K\))}
\end{aligned}\)
\hspace{20pt}
\(\begin{aligned}[t]
	% Local Canonical Form
    & \text{\underbar{Local Canonical Form at \(t=0\)}}:\\[5pt]
    & \bullet\ (\hat{t},\hat{n},\hat{b}) = (\hat{x},\hat{y},\hat{z})\\
    & \bullet\ \hsvec{r}(s) - \hsvec{r}(0) 
        \approx ( s - \tfrac{k^2 s^3}{6},\ \tfrac{k}{2} s^2 + \tfrac{k' s^3}{6},\ \tfrac{-k\tau}{6} s^3 )
        \\
    & \bullet\ \tau < 0 \ \Rightarrow\ \tfrac{dz}{ds} > 0
		\\[5pt]
	% Isoperimetric Ineq.
	& \text{\underline{Isoperimetric Inequality}}:\ 0 \leq l^2 - 4\pi A
		\\[5pt]
	% 4-Vertex Theor.
	& \text{\underline{Four-Vertex Theorem}}:\ \text{\scriptsize A simple closed curve has \(\geq\) 4 vertices}
\end{aligned}\)

% Cauchy-Crofton Formula
\vspace{15pt}\noindent
\(\begin{aligned}[t]
	& \text{\underline{Cauchy-Crofton Formula (measure of number of times lines intersect a curve)}}:\\[5pt]
	& \bullet\ \text{Tangent line at } (\rho, \theta) :\ x\cos\theta + y\sin\theta = \rho
		\hspace{15pt} \bullet\ \text{Curve } c:\ y=0,\ x \in (-l/2, l/2)
		\hspace{5pt} , \hspace{10pt} C = \mss{\sum}\hs c_i
		\\
	& \bullet\ \mss{\int}\ \text{\scriptsize Lines that cross } c 
		= \mss{ \int_0^{2\pi} \int_0^{|\cos\theta| l/2} } d\rho \hs d\theta = 2l
		\ \Rightarrow\
		\mss{ \int_0^{2\pi} \int_0^\infty } n_C \ d\rho \hs d\theta = 2l
\end{aligned}\)

%--------------------------------------------------------------------------------------------------------------------------------------
%--------------------------------------------------------------------------------------------------------------------------------------
%--------------------------------------------------------------------------------------------------------------------------------------
%--------------------------------------------------------------------------------------------------------------------------------------
\newpage
% Differential
\section{Jacobian/Differential, \( dF_{\alpha(0)} : \underline{ \mathbb{R}^n \rightarrow \mathbb{R}^m } \) }

\(\begin{aligned}
	& \bullet\ \boxed{ \alpha(0) = \beta(0) } \ \Rightarrow\ \underline{F(t=0)} 
		= F \circ \alpha \big|_{t=0} = F \circ \beta \big|_{t=0} 
		\\[5pt]
	& \bullet\ \boxed{ \alpha'(0) = \beta'(0) } \ \Rightarrow \tfrac{\partial x}{\partial \alpha_i} \big|_{t=0} 
		= \tfrac{\partial x}{\partial \beta_i} \big|_{t=0} 
		\cdot \bcancel{ \tfrac{d \beta_i / dt}{d \alpha_i / dt} \big|_{t=0} }
		\ \Rightarrow\ \nhs\nhs \boxed{ dF_{\alpha(0)}(\alpha'\mss{(0)}) = dF_{\beta(0)}(\beta'\mss{(0)}) }
		\hspace{5pt} \underline{ \text{\scriptsize(doesn't depend on \(\alpha\))} }
		\\[5pt]
	& \ast\ F = (f_0, f_1, \dots, f_m) 
		\ \Rightarrow\ \underline{ dF_{\alpha(0)} (\alpha'\mss{(0)}) } \hs \equiv\hs \tfrac{d}{dt}(F \circ \alpha) \big|_{t=0}
		= \left[ 
			\mss{ \arraycolsep=3pt \begin{matrix}
				\tfrac{\partial f_0}{\partial \alpha_0} & | & \dots \\[7pt]
				\tfrac{\partial f_1}{\partial \alpha_0} & F_{\alpha_1} & \dots \\
				\vdots & |
			\end{matrix} }
		\right]_{t=0}
		\left[ 
			\mss{ \begin{matrix}
				\tfrac{d \alpha_0}{dt}\\[7pt]
				\tfrac{d \alpha_1}{dt}\\
				\vdots
			\end{matrix} }
		\right]_{t=0}
		\hspace{-10pt} = \boxed{ J_F\mss{(0)} \cdot \alpha'\mss{(0)} }
		\\
	& \ast\ 
		\begin{gathered}
			\text{\scriptsize Surface Tangent}:\\[-5pt]
			\text{\scriptsize(see below)}
		\end{gathered}
		\
		\begin{aligned}
			q & = \gamma\mss{(t=0)} = (u\mss{(0)}, v\mss{(0)}) = X^{-1} \circ \alpha\mss{(0)}\\
			X\mss{(q)} & = X \circ \gamma \mss{(0)} = \alpha\mss{(0)} \in S
				\ \Rightarrow \ dX_{q}(\gamma'\mss{(0)}) = \alpha'\mss{(0)}
		\end{aligned}
		\\[10pt]
	& \bullet d(G \circ F)_p = dG_{F(p)} \circ dF_p
		\hspace{15pt} \bullet\ \underline{ \text{\scriptsize Regular \textbf{Value}, \(F(p)\)} }:\ dF_p \neq 0
		\hspace{15pt} \bullet\ \underline{ \text{\scriptsize Critical \textbf{Point}, \(p\)} }:\ dF_p = 0
\end{aligned}\)

\vspace{15pt}\noindent
% F is a Homeomorphism
\(
	\mss{ \begin{gathered}
		{F\ \text{is a}}\\[-2pt]
		\underline{\text{Homeomorphism}}\\[-2pt]
		\text{onto image}\ F(X)
	\end{gathered} }:\ 
	\begin{aligned}
		& \bullet\ F \ \text{\scriptsize is bijective between \(X\ \&\ F(X)\) }\\
		& \bullet\ F \ \text{\scriptsize is cont.}
			\hspace{10pt} \bullet\ F^{-1} \ \text{\scriptsize is cont.}
	\end{aligned}
\)
\hfill
% F is a Diffeomorphism
\(
	\mss{ \begin{gathered}
		{F\ \text{is a}}\\[-2pt]
		\text{\underline{Diffeomorphism}}\\[-2pt]
		\text{onto image}\ F(X)
	\end{gathered} }:\ 
	\begin{aligned}
		& \bullet\ F \in C^{\infty} \ \ \text{\scriptsize(cont. part. deri. of all orders)}\\
		& \bullet\ F^{-1} \in C^\infty \hspace{10pt} \bullet\ F \text{\scriptsize\ is a bijection}
	\end{aligned}
\)

% Inverse Function Theorem
\vspace{15pt}\noindent
\(
	\begin{gathered}
		\text{\underline{Inverse Function}}\\
		\text{\underline{Theorem} (IFT)} 
	\end{gathered}:\ 
	\begin{aligned}
		& \bullet\ F: \underline{ \mss{ \mathbb{R}^n \rightarrow \mathbb{R}^n } } ,\ F \in C^{\infty} \\
		& \bullet\ \exists dF_p^{-1} \ \ \text{(\scriptsize sq. matrix \(dF_p\) is an isomorphism/non-zero det.)} 
	\end{aligned}
	\ \Rightarrow\ \exists F^{-1} \in C^\infty \hspace{10pt} \text{\scriptsize(locally at F(p))}
\)

\vspace{5pt}
%-----------------------------------------------------------------------------------------------------------------------------------
%-----------------------------------------------------------------------------------------------------------------------------------
\section{Surfaces, \(S:\ 
	X\mss{(q)} = X\mss{(u,v)} = \bigl( x\mss{(u,v)}, y\mss{(u,v)}, z\mss{(u,v)} \bigr) 
	= p \in S \subset \mathbb{R}^3
\)}

% Surfaces
\noindent
\(\begin{aligned}
	% Regular Parametrized Surface
	& \underline{ \text{Regular Parametrized Surface} } \\
	& - \forall p \in S,\ \underline{\exists X \in C^\infty} 
		,\ X: V_q\ \text{\scriptsize(neighborhood of q)} \rightarrow V_p \cap S
		\hspace{15pt} \text{\scriptsize(diff. parametrizations are possible, btw)}
		\\
	& - dX_q \ \ \text{\scriptsize is one-to-one = (maybe non sq.) matrix col. are lin. ind. 
		= any 2x2 \(\big| \text{sub-}J_X \big|\) \(\neq 0\)}
		\ \Rightarrow\ \exists(\text{\scriptsize tangent at all points})
		\\[5pt]
	% Regular Surface
	& \underline{ \text{Regular Surface ({\scriptsize is reg. param. surface})} } \\
	& - \begin{gathered}
			X \ \text{\scriptsize is a homeo. in } V_q\\[-3pt]
			( \text{\scriptsize or}\ X \ \text{\scriptsize is one-to-one} )
		\end{gathered}
		\ \rightarrow\ 
		\begin{gathered}
			\underline{ X^{-1} \in C^0 } \hspace{5pt} \text{\scriptsize(is cont.)}\\[-5pt]
			\mss{ \forall p \in S,\ X^{-1}(V_p) = V_q }
		\end{gathered}
		\ \Rightarrow\ \text{\scriptsize \(\exists\) no self-intersections; cont. = } 
		\begin{gathered}
			\text{\scriptsize doesn't depend on parametrization}\\[-7pt]
			\text{\scriptsize(see coor. change below)}
		\end{gathered} 
\end{aligned}\)

% Coordinate Change between Two Parametrizations
\vspace{10pt}\noindent
\(\begin{aligned}
	& \bullet\ \text{\small Coordinate Change, \(h\), between Two Param. %
		\underline{is a Diffeomorphism (need for diff. func. on \(S\))}}: 
		\\[5pt]
	& \ast\ X^{-1} \text{\scriptsize\ is a homeomorphism} 
		\ \rightarrow\ \underline{ h = X^{-1} \circ Y \text{\scriptsize\ is is a homeomorphism from \(Y\) to \(X\)} }
		\ \Rightarrow\Rightarrow\ \underline{ h^{-1} \ \text{\scriptsize is a homomorphism} }
		\\[5pt]
	& \ast\ p \in S \ ,\ \ p = Y\mss{(\epsilon, \eta)} = X\mss{(u, v)} = \bigl( x\mss{(u,v)}, y\mss{(u,v)}, z\mss{(u,v)} \bigr) 
		\ , \ \ \tfrac{\partial(x,y)}{\partial(u,v)} \neq 0 
		\hspace{10pt} \text{\scriptsize(can change axes to make this true)}
		% \hspace{10pt} \text{\scriptsize(see Impl. Func. Theo. below)}
		\\
	& \hspace{13pt} F\mss{(u,v,t)} = \bigl( x\mss{(u,v)}, y\mss{(u,v)}, z\mss{(u,v)} + t \bigr) 
		: \ F \mss{(u,v,t)},\ X\mss{(u,v)} \in C^\infty
		\ , \ \ \exists dF^{-1} \ \stackrel{(IFT)}{\Rightarrow}\ F^{-1} \in C^\infty
		\\[3pt]
	& \hspace{13pt} \mss{(u,v)} = \mss{X^{-1} \circ Y} \mss{(\epsilon, \eta)} = h \mss{(\epsilon, \eta)} 
		\stackrel{\sim}{=} ( \mss{F^{-1} \circ Y} ) \mss{(\epsilon,\eta)} 
		\ \Rightarrow\ \underline{ h \in C^\infty } 
		\ \Rightarrow\Rightarrow\ \underline{ h^{-1} \in C^\infty } 
		\hspace{10pt} \text{\scriptsize(same for \(Y^{-1} \circ X\))}
		\\[2pt]
	& \ast \text{\scriptsize\ Needed that \(X^{-1} \in\) 
		\textbf{\(C^0\) on a [3D] neigh. for every point} \( [\forall p \in S,\ X^{-1}(V_p) = V_q \stackrel{\sim}{=} F^{-1}(V_p)] \)
		, to avoid \( (t \neq 0,\ F^{-1} \circ Y \neq h) \)
		}
		\\[2pt]
	& \ast \text{\scriptsize\ Ex:}\ \ \mss{ 
			\begin{gathered}
				\gamma(t) = (\cos t, \sin 2t) \\[-2pt]
				\gamma(\mathbb{R}) = \alpha(I_1) = \beta(I_2)\\[-2pt]
				{(\infty \text{ - graph {\scriptsize \underline{not reg.}}})}
			\end{gathered}
			\ ,\ \arraycolsep=1pt \begin{array}{c c c c c}
				I_1 & = 
					& (-\tfrac{\pi}{2} , \tfrac{3\pi}{2}) 
					& = 
					& (-\tfrac{\pi}{2} , \tfrac{\pi}{2}) \cup \hs\hs\tfrac{\pi}{2}\hs\hs \cup (\tfrac{\pi}{2} , \tfrac{3\pi}{2})
					\\[2pt]
				I_2 & = 
					& (\tfrac{\pi}{2} , \tfrac{5\pi}{2}) 
					& = 
					& \underline{ (\tfrac{3\pi}{2} , \tfrac{5\pi}{2}) \cup \tfrac{3\pi}{2} \cup (\tfrac{\pi}{2} , \tfrac{3\pi}{2}) }
			\end{array}
			\ \Rightarrow\ 
			\begin{aligned}
				& \beta^{-1} \text{ is 1:1 but not cont. for any \textbf{[2D] neigh.} of } (0,0)\\[-2pt]
				& F^{-1}(x,y) = (t', u) \ \neq\ \beta^{-1}(x,y) \stackrel{\sim}{=} (t,0) \ \text{ near } (0,0) \\[-1pt]
				& \underline{\beta^{-1} \circ \alpha (I_1)} \text{ is 1:1 but not cont., so not diffeo.}
			\end{aligned}
		}
		\\
	&
\end{aligned}\) 

%--------------------------------------------------------------------------------------------------------------------------------------
%
%
%
\newpage

% Theorems about Reg Surfaces
\vspace{15pt}\noindent
\(\begin{aligned}	
	& \bullet\ \underline{ f \in C^\infty } 
		\ \Rightarrow\ \boxed{ \big( \hsvec{x}, f\mss{(\hsvec{x})} \big) \ \text{\scriptsize is a reg. surf.} }
		\\[10pt]
	% Regular Value Theorem
	& \bullet\ 
		\begin{aligned}
			& f : \mss{ \mathbb{R}^n \rightarrow \mathbb{R} }\\
			& f \mss{ (X) } = c \\[-7pt]
			& \underline{ \text{\scriptsize is a reg. val.} }
		\end{aligned}
		\ , \ \ \begin{aligned}
			& f \in C^\infty \\
			& F\mss{(X)} = ( \mss{ x_1,\hs ...\ , x_{n-1} , f(X) } )\\
			& \exists dF_p^{-1}
		\end{aligned}
		\ \ \stackrel{\text{(IFT)}}{\Rightarrow}\ \
		\begin{aligned}
			& \exists F^{-1} \in C^\infty \\
			& F^{-1}( \mss{ f_1,\hs ...\ , f_{n-1} , f(\nhs\hsvec{x}) } ) = X
		\end{aligned}
		\ , \ \ 
		\begin{aligned}
			& x_n\ \mss{ = f^{-1}_n }: \mss{ \mathbb{R}^n \rightarrow \mathbb{R} }\\
			& \underline{ x_n\ \mss{ = f^{-1}_n } \in C^\infty }
		\end{aligned}
		\\[5pt]
	& \hspace{10pt} \rightarrow\ 
		\begin{aligned}
			x_n & = \mss{ {f}^{-1}_n }( \mss{ x_1,\hs ...\ , x_{n-1} , f(\nhs\hsvec{x}) = c } )\\
			& = \underline{ \mss{ {f'}^{-1}_n } ( \mss{ x_1,\hs ...\ , x_{n-1} } ) }
		\end{aligned}
		\ \Rightarrow \ 
		\begin{aligned}
			& S = \underline{ ( \mss{ x_1,\hs ...\ , x_{n-1} , {f'}^{-1}_n } ) }
				\ \text{\scriptsize where}\ \mss{ f(\nhs\hsvec{x}) = c } 
				\\
			& S \ ==\ \text{\scriptsize Surface}\ f^{-1}{(c)}
		\end{aligned}
		\Rightarrow \boxed{ \begin{gathered}
			\underline{ \text{\scriptsize Regular Value Theorem} }\\
			\text{\scriptsize Surface}\ f^{-1}{(c)}\ \text{\scriptsize is reg.} 
		\end{gathered} }
		\\[5pt]
	% Implicit Function Theorem
	& \bullet\ \underline{ \tfrac{\partial(x,y)}{\partial(u,v)} \neq 0 }
		\ \Rightarrow\ { \pi_{\text{proj.}} \circ X } \mss{(u,v)} 
		\equiv  ( x\mss{(u,v)}, y\mss{(u,v)}, \bcancel{ z\mss{(x,y)} } )
		\stackrel{(IFT)}{\Rightarrow} \ (\mss{\pi \circ X})^{-1}\mss{( x, y )}
		= (u\mss{(x,y)}, v\mss{(x,y)})
		\\
		% \begin{aligned}
		% 	\text{\scriptsize(For coord. change, \(\gamma(t)\) isn't 1:1)} \hspace{10pt} 
		% 	& ( \text{\scriptsize\& } X \ \text{\scriptsize is \underline{one-to-one}} )
		% \end{aligned}
		% ,\ 
	& \hspace{10pt} 
		\begin{aligned}
			% Implicit Function Theorem
			& \ast\ X\mss{(u,v)} = ( x\mss{(u,v)}, y\mss{(u,v)}, \underline{ z\mss{(u,v)} } ) 
				\ \Rightarrow \ z( u(\mss{x,y}), v(\mss{x,y})) 
				= z \circ (\mss{\pi \circ X})^{-1}\mss{(x,y)} 
				= \boxed{ \begin{gathered}
					\underline{ \text{\scriptsize Implicit Func. Theor.} } \\[-5pt]
					\text{\scriptsize(locally orientable)}\\[-3pt]
					 f\mss{(x,y)} = z \ \mss{\in C^\infty} 
				\end{gathered} }
				\\[-3pt]
			% 1:1 -> Continuous
			& \ast\ \begin{gathered}
					\underline{ \text{\scriptsize Know } S \text{\scriptsize\ is reg. sur.} }\\[-2pt]
					X \text{\scriptsize\ is param?}
				\end{gathered}
				,\ \begin{gathered}
					X \in C^\infty\\[-2pt]
					dX_q \text{\scriptsize\ is 1:1} 
				\end{gathered}
				,\ \underline{ X \text{\scriptsize\ is 1:1} }
				\ \Rightarrow \ \underline{ (\mss{\pi \circ X})^{-1} \circ \pi } \circ X \mss{(u,v)} 
				= \underline{X^{-1}} \circ X \mss{(u,v)}
				\ \Rightarrow\ \boxed{X^{-1} \in C^0}
		\end{aligned}
\end{aligned}\)

% Regular Surface + Differentials
\vspace{15pt}\noindent
\(\begin{aligned}
	% Surface Tangents
	& \bullet\ \begin{gathered}
			\underline{ \text{Surface} }\\
			\underline{ \text{Tangent} }
		\end{gathered} 
		:\ \
		\begin{aligned}
			q & = \gamma\mss{(t=0)} = (u\mss{(0)}, v\mss{(0)}) = X^{-1} \circ \alpha\mss{(0)}\\
			X\mss{(q)} & = X \circ \gamma \mss{(0)} = \alpha\mss{(0)} \in S
				\ \Rightarrow \ dX_{q}(\gamma'\mss{(0)}) = \alpha'\mss{(0)}
				= \tfrac{\partial X}{\partial u}\mss{(q)} \hs u'\mss{(0)} + \tfrac{\partial X}{\partial v}\mss{(q)} \hs v'\mss{(0)}
		\end{aligned}
		\\[5pt]
	% First Form
	& \bullet\ \text{``\underline{First Form}''}:\ \left< \alpha'\mss{(0)} , \alpha'\mss{(0)} \right> 
		= \lVert \mss{ \alpha' } \rVert^2 
		= \mss{
			\left[ u' \ v' \right]
			\left[ \begin{matrix}
				X_u\\
				X_v
			\end{matrix} \right]
			\left[ X_u \ X_v \right]
			\left[ \begin{matrix}
				u'\\
				v'
			\end{matrix} \right]
		}
		= \begin{gathered}
			\mss{ \lVert X_u \rVert^2 (u')^2 + 2 \left< X_u, X_v \right> u'v'  + \lVert X_v \rVert^2 (v')^2 }\\[-5pt]
			\boxed{ \mss{ E (u')^2 + 2 F u'v'  + G (v')^2 } }
		\end{gathered}
		\\[5pt]
	% Line Element
	& \bullet\ \begin{gathered}
			\underline{ \text{Line} }\\[-3pt]
			\underline{ \text{Element} }
		\end{gathered}
		:\ ds = \lVert \alpha'\mss{(t)} \rVert dt
		% Area Element
		\hspace{15pt} \bullet\ \begin{gathered}
			\underline{ \text{Area} }\\[-3pt]
			\underline{ \text{Element} }
		\end{gathered}
		:\ dA = { \lVert X_u \times X_v \rVert } du\hs dv 
		= \sqrt{ EG - F^2 } \hs du\hs dv 
\end{aligned}\)

%------------------------------------------------------------------------------------------------------------------------------------

\vspace{15pt}\noindent
% Regular Curves
\(\ast\ \begin{aligned}[t]
	& \underline{ \text{Regular Curves, \(C \in R^3\) (instead of Regular Parametrized Curves)} }\\[5pt]
	& \bullet\ \forall p \in C
		,\ \exists \alpha \in C^\infty
		,\ \alpha: I_t \ \text{\scriptsize(neighborhood of \(t\))} \subset R 
		\rightarrow V_p \cap C \ \text{\scriptsize(neighborhood of \(p\))}
		\\
	& \bullet\ \forall t \in I\ ,\ \ d\alpha_t \ \text{\scriptsize is 1:1}
		\hspace{20pt} \bullet\ \alpha \text{\scriptsize\ is a homeo. in } I_t
		\\
	& \ast\ \text{\scriptsize Change of param. are homeomorphisms} 
		\ \Rightarrow\ \text{\scriptsize Properties like arc length, curvature, torsion, etc. aren't param. dependent}
\end{aligned}\)

% Coordinate Curves
\vspace{10pt}\noindent
\(
	\ast\ \underline{\text{Coordinate Curves}}:\ 
	\alpha\mss{(t)} = X \circ \gamma\mss{(t)} \ \big|\ 
	\gamma \in \{ ( u\mss{(t)}, v_0 ) ,\ ( u_0, v\mss{(t)} ) \}
	\hspace{15pt} \text{\scriptsize(maps of parallels and meridians)}
\) 

%------------------------------------------------------------------------------------------------------------------------------------

% Function on S
\vspace{15pt}\noindent
\(\begin{aligned}
	& \underline{ \text{Function, \(f: S \subset \mathbb{R}^n \rightarrow \mathbb{R}\)} }\\[5pt]
	& \bullet\ \big( \forall p \in S,\ \underline{ f(p) \neq 0 } \big)
		\ \Rightarrow\ \big( \forall p \in S,\ \underline{ f(p) > 0 } \big) 
		\text{ or } \big( \forall p \in S,\ \underline{ f(p) < 0 } \big)
		\\
	& \bullet\ \underline{ \textit{Differentiable on } S }:\ f \circ X \in C^\infty
		\hspace{10pt} \text{\scriptsize(doesn't depend on param./coord. change)}
		\\
	& \bullet\ \text{E.g., } X^{-1}\mss{(p)},\ \hsvec{v} \cdot p,\ |p-p_0|^2 
		\ \Rightarrow\ \boxed{X^{-1} \in C^\infty} \ , \ \ \boxed{U \ \text{\scriptsize is diffeo. to } X(U)}
\end{aligned}\)

% Function from S_1 to S_2
\vspace{10pt}\noindent
\(\begin{aligned}
	& \underline{ \text{Function, \(\phi: S_1 \rightarrow S_2\) is a Diffeomorphism from \(S_1\) to \(S_2\)} }\\[5pt]
	& \bullet\ \underline{ \textit{Differentiable}}:\ X_2^{-1} \circ (\phi \circ X_1) \in C^\infty
		\hspace{10pt} \text{\scriptsize(doesn't depend on param./coord. change)}
		\\
	% ?????
	& \bullet\ \beta'(0) = d\phi_p(w) = d\phi_p \hs \alpha'\mss{(0)} = d\phi_p\hs dX_q (u'\mss{(0)}, v'\mss{(0)})^T 
		\hspace{25pt} \text{(p.85???)}
		\\
	% p.86 Inverse Function Theorem
	& \bullet\ \underline{ \text{Inverse Function Theorem} }:\ \phi \in C^\infty ,\ \exists d\phi_p^{-1} 
		\ \Rightarrow\ \phi^{-1} \in C^\infty \hspace{10pt} \text{\scriptsize(Diffeomorphism from \(S_1 \rightarrow S_2\)??????)}
\end{aligned}\)


%-------------------------------------------------------------------------------------------------------------------------------------



%----------------------------------------------------------------------------------------------------------------------------------
%
%
%
\newpage
% Gauss Map/Normals
\section{Gauss Map (Normals),\ \( N(p) = \tfrac{X_u \times X_v}{| X_u \times X_v |} 
= \tfrac{X_u \times X_v}{EG-F^2} : S \rightarrow S^2 \)}

\noindent
\(\begin{aligned}
	% Derivative of Gauss Map
	& N'\mss{(p)} = \underline{ dN_p \hs \alpha'\mss{(0)} }
		= \Big[ \begin{gathered}
			\mss{(dN_p)}\\[-6pt]
			\mss{ N_x \hs\hs N_y \hs\hs N_z }
		\end{gathered} \Big]
		\Big[ \begin{gathered}
			\mss{(dX_q)}\\[-6pt]
			\mss{ X_u \hs\hs X_v } ]
		\end{gathered}
		\mss{ \left[ \begin{gathered}
			u'\\[-2pt]
			v'	
		\end{gathered}\right] }
		\equiv \Big[ \mss{ N_u \ N_v } ] 
		\mss{ \left[ \begin{gathered}
			u'\\[-2pt]
			v'	
		\end{gathered}\right] }
		\\[3pt]
	% Orientation
	& \bullet\ \boxed{ S = f^{-1}(c) \ \Leftrightarrow\ { \textit{Orientated}} \hs\hs } 
		= \underline{ 
			\text{\scriptsize normals \(N(p)\) are in same dir. \((\pm 1)\)} 
			}
		= \boxed{ \exists \tfrac{\partial(\hat{u},\hat{v})}{\partial(u,v)} > 0 \text{ over all } S }
\end{aligned}\)

\vspace{10pt}\noindent
\(\begin{aligned}
	% Second Fundamental Form
	& \underline{\text{Second Fundamental [Quadratic] Form}}:\ 
		\left< -dN_p(\alpha'\mss{(0)}), \alpha'\mss{(0)} \right> 
		= \underline{\left< \alpha'\mss{(0)}, -dN_p(\alpha'\mss{(0)}) \right>}
		\hspace{15pt} \text{\scriptsize(is self-adjoint)}
		\\[3pt]
	%% Normal Curvature = Second Form
	& \ast\ { \mss{ \left< N\mss{(s)}, \alpha'\mss{(s)}\right> = 0 } }
		\ \Rightarrow
		\mss{ \begin{aligned}
			& \boxed{ \left< N\mss{(s=0)}, \alpha''\mss{(0)} \right> }\\
			& = -\left< N'\mss{(0)}, \alpha'\mss{(0)} \right>
		\end{aligned} }
		= \begin{gathered}
			\text{\scriptsize(depends on \(\alpha'(0)\))}\\[-2pt]
			\underline{ - \left< dN_p \hs \alpha'\mss{(0)}, \alpha'\mss{(0)} \right> }
		\end{gathered}
		= \begin{gathered}
			\text{\scriptsize(\underline{Normal Curvature} of \(\alpha\) at \(p\))}\\[-2pt]
			\boxed{ \left< N, kn \right> \mss{(p)} \equiv k_n\mss{(p)} }
		\end{gathered}
		= \ \begin{gathered}
			\text{\scriptsize\(k\) of \(\alpha\) from a}\\[-9pt]
			\text{\scriptsize\underline{normal (cross)}}\\[-6pt]
			\text{\scriptsize\underline{section} of \(S\)}
		\end{gathered}
		\\[3pt]
	%% Second Form in terms of efg
	& \ast\ \begin{aligned}[t]
			& \hspace{16pt} \underline{ \mss{ \left< -dN_p \hs \alpha', \alpha' \right> } } = \mss{ 
					-[ N_u \ N_v ] 
					\left[ \begin{gathered}
						u'\\[-5pt]
						v'
					\end{gathered} \right]
					[ u' \ v' ]
					\left[ \begin{gathered}
						X_u\\[-5pt]
						X_v
					\end{gathered} \right] 
				}
				= \mss{ \underline{ 
					\Big( \underbrace{ -\left< N_u, X_u \right> }_{e} , 
					\underbrace{ -\left< N_u , X_v \right> - \left< N_v , X_u \right> }_{2f = 2 \left< N_u , X_v \right> } , 
					\underbrace{ -\left< N_v , X_v \right> }_{g} \Big) 
					\cdot \Big( (u')^2, u'v' ,(v')^2 \Big)
				} }
				\\[-10pt]
			& \boxed{ \begin{aligned}
					k_n\mss{(p, \alpha')} & = \mss{ e (u')^2 + 2f u'v' + g (v')^2 }\\[-5pt]
					\text{\scriptsize(locally, \(\leq 2\) sol.)} & = \mss{ (Au' + Bv')(Cu' + Dv') }
				\end{aligned}
			}
		\end{aligned}
		\\
	% Tangent Plane	Principal Directions/Eigenbasis
	& \bullet\ \begin{gathered}
			\text{\scriptsize(\underline{Prin. dir.} at \(p\))}\\[-7pt]
			\text{\scriptsize\underline{Eigenbasis}}
		\end{gathered} 
		: \exists e_1, e_2 
		\hs\hs \big| \hs\hs \mss{span(e_1, e_2)} = T_p(S) \hs \ni \hs \underline{ -dN_p(xe_1 + ye_2) = k_1 x e_1 + k_2 y e_2 }
		\hspace{15pt} \begin{gathered}[b]
			\text{\scriptsize(\underline{Prin. curv.} at \(p\))}\\[-5pt]
			\text{\scriptsize(eigenvalues, \(k_1 \geq k_2\))}
		\end{gathered}
		\\[3pt]
	%% Euler's Formula for 2nd Form
	& \ast\ \underline{\text{Euler's Formula (for 2nd Form)}}:\ 
		\underline{ \left< -dN_p \hs \hsvec{t} , \hsvec{t} = \mss{e_1 \cos\theta + e_2 \sin\theta} \right> }
		= \boxed{ 
			\begin{aligned}[t]
				k_1 \cos^2\theta + k_2 \sin^2\theta & = k_n\mss{(p,\theta)} \\[-6pt]
				% \mss{e (u')^2 + 2f u'v' + g (v')^2} & =
			\end{aligned}
		}
		\\[5pt]
	% Gaussian Curvature, K
	& \bullet\ \begin{gathered}
			\text{\underline{Gaussian Curvature}}:\\[-3pt]
			( \text{\scriptsize \(2n\) dim. !\(\Delta\) det. w/ orien. flip} )
		\end{gathered} 
		\ \ \boxed{ \begin{aligned}
			K\mss{(p)}\ & \mss{ = \det(dN_p) }\\[-6pt]
			& \mss{ = (-k_1)(-k_2) }
		\end{aligned} }
		\hspace{15pt}
		% Mean Curvature
		\bullet\ \text{\underline{Mean Curvature}}:\ \boxed{ H\mss{(p)} = \tfrac{-\text{Tr}(dN_p)}{2} = \tfrac{k_1 + k_2}{2} }
		\\[3pt]
	%& Quadradic Type for K
	& \ast\ \text{\scriptsize Planar: \(dN_p = 0\), Ellip.\(\rightarrow K > 0\), Para.\(\rightarrow K = 0\), ...} 
		\hspace{15pt} 
		\ast\ \mss{ K>0 \Rightarrow \exists V_p: p+T_p(S) \text{ !div.}\hs\hs V_p} 
		\ , \ \mss{ K<0 \Rightarrow \forall V_p: p+T_p(S) \text{ div.}\hs\hs V_p} 
		\\[3pt]
	% Interpretation of Gaussian Curvature
	& \ast\ \begin{gathered}
			\mss{ | Av \times Aw | = |v \times w| k_1k_2 } \hspace{7pt} \text{\scriptsize(2D)} \\[-5pt]
			\underline{ \mss{ | dN_p X_u \times dN_p X_v | = |X_u \times X_v| \cdot K } }
		\end{gathered}
		\ \Rightarrow\ K \mss{\neq 0} = 
		\begin{gathered}
			\underline{ 
				\mss{ \lim_{\int dudv \rightarrow 0} \int | dN_p X_u \times dN_p X_v | \hs du dv } \ /\ \mss{ \int dudv } 
				}
				\\[-2pt]
			\mss{ \lim_{\int dudv \rightarrow 0} \int | X_u \times X_v | \hs du dv } \ /\ \mss{ \int dudv }
		\end{gathered}
		\hspace{15pt} \text{\scriptsize(See Indi. of Tan. \(k\))}
		\\
	% dN in basis of X_u, X_v
	& \bullet\ \begin{aligned}[t]
			& \underline{N_u, N_v \in T_p(S)} \ \Rightarrow\ dN_p \hs \alpha'\mss{(0)}
				= \underline{ [ \mss{ N_u \ N_v } ] }
				\mss{ \left[ \begin{gathered}
					u'\\[-2pt]
					v'	
				\end{gathered}\right] }
				\equiv \underline{
					[ \mss{X_u \ X_v} ] 
					\Big[ dN \Big] 
				}
				\mss{ \left[ \begin{gathered}
					u'\\[-2pt]
					v'	
				\end{gathered}\right] }
				\\
			%% continuation
			& \begin{gathered}
					\text{\scriptsize{\underline{General Basis} for \(N_u, N_v\)}}\\[-5pt]
					% \text{\scriptsize{(Weingarten Eq.)}}
				\end{gathered}:\ 
				\mss{ \left[ \arraycolsep=3pt \begin{matrix}
					X_u \cdot N_u = -e & X_u \cdot N_v = -f\\
					X_v \cdot N_u = -f & X_v \cdot N_v = -g
				\end{matrix} \right] }
				= \mss{ \left[ \arraycolsep=2pt\begin{matrix}
					X_u^2 = E & X_u \cdot X_v = F\\
					X_v \cdot X_u = F & X_v^2 = G
				\end{matrix} \right] } 
				\Big[ dN \Big]
				\hspace{15pt} \underline{ \mss{ \begin{aligned}
					\left< N, X_{ij} \right> & = - \left< N_i, X_j \right> \\
					& = - \left< N_j, X_i \right>
				\end{aligned} } }
		\end{aligned}
		\\
	%% dN entries with efg,EFG
	& \ast\ \begin{gathered}
			\text{\scriptsize{(Weingarten Eq.)}}\\[-3pt]
			\boxed{ 
				\big[ dN \big] = \tfrac{-1}{EG-F^2}
				\mss{ 
					\left[ \arraycolsep=2pt\begin{matrix}
						G & -F\\
						-F & E
					\end{matrix} \right]
					\left[ \arraycolsep=2pt\begin{matrix}
						e & f\\
						f & g
					\end{matrix} \right]
				}
			}
		\end{gathered}
		\hspace{15pt}
		% Gaussian + Mean Curvature
		\ast\ \begin{gathered}
		\mss{ (k^2 - 2Hk + K = 0) } \\[-3pt]
		\boxed{ k_\pm = \mss{ H \pm \sqrt{H^2 - K} } } 
		\end{gathered} : \
		\boxed{ K = \tfrac{ eg - f^2 }{ EG - F^2 } }
		\hspace{5pt} , \hspace{5pt} 
		\boxed{ H = \tfrac{1}{2} \tfrac{ e G - 2fF + g E }{ EG - F^2 } }
\end{aligned}\)

\vspace{10pt}\noindent
% Umbilical Point
\(
	\underline{\text{Umbilical Point}}:\ \begin{gathered}
		p \in S \ \big|\ k_1 = k_2 \ \Rightarrow\ H^2 = K \\[-5pt]
		\text{\scriptsize(only spheres \& planes have all umb. pts.)}
	\end{gathered}
\)
\hspace{20pt}
%$ Asymptotic Directions
\( \underline{\text{Asymptotic Direction}}:\ k_n \mss{(p,\theta)} = 0 \)

\vspace{10pt}\noindent
\(\begin{aligned}
	% Conjugate Directions
	& \underline{\text{Conjugate Directions}}:\ 
		(\theta, \phi) \ \text{\scriptsize from \(e_1\)} \hs\hs \big| \hs\hs 
		\left< -dN_p\hs \hsvec{t}_1\mss{(\theta)} , \hsvec{t}_2 \mss{(\phi)}\right> 
		\equiv \boxed{ 0 = -k_1 \cos\theta \cos\phi - k_2 \sin\theta \sin\phi }
\end{aligned}\)

\vspace{10pt}\noindent
\(\begin{aligned}
	% Dupin Indicatrix
	& \text{\underline{Dupin Indicatrix}}:\ 
		\left< -dN_p\hs\hsvec{t}, \hsvec{t} \hs \hs \right> = \pm \tfrac{1}{\rho^2} = k_n
		\ \Rightarrow\ 
		\begin{aligned}[t]
			\left< -dN_p(\rho\hsvec{t}\hs\hs), (\rho\hsvec{t}\hs\hs) \right> & 
				= k_1 \hs \underline{ \rho^2 \cos^2\theta } + k_2\hs \underline{ \rho^2 \sin^2\theta }
				\\
			\underline{\text{\scriptsize Conic Graph \( (\xi,\eta) \)}} & 
				= \boxed{ k_1 \underline{\xi^2} + k_2 \underline{\eta^2} = \pm 1 }
		\end{aligned}
		\\[-15pt]
	%% Asymptotic Direction with Dupin Indicatrix for Elliptic Curvature (There's none)
	& \bullet\ K > 0 \ \Rightarrow\ \forall \theta,\ k_n\mss{(\theta)} > 0
		\\[5pt]
	%% Asymptotic Direction with Dupin Indicatrix for Hyperbolic Curvature
	& \bullet\ K < 0 \ \Rightarrow\ \exists \theta_{\underline{1,2}} \ \big|\ 
		k_n\mss{(\theta)} = 0 
		= (\mss{ k_1 \cos^2\theta + k_2 \sin^2\theta })\rho^2 = k_1 \xi^2 + k_2 \eta^2 \neq \pm 1 
		\hspace{15pt} \text{\scriptsize(\(\theta_{\underline{1,2}}\) are asymptotes of \( (\xi, \eta) \))}
		\\[5pt]
	%% Conjugate Direction with Dupin Indicatrix
	& \bullet\ \text{Conj. Dir.}\ (\phi_1, \phi_2):\ \phi_{{2,1}}
		= \arctan \tfrac{d\eta}{d\xi}\big|_{(\xi,\eta) \hs\hs \cap\hs\hs \theta = \phi_{1,2}}
\end{aligned}\)

%---------------------------------------------------------------------------------------------------------------------------------
%
%
%
\newpage
\noindent
\(\begin{aligned}
	% Line of Curvature
	& \underline{\text{Line of Curvature}}:\ \alpha\mss{(t)} \ \big|\ 
		N'\mss{(t)} = \underline{ dN_p \hs \alpha'\mss{(t)} } = \underline{ \lambda\mss{(t)} \hs \alpha'\mss{(t)} }
		\hspace{15pt} \text{\scriptsize(curve s.t. tangent is always in a princ. dir.)}
		\\[3pt]
	% Determinant Condition for Line of Curvature
	& \bullet\ [ \mss{ u'\ v' } ]
		\mss{ \left[ \begin{matrix}
			0 & 1\\
			-1 & 0
		\end{matrix} \right] }
		\Big[ {dN} \Big]
		\mss{ \left[ \begin{matrix}
			u'\\
			v'
		\end{matrix} \right] }
		= [ \mss{ -v'\ u' } ] \lambda\mss{(t)} 
		\mss{ \left[ \begin{matrix}
			u'\\
			v'
		\end{matrix} \right] }
		= 0
		\ \stackrel{\text{(expand)}}{\Rightarrow} \ 
		\boxed{
			\left\vert \mss{ \arraycolsep=2pt \begin{matrix}
				(v')^2 & -u'v' & (u')^2\\
				e & f & g \\
				E & F & G
			\end{matrix} } \right\vert = 0
		}
		\\[3pt]
	%% Asymptotic Curves
	& \ast\ \begin{gathered}
			\underline{ \text{Asymp. Curve} }
		\end{gathered} :\ 
		\begin{aligned}
			& \alpha\mss{(t)} \hs\hs \big|\hs\hs 
				\mss{ 
					\lambda(t) = k_n \mss{(p,\theta)} 
					= \boxed{
						k_1 \cos^2\theta + k_2 \sin^2\theta = e (u')^2 + 2f u'v' + g (v')^2 = (Au' + Bv')(Cu' + Dv') = 0
					} 
				}
				\\
			& \mss{ 
				(ef-g),\ \underline{K < 0} 
				\ \Rightarrow\ 0 = \underline{ (Au' + Bv')(Au' + Dv') } :\ A^2 = e,\ A(B+D) = f,\ BD = g 
				\ \Rightarrow\ \boxed{ \exists \alpha_1, \alpha_2 }
				}
		\end{aligned}
		\\[3pt]
	& \ast\ \underline{e=g=0} \ \Leftrightarrow\ \boxed{ \alpha \circ (c, v\mss{(t)}) } \wedge \boxed{ \alpha \circ (u\mss{(t)}, c) }
		\ \ \text{\scriptsize are asympt. curves}
\end{aligned}\)

\vspace{15pt}\noindent
\(\begin{aligned}
	% Surface of Revolution
	& \text{\underline{Surface of Revolution}}:\ X\mss{(u,v)} 
		= \big( \rho\mss{(v)} \cos u ,\hs \rho\mss{(v)} \sin u,\hs z\mss{(v)} \big)
		\hspace{5pt} \big| \hspace{5pt} \alpha_u\mss{(v)} = \big( z\mss{(v)} ,\ \rho\mss{(v)} \big)
		\hspace{5pt} , \hspace{5pt} \Vert \alpha_u' \Vert = 1
		\\[5pt]
	% First Fundamental Form
	& \bullet\ \left< \alpha', \alpha' \right> 
		= \begin{gathered}[t]
			\underline{ \big[ \rho^2,\ 0,\ (\rho')^2 + (z')^2 = 1 \big] } \\
			\mss{(\rho' \rho'' + z' z'' = 0)}
		\end{gathered}
		\left[ \hs \mss{ \begin{matrix}
			(u')^2 \\
			2u'v'\\
			(v')^2
		\end{matrix} } \hs \right]
		\hspace{15pt}
		% Second Fundamental Form
		\bullet\ \left< N, \alpha'' \right> 
		= \underline{ \big[ -\rho z',\ 0,\ \rho'' z' - \rho' z'' \hs \big] } 
		\left[ \hs \mss{ \begin{matrix}
			(u')^2 \\
			2u'v'\\
			(v')^2
		\end{matrix} } \hs \right]
		\\
	% Eigenvalues and Curvature
	& \bullet\ \boxed{ k_1 = \tfrac{e}{E} = - \tfrac{z'}{\rho} }
		\ , \ \ \boxed{ k_2 = \tfrac{g}{G} = \rho'' z' - \rho' z'' }
		\ , \ \ \boxed{ K = - \tfrac{z' ( \rho'' z' - \rho' z'' ) }{\rho} = - \tfrac{\rho''}{\rho} }
\end{aligned}\)


\vspace{15pt}\noindent
\(\begin{aligned}
	% Graph of a Differential Function
	& \underline{ \text{Graph of a Differentiable Function} }:\ X\mss{(u,v)} = \big( u,v, z\mss{(u,v)} \big) 
		\hspace{20pt}
		% Normal
		\bullet\ N\mss{(p)} = \tfrac{(- z_u, -z_v, 1)}{\sqrt{z_u^2 + z_v^2 + 1}}
		\\
	% First Fundamental Form
	& \bullet\ \left< \alpha', \alpha' \right> 
		= \underline{ \big[ 1 + z_u^2,\ z_u z_v,\ 1 + z_v^2 \big] } 
		\left[ \hs \mss{ \begin{matrix}
			(u')^2 \\
			2u'v'\\
			(v')^2
		\end{matrix} } \hs \right]
		\hspace{15pt}
		% Second Fundamental Form
		\bullet\ \left< N, \alpha'' \right> 
		= \underline{ \tfrac{1}{\sqrt{z_y^2 + z_v^2 + 1}} \big[ z_{uu},\ z_{uv},\ z_{vv} \big] } 
		\left[ \hs \mss{ \begin{matrix}
			(u')^2 \\
			2u'v'\\
			(v')^2
		\end{matrix} } \hs \right]
		\\[5pt]
	% Local Form Hessian
	& \bullet\ z\mss{(0,0)} = p \ , \ \ N(p) = \mss{(0,0,1)} 
		\ \Rightarrow\ \underline{\text{Hessian}}:\ 
		k_n\mss{(p)} = \underline{ \big[ z_{xx},\ z_{xy},\ z_{yy} \big] } 
		\left[ \hs \mss{ \begin{matrix}
			x^2 \\
			2xy\\
			y^2
		\end{matrix} } \hs \right] 
		\ , \ \hsvec{v} = \mss{(x,y)} 
		\\[3pt]
	%% Hessian to Dupin Indicatrix
	& \ast\ \hsvec{v} = \mss{xe_1 + ye_2} 
		\ \Rightarrow\ \begin{gathered}[t]
			z\mss{(x,y)} - z\mss{(0,0)}\\[-5pt]
			\text{\scriptsize(\(p\) is non-planer!!)}
		\end{gathered} 
		= \tfrac{1}{2} k_n\mss{(p)} + \mathcal{O}(r^3)
		\hs \approx\hs 
		\mss{ \begin{aligned}[t]
			\tfrac{1}{2}(z_{xx} x^2 + x_{yy} y^2) & = \epsilon\\
			k_1x^2 + k_2 y^2 & = 2\epsilon
		\end{aligned} }
		\rightarrow
		\mss{ \begin{gathered}[t]
			k_1\chi^2 + k_2 \eta^2 = \pm 1\\[-2pt]
			\boxed{ \text{\scriptsize(Dupin Indicatrix)} }
		\end{gathered} }
\end{aligned}\)

\vspace{10pt}\noindent
\(
	\begin{aligned}
		% Vector Field : w
		& \text{\underline{{\scriptsize(Diff.)} Vector Field over \(S\)}}:\ 
			\boxed{ w\mss{(p)} = a\mss{(u,v)} X_u + b\mss{(u,v)} X_v }
			\hspace{15pt} \big( \text{e.g.}\ \underline{ \gamma\mss{(t)} \rightarrow w_\gamma\mss{(p)} = u' X_u + v' X_v } \hs \big)
			\\[5pt]
		% Trajectory of w
		& \textit{\underline{Trajectory of \(w\)}}:\ 
			\alpha\mss{(t)} \subset S \ \big| \ \boxed{ \alpha\mss{(0)} = p,\ \alpha'\mss{(t)} = w( \alpha\mss{(t)} ) }
			% \hspace{20pt}
			\\
		% Local Flow of w
		& \textit{\underline{(Local) Flow of \(w\)}}:\ 
			\alpha\mss{(p,t)} \equiv \alpha_p\mss{(t)} \ \big| \ 
			\boxed{ \alpha_p\mss{(0)} = p }
			,\ \boxed{ \alpha_p'\mss{(t)} = w( \alpha_p\mss{(t)} ) }
			\ \Rightarrow\ \boxed{ \alpha_p\mss{(t)} = p + ( a_0^1\mss{(t)}, a_0^2\mss{(t)}, a_0^3\mss{(t)} )}
			\\[5pt]
		%% exists some local inverse that relates to the first integral of w
		& \bullet\ \mss{ \begin{aligned}
				& \boxed{ w\mss{(p_0)} \neq 0 }\\
				& w\mss{(p_0)} \cdot \hat{x} = |w|\\
				& \tilde{\alpha}_{p_0}\mss{(t)} = \alpha_{p_0}\mss{(t)} \big|_{x = x_0}
			\end{aligned} }
			\ \Rightarrow\
			\begin{aligned}
				& d\alpha_{p_0} = [ \mss{ \mathbbm{1}_3\ w(\alpha) } ]\\
				& d\tilde{\alpha}_{p_0} = [ \mss{ \mathbbm{1}_3\ w(\alpha) } ] 
					\left[ \mss{ \begin{matrix}
							0\\
						\mathbbm{1}_3
					\end{matrix} } \right]
					= \Big[ \mss{e_2\ e_3\ w(\alpha)} \Big]
			\end{aligned}			
			\ \Rightarrow\ 
			\mss{ \begin{aligned}
				& \det( d\tilde{\alpha}_{p_0} ) = w\mss{(p_0)} \neq 0 \\
				& \boxed{ \exists \tilde{\alpha}^{-1} :\ V_{\alpha(p_0)} \subset S \rightarrow V_{p_0} \big|_{x = x_0} } 
					\hspace{5pt} \text{(IFT)} 
					\\ 
				& \forall p \in \alpha_{p_0}(t)
					,\ \underline{g\mss{(p)}} \equiv \pi_t \circ \tilde{\alpha}^{-1}_{p_0} \mss{(p)} = p_0 
			\end{aligned} }
			\\[10pt]
		% First Integral of w
		& \textit{\underline{(Local) First Integral of \(w\)}}:\ 
			f\mss{(p)} \ \big|\ \forall p \in \alpha_{p_0}\mss{(t)}
			,\ \boxed{ f\mss{(p)} = c},\ \boxed{ df_{p} \neq 0 }
			\hspace{15pt} \big( \ \begin{gathered}
				\text{\scriptsize \(f(p) = \) arcdist\(( p_0, \underline{g(p)} )\)}\\[-8pt]
				\text{\scriptsize along \(S|_{x=x_0}\)}
			\end{gathered} \ \big)
			\\
		%% if w != 0, there always exists a first integral function
		& \bullet\ \boxed{ w\mss{(p_0)} } \neq 0 \hspace{5pt} \text{\scriptsize(see above)}
			\ \Rightarrow\ \big( \exists V_{p_0} \subset S \big) 
			\hs\big(\hs \forall p \in V_{p_0},\ \underline{ \exists f\mss{(p)} } \hs\big)
			\\[-10pt]
		%% there exists coordinate curves for the trajectories of two lin. ind. w_1 and w_2 
		& \bullet\ \mss{ w_1\mss{(p_0)} }\ \underline{ \neq }\ \mss{ Aw_2\mss{(p_0)}  } 
			,\ \phi\mss{(p_0)} 
			= \left[ \mss{ \begin{matrix}
				f_1\mss{(p_0)} = u_0\\
				f_2\mss{(p_0)} = v_0
			\end{matrix} } \right]
			\ \Rightarrow\ \big[ d\phi_p \big] 
			\big[ \mss{ w_1\mss{(p_0)}\ w_2\mss{(p_0)} } \big]
			= \left[ \mss{ \arraycolsep=2pt \begin{matrix}
				0 & a\\
				b & 0
			\end{matrix} } \right]
			\hs\hs \underline{ \neq }\ 0
			\ \stackrel{ \text{\scriptsize(IFT)} }{\Rightarrow} \ 
			\mss{ \begin{gathered}
				\exists \underline{X} : \underline{ \phi^{-1} }\mss{(u_0,v_0)} = p_0\\
				\underline{\ast}\ \boxed{ \begin{aligned}
					& X\mss{(u_0, v)} \in \alpha_1\\ 
					& X\mss{(u, v_0)} \in \alpha_2
				\end{aligned} }
			\end{gathered} }
			\\[-10pt]
		& \bullet\ w_1 \equiv X_u,\ w_2 \equiv -\tfrac{X_u \cdot X_v}{X_u \cdot X_u} X_u + X_v 
			\ \Rightarrow\ \underline{ 
				\exists (w_1,w_2) 
				\big( w_2\mss{(p_0)} \cdot w_2\mss{(p_0)} = 0 ,\ \underline{ \exists \ast } \big)
			}
			\\[5pt]
		& \bullet\ \underline{K<0} \ \Rightarrow\ \underline{ \exists \alpha_1,\alpha_2 (k_n = 0) }
			\ \rightarrow\ \underline{ \exists (w_1, w_2) (\exists \ast) }
			\hspace{15pt}
			\ast\ k_1 \neq k_2,\ \exists (\alpha_1,\alpha_2)_{k_n} 
			\ \rightarrow\ \underline{ \exists (w_1, w_2) (\exists \ast) }
	\end{aligned}
\)

%----------------------------------------------------------------------------------------------------------------------------------
%
%
%
\newpage
\(\begin{aligned}
	% Direction/Ray Field
	& \underline{\textit{Direction/Ray/Line Field}}:\ \boxed{ r_w = c_{\neq0} \big( b\mss{(u,v)}, -a\mss{(u,v)} \big) }
		\ \rightarrow\ \tfrac{y'}{x'} = \tfrac{-a}{b}
		\\
	% Orthogonal Ray Field
	& \underline{\textit{Orthogonal Field to \(r\)}}:\ \boxed{ 
		\overline{r}_w \equiv r_{\overline{w}} : \ 
		\overline{w} \cdot w = (\overline{a} X_u + \overline{b} X_v ) \cdot (aX_u + bX_v) = 0 
		}
		\\[5pt]
	%% example of orthog ray field
	& \text{\scriptsize E.g.}:\ 
		\begin{gathered}
			X\mss{(q)} = (u,v,u^2 - v^2)\\
			\gamma\mss{(t)} : \mss{u^2 - v^2 = c} \rightarrow \tfrac{v'}{u'} = \tfrac{-u}{v}
		\end{gathered} 
		\ \Rightarrow\ 
		\begin{aligned}
			& w_\gamma = u'\mss{(t)}X_u + v'\mss{(t)}X_v \hs\stackrel{\rightarrow}{=}\hs v X_u - u X_v\\
			& \overline{w}_\gamma \cdot w_\gamma = \overline{a} v - \overline{b} u 
				= \underline{ u'\mss{(\hs \overline{t} \hs)} v - v'\mss{(\hs \overline{t} \hs)} u = 0 }
		\end{aligned}
		\ \Rightarrow\
		\begin{aligned}
			& \overline{\gamma}\mss{(\hs \overline{t} \hs)} :
				\underline{ u\mss{(\hs \overline{t} \hs)} v\mss{(\hs \overline{t} \hs)} = c }
				\\
			& X_c = \mss{ (u, \tfrac{c}{u}, u^2 - \tfrac{c^2}{u^2}) }
		\end{aligned}
\end{aligned}\)

\end{document}