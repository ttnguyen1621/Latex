\documentclass[12pt]{article}
\usepackage[utf8]{inputenc}
\usepackage[left=.75in, right=.75in, top=1in, bottom = 1in]{geometry}
\usepackage{amssymb, amsmath, amsfonts, mathtools}
\usepackage{array}
\usepackage{multirow}
\usepackage{cancel}
\newcommand{\tabitem}{~~\llap{\textbullet}~~}
\newcommand{\checkedbox}{\mbox{\ooalign{$\checkmark$\cr\hidewidth$\square$\hidewidth\cr}}} % checked box
\newcommand*{\dotP}{\boldsymbol \cdot}	% dot product
\newcommand*\calign[1]{\omit\hfil$\displaystyle#1$\hfil\ignorespaces} % centers the math cell in alignat
\newcommand*\lalign[1]{\omit$\displaystyle#1$\hfil\ignorespaces} % centers the math cell in alignat


% \title{Thermal}
% \author{ringoffire0 }
% \date{November 2022}

\begin{document}

% 1st Law Thermo
\section{Internal Energy and Introduction}

% Definition
\noindent
\(\begin{aligned}
    &\begin{gathered}
        \text{\underline{Internal Energy}:}\\
        \text{(1st Law)}
    \end{gathered} \ \ \ \boxed{ \begin{aligned}
        \Delta U &= Q + W\\[5pt]
        dU &= \delta Q + \delta W
    \end{aligned} }\\[5pt]
    &\boxed{dU = T\ dS - P\ dV + \sum \mu_i\ dN_i + \dots}
\end{aligned}\)
\hspace{18pt}
% Maxwell Identity
\(\begin{aligned}
    &\text{\underline{Maxwell Identity}:} \ \ \
        \frac{\partial}{\partial V} \left( \frac{\partial U}{\partial S} \right)
        = \frac{\partial}{\partial S} \left( \frac{\partial U}{\partial V} \right)\\[5pt]
    &\rightarrow\ \boxed{ \left( \frac{\partial T}{\partial V} \right)_{N,S} 
        = - \left( \frac{\partial P}{\partial S} \right)_{N,V} } 
        \indent \text{\scriptsize(other iden. with \(\mu\),\(N\))}
\end{aligned}\)

% \vspace{5pt}
\subsection{Initial Assumptions}
\noindent
\begin{minipage}[t]{.49\textwidth}
    % Equipartition Theorem
    \underline{Equipartition Theorem (ET)}: \\[10pt]
    \(\boxed{ U_\text{therm} = f \left( \tfrac{1}{2} N k_b T \right) }\) 
        \hspace{20pt}\(\scriptstyle f = x^2\ \text{DOF}\) \\[10pt]
    -Solid: 3 [vib.] = 3 trans. + 3 pot. \\
    -Mono Gas: 3 trans.\\
    -Dia Gas: \(\begin{gathered}[t]
            \begin{aligned}[t]
                    &\text{3 trans.} + \text{2 rot.} + \text{1 vib.}\\[-6pt]
                    &\hspace{37pt} \text{\scriptsize(1 K)} \hspace{17pt} \text{\scriptsize(1000 K)} 
                \end{aligned}\\[-5pt]
            \text{\scriptsize(Some DOF "freeze" at low temp)}
        \end{gathered}\)
\end{minipage}
\begin{minipage}[t]{.49\textwidth}
    % Ideal Gas Law
    \underline{Ideal Gas Law (IG)}: \hspace{5pt} \(\boxed{ PV = N k_b T }\)

    % Quasistatic Approx
    \vspace{20pt}\noindent
    \underline{Quasistatic Approx. (QS)}: \hspace{5pt} \begin{minipage}{3cm}
            \scriptsize 
            slow enough that \\ conditions are equal \\ throughout
        \end{minipage}\\[10pt]
    \(\bullet\ \boxed{ W_\text{cpr} = - \int P\ dV }\) \ \ \ \ QS Work (QW)
\end{minipage}

\vspace{15pt}\noindent
\begin{minipage}[t]{.46\textwidth}
    % Reversible
    \subsection{Reversible (RV)}
    {\scriptsize Infinitesmal small changes can be reversed back}\\
    {\scriptsize to the original state}
    
    \vspace{10pt}
    \(\begin{aligned}
        &\bullet\ \delta W = P\ dV = \delta W_\text{cpr} + \cancel{ \delta W_\text{other} }\\
        &\bullet\ \delta Q = T\ dS
    \end{aligned}\)

    % Isothermal Expansion
    \subsection{Isothermal (iT): \(\boxed{ \tfrac{dT}{dt}=0 }\)}
    
    % Internal Energy
    \(\rightarrow\ \begin{aligned}[t]
        &\text{ET}: \ \begin{aligned}[t]
                &\Delta U \sim \Delta T = 0\\
                &\boxed{dU \sim dT = 0}
            \end{aligned}\\[5pt]
        &\text{\scriptsize(Energy as heat moves to and from environment)}
    \end{aligned}\)

    % PV Equation
    \vspace{10pt}
    \(\bullet\ \begin{aligned}[t]
            &\text{IG}: \ \boxed{ P = \frac{N k_b T}{V} } 
                \hspace{10pt} \text{\scriptsize(Isotherm \(1/V\) level-curves)}\\[5pt]
            &\Rightarrow\ \boxed{ W = N k_b T \ln\left( \frac{V_i}{V_f} \right) = -Q }
        \end{aligned}\)    

    % Entropy
    \vspace{10pt}
    \(\bullet\ \boxed{ \Delta S = \int \frac{PdV}{T} = N k_b \ln \big( \tfrac{V_f}{V_i} \big) 
        = \int \frac{\delta Q}{T}}\)
\end{minipage}
\begin{minipage}[t]{.52\textwidth}
    % Free Expansion
    \subsection{Free Expansion}
    \(\boxed{ \Delta U = Q = W = 0 }\)\\[5pt]
    \(\bullet\) IG: \ \(\boxed{ 0 = \Delta T } \ , \ \ 
        \boxed{ \Delta S = \tfrac{{\scriptstyle \int} P dV}{T} = N k_b \ln (V_f / V_i) }\)\\[5pt]
    \(\bullet\) Else: See Throttling

    % Adiabatic Expansion
    \subsection{Adiabatic (aB): \(\boxed{ Q=0 }\)}
    
    % Internal Energy
    \(\rightarrow\ \begin{aligned}[t]
            &\Delta U = W\\
            &\boxed{ dU = \delta W }
        \end{aligned}\) \ \ {\scriptsize(Compressed fast enough so no heat escapes)}

    % PV Equation
    \vspace{10pt}
    \(\bullet\ \begin{aligned}[t]
            &\text{ET \& IG}: WRONG%\ \boxed{ dU = \tfrac{f N k}{2}\ dT 
                % = - P\ dV = \tfrac{- N k T}{V}\ dV )
                \\[5pt]
            &\Rightarrow\ V_\text{f}\ T_\text{f}^{f/2} 
                = \left( V_\text{i}\ T_\text{i}^{f/2} = \text{constant} \right)\\[5pt]
            &\Rightarrow\ \begin{aligned}[t]
                    T V^{\frac{2}{f}} &= \left( \frac{PV}{Nk_b} \right) V^{\frac{2}{f}}\\[5pt]
                    PV^{\left( 1+\frac{2}{f} \right)} &= \boxed{ PV^{\gamma} = \text{constant} }
                        = {\scriptstyle V_\text{i}^{\frac{2}{f}} N k_b T_\text{i}}\\
                    &\hspace{5pt} \text{\scriptsize(connects isotherm level-curves)}
                \end{aligned}
        \end{aligned}\)    

    % Entropy    
    \vspace{5pt}
    \(\bullet\) QS : \ \(\boxed{ \Delta S = 0  = \int \frac{\delta Q}{T} }\)\ \ {\scriptsize(See Isentropic)}
\end{minipage}

%-----------------------------------------------------------------------------------------------------------------------------
%-----------------------------------------------------------------------------------------------------------------------------
%-----------------------------------------------------------------------------------------------------------------------------
%-----------------------------------------------------------------------------------------------------------------------------
\newpage
% Enthalpy and Heat Capacity
\section{Enthalpy and Heat Capacity}

% Definition
\noindent
\(\begin{aligned}[t]
    &\text{\underline{Enthalpy}:}\ \ \ \boxed{ H \equiv U + PV }\\[5pt]
    &\boxed{ \begin{aligned}[t]
        dH &= dU + P\ dV + V\ dP\\[5pt]
        &= T\ dS + V\ dP + \sum \mu_i\ dN_i + \dots
    \end{aligned} }
\end{aligned}\)
\hspace{1.1cm}
% Maxwell Identity
\(\begin{aligned}[t]
    &\text{\underline{Maxwell Identity}:}\ \ \ 
        \frac{\partial}{\partial P} \left( \frac{\partial H}{\partial S} \right)
        = \frac{\partial}{\partial S} \left( \frac{\partial H}{\partial P} \right)\\[10pt]
    &\rightarrow\ \boxed{ \left( \frac{\partial T}{\partial P} \right)_{N,S} 
        = \left( \frac{\partial V}{\partial S} \right)_{N,P} }
        \indent \text{\scriptsize(other iden. with \(\mu\),\(N\))}
\end{aligned}\)

% Heat Capacities
\vspace{15pt}\noindent
\underline{Heat Capacity}: \ \(\boxed{ C = \frac{Q}{\Delta T} = \frac{\Delta U-W}{\Delta T} }\)
    \indent\indent \(\begin{aligned}
        &\text{\underline{Specific Heat Capacity}}: \ c \equiv C/m\\[5pt]
        &\text{\underline{3rd Law}}: \ \boxed{ \lim_{T \rightarrow 0}\ S(T) \rightarrow 0 
            = C(T) \ \Rightarrow \ T \nless 0}
    \end{aligned}\)

% iNV Heat Capacity
\noindent
\subsection{Isochoric (iV): \(\boxed{ \tfrac{dV}{dt} = 0 }\)
    \ \ \& \ ``Isomolic'' (iN): \(\boxed{ \tfrac{dN}{dt} = 0 }\)}
\vspace{5pt}\noindent
\begin{minipage}[t]{.49\textwidth}
    % U Relation
    \(\rightarrow \ \begin{aligned}[t]
        \Delta U &= Q + W_\text{other} + \cancel{W_\text{cpr}} \ \ \ \ \text{\scriptsize(QW)} \\[5pt]
        \Aboxed{ dU &= \delta Q + \delta W_\text{other}}
    \end{aligned}\)

    % U Capacity
    \vspace{15pt}
    \(\bullet \ \begin{aligned}[t]
        &\text{IG Energy Capacity}:\\[10pt]
        &\begin{aligned}[t]
            C_V &= \tfrac{Q}{\Delta T} = \tfrac{\Delta U}{\Delta T} 
                = \left( \tfrac{\partial U}{\partial T} \right)_{N,V} \\[5pt]
            &= \partial_t \left( \tfrac{f}{2} nRT \right)
                \hspace{20pt} \text{\scriptsize(normal temp. ET)}\\[5pt]
            \Aboxed{ C_V &= \left( \tfrac{\partial U}{\partial T} \right)_{N,V} = \tfrac{f}{2} nR }
        \end{aligned}
    \end{aligned}\)    
\end{minipage}
\begin{minipage}[t]{.49\textwidth}
    % U Change
    \vspace{1pt}
    \(\bullet \ \boxed{ \Delta U = Q = \tfrac{f}{2} nRT }\)

    % S Change
    \vspace{15pt}
    \(\bullet \ \boxed{ \begin{aligned}[t]
        \Delta S &= \int \frac{\delta Q}{T} = \int \frac{C_V\ dT}{T}\\[5pt]
        &= \tfrac{f}{2} nR \ln{ \left( \frac{T_f}{T_i} \right) }\\[5pt]
    \end{aligned} }\) 
    \hspace{5pt} \begin{minipage}[t]{75pt}
        \(\begin{gathered}[t]
            \text{(3rd Law)}\\[5pt]
            \lim_{T \rightarrow 0}\ S(T) = 0\\
            \text{\scriptsize(ignoring residual)}
        \end{gathered}\)
    \end{minipage} 
\end{minipage}

% iPN Heat Capacity
\vspace{5pt}\noindent
\subsection{iN \& Isobaric (iP): \(\boxed{ \tfrac{dP}{dt} = 0 }\)}
\begin{minipage}[t]{.49\textwidth}
    % H relation
    \(\rightarrow \ \begin{aligned}[t]
        \Delta H &= \Delta U + P \Delta V\\[5pt]
        &= Q + W_\text{other} + \cancel{ W_\text{cpr} } + \cancel{ P \Delta V } 
            \ \ \ \text{\scriptsize(QW)}\\[5pt]
        \Aboxed{ dH &= \delta Q + \delta W_\text{other} }
    \end{aligned}\)    

    % H Capacity
    \vspace{15pt}
    \(\bullet \ \begin{aligned}[t]
        &\text{IG Enthalpy Capacity}:\\[10pt]
        &\begin{aligned}[t]
            C_P &= \tfrac{Q}{\Delta T} = \tfrac{\Delta U + P \Delta V}{\Delta T} 
                = \left( \tfrac{\partial H}{\partial T} \right)_{N,P}\\[5pt]
            &= \left( \tfrac{\partial U}{\partial T} \right)_{N,P} 
                + P \left( \tfrac{\partial V}{\partial T} \right)_{N,P}\\[5pt]
            &= C_V + P\ \partial_t \left( \tfrac{nRT}{P} \right)
                \hspace{20pt} \text{\scriptsize(normal temp. ET)}\\[5pt]
            \Aboxed{ C_P &= \left( \tfrac{\partial H}{\partial T} \right)_{N,P} 
                = \tfrac{f}{2} nR + nR \ =\ C_V + nR }
        \end{aligned}
    \end{aligned}\) 
\end{minipage}
\begin{minipage}[t]{.49\textwidth}
    % Q Change
    \vspace{1pt}
    \(\bullet \ \boxed{ \Delta H = Q = (C_V + nR) T }\)
    
    % S Change
    \vspace{15pt}
    \(\bullet \ \boxed{ \begin{aligned}[t]
            \Delta S &= \int \frac{\delta Q}{T} = \int \frac{C_P\ dT}{T} \\[5pt]
            &= (C_V + nR) \ln{ \left( \frac{T_f}{T_i} \right) } 
        \end{aligned} } 
    \) 
    \hspace{5pt} \begin{minipage}[t]{75pt}
        \(\begin{gathered}[t]
            \text{(3rd Law)}\\[5pt]
            \lim_{T \rightarrow 0}\ S(T) = 0\\
            \text{\scriptsize(ignoring residual)}
        \end{gathered}\)
    \end{minipage}

    % Latent Heat
    \vspace{15pt}
    \(\bullet\ \begin{gathered}[t]
        \text{\underline{Latent Heat of Transformation}}:\\
        \text{\scriptsize(phase change at constant 1 atm)}
    \end{gathered} \ \ L \equiv Q/m\) 
\end{minipage}

\par\vspace{20pt}\noindent

% THe rates of processes Sec. 1.7

%---------------------------------------------------------------------------------------------------------------------------
%---------------------------------------------------------------------------------------------------------------------------
%---------------------------------------------------------------------------------------------------------------------------
%---------------------------------------------------------------------------------------------------------------------------
\newpage
% Entropy
\section{Entropy and Engine Efficiency}

% Definition and 2nd Law
\(\begin{aligned}
    &\text{\underline{Entropy}}:&& \boxed{ S \ \equiv\ k_b \ln{\Omega} }\\[5pt]
    &\text{\underline{2nd Law}}:&& \boxed{ \text{System Action s.t. } \Omega \uparrow \ \text{\&} \ S \uparrow }
\end{aligned}\)
\indent\indent\(\boxed{ dS = \frac{1}{T} \ dU + \frac{P}{T} \ dV - \sum \frac{\mu_i}{T} \ dN_i + \dots}\) 

% S for U and T
\vspace{10pt}\noindent
\begin{minipage}[t]{.49\textwidth}
    \setlength{\parindent}{.5cm}
    \noindent
    % Temperature Definition from S
    \(\displaystyle \bullet\ \ \boxed{ \left( \frac{\partial S}{\partial U} \right)_{V,N} \equiv \frac{1}{T} }\)

    \vspace{15pt}
    \indent\(\begin{aligned}[t]
        &0 < \left( \frac{1}{T_1} = \frac{\Delta S_1}{\Delta U_1} \right)
            < \left( \frac{1}{T_2} = \frac{\Delta S_2}{\Delta U_2} \right)\\[10pt]
        &\ \ \Rightarrow \ \left( \Delta U_1 < 0 < \Delta U_2 \right)\\[10pt]
        &\ \ \Rightarrow \ \left( U_1 > U_2 \right) 
            \ \ \Leftrightarrow \ \ \boxed{ \text{``} U_1 \rightarrow U_2 \text{"} }\\[10pt]
        &\ \ \Rightarrow \ \begin{aligned}[t]
            &\left( T_1 > T_2 \right) 
                \ \ \Leftrightarrow \ \ \boxed{ \text{``} T_1 \rightarrow T_2 \text{"} }\\[5pt]
            &\text{\scriptsize(Ener. flows from high temp. to low temp.)}
        \end{aligned}
    \end{aligned}\)    
\end{minipage} 
% S for V and P
\begin{minipage}[t]{.49\textwidth}
    \setlength{\parindent}{.5cm}
    \noindent
    % Pressure def. from S
    \(\displaystyle \bullet\ \ \boxed{ T \left( \frac{\partial S}{\partial V} \right)_{U,N} = P }\)

    \vspace{15pt}
    \indent\(\begin{aligned}[t]
        &0 < \left( P_1 \sim \frac{\Delta S_1}{\Delta V_1} \right)
            < \left( P_2 \sim \frac{\Delta S_2}{\Delta V_2} \right)\\[10pt]
        &\ \ \Rightarrow \ \left( \Delta V_1 < 0 < \Delta V_2 \right)\\[10pt]
        &\ \ \Rightarrow \ \left( V_1 > V_2 \right) 
            \ \ \Leftrightarrow \ \ \boxed{ \text{``} V_1 \rightarrow V_2 \text{"} }\\[10pt]
        &\ \ \Rightarrow \ \begin{aligned}[t]
            &\left( P_1 < P_2 \right) 
                \ \ \Leftrightarrow \ \ \boxed{ \text{``} P_1 \rightarrow P_2 \text{"} }\\[5pt]
            &\text{\scriptsize(Vol. flows from low pres. to high pres.)}
        \end{aligned}
    \end{aligned}\)      
\end{minipage}

% Chemical Potential def. from S
\vspace{15pt}\noindent
\(\bullet \ \) \underline{Chemical Potential}: \ \(\boxed{ \mu 
    \equiv - T \left( \frac{\Delta S}{\Delta N} \right)_{U,V} 
    = \left( \frac{\Delta U}{\Delta N} \right)_{S,V}} 
    \indent \begin{minipage}{6.5cm}
        \scriptsize
        (usually negative, but particle may include kinetic energy that converts
            to potential, for it to be added into the system)
    \end{minipage}\)

% S for N and mu
\vspace{15pt}
\(\begin{aligned}[t]
    &\mu_2 < \mu_1 < 0 < \left( \Vert \mu_1 \Vert \sim \frac{\Delta S_1}{\Delta N_1} \right)
        < \left( \frac{\Delta S_2}{\Delta N_2} \sim \Vert \mu_2 \Vert \right)\\[10pt]
    &\Rightarrow \ \left( \Delta N_1 < 0 < \Delta N_2 \right)\\[10pt]
    &\Rightarrow \ \left( N_1 > N_2 \right) 
        \ \ \Leftrightarrow \ \ \boxed{ \text{"} N_1 \rightarrow N_2 \text{"} }\\[10pt]
    &\Rightarrow \ \left( \mu_1 > \mu_2 \right) 
        \ \ \Leftrightarrow \ \ \boxed{ \text{"} \mu_1 \rightarrow \mu_2 \text{"} }
        \indent \text{\scriptsize(Part. flows from high \(\mu\) to low \(\mu\))}
\end{aligned}\)   

\vspace{20pt}
\noindent
% Quasistatic and Constant Particle Number
\begin{minipage}[t]{.48\textwidth}
    \setlength{\parindent}{.5cm}

    \subsection{QW}
    \vspace{10pt}
    \(\begin{aligned}
        &\rightarrow \ \begin{aligned}[t]
            T\ dS &= dU + P\ dV - \mu\ dN + \dots\\[5pt]
            \Aboxed{ dS &= \frac{\delta Q}{T} + \frac{\delta W_\text{other}}{T} - \frac{\mu\ dN}{T} + \dots}
        \end{aligned}\\[10pt]
        &\bullet \ \textbf{(iN)} \ \Rightarrow \ \begin{gathered}[t]
            \boxed{ \delta Q = T\ dS }\\
            \text{\scriptsize(if no other work)}\\
            \text{\scriptsize(means RV)}
        \end{gathered}
    \end{aligned}\)    
\end{minipage}
% Isentropic
\begin{minipage}[t]{.48\textwidth}
    \setlength{\parindent}{.5cm}

    \subsection{Isentropic (iS): \ \(\boxed{ \Delta S = 0 }\)}
    \vspace{10pt}
    \(\bullet \ \ \big( \textbf{QS, aB, iN} \big) \ \Rightarrow\ \)
    \(\begin{gathered}[t]
        \boxed{\delta Q = T\ dS = 0 = \Delta S}\\
        \text{\scriptsize(if no other work)}\\
        \text{\scriptsize(means RV)}
    \end{gathered}\)
\end{minipage}

%---------------------------------------------------------------------------------------------------------------------------
\newpage
% Engines
\subsection{Engines and Efficiency}

% Engine Efficiency Limit
\noindent
\begin{minipage}[t]{.37\textwidth}
    \setlength{\parindent}{.5cm}
    \noindent
    \underline{Engine Efficiency, \(e\)}: \\[10pt]
     \(\frac{Q_c}{T_c} \geq \frac{Q_h}{T_h} \ \ \rightarrow \)\\[10pt]
     \(\begin{aligned}[t]
        &\boxed{ e = \frac{W}{Q_h} = \frac{Q_h - Q_c}{Q_h} = 1 - \frac{Q_c}{Q_h}}\\[5pt]
        &\boxed{ e \leq 1 - \frac{T_c}{T_h} = \frac{T_h - T_c}{T_h} }
    \end{aligned}\)    
\end{minipage}
% Carnot Cycle For Engine
\begin{minipage}[t]{.27\textwidth}
    \scriptsize
    \underline{Carnot Engine Cycle}:\\[5pt]
    1. iT Expansion at \(\sim_< T_h\) to \\ Absorb \(Q_h\)\\[5pt]
    2. aB Expansion to Reduce \\ Temp to \(\sim_> T_c\) and Expel \(W\)\\[5pt]
    3. iT Compression to Expel \(Q_c\)\\[5pt]
    4. aB Compression to Increase \\ Temp to \(\sim_< T_h\)\\[5pt]
    \(\bullet \ \boxed{ \int P\ dV = W = Q = \Delta T \cdot \Delta S }\)
\end{minipage}
\hspace{9pt}
% Carnot Cycle For Fridge
\begin{minipage}[t]{.3\textwidth}
    \scriptsize
    \underline{Carnot Fridge Cycle}:\\[5pt]
    1. iT Compression at \(\sim_> T_h\) to \\ Expel \(Q_h\)\\[5pt]
    2. aB Expansion Absorbing \(W\) \\ to Reduce Temp to \(\sim_< T_c\)\\[5pt]
    3. iT Expansion to Absorb \(Q_c\)\\[5pt]
    4. aB Compression to Increase \\ Temp to \(\sim_< T_h\)
\end{minipage}

\vspace{20pt}\noindent
% Refrigerator Efficiency
\begin{minipage}[t]{.52\textwidth}
    \setlength{\parindent}{.5cm}
    \noindent
    \underline{Refrigerator Coefficient of Performance (COP)}: \\[10pt]
    \(\frac{Q_h}{T_h} \geq \frac{Q_c}{T_c} \ \ \rightarrow \)\\[10pt]
    \(\begin{aligned}[t]
            &\boxed{ \text{COP} \ = \ \frac{Q_c}{W} = \frac{Q_c}{Q_h - Q_c} = \frac{1}{Q_h / Q_c - 1} }\\[5pt]
            &\boxed{ \text{COP} \ \leq \ \frac{1}{T_h / T_c - 1} = \frac{T_c}{T_h - T_c} }\\[5pt]
            &\big( \text{\scriptsize \(W + Q_c = W + [\text{COP}] \cdot W\)} \big)
        \end{aligned}\)    
\end{minipage}
% Real Fridge Example
\begin{minipage}[t]{.46\textwidth}
    \scriptsize 
    \underline{Real Fridge:}\\[5pt]
    1. Reduce High Pressure Liquid to Low Pressure and Temp Liquid/Gas By Throttling 
        {\scriptsize (see below)}\\[5pt]    
    2. Absorb \(Q_c\) In Evaporator to Complete Phase Change \\
    (Maybe Increase Temp A Little)\\[5pt]
    3. Adiabatically Compress Low Pressure Gas To High Pressure and Temp in Compressor\\[5pt]
    4. Cool Down Gas To Liquid in Condenser (Pipes to Outside Heat Reservoir)
\end{minipage}

% Joule- Thompson Effect (Throttling)
\vspace{15pt}\noindent
\begin{minipage}{.98\textwidth}
    \scriptsize
    {\centering --- Joule-Thompson Effect (Throttling) --- \par} 

    \vspace{5pt} \(\bullet \ H\) is conserved \\[5pt]
    \(\bullet \ \)``The force between any two molecules is weakly attractive at long distances and
    strongly repulsive at short distances. Under most (though not all) conditions the
    attraction dominates; then \(U_\text{potential}\) is negative, but becomes less negative as the
    pressure drops and the distance between molecules increases. To compensate for
    the increase in potential energy, the kinetic energy generally drops, and the fluid
    cools as desired.'' - Schroeder, P.140
    
    ``...Starting from room temperature, the Hampson-Linde cycle can be used to 
    liquefy any gas except hydrogen or helium. These gases, when throttled starting at
    room temperature and any pressure, actually become hotter. This happens because
    the attractive interactions between the molecules are very weak; at high temperatures 
    the molecules are moving too fast to notice much attraction, but they still
    suffer hard collisions during which there is a large positive potential energy. When
    the gas expands the collisions occur less frequently, so the average potential energy
    decreases and the average kinetic energy increases.
    To liquefy hydrogen or helium, it is therefore necessary to first cool the gas
    well below room temperature, slowing down the molecules until attraction becomes
    more important than repulsion.''  
\end{minipage}

%---------------------------------------------------------------------------------------------------------------------------
\newpage
% Shannon Entropy
\subsection{Shannon Entropy: \(\boxed{S = -k_b \sum_s P(s) \ln{P(s)}}\)}
\vspace{5pt}\noindent
\(\begin{aligned}
    S \ &=\ -k \sum_s P(s) \ln{P(s)}
        \ =\ -k \sum_s \frac{ e^{- \beta E(s)} }{Z} \left( \ln{e^{- \beta E(s)}} - \ln{Z} \right)
        \indent\indent \text{\scriptsize(see Canonical Ensemble)}\\
    \ &=\ k \beta \sum_s E(s) \frac{ e^{- \beta E(s)} }{Z} \ +\ k \ln{Z} \sum_s \frac{ e^{- \beta E(s)} }{Z}\\
    \ &=\ \frac{ \overline{E} }{T} + k \ln{Z} = \frac{U-F}{T} \ \ \checkedbox
        \indent \text{\scriptsize(see Helmholtz Free Energy)}
\end{aligned}\)

\vspace{20pt}\noindent
\underline{Form Motivation}\\[10pt]
\(\left. \begin{aligned}
    &\begin{aligned}
        &\bullet \ \text{Extensive}:&   &S_{AB} = S_A + S_B\\[5pt]
        &\bullet \ \text{Probability}:& &P_{A_B} = P_A P_B
    \end{aligned} \ \Leftarrow \ S \ \propto\ \sum_s \ln{P(s)}\\[5pt]
    &\bullet \ \text{Positive}: \indent \ S \ \sim\ - \sum_s \ln{P(s)}\\[5pt]
    &\bullet \ \text{Bounded}: \indent S \ \sim\ - \sum_s P(s) \ln{P(s)}
\end{aligned} \right\} \ \ \ 
\begin{aligned}
    S_{AB} \ &\propto - \sum P_{AB} \ln{P_{AB}} \\[5pt]
    &= - \sum P_{A} \ln{P_{A}} - \sum P_{B} \ln{P_{B}}\\[5pt]
    &= S_A + S_B
\end{aligned}\)


%---------------------------------------------------------------------------------------------------------------------------
%---------------------------------------------------------------------------------------------------------------------------
%---------------------------------------------------------------------------------------------------------------------------
%---------------------------------------------------------------------------------------------------------------------------
% Microcanonical (Examples for Entropy)
\newpage
\section{Microcanonical Ensemble (NVE*) \(\hspace{30pt} {\scriptstyle^*(E \text{ means } U \text{ here})}\)}
\begin{minipage}[t]{.58\textwidth}
    % Multiplicity (General Partition Func.)
    \underline{Generalized Partition Function}: \\[5pt]
    1.) \ \(\Omega\) \ \ {\scriptsize(multiplicity)}

    % Characteristic State Function
    \vspace{10pt}
    \underline{Characteristic State Function, \(TS\)}: \\[7pt]
    \(\begin{aligned}
        &1.)\ \ [S = k \ln{\Omega}] \ \leftrightarrow \ [\beta TS = \ln{\Omega}]
            \ \leftrightarrow \ \Big[ \Omega = e^{\beta TS} \Big]\\[2pt]
        &2.)\ \ dS = \tfrac{1}{T} dU + \tfrac{P}{T} dV - \tfrac{\mu}{T} dN\\[5pt]
        &3.)\ \ \Delta S = {\textstyle\int} \tfrac{\delta Q}{T}
    \end{aligned}\)
\end{minipage}
\begin{minipage}[t]{.40\textwidth}
    % U Energy
    \underline{Total [Relevant] Energy}: \\[5pt]
    \(\begin{aligned}
        &1.)\ \left[ \tfrac{\partial S}{\partial U} = \tfrac{1}{T} \right] \ \ \rightarrow \ \ 
            \left[ \tfrac{\partial U}{\partial q_i} = T \tfrac{\partial S}{\partial q_i} \right] \\[1pt]
        &2.)\ \ \beta \tfrac{\partial U}{\partial \beta} = \tfrac{\partial (\beta TS)}{\partial \beta} 
            = \tfrac{\partial \ln{\Omega}}{\partial \beta} 
            = \tfrac{1}{\Omega} \tfrac{\partial \Omega}{\partial \beta}
    \end{aligned}\)

    % Stirling's Approx
    \vspace{12pt}
    \underline{Stirling's Approximation}: \\[7pt]
    \(\begin{aligned}[t]
        &1.)\ \ N! \ \approx\ \sqrt{2\pi N} \left( \tfrac{N}{e} \right)^N\\[2pt]
        &2.)\ \ \ln{N!} \ \approx\ N \ln{N} - N
    \end{aligned}\)
\end{minipage}

% Line divider
\vspace{10pt}\noindent
\rule{1\textwidth}{.5pt}

% Monoatomic Ideal Gas
\vspace{-10pt}
\subsection{Monoatomic Ideal Gas}
\begin{minipage}[t]{.49\textwidth}
    \(\bullet \ \Omega = f(N) V^N U^{3N/2}\)\\[10pt]
    \(\bullet \ \begin{aligned}[t]
        &\text{\underline{Sackur-Tetrode}}\\
        &\begin{aligned}[t]
            S/k &= N \left[ \ln{\left( \frac{V}{N} 
                \left( \frac{4\pi m U}{3N h^2} \right)^{3/2} \right)} + \frac{5}{2} \right]\\[5pt]
            &= N \ln{V} + N \ln{U^{3/2} + f(N)}
        \end{aligned}
    \end{aligned}\)\\[10pt]    
    \(\bullet \ 1/T = \left( \frac{\partial S}{\partial U} \right)_{N,V}
        \ \ \rightarrow \ \ \boxed{ U = \tfrac{3}{2} N k_b T }\)\\[5pt]
    % P-S IG Law
    \(\bullet \ P/T = \left( \frac{\partial S}{\partial V} \right)_{U,N} \ \rightarrow \ \ \boxed{ PV = Nk_b T }\)
\end{minipage}
\begin{minipage}[t]{.49\textwidth}
    % Free Expansion
    \underline{Free Expansion, Isothermic Expansion}\\[5pt]
    \(\Delta S = \int \tfrac{P dV}{T} = N k_b \ln{ \tfrac{V_f}{V_i} }\)

    % Gas Mixing Entropy
    \vspace{15pt}\noindent
    \underline{Isothermic Mixing Two Different Gases}\\[5pt]
    \(\Delta S = 2 \int \tfrac{P dV}{T} = 2 N k_b \ln{ \tfrac{2V_i}{V_i} } = 2N k_b \ln{2}\)

    % Adiabatic
    \vspace{15pt}\noindent
    \underline{Adiabatic \& QS}\\[5pt]
    \(\Delta S = \int \tfrac{0}{T} = 0\)
\end{minipage}

% Einstein Solid/Oscillator
\subsection{Einstein Solid/Oscillator/Vibration \ \
    \(\bullet \ \boxed{ \text{\scriptsize\(\left(\frac{\partial U}{\partial q} = \hbar \omega\right)\)} }\)}

{\setlength{\arraycolsep}{5pt}
    \(\begin{array}{l c c c c l}
        \bullet & \left[ \Omega(N,q) = \tfrac{(q+N-1)!}{q! (N-1)!} \right] 
            &\approx& \left( \tfrac{q+N}{q} \right)^q \left( \tfrac{q+N}{N} \right)^N
            &\approx& \left[ \left( \frac{eq}{N} \right)^N e^{N^2/q} \ \approx \ 
                \left( \frac{eq}{N} \right)^N \right] \indent {\scriptstyle q \gg N}\\[15pt]
        \bullet & S/k 
            &\approx& q \ln \left( \tfrac{q+N}{q} \right) + N \ln \left( \tfrac{q+N}{N} \right)
            &\approx& N + N \ln{\left( \frac{q}{N} \right)} \indent {\scriptstyle q \gg N}\\[10pt]
        \bullet & \left( T \tfrac{\partial S}{\partial q} = \tfrac{\partial U}{\partial q} \right)_{N,V}
            &\rightarrow& \boxed{ U = \frac{N \hbar \omega}{e^{\hbar \omega / kT} - 1} } 
            &\rightarrow& \boxed{ C_V = N k_b } \indent {\scriptstyle q \gg N,\ kT \gg \hbar \omega}
    \end{array}\)
}

% 2 State Paramagnet    
\subsection{Two-State Paramagnet}
\begin{minipage}[t]{.58\textwidth}
    \(\bullet \ \Omega(N, N_\uparrow) = \frac{N!}{N_\uparrow! N_\downarrow!}\)\\[10pt]
    \(\bullet \ S/k \ \approx \ N \ln{N} - N_\uparrow \ln{N_\uparrow} 
        - (N-N_\uparrow) \ln{(N-N_\uparrow)}\)\\[10pt]
    \(\bullet \ \begin{aligned}[t]
        \boxed{ \big( U = -MB \big) } &= -\mu B N_\uparrow + \mu B N_\downarrow = \mu B (N - 2N_\uparrow)
    \end{aligned}\)    
\end{minipage}
\begin{minipage}[t]{.4\textwidth}
    \(\bullet \ \begin{aligned}[t]
        &\left( T \tfrac{\partial S}{\partial N_\uparrow} 
            = \tfrac{\partial U}{\partial N_\uparrow} \right)_{N,V,B}\\[5pt]
        &\boxed{ M = N\mu \tanh{ \frac{\mu B}{k T} } }
            \hspace{.75cm} {\scriptstyle(\tanh{x}\ \approx\ x\ \approx\ 0)}
    \end{aligned}\)
\end{minipage}

%--------------------------------------------------------------------------------------------------------------------------------
%--------------------------------------------------------------------------------------------------------------------------------
%--------------------------------------------------------------------------------------------------------------------------------
%--------------------------------------------------------------------------------------------------------------------------------
% Gibbs Free Energy
\newpage
\section{Gibbs Free Energy}
\vspace{-10pt}
\indent{\scriptsize(see Helmholtz initally for easier comparison first)}

% Definition
\vspace{15pt}\noindent
\(\begin{aligned}[t]
    &\text{\underline{Gibbs Free Energy}}:\ \boxed{ \begin{aligned}
            G &\equiv H - TS\\[5pt]
            &= U + PV - TS
        \end{aligned} }\\[5pt]
    &\boxed{ \begin{aligned}
            dG &= dH - T\ dS - S\ dT \\[5pt]
            &= -S\ dT + V\ dP + \sum \mu_i\ dN_i + \dots
        \end{aligned} }
\end{aligned}\)
\hspace{20pt}
% Maxwell Identity
\(\begin{aligned}[t]
    &\text{\underline{Maxwell Identity}:} \ \ \
        \frac{\partial}{\partial V} \left( \frac{\partial H}{\partial T} \right)
        = \frac{\partial}{\partial T} \left( \frac{\partial H}{\partial V} \right)\\[5pt]
    &\rightarrow\ \boxed{ - \left( \frac{\partial S}{\partial P} \right)_{N,T} 
        = \left( \frac{\partial V}{\partial T} \right)_{N,P} } 
        \indent \text{\scriptsize(other iden. with \(\mu\),\(N\))}
\end{aligned}\)

% Chemical Potential def from G
\vspace{20pt}\noindent
\(\bullet \ \begin{aligned}[t]
    &\boxed{ \mu_i = \left( \frac{\partial G}{\partial N_i} \right)_{T,P} }
    \ \ \Rightarrow \ \ \boxed{G = \sum \mu_i N_i} 
    \indent \text{\scriptsize(\underline{\(T\) and \(P\) are intensive, so \(\mu \neq \mu(N)\)})}\\[10pt]
    &\text{IG}: \ \frac{\partial \mu}{\partial P} = \frac{1}{N} \frac{\partial G}{\partial P} 
        = \frac{V}{N} = \frac{k_b T}{P} \ \ \Rightarrow \ \ 
        \boxed{ \mu(T, P) = \mu^\circ (T) + k_b T \ln{\left( \tfrac{P}{P^\circ} \right)} }
\end{aligned}\)

% Isothermic + Isobaric Case
\vspace{10pt}
\subsection{iPT}
\(\begin{aligned}
    \Aboxed{ \Delta G &= \Delta H + T \Delta S }\\[5pt]
    &= \Delta U + P \Delta V - T \Delta S\\[5pt]
    &= Q + \cancel{ W_\text{cpr} } + W_\text{other} + \cancel{ P \Delta V } - T \Delta S\\[5pt]
    &= W_\text{other} - ( W_\text{other} - \sum \mu_i \Delta N_i + \dots )\\
    &= \sum \mu_i \Delta N_i - \dots
\end{aligned}
\hspace{1cm} \rightarrow \hspace{1cm} \boxed{ \begin{aligned}
    dG \ \ &\leq \ \ \delta W_\text{other}\\
    &\text{\scriptsize(and)}\\
    dG \ \ &\leq \ \ \sum \mu_i\ dN_i
\end{aligned} } \)

% Isobaric, Isothermal case
\vspace{5pt}
\subsection{iNPT}
\(\begin{aligned}
    dS_\text{total} &= dS + dS_R = dS + \frac{dU_R}{T_R} \\[5pt]
    &= dS - \frac{dU}{T} - \frac{P\ dV}{T}= -\frac{1}{T} (dU + P\ dV- T\ dS)\\[5pt]
    \Aboxed{ dS_\text{total}\uparrow &= - \frac{dG\downarrow}{T} } 
    \indent \indent G\downarrow = \boxed{U\downarrow} + \boxed{PV\downarrow} - TS\uparrow
\end{aligned}\)

%-----------------------------------------------------------------------------------------------------------------------------------
% Examples: Electrolysis + Lead Battery
\newpage
\subsection{Example: Electrolysis Chemical Reaction}
\vspace{10pt}
\(\begin{gathered}
    {\setlength{\tabcolsep}{2pt}
        \begin{tabular}{c|c c c c c|l c l}
            %empty%
                &\(\text{H}_2\text{O}\)
                &\(\longrightarrow\)& \(\text{H}_2\)
                &+&\(\tfrac{1}{2} \text{O}_2\)\\[5pt]
            \hline &&&&&\\
            S 
                &(70 J/K)
                &\(\longrightarrow\)& (131 J/K)
                &+&\(\tfrac{1}{2}\)(205 J/K)
                &\ \(\Delta S\)
                &=& 163 J/K\\[10pt]
            \ \(\Delta_f H\) \
                &(-286 J)
                &\(\longrightarrow\)& (0 J)
                &+&\(\tfrac{1}{2}\)(0 J)
                &\ \(\Delta H\) 
                &=& 286 J\\
            &&&&&
        \end{tabular} }\\[10pt]
    % Energy Fill in the Blanks
    \begin{alignedat}{3}
        \calign{\Delta G} 
            &+{}& \calign{T \Delta S}   
            &={}& \calign{\Delta U} 
            &+{}& \calign{P \Delta V}\\
        \calign{(237\ \text{kJ})}
            &+{}& \calign{(298\ \text{K})(163\ \text{J/K})}
            &={}& \calign{(282\ \text{kJ})}    
            &+{}& \calign{(4\ \text{kJ})}
    \end{alignedat}
\end{gathered}\)

% Clausius-Clapeyron Relation
\vspace{5pt}\noindent
\subsection{Clausius-Clapeyron Relation (Phase Boundaries)}
\vspace{5pt}
\(\left.\begin{array}{l c}
    \text{At Phase Boundary:}  & G_g = G_l \\[5pt]
    \text{Remain on Boundary:} & dG_g = dG_l
\end{array}\right\}
\hspace{10pt} \Rightarrow \hspace{5pt}
\begin{aligned}[t]
    &-S_l\ dT + V_l\ dP = -S_g\ dT + V_g\ dP\\[10pt]
    &\Rightarrow \boxed{ \frac{dP}{dT} = \frac{S_g - S_l}{V_g - V_l} 
        = \left( \frac{\Delta S}{\Delta V} \right)_{gl} = \left( \frac{L}{T \Delta V} \right)_{gl} }
\end{aligned}\)

% van der Waals Formula
\subsection{van der Waals Model (Liquid-Gas)}
\vspace{5pt}
\begin{minipage}[t]{.42\textwidth}
    \(\begin{aligned}[t]
        &\begin{aligned}[t]
                P &= P_\text{no attraction} + P_\text{vdW}\\[7pt]
                &= \frac{NkT}{V - Nb} - \frac{\partial}{\partial V} \big( U_\text{pot. e. attraction} \big)\\[5pt]
                &= \frac{NkT}{V - Nb} - \frac{\partial}{\partial V}  
                    \begin{aligned}[t]
                        \big( - & a \left[ \tfrac{N}{V} \right] \cdot N \big) \\[-4pt]
                        &\text{\scriptsize(\(\uparrow\) dens. = \(\uparrow\) attract.)}
                    \end{aligned} 
                    \\[-1pt]
                &= \frac{NkT}{V - Nb} - \frac{aN^2}{V^2}
            \end{aligned}\\[3pt]
        &\Rightarrow \boxed{ \left( P + \frac{aN^2}{V^2} \right) \left( V - Nb \right) = Nk_b T }\\[20pt]
        &\left( \tfrac{\partial G}{\partial P} \right)_{N,T} = V \ \Rightarrow \ 
            \boxed{ \left( \tfrac{\partial G}{\partial V} = V \tfrac{\partial P}{\partial V} \right)_{N,T} }
    \end{aligned}\)   
\end{minipage}
\begin{minipage}[t]{.56\textwidth}
    \setlength{\parindent}{.5cm}
    \scriptsize
    An isotherm \((V,P)\) is either a monotonic decrease like an exponential curve for \(T \geq T_c\) (one-to-one), 
    or like a mexican-hat curve for smaller \(T < T_c\) (not one-to-one). The mexican-hat curve could be misinterpreted
    since it looks like decreasing volume will start decreasing pressure at the hill/peak of the curve. Graphing 
    \((G,P)\) parametrically with decreasing \(V\), one sees that the line does a triangular loop and intersects itself 
    at the point of vapor pressure. Since the thermodynamic stable state is at the lowest \(G\), the system will skip 
    the loop (representing unstable states). On the PV diagram, this corresponds to a flat, straight line at the 
    vapor pressure between the curve's two critical points (in the differential sense) - the peak and trough - 
    connecting a higher volume to a lower volume (the mexican-hat is not one-to-one). This sharp decrease in volume at 
    the same pressure describes a phase change from higher volume gas to lower volume liquid at vapor pressure. The 
    curve between these two volumes represents the thermodynamic state if the fluid were homogenous. But these ``are
    unstable, since there is always another state (gas or liquid) at the same pressure with a lower Gibbs
    free energy.''

    \vspace{10pt}

    \underline{Maxwell Construction}:
    To find the vapor pressure without graphing \((G,P)\) and finding the intersection, note that
    \[0 = \int_\text{loop} dG = \int_\text{loop} \left( \tfrac{\partial G}{\partial P} \right)_{N,T} dP
        = \int_\text{loop} V dP.\] 
\end{minipage}

\vspace{15pt}
{\scriptsize \noindent
On the \((P,V)\) graph, this corresponds to the summing the area between the 
mexican-hat curve and the flat vapor pressure line. The area is divided in two: one beneath the curve's hill/peak 
(positive area) and above the line, and the other beneath the line and above the curve's valley/trough (negative area). The line
should be chosen so that the sum of the two is zero. 

\vspace{5pt}

Such a line can't be made when the isotherm is in a monotonic decrease (or increase at a decreasing volume) when 
the temperature exceeds the critical point, \(T \geq T_c\). Here, decreasing volume consistently means an 
increasing pressure, and there is no loop in the \((G,P)\) parametric graph. Above \(T_c\), there is no phase 
change between liquid and gas, and there is little distinction between them.

\vspace{5pt}

The model is good qualitatively, but not quantitatively in practice.

}



%----------------------------------------------------------------------------------------------------------------------------
%----------------------------------------------------------------------------------------------------------------------------
%----------------------------------------------------------------------------------------------------------------------------
%----------------------------------------------------------------------------------------------------------------------------
\newpage
% Helmholtz Free Energy
\section{Helmholtz Free Energy}

% Definition
\noindent
\(\begin{aligned}[t]
    &\text{\underline{Helmholtz Free Energy}}: \ \boxed{ F \equiv U - TS } \\[5pt]
    &\boxed{ \begin{aligned}
        dF &= dU - T\ dS - S\ dT\\[5pt]
        &= - S\ dT - P\ dV + \sum \mu_i\ dN_i + \dots
    \end{aligned} }
\end{aligned}\)
\hspace{30pt}
% Maxwell Identity
\(\begin{aligned}[t]
    &\text{\underline{Maxwell Identity}:} \ \ \
        \frac{\partial}{\partial V} \left( \frac{\partial F}{\partial T} \right)
        = \frac{\partial}{\partial T} \left( \frac{\partial F}{\partial V} \right)\\[5pt]
    &\rightarrow\ \boxed{ \left( \frac{\partial S}{\partial V} \right)_{N,T} 
        = \left( \frac{\partial P}{\partial T} \right)_{N,V} } 
        \indent \text{\scriptsize(other iden. with \(\mu\),\(N\))}
\end{aligned}\)

% Isothermic case
\vspace{10pt}
\subsection{iT}
\(\begin{aligned}
    \Delta F &= \Delta U - T \Delta S\\[5pt]
    &= W + Q - T \Delta S\\[5pt]
    &= W_\text{cpr} + W_\text{other} - (W_\text{other} - \sum \mu_i \Delta N_i + \dots)\\
    &= W_\text{cpr} + \sum \mu_i \Delta N_i - \dots
\end{aligned}
\hspace{1cm} \rightarrow \hspace{1cm} \boxed{ \begin{aligned}
    dF \ \ &\leq \ \ \delta W\\
    &\text{\scriptsize(and)}\\
    dF \ \ &\leq \ \ \delta W_\text{cpr} + \sum \mu_i\ dN_i
\end{aligned} } \)

% iVT
\vspace{5pt}
\subsection{iVT}
\(\begin{aligned}
    \Delta F &= \sum \mu_i \Delta N_i - \dots
\end{aligned}
\hspace{1cm} \rightarrow \hspace{1cm} \boxed{ \begin{aligned}
    dF \ \ &\leq \ \ \sum \mu_i\ dN_i
\end{aligned} } \)

% iNVT
\vspace{5pt}
\subsection{iNVT}
\(\begin{aligned}
    dS_\text{total} &= dS + dS_R = dS + \frac{dU_R}{T_R} + \cancel{\frac{P\ dV - \sum \mu_i N_i}{T}_R}\\[5pt]
    &= dS - \frac{dU}{T} = -\frac{1}{T} (dU - T\ dS)\\[5pt]
    \Aboxed{ dS_\text{total}\uparrow &= - \frac{dF\downarrow}{T} }
    \indent\indent F\downarrow = \boxed{U\downarrow} - TS\uparrow
\end{aligned}\)

% iNV
\vspace{5pt}
\subsection{iNV (see Canonical Ensemble)}
% Proof
\(\begin{aligned}[t]
    &1.)\ \tilde{F} \equiv -k T \ln{Z}:\\[5pt]
    &\begin{aligned}
        \left( \tfrac{\partial \tilde{F}}{\partial T} \right)_{V,N} 
            &= -k \ln{Z} - k T\ \frac{\partial \beta}{\partial T} \frac{\partial \ln{Z}}{\partial \beta}\\
            &= \frac{\tilde{F}}{T} - k T \left( \frac{-1}{k T^2} \right) \frac{1}{Z} \frac{\partial Z}{\partial \beta}\\
            &= \frac{\tilde{F}}{T} - \frac{\overline{E}}{T} \equiv \frac{\tilde{F}}{T} - \frac{U}{T}
                \ \ \ \text{\scriptsize(\underline{\(U\) is now \(\overline{E}\)?})}\\
    \end{aligned}
\end{aligned}\)
\indent
\(\begin{aligned}[t]
    2.)\ \left( \tfrac{\partial F}{\partial T} \right)_{V,N} &= -S \\[5pt]
    &= \frac{F-U}{T}
\end{aligned}\)
\indent
\(\begin{aligned}[t]
    3.)\ &T = 0\\[5pt]
    &\begin{aligned}
        \tilde{F}(0) &= -kT \ln{Z(0)} \\[5pt]
        &= -kT \ln{ e^{-U_0 / kT} }\\[5pt]
        &= U_0 = F(0)
    \end{aligned}
\end{aligned}\)

\vspace{15pt}\noindent
\(4.)\ \Rightarrow \ \boxed{ F = - k_b T \ln{Z} \ \ \Leftrightarrow\ \ Z(T) = e^{- \beta F} }\)

%-----------------------------------------------------------------------------------------------------------------------------------
%-----------------------------------------------------------------------------------------------------------------------------------
%-----------------------------------------------------------------------------------------------------------------------------------
%-----------------------------------------------------------------------------------------------------------------------------------
% Canonical Ensemble
\newpage
\section{Canonical Ensemble (NVT)}

\noindent
% Single Particle
\(\begin{aligned}[t]
    &\text{\underline{Single Particle}:}\\
    &\text{\scriptsize - Particle Energy, \(E\)}\\[-7pt]
    &\text{\scriptsize - Ext. Heat Reservoir Energy, \(U_R\)}\\[1pt]
    &\text{-\ }U_R{\scriptstyle(s_1)} + E{\scriptstyle(s_1)} = U_R{\scriptstyle(s_2)} + E{\scriptstyle(s_2)}
\end{aligned}\)
\hspace{1cm}
% Boltzmann Factor Motivation
\(\begin{aligned}[t]
    % Motivation
    &\\[-18pt]
    &\frac{ P{\scriptstyle(s_2)} }{ P{\scriptstyle(s_1)} } 
        = \frac{ \Omega{\scriptstyle(s_2)} }{ \Omega{\scriptstyle(s_1)} }
        = \frac{d{\scriptstyle(s_2)}\ e^{S_R(s_2) / k}}{d{\scriptstyle(s_1)}\ e^{S_R(s_1) / k}}\\[10pt]
    &\Rightarrow \ \frac{ P{\scriptstyle(s_2)} }{ P{\scriptstyle(s_1)} } 
        = \frac{ d{\scriptstyle(s_2)}\ e^{\beta U_R(s_2)} }{ d{\scriptstyle(s_1)}\ e^{\beta U_R(s_1)} }
        = \frac{ d{\scriptstyle(s_2)}\ e^{-\beta E(s_2)} }{ d{\scriptstyle(s_1)}\ e^{-\beta E(s_1)} }
        \hspace{5pt} , \hspace{10pt} \boxed{ \beta \equiv \tfrac{1}{k_b T} }\\[10pt]
    % Boltzmann Factor
    &\Rightarrow \ \text{\underline{Boltzmann Factor}}: \ \boxed{ e^{-\beta E_i} } \sim \Omega_i(U) = e^{S(U)/k_b}
\end{aligned}\)

\vspace{15pt}\noindent
% Partition Function
\underline{Partition Function}:\\[5pt]
\(\begin{aligned}[t]
    % Single Particle/DOF
    &1.)\ \begin{aligned}[t]
        &\text{{Single Particle/DOF}}: \\
        &\boxed{ Z(T) \equiv \sum_{s_i} d_i\ e^{-E_i / k_b T} }\\
        &\text{\scriptsize(DOF like \(x,y,z\) directions, rot., vib., etc.)}
    \end{aligned}\\[10pt]
    % Average Energy
    &2.)\ \begin{aligned}[t]
        &\text{{Energy State Probability}}:\\
        &\boxed{ P{\scriptstyle(s_i)}= \frac{d_i\ e^{-E_i / k_b T}}{Z} }
    \end{aligned}
\end{aligned}
\hspace{.75cm}
\begin{aligned}[t]
    % Multiple Particles Distinguishable
    &3.)\ \begin{aligned}[t]
        &\text{{Nonint., Mult. Dist. Particles/DOF}}: \\
        &\boxed{ Z^{(N)} = Z_1 Z_2 \dots Z_N = \prod_i^N Z_i }
    \end{aligned}\\[20pt]
    % Indistinguishable
    &4.)\ \begin{aligned}[t]
        &\text{{Nonint., Mult. Indist. Particles}}: \\
        &\boxed{ Z^{(N)} \approx \frac{1}{N!} Z_1^N} \ \ \
            \begin{gathered}
                \text{\scriptsize(same-state particles are overcounted,} \\[-5pt]
                \text{\scriptsize but are less likely in non-dense systems)}            
            \end{gathered}\\[5pt]
        &\begin{aligned}[t]
                \mu = -kT \tfrac{\partial}{\partial N} \ln{Z^{(N)}}
                    \ &\approx\ -kT \tfrac{\partial}{\partial N} (N \ln{Z_1} -N\ln{N} + N)\\[5pt]
                \Aboxed{ \mu &= -k_b T \ln{\tfrac{Z_1}{N}} }
            \end{aligned} 
    \end{aligned}
\end{aligned}\)

% Average Energy
\vspace{20pt}\noindent
\(\begin{aligned}[t]
    &\text{\underline{Average Energy}}:\\[5pt]
    &1.)\ \boxed{ \overline{E} \equiv \sum_s E{\scriptstyle(s)} P{\scriptstyle(s)} \equiv U_1}\\[5pt]
    &2.)\ \boxed{ \overline{E}, U^{(N)} = - \frac{1}{Z} \frac{\partial Z}{\partial \beta} 
        = - \frac{\partial \ln{Z}}{\partial \beta} = \frac{\partial (\beta F)}{\partial \beta} }
\end{aligned}
\hspace{1cm}
% Helmholtz
\begin{aligned}[t]
    &\text{\underline{Characteristic State Function, \(F\) (see \(F\))}}:\\[5pt]
    &1.)\ \boxed{ \begin{gathered}[t]
            [F = -k_b T \ln{Z}] \leftrightarrow \big[ -\beta F = \ln{Z} \big]\\[2pt]
            \leftrightarrow \Big[ Z = e^{-\beta F} \Big] 
        \end{gathered} }\\[5pt]
    &2.)\ \boxed{ dF = -S dT - P dV + \mu dN } \\[5pt]
    &3.)\ \boxed{ S = \tfrac{\partial(kT \ln{Z})}{\partial T} }
\end{aligned}\)

% % Line Divider
% \vspace{15pt}\noindent
% \rule{1\textwidth}{.5pt}

%----------------------------------------------------------------------------------------------------------------------------------
\newpage

% Equipartition Theorem Proof
\subsection{Equipartition Theorem Proof}
\(d_q = 1 \ , \ \ E_q = cq^2\)\\[10pt]
\(\begin{aligned}[t]
    1.) \ \ Z = \sum_q e^{-\beta cq^2} &\approx \frac{1}{\Delta q} \int_0^\infty e^{-\beta cq^2}\ dq\\[5pt]
    &= \frac{1}{\Delta q \sqrt{\beta c}} \int_0^\infty e^{-q^2}\ dq \\[5pt]
    &= \frac{\sqrt{\pi} / 2}{\Delta q \sqrt{c}}\ \beta^{-1/2}
\end{aligned}
\indent \indent \indent \begin{gathered}[t]
    2.)\ \ \overline{E} = - \frac{1}{Z} \frac{\partial Z}{\partial \beta} = \tfrac{1}{2} k_b T \ \ \ \checkedbox\\[5pt]
    \begin{minipage}{4cm}
        \begin{center}
            \({\scriptstyle(kT\ \gg\ \Delta q)}\)\\[5pt]
            \scriptsize not good for quantum systems\\
            or low temp.
        \end{center}
    \end{minipage}
\end{gathered}\)

% Diatomic Gas Example
\vspace{10pt}
\subsection{Diatomic Gas/Rotation}
% Initial Info
\(\begin{aligned}
    d_j &= 2j + 1 \ , & \quad E_j &= j(j+1) \epsilon \indent {\scriptstyle(\epsilon\ \sim\ 1/2I)}
\end{aligned}\)

% Distinguishable atoms
\vspace{15pt}\noindent
\begin{minipage}[t]{.45\textwidth}
    \underline{Dist. Atoms (Like CO, \ \(\epsilon = 2.4\)E-4\ eV)}\\[5pt]
    \(\begin{aligned}[t]
        % Partition Function
        &1.)\ \ Z_\text{rot} \begin{aligned}[t]
            &= \sum d_j\ e^{-\beta E_j} \\[5pt]
            &\approx \displaystyle \int d_j\ e^{-\beta E_j}\ dj = \frac{\beta}{\epsilon} 
            \indent {\scriptstyle(\beta\ \gg\ \epsilon)}
        \end{aligned}\\[10pt]
        % Internal Energy
        &2.)\ \ \overline{E}_\text{rot} = - \frac{1}{Z_\text{rot}} \frac{\partial Z_\text{rot}}{\partial \beta} 
            = k_b T \ \ \ \ \text{\scriptsize(ET)}\ \checkedbox\\[10pt]
        % 3rd Law Check for C_V
        &3.)\ \begin{aligned}[t]
            &\lim_{T \rightarrow 0}\ Z_\text{rot} = 1
                + 3 e^{-2\epsilon / kT} + \dots\\[10pt]
            &\Rightarrow \ \lim_{T \rightarrow 0}\ C_\text{rot}(T) 
                = \lim_{T \rightarrow 0}\ \frac{\partial \overline{E}}{\partial T} = 0 \ \ \ \checkedbox
        \end{aligned}
    \end{aligned}\) 
\end{minipage}
% Indistringuishable atoms
\begin{minipage}[t]{.53\textwidth}
    \underline{Indist. Atoms (H2 Example, \ \(\epsilon = 7.6\)E-3\ eV)}\\[5pt]
    \(\begin{aligned}[t]
        % Partition Function
        &1.) \ \ Z_\text{rot} \approx \frac{\beta}{2 \epsilon} 
            \indent {\scriptstyle(\beta\ \gg\ \epsilon)}\\[10pt]
        % Internal Energy
        &2.)\ \ \overline{E}_\text{rot} = - \frac{1}{Z_\text{rot}} \frac{\partial Z_\text{rot}}{\partial \beta} 
            = k_b T \ \ \ \ \text{\scriptsize(ET)}\ \checkedbox\\[10pt]
        % 3rd Law Check for C_V
        &3.) \begin{aligned}[t]
            &\ \lim_{T \rightarrow 0} Z_\text{rot} = \sum d_j\ e^{-\beta E_j} \
                \begin{cases}
                    \begin{gathered}
                        {\scriptstyle j = 2n}\\
                        \text{\scriptsize(sym.)}
                    \end{gathered}  & \begin{gathered}
                            \text{\scriptsize para-H\(_2\)}\\[0pt]
                            \text{\scriptsize(sing. nuclei)}
                        \end{gathered}\\[15pt]
                    \begin{gathered}
                        {\scriptstyle j = 2n+1}\\
                        \text{\scriptsize(anti-sym.)}
                    \end{gathered}  & \begin{gathered}
                        \text{\scriptsize ortho-H\(_2\)}\\[0pt]
                        \text{\scriptsize(trip. nuclei)}
                    \end{gathered}
                \end{cases}\\[10pt]
            &\Rightarrow \ \text{Graph}\ \frac{C_\text{rot}{\scriptstyle(kT / \epsilon)}}{k}\ \text{with a few terms}
        \end{aligned}
    \end{aligned}\)
\end{minipage}

% Einstein Solid Oscillator
\vspace{10pt}
\subsection{Einstein Solid/Oscillator/Vibration}

% Partition Functions
\vspace{5pt}
\(
    % Full Harmonic Partition
    \begin{aligned}
        &1.)\ \ E_n = \hbar \omega (\tfrac{1}{2} + n)\\[5pt]
        &2.)\ \ Z_1 = \sum_n e^{-\beta (n + 1/2)\hbar \omega} 
            = \frac{e^{-\beta \hbar \omega / 2}}{1-e^{-\beta \hbar \omega}}
    \end{aligned} 
    \ \ \ \rightarrow \ \ \
    % Ignoring Ground State Energy Partition
    \begin{aligned}
        \Delta E_n &= n \hbar \omega \ \ \ \ \ \text{\scriptsize(only care about exchangable energy)}\\[5pt]
        Z_1^\Delta &= \sum_n e^{-\beta \Delta E} = \frac{1}{1-e^{-\beta \hbar \omega}}
    \end{aligned}
\)

% Potential Energy
\vspace{10pt}\noindent
\(
    \begin{aligned}
        3.) \ \ U^\Delta = N \overline{E} &= - \frac{N}{Z_1^\Delta} \frac{\partial Z_1^\Delta}{\partial \beta} 
            = \frac{N \hbar \omega e^{-\beta \hbar \omega}}{1-e^{-\beta \hbar \omega}}
            = \frac{N \hbar \omega}{e^{\beta \hbar \omega} - 1}\\[10pt]
        U &= - \frac{N}{Z_1} \frac{\partial Z_1}{\partial \beta}
            = \frac{N \hbar \omega}{e^{\beta \hbar \omega} - 1} 
            + \frac{N\hbar\omega}{2} e^{-\beta \hbar \omega/2}\\
    \end{aligned}
    \ \ \rightarrow \ \ C_V = Nk_bT \indent \checkedbox 
\)

%----------------------------------------------------------------------------------------------------------------------------------
\newpage

% Two-State Paramagnet
\subsection{Two-State Paramagnet}
\(\begin{aligned}[t]
    \big( - M B \equiv U \big) &= N \sum_s E(s) P(s) \\[5pt]
    &= N \big[ (-\mu B) P_\uparrow + (\mu B) P_\downarrow \big] 
        \ \ \ = \ \ \ \mu B N(1 - 2P_\uparrow)  \indent \text{\scriptsize(compare to Microcanonical)}\\[7pt]
    &= - \mu B N \left( \frac{ e^{\mu B / kT} - e^{-\mu B / kT} }{ e^{\mu B / kT} + e^{- \mu B / kT} } \right)
        = -\mu B N \tanh{ \frac{\mu B}{k T} } \ \ \ \checkedbox
\end{aligned}\)

% Ideal Gas
\vspace{10pt}
\subsection{Ideal Gas}
\(\begin{aligned}[t]
    % Single Particle Kinetic Energy Partition
    &1.)\ \ \begin{aligned}[t]
            Z_\text{kin} &= \left( \sum_n e^{-\beta E_\text{inf. sq. well}} \right)^3 
                \ \approx \ \left( \int_0^\infty e^{-\beta \frac{\hbar^2 k_n^2}{2m}} \ dn \right)^3
                = \left( \tfrac{1}{2} \int_{-\infty}^\infty e^{-\beta \frac{\hbar^2 (\pi / L)^2}{2m}} n^2 \ dn \right)^3\\[5pt]
            &= \left( \frac{\sqrt{\pi}}{2 \sqrt{\beta \frac{\hbar^2 (\pi / L)^2}{2m} }} \right)^3 
                = \left( \frac{1}{\sqrt{\pi}} \frac{L}{2\pi / \sqrt{2 m k T / \hbar^2} } \right)^3\\[5pt]
            &= \left( \frac{1}{\sqrt{\pi}} \frac{L}{h / \sqrt{2m k T}} \right)^3 
                = \left( \frac{1}{\sqrt{\pi}} \frac{L}{\lambda_\beta} \right)^3 
                \equiv \left( \frac{L}{l_Q} \right)^3 = \frac{V}{v_Q} \indent \indent\begin{gathered}
                    \text{\scriptsize(\(l_Q\), quantum length)}\\[-5pt]
                    \text{\scriptsize(\(v_Q\), quantum volume)}
                \end{gathered}
        \end{aligned}\\[10pt]
    % Multiple Particle Partition
    &2.)\ \ Z^{(N)} = \frac{(Z_\text{kin} Z_\text{int})^N}{N!} 
        = \frac{1}{N!} \left( \frac{V Z_\text{int}}{v_Q} \right)^N \ \ \Rightarrow \ \ 
        \ln{Z^{(N)}} \approx N \big( \ln{V} + \ln{Z_\text{int}} - \ln{v_Q} - \ln{N}  + 1 \big)\\[10pt]
    % Total Internal Energy
    &3.)\ \ U^{(N)} = - \frac{\partial \ln{Z^{(N)}}}{\partial \beta} 
        = \tfrac{3}{2} NkT + NU_\text{int}^{(1)} \indent\indent \checkedbox
        \indent \text{\scriptsize(rot. and vib. each add an \(Nk\) to the \(C_V\))}\\[10pt]
    % Helmholtz Free Energy
    &4.)\ \ \begin{aligned}[t]
            F^{(N)} = -kT \ln{Z^{(N)}} &= -NkT \big( \ln{V} + \ln{Z_\text{int}} - \ln{v_Q} - \ln{N}  + 1 \big)\\[10pt]
            &= -NkT \big( \ln{V} - \ln{v_Q} - \ln{N}  + 1 \big) + NF_\text{int}^{(1)}
        \end{aligned}\\[5pt]
    % Ideal Gas Law
    &5.)\ \ P = - \left( \tfrac{\partial F^{(N)}}{\partial V} \right)_{T,N} 
        = \tfrac{N k_b T}{V} \indent\indent \checkedbox\\[5pt]
    % IG Chemical Potential 
    &6.)\ \left( \tfrac{\partial F}{\partial N} \right)_{T,V} 
        = \boxed{ \mu = -k_b T \ln{\left( \tfrac{V Z_\text{int}}{N v_Q} \right)}
        = -kT \ln{\left( \tfrac{Z_1}{N} \right)} }
\end{aligned}\)

%----------------------------------------------------------------------------------------------------------------------------------
%----------------------------------------------------------------------------------------------------------------------------------
%----------------------------------------------------------------------------------------------------------------------------------
%----------------------------------------------------------------------------------------------------------------------------------
\newpage
% Grand Canonical Ensemble
\section{Grand Canonical Ensemble (\(\boldsymbol{\mu}\)VT)}

% Intro
\(\begin{aligned}
    &\underline{\text{Gibb's Factor}}: \ \boxed{ e^{- \beta [ E(s) - \mu N(s)] } 
        \ \ \rightarrow\ \ e^{- \beta n_s(\epsilon_s - \mu_s) } = e^{-nx} }
        \hspace{1cm} \begin{gathered}[b]
            {\scriptstyle n \ \equiv\ \text{\scriptsize part. per state,}\ s}\\[-7pt]
            {\scriptstyle x \ \equiv\ \beta(\epsilon - \mu)}
        \end{gathered}\\[5pt]
    &\underline{\text{Grand Partition Function}}: \ 
        \boxed{ \mathcal{Z} = \sum_s e^{- \beta [ E(s) - \sum_i \mu_i N_i (s)] } 
        \ \ \rightarrow\ \ \sum_{s} e^{- \beta n_s (\epsilon_s - \mu_s) } = \sum_{s} e^{- n x } }\\[5pt]
    &\underline{\text{Grand Potential}}: \ \boxed{ \Phi \equiv U - TS - \mu N = -kT\ln{\mathcal{Z}} }\\[5pt]
    &\underline{\text{Average Particles Per State}}: \ \ 
        \sum_n n P_n = \sum_n n \frac{e^{-nx}}{\mathcal{Z}} 
        = - \frac{1}{\mathcal{Z}} \sum_n \tfrac{\partial}{\partial x} e^{-nx}
        = \boxed{ \overline{n} = - \frac{1}{\mathcal{Z}} \frac{\partial \mathcal{Z}}{\partial x} }\\[5pt]
    &\underline{\text{Average Total Particles}}: \ \ 
        \boxed{ \overline{N} = \sum_i d{\scriptstyle(\epsilon_i)} \overline{n}{\scriptstyle(\epsilon_i)} 
        \ \approx \int \rho{\scriptstyle(\epsilon)} \overline{n}{\scriptstyle(\epsilon)} \ d\epsilon} 
        \indent \begin{gathered}
            \text{\scriptsize(\(d{\scriptstyle(\epsilon_i)}\) degeneracies)}\\[-5pt]
            \text{\scriptsize(\(\rho{\scriptstyle(\epsilon)}\) state density)}
        \end{gathered}
\end{aligned}\)

% Line Divider
\vspace{10pt}\noindent
\rule{1\textwidth}{.5pt}

% O2 and CO Example
\subsection{Example: O\(_2\) and CO Bonding to Hemaglobin}
\vspace{5pt}
% Variables
\(1.)\ \text{Variables}:\)

\vspace{5pt}
\(\begin{aligned}[t]
    &\epsilon_{\text{O}_2} \ \approx \ -0.7 \ \text{eV}\\[5pt]
    &\left.\begin{aligned}
            T &= 310 \ \text{K}\\[5pt]
            P_{\text{O}_2} &= .2 \ \text{atm} \\[5pt]
        \end{aligned}\right\}
        \ \rightarrow \ \begin{aligned}
            \mu_{\text{O}_2} &= -k_b T \ln{\left( \tfrac{V Z_\text{int}}{N v_Q} \right)}\\[5pt]
            &\approx -0.6 \ \text{eV}\\[5pt]
        \end{aligned}
\end{aligned}
\hspace{1.2cm}
\begin{aligned}[t]
    &\epsilon_{\text{CO}} \ \approx \ -0.85 \ \text{eV}\\[5pt]
    &P_{\text{CO}} = P_{\text{O}_2} / 100 
        \ \rightarrow \ \begin{aligned}[t]
            \mu_{\text{CO}} \ &\approx\  \mu_{\text{O}_2} - kT \ln{100} \\[5pt]
            &\approx \ -0.72 \ \text{eV}
        \end{aligned}
\end{aligned}\)

\vspace{20pt}\noindent
% O2 Alone
\begin{minipage}[t]{.43\textwidth}
    \setlength{\parindent}{.5cm}
    \noindent
    {2.)\ O\(_2\):}
    
    \vspace{10pt}
    \indent \(\mathcal{Z} = \begin{gathered}
        \text{\scriptsize(No Bonding)}\\[5pt]
        e^{ -\beta(0)(\epsilon - \mu) }\\[5pt]
        1
    \end{gathered} \ \ + \ \ \begin{gathered}
        \text{\scriptsize(O\(_2\) Bonding)}\\[5pt]
        e^{ -\beta(\epsilon_{\text{O}_2} - \mu_{\text{O}_2}) }\\[5pt]
        40
    \end{gathered}\)\\[20pt]
    \indent \(P(\text{\scriptsize O\(_2\) Bonding}) = \frac{40}{1 + 40} = 98\%\)
\end{minipage}
% CO and O2
\begin{minipage}[t]{.55\textwidth}
    \setlength{\parindent}{.5cm}
    \noindent
    {3.)\ O\(_2\) and CO:}
    
    \vspace{10pt}
    \indent \(\mathcal{Z} = \begin{gathered}
        \text{\scriptsize(No Bonding)}\\[5pt]
        e^{ -\beta(0)(\epsilon - \mu) }\\[5pt]
        1
    \end{gathered} \ \ + \ \ \begin{gathered}
        \text{\scriptsize(O\(_2\) Bonding)}\\[5pt]
        e^{ -\beta(\epsilon_{\text{O}_2} - \mu_{\text{O}_2}) }\\[5pt]
        40
    \end{gathered} \ \ + \ \ \begin{gathered}
        \text{\scriptsize(CO Bonding)}\\[5pt]
        e^{ -\beta(\epsilon_{\text{CO}} - \mu_{\text{CO}}) }\\[5pt]
        120
    \end{gathered}\)\\[20pt]
    \indent \(P(\text{\scriptsize O\(_2\) Bonding}) = \frac{40}{1 + 40 + 120} = 25\%\)
\end{minipage}

%-------------------------------------------------------------------------------------------------------------------------
\newpage

% Fermi-Dirac Distribution for Fermions Particles
\subsection{Fermi-Dirac Distribution (Fermions)}
\(\begin{aligned}
    \overline{n} &= \frac{ (0) e^{-\beta(0)(\epsilon-\mu)}
        + (1) e^{-\beta (1)(\epsilon-\mu)} }{ e^{-\beta(0)(\epsilon-\mu)} 
        + e^{-\beta (1)(\epsilon-\mu)} }\\[5pt]
    &= \frac{e^{-\beta (\epsilon-\mu)}}{ 1 + e^{-\beta (\epsilon-\mu)} }\\[5pt]
    \Aboxed{ \overline{n}(\epsilon) &= \frac{1}{ e^{\beta (\epsilon-\mu)} + 1 } }
\end{aligned}
\hspace{.75cm}
\begin{cases}
    \epsilon \ll \mu  & \hspace{10pt} \overline{n} \rightarrow 1\\[5pt]
    \epsilon = \mu  & \hspace{10pt} \overline{n} = 1/2 \ \ \ \Rightarrow\\[5pt]
    \epsilon \gg \mu  & \hspace{10pt} \overline{n} \rightarrow 0_+
\end{cases}
\hspace{.4cm}
\left\{\begin{aligned}
    \lim_{T \rightarrow 0} \
        &\left.\frac{\partial \overline{n}}{\partial \epsilon}\ \right|_\mu \ =\ -\infty\\[5pt]
    \lim_{T \rightarrow \infty} \
        &\left.\frac{\partial \overline{n}}{\partial \epsilon}\ \right|_\mu \ =\ 0
\end{aligned}\right.\)

% Total Average Number
\vspace{10pt}\noindent
\(\begin{aligned}
    \boxed{ \overline{N} = \sum d_\epsilon \overline{n}\ d\epsilon 
        = \int \rho_\epsilon \overline{n}\ d\epsilon }
\end{aligned} 
\hspace{10pt}
\Rightarrow
\hspace{10pt}
\begin{aligned}
    \boxed{ \lim_{T \rightarrow0} \overline{N} = \int_0^\infty \rho_\epsilon \overline{n}\ d\epsilon 
        = \int_0^\mu \rho_\epsilon\ d\epsilon }
\end{aligned}\)

% Bose-Einstein Distribution for Bosons Particles
\vspace{5pt}
\subsection{Bose-Einstein Distribution (Bosons)}
\(\begin{aligned}
    \mathcal{Z} &= 1 + e^{-\beta(1)x} + e^{-\beta(2)x} + \dots \\[5pt]
    &= \frac{1}{1-e^{-\beta x}} \indent {\scriptstyle(\epsilon \ \geq\ \mu)}\\
    &\downarrow\\
    \Aboxed{ \overline{n} &= \frac{1}{ e^{\beta (\epsilon-\mu)} - 1 } }
\end{aligned}
\hspace{1cm}
\begin{cases}
    \epsilon = \mu_+  & \hspace{10pt} \overline{n} \rightarrow \infty\\[5pt]
    \epsilon \gg \mu  & \hspace{10pt} \overline{n} \rightarrow 0_+ 
\end{cases}\)

% Total Average Number
\vspace{10pt}\noindent
\(\begin{aligned}
    \boxed{ \overline{N} = \sum d_\epsilon \overline{n}\ d\epsilon 
        = \int \rho_\epsilon \overline{n}\ d\epsilon }
\end{aligned} 
\hspace{10pt}
\Rightarrow
\hspace{10pt}
\begin{aligned}
    \boxed{ \lim_{T \rightarrow0} \overline{N} = ? \text{\scriptsize(split between ground state and higher states)} }
\end{aligned}\)

% Maxwell-Boltzmann Distribution for Distinguishable Particles
\vspace{5pt}
\subsection{Maxwell-Boltzmann Distribution (Distinguishable)}
\vspace{10pt}\noindent
\begin{minipage}{.53\textwidth}
    \(\begin{aligned}
        \overline{n} &= NP_1(\epsilon) \\[5pt]
        &= \frac{N e^{-\beta \epsilon}}{Z_1} = \frac{N e^{-\beta \epsilon}}{N e^{-\beta\mu}}\\[5pt]
        \Aboxed{ \overline{n} &= e^{-\beta(\epsilon-\mu)} }
    \end{aligned}
    \hspace{.5cm}
    \begin{cases}
        \epsilon \ll \mu  & \hspace{10pt} \overline{n} \rightarrow \infty\\[5pt]
        \epsilon = \mu  & \hspace{10pt} \overline{n} = 1 \\[5pt]
        \epsilon \gg \mu  & \hspace{10pt} \overline{n} \rightarrow 0_+
    \end{cases}\)    
\end{minipage}
% Condition to Use MBD approx.
\begin{minipage}{.45\textwidth}
    \underline{FDD, BED \(\rightarrow\) MBD Condition}:\\[10pt]
    \(\begin{aligned}
        &\bullet\ \beta(\epsilon - \mu) \ \gg\ (1>0)\\[10pt]
        &\rightarrow \ \Big[ -kT + \epsilon \approx -kT \Big]
            \gg \Big[ \mu = -kT \ln{\tfrac{Z_1}{N}} \Big]\\[5pt]
        &\rightarrow \ \boxed{ Z_1 \gg Ne > N }
    \end{aligned}\)
\end{minipage}

% Total Average Number
\vspace{10pt}\noindent
\(\begin{aligned}
    \boxed{ \overline{N} = \sum d_\epsilon \overline{n}\ d\epsilon 
        = \int \rho_\epsilon \overline{n}\ d\epsilon }
\end{aligned}\)

\end{document}
